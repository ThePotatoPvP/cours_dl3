\documentclass[french]{report}
\usepackage{../exercices}

\begin{document}

\begin{center}
    \huge{\textbf{S6-Algebre - TD2}}
\end{center}

\section*{Extension de corps, éléments algébriques}

\begin{exo}
    Voici trois nombres : \(\sqrt{2}, \sqrt{4+2\sqrt{2}}, \sqrt[4]{2}\)
    \begin{q}{1}
        Montrer que ces nombres sont algébriques sur \(\Q(\sqrt{2})\) et donner leur
        polynôme minimal.
        \boxans{Les polynômes \(X-\sqrt{2},\ X^2-(4+2\sqrt{2})\) et \(X^2-\sqrt{2}\). admettent les nombres en racine et sont
        irréductibles, ce sont donc leur minimaux sur \(\Q(\sqrt{2})\).}
    \end{q}
    \begin{q}{2}
        Même question pour \(\Q\).
        \boxans{Les polynômes \(X^2-2,\ X^4-(4+2\sqrt{2})\) et \(X^4-2\). admettent les nombres en racine et sont
        irréductibles, ce sont donc leur minimaux sur \(\Q(\sqrt{2})\).}
    \end{q}
\end{exo}

\begin{exo}
    Le corps des fractions \(\R(X)\) de \(\R[X]\) est une extension (infinie) de \(\Q\).
    \begin{q}{1}
        Donner un élément algébrique sur \(\Q\) dans \(\R(X)\backslash\Q\)
        \boxans{L'élément \(\sqrt{2}\) fonctionne.}
    \end{q}
    \begin{q}{2}
        Donner un élément non algébrique sur \(\Q\) de \(\R(X)\).
        \boxans{On choisit \(e\) qui est notre nombre transcendant préféré.}
    \end{q}
\end{exo}

\begin{exo}
    Autour de la clotûre.
    \begin{q}{1}
        Le corps \(\C(X)\) est-il algébriquement clos ?
        \boxans{Non, ce corps n'est pas algébriquement clos, soit \(P(Y)=Y^2-X\) à
        coefficients dans \(\C(X)\) ses racines sont \(\pm\sqrt{X}\) qui ne sont pas
        dans \(\C(X)\).}
    \end{q}
    \begin{q}{2}
        Un corps fini peut-il être algébriquement clos ?
    \end{q}
\end{exo}

\begin{exo}
    Soit \(\alpha\in\C\) un nombre algébrique et \(P\) son
    minimal sur \(\Q\).
    \begin{q}{1}
        Montrer que \(\alpha\) est une racine simple de \(P\) dans \(\C\).
    \end{q}
    \begin{q}{2}
        Montrer que \(P\) n'a que des racines simples dans \(\C\).
    \end{q}
\end{exo}

\begin{exo} On bosse un peu maintenant.
    \begin{q}{1}
        Montrer que \(\C\) est une clotûre algébrique de \(\R\).
    \end{q}
    \begin{q}{2}
        Montrer que le corps \(\C\) n'a pas d'extension finie non triviale.
    \end{q}
    \begin{q}{3}
        Montrer que les seules extensions finies du corps \(\R\) sont \(\R\) et \(\C\)*
        à isomorphisme près.
    \end{q}
    \begin{q}{4}
        Montrer que \(\C\) n'est pas une clotûre algébrique de \(\Q\).
    \end{q}
\end{exo}

\section*{Corps de rupture, degrés d'extension, corps de décomposition}

\begin{exo}
    On considère \(K=\Q[X]/(P)\) où \(P=X^3+3X+3\), et on note \(\alpha\)
    la classe de \(X\) dans \(K\).
    \begin{q}{1}
        Expliquer pourquoi \(K\) est un corps.
    \end{q}
    \begin{q}{2}
        Déterminer \(k\) tel que \(\left(1,\alpha,\dots,\alpha^k\right)\) soit
        une base de \(\Q\) espace vectoriel \(K\).
    \end{q}
    \begin{q}{3}
        Exprimer \(a^{-1}\) dans cette base.
    \end{q}
\end{exo}

\begin{exo}
    Soit \(P=\sum^n a_iX^i\), où \(a_n\neq0\), un polynôme irréductible sur un corps \(K\).
    \begin{q}{1}
        Montrer que \(Q = \sum a_{n-i}X^i\) est irréductible sur \(K\).
    \end{q}
    \begin{q}{2}
        Montrer que \(K[X]/(P)\) et \(K[X]/(Q)\) sont isomorphes.
    \end{q}
\end{exo}

\begin{exo}
    Montrer que si \([L\colon K]\) est un nombre premier, alors la \(K\)-extension
    \(L\) est de la forme \(K(\alpha)\).
\end{exo}

\begin{exo}
    Soit \(L\) une extension d'un corps \(K\) et \(\alpha,\beta\in L\) deux éléments
    algébriques sur \(K\).
    \begin{q}{1}
        Montrer que si les degrés d'extension \([K(\alpha)\colon K]\) et \([K(\beta)
        \colon K]\) sont premiers entre eux, alors \([K(\alpha,\beta)\colon K]=
        [K(\alpha)\colon K][K(\beta)\colon K]\)
    \end{q}
    \begin{q}{2}
        A-t-on toujours \(K(\alpha\beta)=K(\alpha,\beta)\) lorque les degrés sont
        premiers entre eux ?
    \end{q}
\end{exo}


\begin{exo}
    Atour des racines de \(2\).
    \begin{q}{1}
        Déterminer pour tout enter \(n\geq 2\) le polynôme minimal de \(\sqrt[n]{2}\)
        sur \(\Q\) et le degré \([\Q(\sqrt[n]{2}):\Q]\).
    \end{q}
    \begin{q}{2}
        En déduire pour tous \(k,n\geq 2\) entiers, que l'on a \(\sqrt[k]{2}\in\Q
        (\sqrt[n]{2})\) si et seulement si \(k\) divise \(n\).
    \end{q}
\end{exo}

\begin{exo}
    Soit \(\alpha=\sqrt{2}+\sqrt{3}\).
    \begin{q}{1}
        Trouver une expression simple de \(\alpha+\alpha^{-1}\) et en déduire
        que \(\Q(\sqrt{3})\subset\Q(\alpha)\).
    \end{q}
    \begin{q}{2}
        Montrer par l'absure que \(\sqrt{2}\notin\Q(\sqrt{3})\) en supposant que l'on
        peut écrire \(\sqrt{2}\) sous la forme \(a+b\sqrt{3}\) avec \(a,b\in\Q\).
        En déduire le degré de l'extension \([\Q(\alpha):\Q(\sqrt{3})]\).
    \end{q}
    \begin{q}{3}
        Déterminer \([Q(\alpha):\Q]\).
    \end{q}
    \begin{q}{4}
        Trouver dans \(\Q[X]\) un polynôme unitaire de degré \(4\) ayant \(\alpha\)
        comme racine et déduire de la question précédente qu'il s'agit nécessairement
        d'un polynôme minimal de \(\alpha\) sur \(\Q\).
    \end{q}
\end{exo}

\begin{exo}
    Du fun avec les complexes.
    \begin{q}{1}
        On pose \(j=e^{2i\pi/3}\). En s'inspirant de l'exercice précédent déterminer
        le degré des nombre algébriques suivants, sans chercher leur minimal sur \(\Q\):
        \(j\sqrt{2}, i+\sqrt{2}, j+\sqrt{3}, i+j\).
    \end{q}
    \begin{q}{2}
        Expliciter leur polynôme minimal sur \(\Q\).
    \end{q}
\end{exo}

\begin{exo}
    Soit \(K=\Q(\sqrt{3}, i)\) et \(\xi=e^{i\pi/6}\)
    \begin{q}{1}
        Calculer \([K:\Q]\).
    \end{q}
    \begin{q}{2}
        Montrer que l'ensemble \(B=\{1,\sqrt{3}, i, i\sqrt{3}\}\) est une base de \(K\)
        en tant que \(\Q\)-espace vectoriel.
    \end{q}
    \begin{q}{3}
        Mettre \(\xi\) sous forme algébrique : \(\xi=a+ib\) avec \(a,b\in\R\), et conclure
        que \(\xi\in\K\).
    \end{q}
    \begin{q}{4}
        Montrer que \(K=\Q(\xi)\).
    \end{q}
    \begin{q}{5}
        Montrer que \(\xi\notin\Q(i\sqrt{3})\).
    \end{q}
\end{exo}

\begin{exo}
    Soit \(\zeta=e^{2i\pi/7}\in\C\).
    \begin{q}{1}
        Montrer que \(\zeta\) est algébrique sur \(\Q\) et déterminer \([\Q(\zeta):\Q]\).
    \end{q}
    \begin{q}{2}
        Déterminer le polynôme minimal de \(\zeta+\zeta^2+\zeta^4\).
    \end{q}
    \begin{q}{3}
        Montrer que \(\Q(\zeta+\zeta^2+\zeta^4)=\Q(i\sqrt{7})\).
    \end{q}
    \begin{q}{4}
        Déterminer le polynôme minimal de \(\zeta+\zeta^6\).
    \end{q}
\end{exo}

\begin{exo}
    Trouver un corps de rupture et un corps de décomposition de chacun des polynômes suivants.
    \begin{q}{1}
        \(X^2-3\in\Q[X]\)
    \end{q}
    \begin{q}{2}
        \(X^3-2\in\Q[X]\)
    \end{q}
    \begin{q}{3}
        \(X^4+X+1\in\mathbb{F}_2[X]\)
    \end{q}
    \begin{q}{4}
        \(X^p-T\in\mathbb{F}_p(T)[X]\)
    \end{q}
\end{exo}

\begin{exo}
    Soit \(P(X)=X^4+2X^2-4, \alpha=\sqrt{\sqrt{5}-1}, \beta=\sqrt{\sqrt{5}+1}\).
    On considère l'extension \(L=\Q(\alpha,\beta)\).
    \begin{q}{1}
        Donner les racines de \(P\) dans \(\C\) et montrer que \(L\) est un corps
        de décomposition de \(P\) sur \(\Q\).
    \end{q}
    \begin{q}{2}
        Calculer \(\alpha\beta\) et en déduire que \(L=\Q(\alpha, i)\).
    \end{q}
    \begin{q}{3}
        Montrer que \(\Q(\sqrt{5})\subset\Q(\alpha)\).
    \end{q}
    \begin{q}{4}
        Montrer par l'absurde que \(\alpha\notin\Q(\sqrt{5})\) en supposant que l'on
        peut écrire \(\alpha\) sous la forme \(a+b\sqrt{5}\) avec \(a,b\in\Q\).
    \end{q}
    \begin{q}{5}
        Déterminer le degré \([\Q(\alpha):\Q]\) et en déduirre que \(P\) est irréductible
        sur \(\Q\).
    \end{q}
    \begin{q}{6}
        Déterminer le degré \([L:\Q]\).
    \end{q}
\end{exo}

\subsection*{"Casus irreductibilis" des équations cubiques}
\begin{exo}
    Convenons de dire qu'un réel \(r\) est \textit{exprimable par radicaux réels} lorsqu'il
    existe une suite finie de sous corps de \(R:\Q=K_0\subset K_1\subset\dots \subset K_n
    \subset \R\) tels que \(r\in K_n\) et tels que pour tout \(i<n, K_{i+1}=K_i(\sqrt[p](\alpha))\)
    avec \(p\) premier et \(\alpha\in K_i\)...
\end{exo}

\subsection*{Construction à la règle et au compas}

On assimile le plan euclidien à \(\C\), et pour tous \(z,z'\in\C\) distincts
et tout \(r>0\) on note ...

\begin{exo}
    Autour de la construction de nombres.
\end{exo}

\begin{exo}
    Quelques problèmes classiques.
    \begin{q}{1}
        \textit{Le problème déliaque} : à partir d'un temple de Delos de longueur 1,
        il faut construire un temple semblable de volume doubme, et donc d'en construire
        la longueur \(\sqrt[3]{2}\) à la règle et au compas. Montrer que c'est impossible.
    \end{q}
    \begin{q}{2}
        \textit{L'ennéagone régulier et la trisection d'un angle} : Un ennéagone est un
        polygone à \(9\) côtés. Justifier le fait que construire à la règle et au
        compas un ennéagone régulier de centre \(0\) dont l'un des sommet est \(1\)
        revient à construire \(z=e^{2i\pi/9}\). Montrer que c'est impossible. En
        déduire qu'il n'existe pas de procédé général de trisection d'un angle à la règle
        et au compas.
    \end{q}
    \begin{q}{3}
        \textit{La quadrature du cercle} : Il s'agit de construire le coté
        (\(c=\sqrt{\pi}\)) d'un carré dont l'aire égale celle du disque unité \(|z\leq 1\).
        Montrer que àa n'est possible que si \([Q(\pi):\Q]\) est une puissance de \(2\).
    \end{q}
\end{exo}
\subsection*{Polynômes symétriques et théorème de d'Alembert-Gauss}

\begin{exo}
    Soit \(A\) un anneau commutatif quelconque. On définit le degré \(\delta\) et le
    poids \(\omega\) d'un polynôme ...
\end{exo}

\begin{exo}
    Théorème fondamental de l'algèbre (\textsc{D'Alembert-Gauss}): \(\C\) est
    algébriquement clos.
    \begin{q}{1}
        Montrer qu'il suffit de démontrer que tout polynôme de degré supérieur à \(1\)
        à coefficients \(textit{réels}\) admet une racine dans \(\C\).
    \end{q}
\end{exo}

\subsection*{Corps finis}

\begin{exo}
    Pour quelles valeurs de \(r\), \(\mathbb{F}_{p^r}\) est-il isomorphe à \(\Z/p^r\Z\).
\end{exo}

\begin{exo}
    Soit \(K=\mathbb{F}_{p^r}\) un corps fini. Montrer que pour tout élément \(\beta\in K\),
    il existe un unique élément \(\alpha\in K\) tel que \(\alpha^p=\beta\).
\end{exo}

\begin{exo}
    Soit \(a\in\mathbb{F}_p\). En comptant \(\{x^2,x\in\mathbb{F}_p\}\) et
    \(\{a-y^2,y\in\mathbb{F}_p\}\) montrer que l'on peur toujours écrire \(a\) comme
    une somme de deux carrés dans \(\mathbb{F}_p\).
\end{exo}

\begin{exo}
    On repasse en CP
    \begin{q}{1}
        Donner les tables d'addition et de multiplication de \(\mathbb{F}_4\).
    \end{q}
    \begin{q}{2}
        Donner les tables d'addition et de multiplication de \(\mathbb{F}_9\).
    \end{q}
\end{exo}
\end{document}