\documentclass[french]{report}
\usepackage{../exercices}

\begin{document}

\begin{center}
    \huge{\textbf{S6-Algebre - TD2}}
\end{center}

\section*{Extension de corps, éléments algébriques}

\begin{exo}
    Voici trois nombres : \(\sqrt{2}, \sqrt{4+2\sqrt{2}}, \sqrt[4]{2}\)
    \begin{q}{1}
        Montrer que ces nombres sont algébriques sur \(\Q(\sqrt{2})\) et donner leur
        polynôme minimal.
    \end{q}
    \begin{q}{2}
        Même question pour \(\Q\).
    \end{q}
\end{exo}

\begin{exo}
    Le corps des fractions \(\R(X)\) de \(\R[X]\) est une extension (infinie) de \(\Q\).
    \begin{q}{1}
        Donner un élément algébrique sur \(Q\) dans \(\R(X)\backslash\Q\)
    \end{q}
    \begin{q}{2}
        Donner un élément non algébrique sur \(\Q\) de \(\R(X)\).
    \end{q}
\end{exo}

\begin{exo}
    Autour de la clotûre.
    \begin{q}{1}
        Le corps \(\C(X)\) est-il algébriquement clos ?
    \end{q}
    \begin{q}{2}
        Un corps fini peut-il être algébriquement clos ?
    \end{q}
\end{exo}

\begin{exo}
    Soit \(\alpha\in\C\) un nombre algébrique et \(P\) son
    minimal sur \(\Q\).
    \begin{q}{1}
        Montrer que \(\alpha\) est une racine simple de \(P\) dans \(\C\).
    \end{q}
    \begin{q}{2}
        Montrer que \(P\) n'a que des racines simples dans \(\C\).
    \end{q}
\end{exo}

\begin{exo} On bosse un peu maintenant.
    \begin{q}{1}
        Montrer que \(\C\) est une clotûre algébrique de \(\R\).
    \end{q}
    \begin{q}{2}
        Montrer que le corps \(\C\) n'a pas d'extension finie non triviale.
    \end{q}
    \begin{q}{3}
        Montrer que les seules extensions finies du corps \(\R\) sont \(\R\) et \(\C\)*
        à isomorphisme près.
    \end{q}
    \begin{q}{4}
        Montrer que \(\C\) n'est pas une clotûre algébrique de \(\Q\).
    \end{q}
\end{exo}

\section*{Corps de rupture, degrés d'extension, corps de décomposition}

\begin{exo}
    On considère \(K=\Q[X]/(P)\) où \(P=X^3+3X+3\), et on note \(\alpha\)
    la classe de \(X\) dans \(K\).
    \begin{q}{1}
        Expliquer pourquoi \(K\) est un corps.
    \end{q}
    \begin{q}{2}
        Déterminer \(k\) tel que \(\left(1,\alpha,\dots,\alpha^k\right)\) soit
        une base de \(\Q\) espace vectoriel \(K\).
    \end{q}
    \begin{q}{3}
        Exprimer \(a^{-1}\) dans cette base.
    \end{q}
\end{exo}

\begin{exo}
    Soit \(P=\sum^n a_iX^i\), où \(a_n\neq0\), un polynôme irréductible sur un corps \(K\).
    \begin{q}{1}
        Montrer que \(Q = \sum a_{n-i}X^i\) est irréductible sur \(K\).
    \end{q}
    \begin{q}{2}
        Montrer que \(K[X]/(P)\) et \(K[X]/(Q)\) sont isomorphes.
    \end{q}
\end{exo}

\begin{exo}
    Montrer que si \([L\colon K]\) est un nombre premier, alors la \(K\)-extension
    \(L\) est de la forme \(K(\alpha)\).
\end{exo}

\begin{exo}
    Soit \(L\) une extension d'un corps \(K\) et \(\alpha,\beta\in L\) deux éléments
    algébriques sur \(K\).
    \begin{q}{1}
        Montrer que si les degrés d'extension \([K(\alpha)\colon K]\) et \([K(\beta)
        \colon K]\) sont premiers entre eux, alors \([K(\alpha,\beta)\colon K]=
        [K(\alpha)\colon K][K(\beta)\colon K]\)
    \end{q}
    \begin{q}{2}
        A-t-on toujours \(K(\alpha\beta)=K(\alpha,\beta)\) lorque les degrés sont
        premiers entre eux ?
    \end{q}
\end{exo}


\end{document}