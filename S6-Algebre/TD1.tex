\documentclass[french]{report}
\usepackage{../exercices}

\begin{document}

\begin{center}
    \huge{\textbf{S6-Algebre - TD1}}
\end{center}

\subsection*{Anneaux}

\begin{exo}
    Soit \(A\) un anneau intègre. Posons \(X=A\times (A\backslash {0})\).
    On définit sur \(X\) la relation \(\sim\) ainsi. Pour tous
    \((a,b), (c,d)\in X\), on a \((a,b)\sim(c,d)\) ssi \(ad = bc\).

    \begin{q}{1}
        Montrer que \(\sim\) est une relation d'équivalence.
        On note \(K=X/\sim\) l'ensemble quotient et on note \(\frac{a}{b}\)
        classe d'équivalence de \((a, b)\) dans \(K\).
    \end{q}
    \begin{q}{2}
        On définit deux lois + et * de composition interne sur \(K\colon
        \frac{a}{b}+\frac{c}{d} \colon=\frac{ad+bc}{bd}\) et \(\frac{a}{b}*\frac{c}{d}
        \colon=\frac{ac}{bd}\). Montrer que ces deux opérations sont bien définies sur
        \(K\) et que \(\left(K, +, *\right)\) est un anneau commutatif et même un corps.
    \end{q}
    \begin{q}{3}
        Montrer que l'applciation \(\iota\colon A\to K\) qui à \(a\) associe \(\frac{a}{1}\)
        est un morphisme d'anneaux injectif.
    \end{q}
    \begin{q}{4}
        On se donne \(K'\) un corps et \(\psi\colon A\to K'\) un morphisme d'anneaux
        injectif.
        \begin{q}{a}
            Montrer qu'il existe un unique morphisme de corps \(\alpha\colon:K\to K'\)
            tel que \(\alpha\iota=\psi\)
        \end{q}
        \begin{q}{b}
            On suppose que \(\eta\colon A\to K'\) est un morphisme d'anneaux injectif qui
            vérifie la propriété suivante : pour tout morphisme d'anneaux injectif \(\varphi
            \colon A\to L\) avec \(L\) un corps il existe un unique morphisme de corps
            \(\beta\colon K'\to L\) tel que \(\beta_\eta=\varphi\).
            Montrer que \(K\) et \(K'\) sont isomorphes.
        \end{q}
    \end{q}
\end{exo}

\begin{exo}
    On considère l'anneau des entiers de \textsc{Gauss} \(\Z[i]\) et l'anneau quotient
    \(A = \Z[i]/(1+3i)\).
    \begin{q}{1}
        Montrer que \(i\equiv 3\) dans \(A\).
        \boxans{On remarque que \(i(1+3i) = i-3\) qui est nul dans \(A\), ainsi
        \(i\equiv 3\) dans \(A\).}
    \end{q}
    \begin{q}{2}
        En déduire que le morphisme d'anneau \(\phi\colon \Z\to A\) est surjectif.
        \boxans{\(\phi\colon a+3b\mapsto a+ib\) est bien défini comme épimorphisme
        d'après la question précédente.}
    \end{q}
    \begin{q}{3}
        Démontrer l'isomorphisme d'anneau \(A\simeq \Z/10\Z\).
        \boxans{Dans \(A\) on a \(0\equiv 1+3i\equiv 10\) ce qui est le résultat recherché.}
    \end{q}
    \begin{q}{4}
        Quelle est la caractéristique de \(A\) ?
        \boxans{D'après la question précédente, \(A\) est de caractéristique 10.}
    \end{q}
\end{exo}

\begin{exo}
    On pose \(j=e^{2i\pi/3}\) et on note \(\Z[j]\) le sous anneau de \(C\) pertinent.
    \begin{q}{1}
        Montrer que \(2\) n'est pas inversible dans \(\Z[j]\).
        \boxans{Tous les éléments de \(\Z[j]\) sont de module entier, donc un élément de
        module non multiplicativement neutre ne peut pas être inversible car le module
        est multilinéaire.}
    \end{q}
    \begin{q}{2}
        Expliquer pourquoi \(\Z[j]\) est un anneau intègre.
        \boxans{L'anneau \(\Z[j]\) est intègre comme sous-anneau de \(\C\).}
    \end{q}
    \begin{q}{3}
        Justifier que pour tout \(P\in\Z[x]\) il existe deux polynômes uniques \(Q\) et \(R\)
        de \(\Z[x]\) tels que \(P=(X^2+X+1)Q+R\) et \(R\) est de degré inférieur ou égal à 1.
    \end{q}
    \begin{q}{4}
        Montrer que l'application \(p\colon\Z[x]\to\Z[j]\colon P\mapsto P(j)\) est un
        morphisme d'anneaux surjectif.
    \end{q}
    \begin{q}{5}
        Montrer que les deux anneaux \(\Z[x]/(X^2+X+1)\) et \(\Z[j]\) sont isomorphes.
    \end{q}
    \begin{q}{6}
        En déduire que l'idéal \((X^2+X+1)\) de \(\Z[X]\) est premier mais pas maximal.
    \end{q}
\end{exo}
\clearpage
\begin{exo}
    On considère dans l'anneau \(A=\Z[X]\), l'idéal \(I\) engendré par \(2\) et \(X\).
    \begin{q}{1}
        Quels sont les irréductibles de \(A\) ?
        \boxans{Les irréductibles de \(A\) sont tous les polynômes
        constant d'image autre que \(1\) et \(-1\). Ainsi que tous les polynômes
        de contenu unitaires irréductibles sur \(\Q[X]\).}
    \end{q}
    \begin{q}{2}
        L'idéal \(I\) est-il engendré par un unique élément ? Que peut-on en déduire ?
        \boxans{Non, \(I\) est généré par deux éléments, ce qui veut dire que
        \(I = \gcd\left(2,X\right)A = A\). On en déduit qu'il n'est pas pertinent.}
    \end{q}
    \begin{q}{3}
        Les idéaux (2) et \((X)\) sont-ils premiers ? maximaux ?
        \boxans{On remarque les isomorphismes d'anneaux suivants:
        \[\Z[X]/2\Z[X]\simeq(\Z/2\Z)[X]\quad\textrm{et}\quad\Z[X]/X\Z[X]\simeq\Z\]
        qui permettent d'affirmer que l'idéal \((X)\) est premier mais non maximal
        car \(\Z\) est un anneau intègre mais pas un corps.
        L'idéal \((2)\) est lui aussi premier mais non maximal, pour la même raison,
        et par le degré d'un produit de polynômes.}
    \end{q}
    \begin{q}{4}
        Montrer que l'identité de \textsc{Bézout} n'est pas valable dans \(A\).
        \boxans{Supposons par l'absure que l'identité de \textsc{Bézout} soit valable
        dans \(A\), on considère les polynômes \(P_1 = X+1\) et \(P_2 = X-1\). Alors par
        hypothèse il existe \(P_a\) et \(P_b\) tels que \(P_aP_1 + P_bP_2 = 1\). On doit
        avoir \(P_a = - P_b\) si on veut espérer que le coefficient \(X\) disparaisse, mais
        alors \(P_aP_1+P_bP_2\) est divisible par \(2\), ce qui est absurde.}
    \end{q}
\end{exo}

\section*{Polynômes}

\begin{exo}
    Effectuer les divisions euclidiennes suivantes:
    \begin{enumerate}
        \itt \(3X^5+4X^2+1\) par \(X^2+2X+3\).
        \itt \(3X^5+2X^4-X^2+1\) par \(X^3+X+2\).
        \itt \(X^4+X^3+X-2\) par \(X^2-2X+4\).
    \end{enumerate}
\end{exo}

\begin{exo} Autour de la division euclidienne
    \begin{q}{1}
        Énoncer la division euclidienne dans \(\R[X]\). Montrer qu'il y a unicité des
        polynômes intervenant dans l'écriture de la division euclidienne.
    \end{q}
    \begin{q}{2}
        Soient \(P_1=X^3-2X^2-2X-3\) et \(P_2=X^2-2X-3\) des polynômes de \(\R[X]\).
        En appliquant l'algorithme d'\textsc{Euclide}, déterminer le pgcd de \(P_1\)
        et \(P_2\).
    \end{q}
    \begin{q}{3}
        Décomposer en produits d'irréductibles de \(\R[X]\) les polynômes
        \(P_1\) et \(P_2\). En déduire une autre méthode pour déterminer leur pgcd.
    \end{q}
\end{exo}

\begin{exo}
    Factoriser sur \(\C\), sur \(\R\)  et sur \(\Q\) les polynômes suivants:
    \begin{enumerate}
        \itt \(X^2-2\)
        \itt \(X^2+2\)
        \itt \(X^2+X\)
        \itt \(X^2+X+1\)
        \itt \(X^4-4\)
        \itt \(X^4+4\)
        \itt \(X^4-X^2+1\).
    \end{enumerate}
\end{exo}

\begin{exo}
    Factoriser sur \(\C\) le polynôme suivant \(X^8+X^4+1\).
\end{exo}

\begin{exo} Autour de la dérivation
    \begin{q}{1}
        Soit \(P\in\C[X]\) un polynôme non-constant. Montrer que \(P\) n'a que des racines
        simple si et seulement si \(P\) et \(P'\) sont premiers entre eux.
    \end{q}
    \begin{q}{2}
        Trouver tous les polynômes \(P\in \C[x]\) vérifiant \(P'\mid P\).
    \end{q}
\end{exo}

\begin{exo}
    Trouver les polynômes \(P\in\R[X]\) tels que \(\left(X+4\right)P(X)=XP(X+1)\).
\end{exo}

\begin{exo}
    Soit \(P\in\C[X]\) vérifiant \(P(X^2) = P(X)P(X+1)\).
    \begin{q}{1}
        Montrer que les racines de \(P\) sont toutes égales à \(0\) ou \(1\).
    \end{q}
    \begin{q}{2}
        Montrer que \(P\) est une puissance de \(X(X-1)\).
    \end{q}
\end{exo}

\begin{exo}
    Soient \(a\) et \(b\) deux entiers positifs et \(d\) leur pgcd. Montrer que le
    pgcd de \(X^a-1\) et \(X^b-1\) dans \(\Z[X]\) est \(X^d-1\).
    \boxans{Le polynômes \(X^a-1\) admet exactement les éléments de \(\U_a\)
    comme racine, de la même façon \(X^b-1\) admet ceux de \(\U_b\). Ainsi le pgcd
    de ces polynômes admet comme racines les éléments de \(\U_a \cap \U_b = \U_d\)}
\end{exo}

\begin{exo}
    Soit \(K\) un corps
    \begin{q}{1}
        Déterminer les couples \(\left(U, V\right)\in K[X]^2\) tels que
        \(X^nU+\left(1-X\right)V=1\) où \(n\geq 1\).
    \end{q}
    \begin{q}{2}
        Démontrer que \(A(X)=X^{3p+2}+X^{3q+1}+X^{3r}\in\R[X]\) est divisible
        par \(X^2+X+1\) pour tout \(\left(p,q,r\right)\in\N^3\).
    \end{q}
\end{exo}

\begin{exo}
    Soit \(n\in\N\).
    \begin{q}{1}
        Montrer qu'il existe un unique polynômes \(P\in\R[X]\) tel que
        \(P(X+1)+P(X) = 2X^n\).
    \end{q}
    \begin{q}{2}
        On note ce polynôme \(E_n\). Montrer que \(E_n'=nE_{n-1}\).
    \end{q}
\end{exo}

\begin{exo}
    Soit \(K\) un corps commutatif. On se donne \(m+1\) éléments distincts de \(K\colon
    q_0\dots q_m\).
    \begin{q}{1}
        Expliciter pour tout \(0\leq j\leq m\) le polynôme \(L_j\in K_m[X]\)
        satisfaisant \(L_j(q_i) = \delta_{i,j}\).
        \boxans{On est ici face à un cas d'école, celui de polynômes de \textsc{Lagrange},
        il suffit de recracher la formule :\[L_j = \prod_{i=0}^m \frac{X-q_i}{q_j-q_i}\]}
    \end{q}
    \begin{q}{2}
        Montrer que la famille des polynômes \(L_j\) forme une base de l'espace vectoriel
        \(K_m[X]\). Exprimer tout polynôme dans cette base.
        \boxans{On commence par montrer la liberté de la famille, soient \(P\in K_m[X]=0\)
        \(\lambda_0\dots \lambda_m\) tels que \(P = \sum_{j=0}^{m} \lambda_j L_j\) alors
        si un des \(\lambda\) est non nul, \(P\) est non nul sur un \(q\) par linéarité
        de la somme et définition de \(L_j\), donc pour \(P=0\) tous les \(\lambda\)
        sont nuls, la famille est libre. C'est donc une base par cardinalité.}
    \end{q}
    \begin{q}{3}
        On note \(\mathcal{P}(\Q, \Q)\) l'ensemble des polynômes à coéfficients rééls prenant des valeurs
        rationelles sur les rationels. Montrer que \(\mathcal{P}(\Q, \Q)=\Q[X]\).
    \end{q}
\end{exo}

\section*{Irréductibilité des polynômes}

\begin{exo}
    Soit \(A\) un anneau principal, \(P\in A[X]\) et \(a\in A\).
    \begin{q}{1}
        Montrer que \(P\) es tirréductible si le poly
    \end{q}
\end{exo}

\end{document}