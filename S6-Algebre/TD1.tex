\documentclass[french]{report}
\usepackage{../exercices}

\begin{document}

\begin{center}
    \huge{\textbf{S6-Algebre - TD1}}
\end{center}

\subsection*{Anneaux}

\begin{exo}
    Soit \(A\) un anneau intègre. Posons \(X=A\times (A\backslash {0})\).
    On définit sur \(X\) la relation \(\sim\) ainsi. Pour tous
    \((a,b), (c,d)\in X\), on a \((a,b)\sim(c,d)\) ssi \(ad = bc\).

    \begin{q}{1}
        Montrer que \(\sim\) est une relation d'équivalence.
        On note \(K=X/\sim\) l'ensemble quotient et on note \(\frac{a}{b}\)
        classe d'équivalence de \((a, b)\) dans \(K\).
    \end{q}
    \begin{q}{2}
        On définit deux lois + et * de composition interne sur \(K\colon
        \frac{a}{b}+\frac{c}{d} \colon=\frac{ad+bc}{bd}\) et \(\frac{a}{b}*\frac{c}{d}
        \colon=\frac{ac}{bd}\). Montrer que ces deux opérations sont bien définies sur
        \(K\) et que \(\left(K, +, *\right)\) est un anneau commutatif et même un corps.
    \end{q}
    \begin{q}{3}
        Montrer que l'applciation \(\iota\colon A\to K\) qui à \(a\) associe \(\frac{a}{1}\)
        est un morphisme d'anneaux injectif.
    \end{q}
    \begin{q}{4}
        On se donne \(K'\) un corps et \(\psi\colon A\to K'\) un morphisme d'anneaux
        injectif.
        \begin{q}{a}
            Montrer qu'il existe un unique morphisme de corps \(\alpha\colon:K\to K'\)
            tel que \(\alpha\iota=\psi\)
        \end{q}
        \begin{q}{b}
            On suppose que \(\eta\colon A\to K'\) est un morphisme d'anneaux injectif qui
            vérifie la propriété suivante : pour tout morphisme d'anneaux injectif \(\varphi
            \colon A\to L\) avec \(L\) un corps il existe un unique morphisme de corps
            \(\beta\colon K'\to L\) tel que \(\beta_\eta=\varphi\).
            Montrer que \(K\) et \(K'\) sont isomorphes.
        \end{q}
    \end{q}
\end{exo}

\begin{exo}
    On considère l'anneau des entiers de \textsc{Gauss} \(\Z[i]\) et l'anneau quotient
    \(A = \Z[i]/(1+3i)\).
    \begin{q}{1}
        Montrer que \(i\equiv 3\) dans \(A\).
        \boxans{On remarque que \(i(1+3i) = i-3\) qui est nul dans \(A\), ainsi
        \(i\equiv 3\) dans \(A\).}
    \end{q}
    \begin{q}{2}
        En déduire que le morphisme d'anneau \(\phi\colon \Z\to A\) est surjectif.
        \boxans{\(\phi\colon a+3b\mapsto a+ib\) est bien défini comme épimorphisme
        d'après la question précédente.}
    \end{q}
    \begin{q}{3}
        Démontrer l'isomorphisme d'anneau \(A\simeq \Z/10\Z\).
        \boxans{Dans \(A\) on a \(0\equiv 1+3i\equiv 10\) ce qui est le résultat recherché.}
    \end{q}
    \begin{q}{4}
        Quelle est la caractéristique de \(A\) ?
        \boxans{D'après la question précédente, \(A\) est de caractéristique 10.}
    \end{q}
\end{exo}

\begin{exo}
    On pose \(j=e^{2i\pi/3}\) et on note \(\Z[j]\) le sous anneau de \(C\) pertinent.
    \begin{q}{1}
        Montrer que \(2\) n'est pas inversible dans \(\Z[j]\).
        \boxans{Tous les éléments de \(\Z[j]\) sont de module entier, donc un élément de
        module non multiplicativement neutre ne peut pas être inversible car le module
        est multilinéaire.}
    \end{q}
    \begin{q}{2}
        Expliquer pourquoi \(\Z[j]\) est un anneau intègre.
        \boxans{L'anneau \(\Z[j]\) est intègre comme sous-anneau de \(\C\).}
    \end{q}
    \begin{q}{3}
        Justifier que pour tout \(P\in\Z[x]\) il existe deux polynômes uniques \(Q\) et \(R\)
        de \(\Z[x]\) tels que \(P=(X^2+X+1)Q+R\) et \(R\) est de degré inférieur ou égal à 1.
    \end{q}
    \begin{q}{4}
        Montrer que l'application \(p\colon\Z[x]\to\Z[j]\colon P\mapsto P(j)\) est un
        morphisme d'anneaux surjectif.
    \end{q}
    \begin{q}{5}
        Montrer que les deux anneaux \(\Z[x]/(X^2+X+1)\) et \(\Z[j]\) sont isomorphes.
    \end{q}
    \begin{q}{6}
        En déduire que l'idéal \((X^2+X+1)\) de \(\Z[X]\) est premier mais pas maximal.
    \end{q}
\end{exo}
\clearpage
\begin{exo}
    On considère dans l'anneau \(A=\Z[X]\), l'idéal \(I\) engendré par \(2\) et \(X\).
    \begin{q}{1}
        Quels sont les irréductibles de \(A\) ?
        \boxans{\quad Soit \(P\) irréductible constant, \(P(0)\) doit être premier
        sans quoi \(P\) se factorise comme produit de deux autre polynômes constants.
        Sinon, si \(P\) est non constant, s'il a un contenu autre que \(1\) il peut se
        factoriser comme une constante multipliée par un polynôme de contenu unitaire.
        Si cont\((P)=1\), supposons qu'il soit réductible sur \(\Q[X]\) alors \(P = AB\)
        avec \(A,B\in\Q[X]\), on peut les multiplier par des constantes \(\alpha,\beta\)
        pour obtenir des polynômes \(A',B'\) dans \(\Z[X]\) de contenu unitaire.


        \quad On a ainsi \(P=\frac{1}{\alpha\beta}A'B'\) est à coefficient entier, donc \(\alpha\)
        et \(\beta\) sont inversibles dans \(\Z[X]\) et finalement \(P\) est réductible dans
        \(\Z[X]\). Donc \(P\) est irréductible sur \(\Z[X]\) si et seulement s'il est
        irréductible sur \(\Q[X]\).}
    \end{q}
    \begin{q}{2}
        L'idéal \(I\) est-il engendré par un unique élément ? Que peut-on en déduire ?
        \boxans{\(I\) est par définition le plus petit idéal contenant \(\langle 2\rangle\)
        et \(\langle X\rangle\). Supposons par l'absurde qu'il soit principal, engendré par
        un polynôme \(P\). Alors il existe \(Q\in\Z[X]\) tel que \(PQ=2\) donc \(P\) est de
        degré nul. de la même façon il existe \(R\in\Z[X]\) tel que \(PR=X\) donc \(P\)
        est de contenu unitaire, finalement \(P=\pm1\), qui n'est pas dans \(I\), on a une
        contradiction.}
    \end{q}
    \begin{q}{3}
        Les idéaux (2) et \((X)\) sont-ils premiers ? maximaux ?
        \boxans{On remarque les isomorphismes d'anneaux suivants:
        \[\Z[X]/2\Z[X]\simeq(\Z/2\Z)[X]\quad\textrm{et}\quad\Z[X]/X\Z[X]\simeq\Z\]
        qui permettent d'affirmer que l'idéal \((X)\) est premier mais non maximal
        car \(\Z\) est un anneau intègre mais pas un corps.
        L'idéal \((2)\) est lui aussi premier mais non maximal, pour la même raison,
        soient \(P,Q\in(\Z/2\Z)[X]\), \(\deg(PQ)=\deg(P)+\deg(Q)\) et donc \(\deg(PQ)\neq 0\)
        si \(P\) ou \(Q\) est non nul, l'anneau est bien intègre.}
    \end{q}
    \begin{q}{4}
        Montrer que l'identité de \textsc{Bézout} n'est pas valable dans \(A\).
        \boxans{Supposons par l'absurde que l'identité de \textsc{Bézout} soit valable
        dans \(A\), on considère les polynômes \(P_1 = X\) et \(P_2 = 2\). Alors par
        hypothèse il existe \(P_a\) et \(P_b\) tels que \(P_aP_1 + P_bP_2 = 1\).
        On remarque que \(P_aP_1\) n'a pas de terme constant, et le terme constant de
        \(P_bP_2\) est pair, donc \(1\) est pair ce qui est absurde.}
    \end{q}
\end{exo}

\section*{Polynômes}

\begin{exo}
    Effectuer les divisions euclidiennes suivantes:
    \begin{enumerate}
        \itt \(3X^5+4X^2+1\) par \(X^2+2X+3\).
        \itt \(3X^5+2X^4-X^2+1\) par \(X^3+X+2\).
        \itt \(X^4+X^3+X-2\) par \(X^2-2X+4\).
    \end{enumerate}
\end{exo}

\begin{exo} Autour de la division euclidienne
    \begin{q}{1}
        Énoncer la division euclidienne dans \(\R[X]\). Montrer qu'il y a unicité des
        polynômes intervenant dans l'écriture de la division euclidienne.
    \end{q}
    \begin{q}{2}
        Soient \(P_1=X^3-2X^2-2X-3\) et \(P_2=X^2-2X-3\) des polynômes de \(\R[X]\).
        En appliquant l'algorithme d'\textsc{Euclide}, déterminer le pgcd de \(P_1\)
        et \(P_2\).
    \end{q}
    \begin{q}{3}
        Décomposer en produits d'irréductibles de \(\R[X]\) les polynômes
        \(P_1\) et \(P_2\). En déduire une autre méthode pour déterminer leur pgcd.
    \end{q}
\end{exo}

\begin{exo}
    Factoriser sur \(\C\), sur \(\R\)  et sur \(\Q\) les polynômes suivants:
    \begin{enumerate}
        \itt \(X^2-2\quad\colon\quad(X-\sqrt{2})(X+\sqrt{2})\) et irréductible
        \itt \(X^2+2\quad\colon\quad(X-i\sqrt{2})(X+i\sqrt{2})\) et irréductible
        \itt \(X^2+X\quad\colon\quad X(X+1)\)
        \itt \(X^2+X+1\quad\colon\quad(X-j)(X-j^2)\) et irréductible
        \itt \(X^4-4\quad\colon\quad(X^2-2)(X^2+2)\) puis voir au dessus
        \itt \(X^4+4\quad\colon\quad(X-\sqrt{2}e^{i\pi/4})(X-\sqrt{2}e^{3i\pi/4})
        (X+\sqrt{2}e^{i\pi/4})(X+\sqrt{2}e^{3i\pi/4})\) et \((X^2-2X+2)(X^2+2X+2)\)
        \itt \(X^4-X^2+1\colon(X-e^{i\pi/6})(X+e^{i\pi/6})(X-e^{5i\pi/6})
        (X+e^{5i\pi/6})\),\((X^2+\sqrt{3}X+1)(X^2-\sqrt{3}X+1)\) et irr
    \end{enumerate}
\end{exo}

\begin{exo}
    Factoriser sur \(\C\) le polynôme suivant \(P=X^8+X^4+1\).
    \boxans{On procède par étapes, le polynôme devient en remarquant que
    \(P[Y:=X^4]\) est trivial et on obtient \(P=(X^4-j)(X^4-j^2)\) ainsi
    \(P=(X^2-j^2)(X^2+j^2)(X^2-j)(X^2+j)\) ce qui se simplifie en:
    \((X-j)(X+j)(X+ij)(X-ij)(X-j^2)(X+j^2)(X-ij^2)(X+ij^2)\).}
\end{exo}

\begin{exo} Autour de la dérivation
    \begin{q}{1}
        Soit \(P\in\C[X]\) un polynôme non-constant. Montrer que \(P\) n'a que des racines
        simple si et seulement si \(P\) et \(P'\) sont premiers entre eux.
        \boxans{D'après la formule de la chaîne, si \(P\) admet une racine double alors
        \(P\) et \(P'\) ne sont pas premiers entre eux. Montrons la réciproque, soit \(P\)
        non premier avec son polynôme dérivé \(P'\) alors ils possèdent un facteur de la
        forme \((X-\lambda)\) en commun, or \(P= \int P' + O(1)\) donc d'après la formule
        de la cahîne \(P\) contient de le facteur \((X-\lambda)\) à une puissance plus
        grande que 1}
    \end{q}
    \begin{q}{2}
        Trouver tous les polynômes \(P\in \C[x]\) vérifiant \(P'\mid P\).
        \boxans{On procède par analyse synthèse, soit \(P\) un tel polynôme que l'on
        suppose sous sa forme factorisée. Les racines de \(P'\) sont racines de \(P\)
        par hypothèse. }
    \end{q}
\end{exo}

\begin{exo}
    Trouver les polynômes \(P\in\R[X]\) tels que \(\left(X+4\right)P(X)=XP(X+1)\).
\end{exo}

\begin{exo}
    Soit \(P\in\C[X]\) vérifiant \(P(X^2) = P(X)P(X+1)\).
    \begin{q}{1}
        Montrer que les racines de \(P\) sont toutes égales à \(0\) ou \(1\).
    \end{q}
    \begin{q}{2}
        Montrer que \(P\) est une puissance de \(X(X-1)\).
    \end{q}
\end{exo}

\begin{exo}
    Soient \(a\) et \(b\) deux entiers positifs et \(d\) leur pgcd. Montrer que le
    pgcd de \(X^a-1\) et \(X^b-1\) dans \(\Z[X]\) est \(X^d-1\).
    \boxans{Le polynômes \(X^a-1\) admet exactement les éléments de \(\U_a\)
    comme racine, de la même façon \(X^b-1\) admet ceux de \(\U_b\). Ainsi le pgcd
    de ces polynômes admet comme racines les éléments de \(\U_a \cap \U_b = \U_d\)}
\end{exo}

\begin{exo}
    Soit \(K\) un corps
    \begin{q}{1}
        Déterminer les couples \(\left(U, V\right)\in K[X]^2\) tels que
        \(X^nU+\left(1-X\right)V=1\) où \(n\geq 1\).
    \end{q}
    \begin{q}{2}
        Démontrer que \(A(X)=X^{3p+2}+X^{3q+1}+X^{3r}\in\R[X]\) est divisible
        par \(X^2+X+1\) pour tout \(\left(p,q,r\right)\in\N^3\).
    \end{q}
\end{exo}

\begin{exo}
    Soit \(n\in\N\).
    \begin{q}{1}
        Montrer qu'il existe un unique polynômes \(P\in\R[X]\) tel que
        \(P(X+1)+P(X) = 2X^n\).
    \end{q}
    \begin{q}{2}
        On note ce polynôme \(E_n\). Montrer que \(E_n'=nE_{n-1}\).
    \end{q}
\end{exo}

\begin{exo}
    Soit \(K\) un corps commutatif. On se donne \(m+1\) éléments distincts de \(K\colon
    q_0\dots q_m\).
    \begin{q}{1}
        Expliciter pour tout \(0\leq j\leq m\) le polynôme \(L_j\in K_m[X]\)
        satisfaisant \(L_j(q_i) = \delta_{i,j}\).
        \boxans{On est ici face à un cas d'école, celui de polynômes de \textsc{Lagrange},
        il suffit de recracher la formule :\[L_j = \prod_{i=0}^m \frac{X-q_i}{q_j-q_i}\]}
    \end{q}
    \begin{q}{2}
        Montrer que la famille des polynômes \(L_j\) forme une base de l'espace vectoriel
        \(K_m[X]\). Exprimer tout polynôme dans cette base.
        \boxans{On commence par montrer la liberté de la famille, soient \(P\in K_m[X]=0\)
        \(\lambda_0\dots \lambda_m\) tels que \(P = \sum_{j=0}^{m} \lambda_j L_j\) alors
        si un des \(\lambda\) est non nul, \(P\) est non nul sur un \(q\) par linéarité
        de la somme et définition de \(L_j\), donc pour \(P=0\) tous les \(\lambda\)
        sont nuls, la famille est libre. C'est donc une base par cardinalité.}
    \end{q}
    \begin{q}{3}
        On note \(\mathcal{P}(\Q, \Q)\) l'ensemble des polynômes à coéfficients rééls prenant des valeurs
        rationelles sur les rationels. Montrer que \(\mathcal{P}(\Q, \Q)=\Q[X]\).
    \end{q}
\end{exo}

\section*{Irréductibilité des polynômes}

\begin{exo}
    Soit \(A\) un anneau principal, \(P\in A[X]\) et \(a\in A\).
    \begin{q}{1}
        Montrer que \(P\) est irréductible si le poly
    \end{q}
\end{exo}

\end{document}