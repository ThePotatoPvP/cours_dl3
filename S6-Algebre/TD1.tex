\documentclass[french]{report}
\usepackage{../exercices}

\begin{document}

\begin{center}
    \huge{\textbf{S6-Algebre - TD1}}
\end{center}

\subsection*{Anneaux}

\begin{exo}
    Soit \(A\) un anneau intègre. Posons \(X=A\times (A\backslash {0})\).
    On définit sur \(X\) la relation \(\sim\) ainsi. Pour tous
    \((a,b), (c,d)\in X\), on a \((a,b)\sim(c,d)\) ssi \(ad = bc\).

    \begin{q}{1}
        Montrer que \(\sim\) est une relation d'équivalence.
        On note \(K=X/\sim\) l'ensemble quotient et on note \(\frac{a}{b}\)
        classe d'équivalence de \((a, b)\) dans \(K\).
    \end{q}
    \begin{q}{2}
        On définit deux lois + et * de composition interne sur \(K\colon
        \frac{a}{b}+\frac{c}{d} \colon=\frac{ad+bc}{bd}\) et \(\frac{a}{b}*\frac{c}{d}
        \colon=\frac{ac}{bd}\). Montrer que ces deux opérations sont bien définies sur
        \(K\) et que \(\left(K, +, *\right)\) est un anneau commutatif et même un corps.
    \end{q}
    \begin{q}{3}
        Montrer que l'applciation \(\iota\colon A\to K\) qui à \(a\) associe \(\frac{a}{1}\)
        est un morphisme d'anneaux injectif.
    \end{q}
    \begin{q}{4}
        On se donne \(K'\) un corps et \(\psi\colon A\to K'\) un morphisme d'anneaux
        injectif.
        \begin{q}{a}
            Montrer qu'il existe un unique morphisme de corps \(\alpha\colon:K\to K'\)
            tel que \(\alpha\iota=\psi\)
        \end{q}
        \begin{q}{b}
            On suppose que \(\eta\colon A\to K'\) est un morphisme d'anneaux injectif qui
            vérifie la propriété suivante : pour tout morphisme d'anneaux injectif \(\varphi
            \colon A\to L\) avec \(L\) un corps il existe un unique morphisme de corps
            \(\beta\colon K'\to L\) tel que \(\beta_\eta=\varphi\).
            Montrer que \(K\) et \(K'\) sont isomorphes.
        \end{q}
    \end{q}
\end{exo}

\begin{exo}
    On considère l'anneau des entiers de \textsc{Gauss} \(\Z[i]\) et l'anneau quotient
    \(A = \Z[i]/(1+3i)\).
    \begin{q}{1}
        Montrer que \(i\equiv 3\) dans \(A\).
    \end{q}
    \begin{q}{2}
        En déduire que le morphisme d'anneau \(\phi\colon \Z\to A\) est surjectif.
    \end{q}
    \begin{q}{3}
        Démontrer l'isomorphisme d'anneau \(A\simeq \Z/10\Z\).
    \end{q}
    \begin{q}{4}
        Quelle est la caractéristique de \(A\) ?
        \boxans{D'après la question précédente, \(A\) est de caractéristique 10.}
    \end{q}
\end{exo}

\begin{exo}
    On pose \(j=e^{2i\pi/3}\) et on note \(\Z[j]\) le sous anneau de \(C\) pertinent.
    \begin{q}{1}
        Montrer que \(2\) n'est pas inversible dans \(\Z[j]\).
    \end{q}
    \begin{q}{2}
        Expliquer pourquoi \(\Z[j]\) est un anneau intègre.
    \end{q}
    \begin{q}{3}
        Justifier que pour tout \(P\in\Z[x]\) il existe deux polynômes uniques \(Q\) et \(R\)
        de \(\Z[x]\) tels que \(P=(X^2+X+1)Q+R\) et \(R\) est de degré inférieur ou égal à 1.
    \end{q}
    \begin{q}{4}
        Montrer que l'application \(p\colon\Z[x]\to\Z[j]\colon P\mapsto P(j)\) est un
        morphisme d'anneaux surjectif.
    \end{q}
    \begin{q}{5}
        Montrer que les deux anneaux \(\Z[x]/(X^2+X+1)\) et \(\Z[j]\) sont isomorphes.
    \end{q}
    \begin{q}{6}
        En déduire que l'idéal \((X^2+X+1)\) de \(\Z[X]\) est premier mais pas maximal.
    \end{q}
\end{exo}

\begin{exo}
    On considère dans l'anneau \(A=\Z[X]\) l'idéal \(I\) engendré par \(2\) et \(X\).
    \begin{q}{1}
        Quels sont les irréductibles de \(A\) ?
    \end{q}
    \begin{q}{2}
        L'idéal \(I\) est-il engendré par un unique élément ? Que peut-on en déduire ?
    \end{q}
    \begin{q}{3}
        Les idéaux (2) et \((X)\) sont-ils premiers ? maximaux ?
    \end{q}
    \begin{q}{4}
        Montrer que l'identité de \textsc{Bézout} n'est pas valable dans \(A\).
    \end{q}
\end{exo}

\section*{Polynômes}

\end{document}