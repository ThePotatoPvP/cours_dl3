\documentclass{report}
\usepackage[T1]{fontenc}
\usepackage[utf8x]{inputenc}
\usepackage{lmodern}
\usepackage{amsmath,amsthm,amsfonts,amssymb}
\usepackage{graphicx}
\usepackage[shortlabels]{enumitem}
\usepackage{xcolor}
\usepackage{accents}
\usepackage{titlesec}
\usepackage{etoolbox}
\usepackage{bookmark}
\usepackage{stmaryrd}
\usepackage{fancyhdr}
\usepackage[margin=25mm]{geometry}
\usepackage[frenchb]{babel}
\usepackage{changepage}
\usepackage{listings}
\usepackage[most,breakable,listings]{tcolorbox}

\date{\today}
\author{Corentin Sallin \thanks{Cours de L3}}
\title{Calcul Différentiel et topologie}


\renewcommand{\theenumi}{\Alph{enumi}}

\newcommand{\N}{\mathbb{N}}
\newcommand{\Z}{\mathbb{Z}}
\newcommand{\R}{\mathbb{R}}
\newcommand{\Q}{\mathbb{Q}}
\newcommand{\C}{\mathbb{C}}
\newcommand{\K}{\mathbb{K}}
\renewcommand{\L}{\mathcal{L}}
\newcommand{\M}{\mathcal{M}}
\newcommand{\T}{\mathcal{T}}
\newcommand{\Diag}{\mathcal{D}}
\newcommand{\curlyv}{\mathcal{V}}

\newcommand{\esp}{\mathbb{E}}
\newcommand{\proba}{\mathbb{P}}


\newcommand{\CC}{\mathcal{C}}
\newcommand{\Int}{\mathrm{Int}}
\newcommand{\id}{\mathrm{id}}
\newcommand{\eps}{\varepsilon}
\newcommand{\mnr}{\mathcal{M}_n(\R)}
\newcommand{\mnk}{\mathcal{M}_n(\K)}
\newcommand{\glnr}{\mathrm{GL}_n(\R)}
\newcommand{\glnk}{\mathrm{GL}_n(\K)}

\newcommand{\D}{\mathop{}\!\mathrm{d}}
\newcommand{\ds}{\displaystyle}
\newcommand*{\ensemble}[3][]{#1\{ #2 \mid #3 #1\}}
\newcommand{\Vvert}{\vert\kern-0.25ex\vert\kern-0.25ex\vert}
\newcommand{\tnorm}[1]{\Vvert #1 \Vvert}

\DeclareMathOperator{\Ker}{Ker}
\DeclareMathOperator{\Isom}{Isom}
\DeclareMathOperator{\Tr}{Tr}

\pagestyle{fancy}
\fancyhf{}
\fancyhead[L]{Corentin Sallin}
\fancyhead[R]{}
\fancyfoot[L]{\leftmark}
\fancyfoot[R]{Page : \thepage}

\renewcommand{\headrulewidth}{2pt}
\renewcommand{\footrulewidth}{1pt}


\theoremstyle{definition}
\newtheorem{exo}{Exercice}
%\newtheorem{q}{\(\quad\)}[exo]

\newtheorem{theorem}{Théorème}[section]
\newtheorem{lemma}[theorem]{Lemme}
\newtheorem*{definition}{Définition}
\newtheorem{lemme}[theorem]{Lemme}
\newtheorem*{corollary}{Corollaire}

\newtheorem{proposition}{Proposition}[theorem]
\newtheorem*{rk}{Remarque}
\newtheorem*{example}{Exemple}

\definecolor{main1white} {RGB}{121, 129, 134}
\definecolor{main1white2} {RGB}{204, 204, 204}

\newenvironment{q}[1]{
    \begin{adjustwidth}{1cm}{}
    \textbf{#1} : }{
    \end{adjustwidth}}

\titleformat{\chapter}[display]
    {\normalfont\bfseries}{}{0pt}{\Huge}

\patchcmd{\chapter}{\thispagestyle{plain}}{\thispagestyle{fancy}}{}{}
\newcommand{\itt}{\item[\(\triangleright\)]}

\newcommand{\boxans}[1]{
    \begin{tcolorbox}[
            breakable,
            enhanced,
            interior style      = {
                left color      = main1white2!65!gray!8,
                middle color    = main1white2!50!gray!7,
                right color     = main1white2!35!gray!6
            },
            %borderline north    = {.3pt}{0pt}{main1white!10},
            %borderline south    = {.3pt}{0pt}{main1white!10},
            frame hidden,
            borderline west     = {2pt}{0pt}{main1white!30},
            sharp corners       = downhill,
            arc                 = 0 cm,
            boxrule             = 0 cm,
            %nobeforeafter,
            %before={},
            %nobeforeSTYLE,
            %noafterSTYLE,
            %after=\par\nointerlineskip
            %source=remy
        ]
        #1
    \end{tcolorbox}
}

\begin{document}

\begin{center}
    \Huge{\textbf{Algebre I}}
\end{center}
\bigskip
mails des teachers : riccardo.brasca@gmail.com | gentiana@math.univ-paris-diderot.fr

\section*{Rappels}

\begin{definition}
    Soit \(E\) un ensemble, une relation \(\mathcal{R}\) sur \(E\) est un sous
    ensemble \(\mathcal{R}\subseteq E\times E\). On écrit \(x\mathcal{R}y\) si \((x,y)\in\mathcal{R}\).
    Si la relation est reflexive, symétrique et transitive elle est dite \textbf{d'équivalence}.
\end{definition}

\begin{example}
    Soit \(f:E_1\to E_2\) telle que \(\forall x,y\in E_1,\ x\mathcal{R}y \Leftrightarrow f(x)=f(y)\)
\end{example}

\begin{definition}
    Soit \(E\) un ensemble, \(\mathcal{R}\) une relation d'équivalence, \(x\in E\), la \textbf{classe d'équivalence} de \(x\) est cl\((x)\)=\(\{y\in E | x\mathcal{R}y\}\).
    L'ensemble \textbf{quotient} \(E/\mathcal{R}\) de \(\mathcal{R}\) est l'ensemble des classes d'équivalences de \(\mathcal{R}\).
\end{definition}

\begin{definition}
    On appelle \(\pi : x \mapsto \text{cl}(x)\) l'application cannonique vers \(E/\mathcal{R}\), elle est surjective.
\end{definition}

\begin{definition}
    Soit \(\omega\in E/\mathcal{R}\), chacun de ses éléments peut être appelé \textbf{représentant}.
    Un système de représentants est un sous ensemble \(M\subseteq E\) formé d'un représentant par classe. (\(\forall \omega\in E, \exists !m\in M, \pi(m)=w\))
\end{definition}

\begin{rk}
    \(E/=\ = \{\{x\}|x\in E\} \neq E\)
\end{rk}

\begin{definition}
    On peut définir les opérations usuelles sur les classes d'équivalences
    Soient \(\omega_1, \omega_2 \in \Z\)
    \begin{enumerate}
        \itt \(\omega_1+\omega_2=\text{cl}(x_1+x_2, y_1+y_2)\)
        \itt \(\omega_1\times\omega_2=\text{cl}(x_1x_2+y_1y_2, x_1y_2+x_2y_1)\)
    \end{enumerate}
\end{definition}

\begin{lemma}
    Il n'y a jamais d'intersection entre des classes non égales.
\end{lemma}

\begin{theorem}
    Soit \(f : E\to F\) telle que \(x\mathcal{R}y \Leftrightarrow f(x)=f(y)\) alors
    \(\exists ! \tilde{f} : E/\mathcal{R} \to F\) telle que \(\tilde{f}\circ \pi = f\).
    \textit{(les petits triangles commutatifs)}
\end{theorem}

\begin{rk}
    Soit \(E\) un ensemble et \(\mathcal{R}\) une relation, il
    existe \(M\subseteq E\) un système de représentants \textit{(axiome du choix ?)}
\end{rk}

\end{document}