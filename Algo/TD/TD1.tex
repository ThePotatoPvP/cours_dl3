\documentclass{report}
\usepackage{../../exercices}

\begin{document}
\begin{exo}
    Vrai- Vrai - Vrai - Faux - Vrai - Vrai - Faux - Vrai
    Vrai - Vrai - Faux - Faux - Vrai(sterling) - Faux - Vrai
\end{exo}

\begin{exo}
    Analyse de complexité.
    \(nlog(n) - nlog(n) - n^2 - log^2(n)\)
\end{exo}

\begin{exo}
    Algorithmes de graphes.
    Algo1 est en \(\Theta(n^2)\) le nombre de sommets si représenté par une matrice
    d'ajacence et en \(\Theta(n)\) le nombre d'arrêtes si en liste d'adjacence. \\
    Algo2 associe à chaque sommet le poids de son arrête incidente la plus légère.
    sa complexité est en \(\Theta(|V|)\) avec constante 3. Pour le rendre plus
    efficace on peut établir un ordre de parcours du graphe afin de ne pas vérifier
    plusieurs fois le poids d'une arrête, on abaisse ainsi la constante
    multiplicative.
\end{exo}

\begin{exo}
    Problème du voyageur de commerce
    \begin{q}{1}
        Il suffit pour cela de faire un graphe complet où chaque ville de France
        est représentée par un sommet et chaque arrête a pour poids la distance entre deux villes.
    \end{q}
    \begin{q}{2}
        On peut calculer le poids de chaque chemin hamiltonien du graphe dans un tableau
        puis le trier pour récupérer le plus court.
    \end{q}
    \begin{q}{3}
        Grâce à un raisonnement combinatoire on obtient que cet algorithme
        a une complexité en \(O(n!)\)
    \end{q}
\end{exo}

\begin{exo}
    Échange de cavaliers.
    \begin{q}{1}
        Si on modélise l'échiquier par un graphe avec une arrête entre deux cases
        si et seulement si le passage de l'une à l'autre est possible on a un graphe pertinent.
    \end{q}
    \begin{q}{2}
        Le début ressemble à 1. N4c3 Na4 2. Nb1 Ndb2 3. Na3 Nc3 4. Nc2 Nba4 5. Na1 Nb1 6. Nb3 Nac3 7. Na1 Na3
        8. Nb3 Ncb1 9. Nc3 Nc2 10. Na4 Nca3 11. Nb2 Nc2 12. Nd3 Nc3 13. Na1 Na3 14. Nc2
        Nab1 15. Na3 Na2 16. Nc2 Nbc3 17. Na3 Na4 18. Nb1 Nb2 19. Nc3S
    \end{q}
    \begin{q}{3}
        Les cases 3 et 9 sont pas utilisées.
    \end{q}
\end{exo}

\end{document}