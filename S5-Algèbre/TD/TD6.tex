\documentclass{report}
\usepackage{../../exercices}

\begin{document}

\begin{center}
    \huge{\textbf{Anneaux et morphismes d'anneaux}}
\end{center}

\begin{exo}
    Familiarisation avec le concept d'anneau.
    \begin{q}{1}
        Quels sont les sous-anneaux de \(\Z\) ?
        \boxans{Soit \(\left(A,+,\times\right)\) un sous-anneau de \(\Z\), alors
        en particulier \(\left(A,+\right)\) est un sous groupe de \(\Z\) donc \(A\)
        est de la forme \(n\Z\). Or pour que ce soit un anneau il faut \(1\in A\) donc
        \(A=\Z\). L'anneau trivial \(A=\{0\}\) mérite aussi d'être mentionné.}
    \end{q}
    \begin{q}{2}
        Quels sont les sous-anneaux de \(\Z/26\Z\) ?
        \boxans{Comme au dessus, en dehors de l'anneau trivial, la classe de \(1\)
        est dans le sous-anneaux et génère ainsi l'anneau entier.}
    \end{q}
    \begin{q}{3}
        Déterminer tous les sous-anneaux de \(Z/4\Z\times \Z/4\Z\).
        \boxans{Soit \(A\) un sous-anneau non trivial, \((1,1)\) est dans \(A\)
        et donc \(\{(x,x)\mid x \in\llbracket 0,4\rrbracket\}\) est un sous-anneau
        recherché. On peut étendre en donnant un ecart de deux entre las deux côtés pour
        avoir un autre sous-anneaux de cardinal \(8\). Enfin on a l'anneau entier.}
    \end{q}
\end{exo}

\begin{exo}
    \begin{q}{1}
        Quels sont les morphismes d'anneaux ...
        \begin{enumerate}
            \itt ... de \(\Z\) dans \(\Z/3\Z\) ?
            \boxans{Le morphisme doit vérifier \(\varphi(1)=\cl(1)\) or c'est un
            morphisme additif donc finalement \(x\mapsto \cl(x)\) est le seul.}
            \itt ... de \(\Z/5\Z\) dans \(\Z/3\Z\) ?
            \boxans{Un tel morphisme doit vérifier \(\varphi(1)=\cl(1)\), or cet élément est d'ordre
            \(3\) d'un côté et \(5\) de l'autre, on a donc pas un morphisme. Il n'y a
            pas de morphisme entre ces deux anneaux.}
            \itt ... de \(\Z/12\Z\) dans \(\Z/3\Z\) ?
            \boxans{Un tel morphisme doit vérifier \(\varphi(1)=\cl(1)\) ainsi la structure
            de groupe additif donne que le seul morphisme possible est \(x\mapsto \cl(x)\)
            qui est bien défini.}
        \end{enumerate}
    \end{q}
    \begin{q}{2}
        Donner une condition nécessaire et suffisante pour qu'il existe un morphisme
        de \(\Z/a\Z\) dans \(\Z/b\Z\) (avec \(a,b\in\N^*\)).
        \boxans{Montrons qu'un tel morphisme existe ssi \(b\mid a\)
        \begin{enumerate}
            \item[\(\Rightarrow\)] Le morphisme \(\cl_a(n) \mapsto \cl_b(n)\) est bien défini.
            \item[\(\Leftarrow\)] L'ordre de \(\varphi(n)\) doit diviser l'ordre de \(n\)
            pour que le morphisme soit bien défini, donc \(b\mid a\).
        \end{enumerate}}
    \end{q}
\end{exo}

\begin{exo}
    Sous-anneaux de \(\R\).
    \begin{q}{1}
        Montrer que \(\Q+\Q\sqrt{2}\colon=\{x+y\sqrt{2}\mid x,y\in\Q\}\) est un
        sous-anneau de \(\R\)
        \boxans{On a bien \(1=1+0\sqrt{2}\in\Q+\Q\sqrt{2}\) et c'est trivialement
        stable par multiplication et somme.}
    \end{q}
    \begin{q}{2}
        Donner une condition nécessaire et suffisante sur le réel \(\alpha\) pour que
        \(\Q+\Q\alpha\) soit un sous-anneau de \(\R\).
        \boxans{Il faut \(\alpha^2\in \Q+\Q\alpha\) et donc \(\alpha^2 = \xi\alpha + \chi\),
        c'est un nombre algébrique de dimension 2.}
    \end{q}
\end{exo}

\begin{exo}
    On définit \(A=Z[j]=\{a+jb\mid a,b\in\Z\}\) où \(j=\exp\left(\frac{2i\pi}{3}\right)\).
    \begin{q}{1}
        Montrer que \(A\) est un sous-anneau de \(\C\).
        \boxans{\(0\in A\). Soient \(x = a+jb, y = c+jd \in A\) alors \(x-y=(a-c)+j(b-d)\in A\) car
        \(\left(\Z,+\right)\) est un groupe, et \(x\times y=ac + j(bc + ad) + j^2(bd)
        =ac +j(bc +ad) + -(1+j)bd\in A\). L'ensemble est donc bien un sous-anneau de \(\C\).}
    \end{q}
    On désigne par \(\mathcal{U}(A)\) le groupe des éléments inversibles de \(A\)
    et on pose, pour tout \(z\in\C\), \(N(z)=z\bar{z}\).
    \begin{q}{2}
        \begin{q}{a}
            Montrer que si \(z\in A\) alors \(N(z)\in\N\).
            \boxans{Soit \(z=a+jb\in A\), \(N(z) = (a+jb)(a+\bar{j}b) = a^2 + (j+\bar{j})ab + j\bar{j}b^2
            = a^2 -ab + b^2\) car \(\bar{j}=j^2\)}
        \end{q}
        \begin{q}{b}
            Soit \(z\in A\). Montrer que \(z\in\mathcal{U}(A)\) ssi \(N(z)=1\).
        \end{q}
        \begin{q}{c}
            Soient \(a,b\in\Z\). Montrer que si \(N(a+jb)=1\) alors
            \(a,b\in\{-1,0,1\}\).
        \end{q}
    \end{q}
    \begin{q}{3}
        Décrire le groupe \(\mathcal{U}(A)\) et en déterminer les éléments d'ordre 3.
    \end{q}
    \begin{q}{4}
        Soit \(\Phi\colon\Q[X]\to C,P\mapsto P(j)\).
        \begin{q}{a}
            Montrer que \(\Phi\) est un homomorphisme d'anneaux
        \end{q}
        \begin{q}{b}
            Déterminer le noyau de \(\Phi\).
        \end{q}
        \begin{q}{c}
            Montrer que \(\Im(\Phi)=\{a+jb\mid a,b\in\Q\}\) et que c'est un sous-corps
            de \(\C\).
        \end{q}
    \end{q}
\end{exo}

\begin{exo}
    L'ensemble \(\mathcal{M}_n(A)\) des matrices \(n\times n\) à coefficients dans un anneau abélien \(A\).
\end{exo}

\begin{exo}
    Considérons l'ensembke \(A\in\mathcal{M}_2(\R)\) défini par
    \[ A\colon=\left\{ \begin{pmatrix} a&b\\-b&a \end{pmatrix}\mid a,b\in\R\right\}\]
    \begin{q}{1}
        Montrer que \(A\) est un sous-anneau de \(\mathcal{M}_2(\R)\).
    \end{q}
    \begin{q}{2}
        Montrer que \(A\) est abélien.
        \boxans{Soient \(a,b,x,y\in\R\) on a \begin{enumerate}
            \itt \(\begin{pmatrix}a&b\\-b&a\end{pmatrix}\times
            \begin{pmatrix}x&y\\-y&x\end{pmatrix} =
            \begin{pmatrix}ax+by&ax-by\\-bx-ay&-by+ax\end{pmatrix}\)
            \itt \(\begin{pmatrix}x&y\\-y&x\end{pmatrix}\times
            \begin{pmatrix}a&b\\-b&a\end{pmatrix} =
            \begin{pmatrix}ax+by&ax-by\\-bx-ay&-by+ax\end{pmatrix}\)
            \end{enumerate}
            Ainsi \(A\) est abélien.
        }
    \end{q}
    \begin{q}{3}
        Montrer que \(A\) est isomorphe à \(\C\).
        \boxans{On pose \(\varphi\begin{pmatrix}1&0\\0&1\end{pmatrix}=1\) et
        \(\varphi\begin{pmatrix}0&1\\-1&0\end{pmatrix} = i\) on a une application
        linéaire bien définie qui est clairement un isomophisme d'anneaux.}
    \end{q}
\end{exo}

\begin{exo}
    Soit \(A\) un anneau intègre. Soit \(x\in A, x\neq 0\).
    \begin{q}{1}
        Montrer que l'application \(\phi\colon A\to A\) définit par \(\phi(a)=ax\)
        est un morphisme de groupe injectif.
        \boxans{Soient \(a,b\in A\) tels que \(\phi(a)=\phi(b) \Rightarrow ax = bx
        \Rightarrow ax-bx=0\Rightarrow (a-b)x=0\). Ainsi comme \(x\neq 0\) et que \(A\)
        est intègre \(a-b=0\) et donc \(a=b\).}
    \end{q}
    \begin{q}{2}
        En déduire que si \(A\) est fini alors \(A\) est un corps
        \boxans{Soit \(a\in A\) non neutre, par finitude de \(A\) il existe \(n\in\N^*\)
        tel que la famille \(a,a^2,\dots,a^n\) contient des répétitions, ainsi il existe \(m\in\N^*\)
        minimal tel que \(a^m=a\) alors comme \(a^m=a^{m-1}\cdot a = a\). Montrons
        \(a^{m-1}=1\). On sait que \(a^m-a = 0\) donc \((a^{m-1} - 1)\cdot a = 0\) et donc
        comme \(A\) est intègre et que \(a\) est non neutre, \(a^{m-1}-1=0\) et finalement
        \(a^{m-1}=1\).}
    \end{q}
\end{exo}
\end{document}
