\documentclass{report}
\usepackage{../../exercices}

\begin{document}

\begin{center}
    \huge{\textbf{Théorème de Lagrange, sous-groupes distingués,
    groupe quotien}}
\end{center}

\begin{exo}
    Soit \(p\) un nombre premier et \(G\) un groupe d'ordre \(p\). Montrer que \(G\) est cyclique.
    \boxans{Soit \(g\in G\) alors d'après le théormème de \textsc{Lagrange} on
    a \(o(g)\mid p\) donc soit \(o(g)=1\) aucquel cas \(g=e\) soit \(o(g)=p\)
    aucquel cas \(\langle g\rangle\subset G\) et les deux ensembles sont de
    même cardinal, donc \(g\) engendre \(G\) qui est donc cyclique.}
\end{exo}

\begin{exo}
    Soit \(f\colon G\to H\) un morphisme de groupes finis et \(K\) un sous-groupe
    de \(G\)
    \begin{q}{1}
        Montrer que l'ordre de \(f(K)\) divise les ordres de \(K\) et de \(H\)
        \boxans{D'après le premier théorème d'isomorphisme, pour un groupe \(K\) fini
        on a \(|K|=|\Ker(f)|\times|\Im(f)|\) ce qui donne la permière partie de la réponse.
        Enfin on montre que l'ordre de \(f(K)\) divise celui de \(H\) par le théorème de
        \textsc{Lagrange} car c'est un sous-groupe.}
    \end{q}
    \begin{q}{2}
        Montrer que si l'ordre de \(K\) est premier à l'ordre de \(H\) alors
        \(K \subset \Ker(f)\)
        \boxans{Dans ce cas \(|f(K)|\) divise deux nombres premiers entre eux,
        donc \(|f(K)|=1\) et \(e\in f(K)\) donc \(f(K)=\{e\}\)}
    \end{q}
\end{exo}

\begin{exo}
    Soient \(G\) et \(G'\) deux groupes finis. On suppose que les ordres de ces groupes
    sont premiers entre eux. Déterminer tous les morphismes de groupes de \(G\) dans \(G'\)
    \boxans{Soit \(f\in\textbf{Hom}(G,G')\), \(|f(G)|\) divise \(G'\) or
    \(|f(G)|=\frac{|G|}{|\Ker(f)|}\) ainsi \(f(G)\) a un cardinal qui divise deux
    nombres premiers entre eux et donc \(f(G)=e\). La synthèse est immédiate.}
\end{exo}

\begin{exo}
    Montrer que si \(H\) et \(K\) sont deux sous-groupes distinguées d'un groupe \(G\)
    alors leur intersection est distinguée.
    \boxans{On se passera de démontrer que \(H\cap K\) est bien un groupe, montrons qu'il est
    normal. Soient \(x\in\G\) et \(h\in H\cap K\) alors en particulier \(h\in H\) donc
    \(x=hxh^{-1}\). On a ainsi bien que \(H\cap K\) est un sous groupe normal de \(G\).}
\end{exo}

\begin{exo}
    Montrer que si \(N\) est distingué dans \(G\) et \(H\) un sous groupe quelconque de
    \(G\) alors \(N\cap H\) est distingué dans \(H\).
    \boxans{L'énoncé nous indique que \(\forall x\in G, \forall n\in n, nxn^{-1} \in G\), ce qui
    reste vrai si on réduit les ensembles car les quantificateurs sont universels, ainsi on a bien
    \(\forall h\in H, \forall y\in H\cap N, yhy^{-1}\in H\) par stabilité de \(H\).}
\end{exo}

\begin{exo}
    Soient deux sous-groupes distingués \(H\) et \(K\) de \(G\) tels que \(H\cap K=\{e\}\)
    Montrer que \(\forall x\in H,\forall y\in K, xy=yx\).
    \boxans{Soit \(h\in H\) on a par hypothèse de l'énoncé
    \(\forall k\in K, khk^{-1}=h\) ainsi en multipliant à droite par \(k\) on a \(kh=hk\).}
\end{exo}

\begin{exo}
    Soient les éléments \(a=(1,2,3,4)\) et \(b=(2,2)\circ(3,4)\) dans \(\mathfrak{S}_4\)
    et \(G\) le sous groupe de \(\mathfrak{S}_4\) engendré par \(a\) et \(b\). Soit
    \(H=\langle b,a^2\rangle\) et \(K=\langle b\rangle\). Montrer que \(H\) est un sous-groupe
    distingué dans \(G\), que \(K\) est un sous groupe distingué dans \(H\), mais que
    \(K\) n'est pas un sous-groupe distingué dans \(G\).
\end{exo}
\end{document}