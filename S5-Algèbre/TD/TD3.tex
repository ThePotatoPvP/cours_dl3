\documentclass{report}
\usepackage{../../exercices}

\begin{document}

\begin{center}
    \huge{\textbf{TD3: Morphismes de groupes}}
\end{center}

\begin{exo}
    On considère les sous-groupes de \(\text{GL}_2(\R)\) suivants :
    \[ G=\left\{ \begin{pmatrix}1&a\\0&b\end{pmatrix}
        \mid b\neq 0 \right\} \quad
       H=\left\{ \begin{pmatrix}1&0\\0&b\end{pmatrix}
       \mid b \neq 0 \right\} \quad
       K=\left\{ \begin{pmatrix}1&a\\0&1\end{pmatrix}
       \mid a \in \R \right\}
    \] Montrer que :
    \begin{q}{1}
        \(f : A\in G \mapsto b\) est un morphisme et que \(K=\Ker(f)\).
    \end{q}
    \begin{q}{2}
        \(g : A\in G \mapsto a\) n'est pas un morphisme.
    \end{q}
    \begin{q}{3}
        \(h: \R^*\to H\colon b \to B\in H\) est un isomorphisme.
    \end{q}
    \begin{q}{4}
        \(\varphi: \R\to K\colon a \to A\in K\) est un isomorphisme.
    \end{q}
\end{exo}

\begin{exo}
    Soit \(G\) un groupe. Montrer que \(x\mapsto x^{-1}\) est un morphisme
    ssi \(G\) est un groupe abélien.
    \boxans{Soient \(x,y\in G\), \((xy)^{-1}=y^{-1}x^{-1}\) ce qui vaut \(x^{-1}y^{-1}\)
    pour toute paire d'éléments si et seulement si G est abélien}
\end{exo}

\begin{exo}
    Soit \(G, H\) des groupes et \(f\) un morphisme de \(G\) dans \(H\).
    \begin{q}{1}
        Si \(G\) est abélien a-t-on \(f(G)\) abélien ?
        \boxans{\(\forall x,y\in f(G), f(x)f(y) = f(xy) = f(yx) = f(y)f(x)\)}
    \end{q}
    \begin{q}{2}
        Si \(H\) est abélien a-t-on \(f^{-1}(H)\) abélien ?
        \boxans{\(\forall x,y\in f^{-1}(H), xy = f^{-1}(x'y') =
        f^{-1}(y'x') = f^{-1}(y')f^{-1}(x')=yx\)}
    \end{q}
\end{exo}

\begin{exo}
    Soit \(\left(G,\star\right)\) un groupe. Montrer l'équivalence des propriétés :
    \begin{enumerate}
        \item[(i)] \(G\) est abélien
        \item[(ii)] \(\forall a,b\in G, (a\star b)^2 = a^2\star b^2\)
        \item[(iii)] \(\forall a,b\in G, (a\star b)^{-1} = a^{-1}\star b^{-1}\)
        \item[(iv)] \(x\mapsto x^{-1}\) est un automorphisme.
    \end{enumerate}
    \boxans{Procédons par une chaîne d'implications.
    \begin{enumerate}
        \itt \((i)\Rightarrow(ii)\) Si \(G\) est abélien alors \(\left(a\star b\right)^2
        = a\star b\star a\star b = a\star a\star b\star b = a^2\star b^2\)
        \itt \((ii)\Rightarrow(iii)\)
        \itt \((iii)\Rightarrow(iv)\) L'application est un morphisme par hypothèse.
        De plus elle est bijective, deux éléments de même inverse sont égaux.
        \itt \((iv)\Rightarrow(i)\) Voir exercice 2.
    \end{enumerate}}
\end{exo}

\begin{exo}
    Montrer qu'on a un morphisme \(\R\to\text{GL}_2(\R)\) donné par
    \[\theta \mapsto \begin{pmatrix} \cos(\theta)&\sin(\theta)\\
        -\sin(\theta)&\cos(\theta)\end{pmatrix}\]
    \boxans{On remarque que le noyau de ce morphisme est \(2\pi\Z\)}
\end{exo}

\begin{exo}
    Soit \(n\in\N\)
    \begin{q}{1}
        Montrer que l'application \(\det\) est un morphisme surjectif sur \(\R^\times\),
        quel est son noyau ?
        \boxans{Son noyau est le groupe simple linéaire.}
    \end{q}
    \begin{q}{2}
        Les applications qui à \(M\in \glnr\) associent respectivement
        \(^tM, M^{-1}, ^tM^{-1}\) sont-elles des morphismes ?
    \end{q}
\end{exo}

\begin{exo}
    Montrer que \(H=\left\{ (x,y)\in\Z^2\mid x+y\in 2\Z \right\}\)
    est un sous-groupe de \(\Z^2\) et qu'il est isomorphe à \(\Z^2\)
\end{exo}

\begin{exo}
    Soit \(G\) un groupe fini. Déterminer les morphismes de \(G\) dans \(\Z\)
\end{exo}

\begin{exo}
    Quelques remarques sur des groups usuels :
    \begin{q}{1}
        Le groupe \(\left(\R,\times\right)\)est-il isomorphe à \(\left(\C,\times\right)\) ?
        \boxans{Démontrons que non par l'absurde, on suppose \(\varphi\colon\C^*\to\R^*\)
        un isomorphisme, alors \(\varphi(i)^4 = \varphi(i^4)=\varphi(1)=1\) donc \(\varphi(i)^2 = \pm 1\).
        Si \(\varphi(i^2)=1, \varphi(-1)=\varphi(1)\) et \(\varphi\) n'est pas injective, ce
        qui est absurde. Sinon \(\varphi(i^2)=-1\) alors \(\varphi(i)^2=-1\) or
        \(\varphi(i)\in\R\) donc a un carré positif, on a bien une absurdité.}
    \end{q}
    \begin{q}{2}
        Le groupe \(\left(\C,+\right)\) est-il isomorphe au groupe \(\left(\C,\times\right)\) ?
        \boxans{Tout élément non nul de \(\left(\C,+\right)\) est d'ordre infini, ce qui n'est pas
        le cas pour \(\left(\C,\times\right)\). Les deux groupes ne sont donc pas isomorphes.}
    \end{q}
    \begin{q}{3}
        Trouver tous les endomorphismes de \(\left(\Q,+\right)\).
        \boxans{Soit \(\varphi\) un tel endomorphisme, on a \(\varphi
        \left(\frac{p}{q}\right) = p\varphi\left(\frac{1}{q}\right)\)
        car c'est un morphisme, ainsi \(\varphi(1)=q\varphi\left(\frac1{q}\right)\Rightarrow
        \varphi\left(\frac{1}{q}\right)=\frac{\varphi(1)}q\). Donc \(\varphi\left(\frac{p}{q}\right)
        =\varphi(1)\frac{p}{q}\). Ainsi les endomorphismes sont les multiplications
        par un rationel, la synthèses est immédiate. Donc
        \(\left(\textbf{End}(\Q, +),\circ\right)\simeq \left(\Q, +\right)\) et
        \(\left(\textbf{Aut}(\Q, +),\circ\right)\simeq \left(\Q^*, \times\right)\)
        }
    \end{q}
    \begin{q}{4}
        Trouver tous les morphismes de \(\Z\) dans \(\Q\).
        \boxans{Soit \(\varphi\) un tel morphisme, il est entièrement déterminé
        par \(\varphi(1)\) ainsi les morphismes sont les multiplications par un rationel,
        la synthèse est immédiate : \(\textbf{Hom}(\Z, \Q)\simeq \Q\)}
    \end{q}
\end{exo}

\begin{exo}
    Pour tout couple \(\left(a,b\right)\) de \(\R^2\), on pose la matrice
    \(\mathcal{M}_{a,b}=\begin{pmatrix}a&-b\\b&a\end{pmatrix}\).\\ Soit \(S=
    \left\{ \mathcal{M}_{a,b}\colon \left(a,b\right)\in\R^2\backslash(0,0) \right\}\).
    Soit l'application \(f\colon S\to \R\colon \mathcal{M}_{a,b}\mapsto a^2+b^2\)
    \begin{q}{1}
        Montrer que \(S\) est un sous-groupe de \(\text{GL}_2(\R)\)
    \end{q}
    \begin{q}{2}
        Montrer que \(f\) est un morphisme de \(\left(S,\times\right)\) dans \((\R,\times)\).
    \end{q}
\end{exo}

\begin{exo}
    Pour tout couple \(\left(a,b\right)\) de \(\R^2\), on pose \(\mathcal{M}_{a,b}\)
    comme dans l'exercice précédent mais ici \(S=\left\{ M_{a,b}\mid \left(a,b\right)
    \in\R^2 \right\}\) et \(S^\star=S\backslash\{\mathcal{M}_{0,0}\}\). Soit l'application
    \(f\colon S\to\C\colon \mathcal{M}_{a,b}\mapsto a+ib\).
    \begin{q}{1}
        Montrer que \(S\) est un sous-groupe de \(\left(\mathcal{M}_2(\R), +\right)\)
    \end{q}
    \begin{q}{2}
        Montrer que \(f\) est un isomorphisme de \(\left(S,+\right)\) dans \(\left(\C,+\right)\)
    \end{q}
    \begin{q}{3}
        On cherche maintenant à travailler, comme dans l'exercice précédent, sur un groupe
        multiplicatif.
        \begin{q}{a}
            Montrer que \(f\) définit un morphisme de \(\left(S,\times\right)\) dans
            \(\left(\C,\times\right)\)
        \end{q}
        \begin{q}{b}
            Déterminer le noyau et l'image de ce morphisme.
        \end{q}
    \end{q}
\end{exo}
\end{document}