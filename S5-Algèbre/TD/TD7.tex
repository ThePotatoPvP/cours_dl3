\documentclass{report}
\usepackage{../../exercices}

\begin{document}

\begin{center}
    \huge{\textbf{TD7: Idéaux et anneaux quotients}}
\end{center}

\begin{exo} Familiarisation avec le vocabulaire
    \begin{q}{1}
        Quels sont les idéaux premiers de \(\Z\) ?
        \boxans{Les idéaux de \(\Z\) sont tous de la forme \(n\Z\) et
        \(\Z/n\Z\) est intègre ssi \(n\) est premier. Les idéaux premiers de
        \(Z\) sont donc les \(\Z/n\Z\) avec \(n\) premier.}
    \end{q}
    \begin{q}{1}
        Quels sont les idéaux maximaux de \(\Z\) ?
        \boxans{Soit \(n\in\N\), \(n\Z\) est un idéal de \(\Z\), \(n\Z\) contient
        tous les \(kn\Z\), \(k\in\N\). Ainsi \(n\Z\) n'est pas maximal si \(n\)
        admet des diviseurs autres que \(1\) et lui même, ainsi les idéaux maximaux
        sont les \(n\Z\) avec \(n\) premier.}
    \end{q}
\end{exo}

\begin{exo} Reflexion sur les diviseurs de \(0\)
    \begin{q}{1}
        Quels sont les diviseurs de \(0\) de \(\Z\) ? de \(\Z/n\Z\) ?
    \end{q}
    \begin{q}{2}
        Quels sont les diviseurs de \(0\) de l'anneau quotient \(\R[X]/(X^2+1)\) ?
    \end{q}
    \begin{q}{3}
        Soient \(A,B\) deux anneaux. Quels sont les diviseurs de zéros de l'anneau
        produit \(A\times B\) ?
    \end{q}
    \begin{q}{4}
        En déduire qu'un produit de deux anneaux unitaires n'est jamais intègre
        \textit{(et a fortiori n'est jamais un corps)}.
    \end{q}
\end{exo}

\begin{exo} Approche par les polynômes
    \begin{q}{1}
        Soit \(D=\{P\in\R[X]\mid P'(0)=0\}\). Montrer que \(D\) n'est pas un idéal de
        \(\R[x]\) mais que c'est un sous-anneau de \(\R[X]\).
    \end{q}
    \begin{q}{2}
        Soit \(E=\{P\in\R[X]\mid P'(0)=P(0)=0\}\). Montrer que \(E\) n'est pas un
        sous-anneau de \(\R[X]\) et que c'est un idéal de \(\R[X]\) dont on
        donnera un générateur.
    \end{q}
\end{exo}

\begin{exo}
    Notons \(\Z[i]\) l'ensemble des entiers de \textsc{Gauss}
    \begin{q}{1}
        \begin{q}{a}
            Monrer que \(\Z[i]\) est un sous-anneau de \(\C\).
        \end{q}
        \begin{q}{b}
            Est-ce que \(\Z[i]\) est un idéal de \(\C\) ?
        \end{q}
        \begin{q}{c}
            Notons \(I=\{a+ib\in\Z[i]\mid a+b\in2\Z\}\). Est-ce un idél de \(\Z[i]\) ?
        \end{q}
    \end{q}
    \begin{q}{2}
        Considérons maintenant l'anneau \(\Z[X]\).
        \begin{q}{a}
            Est-ce que \(Z\) est un sous-anneau unitaire de \(\Z[X]\) ?
            un idéal de \(\Z[X]\)
        \end{q}
        \begin{q}{b}
            Considérons l'application \(f\colon \Z[X]\to\C\) donnée par \(P\mapsto P(i)\).
            \begin{q}{i}
                Montrer que \(f\) est un morphisme d'anneaux unitaires.
            \end{q}
            \begin{q}{ii}
                Déterminer \(\Ker(f)\) et \(\Im(f)\).
            \end{q}
            \begin{q}{iii}
                Déduire de la question précédente l'existence d'un isomorphisme d'anneaux
                \[\quotient{\Z[X]}{(X^2+1)\Z[X]}\cong Z[i]\]
            \end{q}
        \end{q}
    \end{q}
\end{exo}

\begin{exo}
    Parmi les anneaux suivants, lesquels sont intègres ?
    \begin{enumerate}
        \itt \(\Z[i]\)
        \itt \(\mathcal{M}_n(\R)\)
        \itt \(\CC^0(I, \R)\) avec \(I\) un intervalle réel.
        \itt \(\R[X]/X\)
        \itt \(\R[X]/(X^2)\)
    \end{enumerate}
\end{exo}

\begin{exo}
    Soit \(A\) un anneau isomorphe à \(\Z/n\Z\) comme groupe additif. Montrer que
    \(A\) est isomorphe à \(\Z/n\Z\) en tant qu'anneau.
    \boxans{Soient, \(a,b\in A\) on définit \(a\times b = f^{-1}(f(a)\times f(b))\)
    qui donne la bonne structure d'anneau à \(A\).}
\end{exo}

\begin{exo}
    Soit \(K\) un corps et \(P\in K[X]\) un polynôme de degré \(d\in\N\).
    \begin{q}{1}
        Montrer que \(A=K[X]/P\) est un \(K\)-ev de dimension \(d\).
    \end{q}
    \begin{q}{2}
        Soit \(K\) un corps et \(P\) un polynôme irréductible, montrer que \(K[X]/P\)
        est un corps.
    \end{q}
    \begin{q}{3}
        Si \(K\) est un sous-corps de \(\C\) et \(\alpha\) est une racine de \(P\)
        dans \(\C\), montrer que \(K[X]/P\) est isomorphe au sous-corps \(K[\alpha]\subset C\).
        le plus petit sous-anneau de \(\C\) qui contient \(K\) et \(\alpha\)
    \end{q}
\end{exo}

\begin{exo}
    Montrer que le groupe additif \(\left(\Q/\Z,+\right)\) ne peut pas être
    adjoint d'une loi multiplicative afin de former une structure d'anneau.
    \boxans{Supposons par l'absurde une telle loi, notons \(\alpha\) son neutre qui
    est dans \(\Q/\Z\) par hypothèse, alors pour tous \(x,y\in\Q/\Z\) on a \(\alpha\times x=x\)}
\end{exo}
\end{document}