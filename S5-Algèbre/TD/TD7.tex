\documentclass{report}
\usepackage{../../exercices}

\begin{document}

\begin{center}
    \huge{\textbf{TD7: Idéaux et anneaux quotients}}
\end{center}

\begin{exo} Familiarisation avec le vocabulaire
    \begin{q}{1}
        Quels sont les idéaux premiers de \(\Z\) ?
        \boxans{Les idéaux de \(\Z\) sont tous de la forme \(n\Z\) et
        \(\Z/n\Z\) est intègre ssi \(n\) est premier. Les idéaux premiers de
        \(Z\) sont donc les \(\Z/n\Z\) avec \(n\) premier.}
    \end{q}
    \begin{q}{1}
        Quels sont les idéaux maximaux de \(\Z\) ?
        \boxans{Soit \(n\in\N\), \(n\Z\) est un idéal de \(\Z\), \(n\Z\) contient
        tous les \(kn\Z\), \(k\in\N\). Ainsi \(n\Z\) n'est pas maximal si \(n\)
        admet des diviseurs autres que \(1\) et lui même, ainsi les idéaux maximaux
        sont les \(n\Z\) avec \(n\) premier.}
    \end{q}
\end{exo}

\begin{exo} Reflexion sur les diviseurs de \(0\)
    \begin{q}{1}
        Quels sont les diviseurs de \(0\) dans \(\Z\) ? dans \(\Z/n\Z\) ?
        \boxans{\(\Z\) est intègre et ne possède sont pas de diviseurs de \(0\). Les
        diviseurs de \(0\) dans \(\Z/n\Z\) sont les classe d'équivalences des diviseurs de \(n\).}
    \end{q}
    \begin{q}{2}
        Quels sont les diviseurs de \(0\) de l'anneau quotient \(\R[X]/(X^2+1)\) ?
        \boxans{L'anneau quotient est intègre car le polynôme est irréductible.}
    \end{q}
    \begin{q}{3}
        Soient \(A,B\) deux anneaux. Quels sont les diviseurs de zéros de l'anneau
        produit \(A\times B\) ?
        \boxans{Les diviseurs de l'anneau produit sont les produits d'éléments dont
        les coordonnées ne s'alignent pas.}
    \end{q}
    \begin{q}{4}
        En déduire qu'un produit de deux anneaux unitaires n'est jamais intègre
        \boxans{\((0,1)\times(1,0)=(0,0)\).}
    \end{q}
\end{exo}

\begin{exo} Approche par les polynômes
    \begin{q}{1}
        Soit \(D=\{P\in\R[X]\mid P'(0)=0\}\). Montrer que \(D\) n'est pas un idéal de
        \(\R[X]\) mais que c'est un sous-anneau de \(\R[X]\).
        \boxans{
            Montrons que \(D\) n'est pas un idéal de \(\R[X]\). Soit \(P(X) = 1\)
            dans \(D\). On a \(XP\notin DX\) qui n'est donc pas un idéal car pas stable
            par multiplication externe.

            D'autre part, \(D\) est un sous-anneau de \(\R[X]\) car il est stable par addition
            et multiplication. Si \(P(X)\) et \(Q(X)\) sont dans \(D\), alors \((P+Q)'(0) = P'(0) + Q'(0) = 0\),
            et donc \(P+Q\) est dans \(D\). De plus, \((P \cdot Q)'(0) = P'(0)Q(0) + P(0)Q'(0) = 0\),
            donc \(P \cdot Q\) est également dans \(D\).
        }
    \end{q}
    \begin{q}{2}
        Soit \(E=\{P\in\R[X]\mid P'(0)=P(0)=0\}\). Montrer que \(E\) n'est pas un
        sous-anneau de \(\R[X]\) et que c'est un idéal de \(\R[X]\) dont on
        donnera un générateur.
        \boxans{
            \(E\) n'est pas un sous-anneau de \(\R[X]\) car il ne contient pas le
            neutre multiplicatif.

            En revanche, montrons que \(E\) est un idéal de \(\R[X]\). Soit \(P(X)\) dans \(E\) et \(Q(X)\) dans \(\R[X]\).
            On a \((P \cdot Q)'(0) = P'(0)Q(0) + P(0)Q'(0) = 0\), donc \(P \cdot Q\) est dans \(E\).
            De plus, si \(P(X)\) est dans \(E\), alors \((-P)'(0) = -P'(0) = 0\), donc \(-P\) est dans \(E\).
            Ainsi, \(E\) est un idéal de \(\R[X]\).

            Un générateur de \(E\) est \(X^2\), car tout polynôme \(P(X)\) dans \(E\) peut s'écrire comme \(P(X) = X^2 \cdot Q(X)\)
            où \(Q(X)\) est un polynôme quelconque de \(\R[X]\).
        }
    \end{q}
\end{exo}


\begin{exo}
    Notons \(\Z[i]\) l'ensemble des entiers de \textsc{Gauss}.
    \begin{q}{1}
        \begin{q}{a}
            Montrer que \(\Z[i]\) est un sous-anneau de \(\C\).
            \boxans{
                Pour montrer que \(\Z[i]\) est un sous-anneau de \(\C\), il suffit de vérifier que
                \(\Z[i]\) est stable par l'addition et la multiplication. Soient \(a+bi\) et \(c+di\) deux
                éléments de \(\Z[i]\), alors
                \begin{align*}
                    (a+bi) - (c+di) &= (a-c) + (b-d)i \in \Z[i], \\
                    (a+bi) \cdot (c+di) &= (ac - bd) + (ad + bc)i \in \Z[i],
                \end{align*}
                car \(a-c\) et \(b-d\) sont des entiers (et donc dans \(\Z\)).
            }
        \end{q}
        \begin{q}{b}
            Est-ce que \(\Z[i]\) est un idéal de \(\C\) ?
            \boxans{
                Non, \(\Z[i]\) n'est pas un idéal de \(\C\) car il n'est pas stable par multiplication
                avec les éléments de \(\C\). Par exemple, \(1 \in \Z[i]\), mais \(1 \cdot \frac 12 = \frac 12 \notin \Z[i]\).
            }
        \end{q}
        \begin{q}{c}
            Notons \(I=\{a+ib\in\Z[i]\mid a+b\in2\Z\}\). Est-ce un idéal de \(\Z[i]\) ?
            \boxans{
                Oui, \(I\) est un idéal de \(\Z[i]\). Pour le montrer, il suffit de vérifier que
                \(I\) est stable par multiplication avec les éléments de \(\Z[i]\). Soit \(a+bi \in I\)
                et \(c+di \in \Z[i]\), alors
                \begin{align*}
                    (a+bi) \cdot (c+di) &= (ac - bd) + (ad + bc)i,
                \end{align*}
                où \(ac - bd\) et \(ad + bc\) sont des entiers de même parité, car \(a+b\)
                et \(c+d\) sont dans \(2\Z\). Ainsi, \(I\) est stable par multiplication
                avec les éléments de \(\Z[i]\), donc c'est un idéal de \(\Z[i]\).
            }
        \end{q}
    \end{q}
    \begin{q}{2}
        Considérons maintenant l'anneau \(\Z[X]\).
        \begin{q}{a}
            Est-ce que \(\Z\) est un sous-anneau unitaire de \(\Z[X]\) ? Est-ce un idéal de \(\Z[X]\) ?
            \boxans{
                Oui, \(\Z\) est un sous-anneau unitaire de \(\Z[X]\) car il est stable par l'addition
                et la multiplication. Par contre ce n'est pas un idéal car \(\Z\) n'est pas
                stable par multiplication externe.
            }
        \end{q}
        \begin{q}{b}
            Considérons l'application \(f\colon \Z[X]\to\C\) donnée par \(P\mapsto P(i)\).
            \begin{q}{i}
                Montrer que \(f\) est un morphisme d'anneaux unitaires.
                \boxans{
                    Soient \(P(X), Q(X) \in \Z[X]\). Alors,
                    \begin{align*}
                        f(P + Q) &= (P + Q)(i) = P(i) + Q(i) = f(P) + f(Q), \\
                        f(P \cdot Q) &= (P \cdot Q)(i) = P(i) \cdot Q(i) = f(P) \cdot f(Q),
                    \end{align*}
                }
            \end{q}
            \begin{q}{ii}
                Déterminer \(\Ker(f)\) et \(\Im(f)\).
                \boxans{
                    \itt \(\Ker(f)\) est l'ensemble des polynômes \(P(X)\) tels que \(P(i) = 0\). Cela revient
                    à chercher les polynômes dont \(i\) est une racine. On trouve que \(\Ker(f) = (X^2 + 1)\Z[X]\).

                   \itt\(\Im(f) = \{P(i) \mid P(X) \in \Z[X]\}= \Z[i]\) par induction structurelle.
                }
            \end{q}
            \begin{q}{iii}
                Déduire de la question précédente l'existence d'un isomorphisme d'anneaux
                \[\quotient{\Z[X]}{(X^2+1)\Z[X]}\cong Z[i]\]
                \boxans{
                    Par le premier théorème d'isomorphisme, on a
                    \[\quotient{\Z[X]}{\Ker(f)} \cong \Im(f).\]
                    En utilisant les résultats de la question précédente, on a
                    \[\quotient{\Z[X]}{(X^2+1)\Z[X]} \cong \Z[i].\]
                }
            \end{q}
        \end{q}
    \end{q}
\end{exo}


\begin{exo}
    Parmi les anneaux suivants, lesquels sont intègres ?
    \begin{enumerate}
        \itt \(\Z[i]\) est intègre (par l'exo au dessus)
        \itt \(\mathcal{M}_n(\R)\) n'est pas intègre (nilpotents)
        \itt \(\CC^0(I, \R)\) avec \(I\) un intervalle réel. n'est pas intègre (fonctions nulles sur une partie de \(I\))
        \itt \(\R[X]/X\) est intègre car isomorphe à un corps
        \itt \(\R[X]/(X^2)\) n'est pas intègre, en effet \(X\) est diviseur de \(0\).
    \end{enumerate}
\end{exo}

\begin{exo}
    Soit \(A\) un anneau isomorphe à \(\Z/n\Z\) comme groupe additif. Montrer que
    \(A\) est isomorphe à \(\Z/n\Z\) en tant qu'anneau.
    \boxans{Soient, \(a,b\in A\) on définit \(a\times b = f^{-1}(f(a)\times f(b))\)
    qui donne la bonne structure d'anneau à \(A\).}
\end{exo}

\begin{exo}
    Soit \(K\) un corps et \(P\in K[X]\) un polynôme de degré \(d\in\N\).
    \begin{q}{1}
        Montrer que \(A=K[X]/P\) est un \(K\)-ev de dimension \(d\).
    \end{q}
    \begin{q}{2}
        Soit \(K\) un corps et \(P\) un polynôme irréductible, montrer que \(K[X]/P\)
        est un corps.
    \end{q}
    \begin{q}{3}
        Si \(K\) est un sous-corps de \(\C\) et \(\alpha\) est une racine de \(P\)
        dans \(\C\), montrer que \(K[X]/P\) est isomorphe au sous-corps \(K[\alpha]\subset C\).
        le plus petit sous-anneau de \(\C\) qui contient \(K\) et \(\alpha\)
    \end{q}
\end{exo}

\begin{exo}
    Soit \(K\) un corps et \(P\in K[X]\) un polynôme de degré \(d\in\N\).
    \begin{q}{1}
        Montrer que \(A=K[X]/P\) est un \(K\)-espace vectoriel de dimension \(d\).
        \boxans{
            Soit \(P(X) = a_dX^d + a_{d-1}X^{d-1} + \ldots + a_0\).
            Considérons la classe de \(X\) dans \(A\), notée \(\bar{X}\). On a \(\bar{X}^d = -\frac{1}{a_d}(a_{d-1}\bar{X}^{d-1} + \ldots + a_0)\),
            où \(\frac{1}{a_d}\) est l'inverse de \(a_d\) dans \(K\) (car \(K\) est un corps). Ainsi, \(\{\bar{X}^0, \bar{X}^1, \ldots, \bar{X}^{d-1}\}\)
            est une base de \(A\) sur \(K\), et donc \(A\) est un \(K\)-espace vectoriel de dimension \(d\).
        }
    \end{q}
    \begin{q}{2}
        Soit \(K\) un corps et \(P\) un polynôme irréductible, montrer que \(K[X]/P\) est un corps.
        \boxans{
            Considérons un élément non nul
            \(\bar{f} \in A\), où \(\bar{f}\) est la classe de \(f \in K[X]\) dans \(A\). Puisque \(\bar{f}\) est
            non nul, le polynôme \(f\) n'est pas divisible par \(P\), c'est-à-dire que \(P\) ne divise pas \(f\).
            Ainsi, \(\gcd(f, P) = 1\). Par l'identité de Bézout, il existe des polynômes \(g\) et \(h\) tels que
            \(fg + hP = 1\). En prenant les classes dans \(A\), on a \(\bar{f} \cdot \bar{g} + \bar{h} \cdot \bar{P} = \bar{1}\).
            Cela montre que \(\bar{f}\) est inversible dans \(A\), et donc \(A\) est un corps.
        }
    \end{q}
    \begin{q}{3}
        Si \(K\) est un sous-corps de \(\C\) et \(\alpha\) est une racine de \(P\) dans \(\C\), montrer que \(K[X]/P\) est isomorphe au sous-corps \(K[\alpha]\subset \C\),
        le plus petit sous-anneau de \(\C\) qui contient \(K\) et \(\alpha\).
        \boxans{
            Considérons l'application \(\phi : K[X] \to K[\alpha]\) définie par \(\phi(f) = f(\alpha)\) pour tout \(f \in K[X]\).
            Montrons que \(\phi\) est un morphisme d'anneaux. Soient \(f, g \in K[X]\). Alors,
            \begin{align*}
                \phi(f + g) &= (f + g)(\alpha) = f(\alpha) + g(\alpha) = \phi(f) + \phi(g), \\
                \phi(f \cdot g) &= (f \cdot g)(\alpha) = f(\alpha) \cdot g(\alpha) = \phi(f) \cdot \phi(g).
            \end{align*}
            Ainsi, \(\phi\) est un morphisme d'anneaux. De plus, \(\phi\) est surjectif car \(\alpha\) est une racine de \(P\),
            et donc tout élément de \(K[\alpha]\) est de la forme \(f(\alpha)\) pour un certain \(f \in K[X]\). Le noyau de
            \(\phi\) est l'ensemble des polynômes \(f \in K[X]\) tels que \(f(\alpha) = 0\), c'est-à-dire l'ensemble
            des multiples de \(P\). Donc, \(\ker(\phi) = (P)\), et par le premier théorème d'isomorphisme, on a
            \(K[X]/(P) \cong K[\alpha]\), montrant ainsi que \(K[X]/P\) est isomorphe au sous-corps \(K[\alpha]\).
        }
    \end{q}
\end{exo}


\begin{exo}
    Montrer que le groupe additif \(\left(\Q/\Z,+\right)\) ne peut pas être
    équipé d'une multiplication qui ferait de \(\left(\Q/\Z,+\right)\) un anneau.
    \boxans{
        Supposons par l'absurde qu'une telle multiplication existe. Soit \(a\) un élément
        non nul de \(\left(\Q/\Z,+\right)\), et \(n\) le plus petit entier positif tel
        que \(na = 0\) dans \(\left(\Q/\Z,+\right)\). Puisque \(\left(\Q/\Z,+\right)\) est
        un groupe, l'inverse de \(a\) existe, notons-le \(b\). Alors, \(b\) doit également
        annuler le produit \(na\), c'est-à-dire \(nab = 0\). Cependant, cela signifie
        que l'élément \(ab\) annule \(n\), et donc \(ab = 0\) dans \(\left(\Q/\Z,+\right)\).
        Ainsi, \(a\) est absorbant pour la multiplication, ce qui contredit la propriété
        de groupe commutatif de \(\left(\Q/\Z,+\right)\) (un anneau nécessite un élément neutre
        distinct de zéro pour la multiplication).
    }
\end{exo}
\end{document}