\documentclass{report}
\usepackage{../../exercices}

\begin{document}

\begin{center}
    \huge{\textbf{TD5: Ordre d'un élément, groupes cycliques}}
\end{center}

\begin{exo}
    Soient \(m,n\in\N\backslash\{0\}\).
    \begin{q}{1}
        Démontrer l'équivalence \(\U_m\subseteq \U_n \Leftrightarrow m\mid n\).
    \end{q}
    \begin{q}{2}
        Montrer que \(\U_n\) est cyclique.
    \end{q}
    \begin{q}{3}
        Décrire les sous groupes de \(\U_n\).
    \end{q}
    \begin{q}{4}
        Donner l'ordre des éléments de \(\U_n\).
    \end{q}
\end{exo}

\begin{exo}
    Démontrer que des éléments conjugués d'un groupe sont de même ordre. La réciproque
    est-elle vraie ?
    \boxans{Soit \(G\) un groupe, on pose \(a\) et \(b\) conjugués, il existe
    \(g\in G\) tel que \(a=gbg^{-1}\) alors si \(b\) est d'ordre \(n\in N\) on a
    \(a^n = g\star e\star g^{-1}=e\) donc \(o(a)\mid o(b)\). Aussi en écrivant \(b=g^{-1}ag^{-1}\) on a
    \(o(b)\mid o(a)\) donc \(o(b)=o(a)\). La réciproque est fausse car \(\Z_3\) est
    abélien et compte \(3\) éléments d'ordre \(3\).}
\end{exo}

\begin{exo}
    Soit \(G\) un groupe et \(x\in G\) un élément d'ordre \(n\). Démontrer
    que \(x^m\) est d'ordre \(\frac{n}{\gcd(m,n)}\).
    \boxans{Soit \(G\) un groupe et \(x\in G\) d'ordre \(n\in\N\). Soit \(m\in \N\) on considère
    \(d = n \land m\) alors \((x^m)^{n/d} = (x^n)^{m/d} = e\) donc \(o(x^m) \mid \frac nd\).
    Soit \(p\in\N\) tel que \((x^m)^p=e\) alors \(n\mid np\) donc \(\frac nd \mid \frac md \times p\)
    donc d'après le lemme de \textsc{Gauss} \(\frac nd\mid p\) et donc \(o(x^m)=\frac nd\).}
\end{exo}

\begin{exo}
    Soit \(f\colon G\to H\) un homomorphisme de groupes. Soit \(x\in G\) d'ordre
    fini noté \(n\).
    \begin{q}{1}
        Démontrer que \(f(x)\) est d'ordre fini, et que son ordre divise \(n\).
        \boxans{Notons \(m=o(f(x))\), on a \(f(x)^m = 1 = f(x^n) = f(x)^n\). On peut ensuite
        effectuer la division euclidienne de \(n\) par \(m\) pour obtenir \(n=mq+r\), montrons
        que \(r=0\). On suppose par l'absurde \(r>0\) alors \(f(x)^n = f(x)^r\) or \(f(x)^n=1\) et \(r<m\)
        implique \(f(x)^r\neq 1\) ce qui est absurde, ainsi \(r=0\) et \(m\mid n\).}
    \end{q}
    \begin{q}{2}
        Démontrer que \(f(x)\) est d'ordre exactement \(n\) si \(\Ker(f)\) est
        d'ordre premier à \(n\).
        \boxans{On note \(k\) l'ordre de \(\Ker(f)\) et \(m\mid n\) l'ordre de \(f(x)\), on note
        \(d = \frac nm\in\N\). On sait que \(f(x)^m = (f(x)^m)^d=1\) ainsi \(\langle x^m\rangle \subseteq \Ker(f)\).
        La théorème de \textsc{Langrange} affirme alors que \(d \mid k\) or \(d\mid n\) et
        \(k\land n = 1\) donc \(d= 1\) et finalement \(m=n\)}
    \end{q}
\end{exo}

\begin{exo}
    Montrer que les groupes suivants ne sont pas isomorphes.
    \begin{q}{1}
        \(\left(\Z/3\Z\times\Z/3\Z,+\right)\) et \(\left(\Z/9\Z,+\right)\).
    \end{q}
    \begin{q}{2}
        \(\left(\Z/3\Z\times\Z/8\Z,+\right)\) et \(\left(\Z/4\Z\times\Z/6\Z,+\right)\).
    \end{q}
\end{exo}

\begin{exo}
    Soient \(a\) et \(b\) deux entiers naturels strictement plus grands que \(1\).
    Montrer que le nombre de morphisme de groupes de \(\Z/a\Z\) vers \(\Z/b\Z\)
    est \(\gcd(a,b)\).
\end{exo}

\begin{exo}
    Travail non pertinent sur de la multiplication par 3.
    \begin{q}{1}
        Montrer que les groupes \(\left(3\Z/18\Z,+\right)\) et \(\left(\Z/6\Z,+\right)\)
        sont isomorphes.
    \end{q}
    \begin{q}{2}
        Décrire les sous-groupes, l'ordre des éléments et les générateurs de
        \(\left(3\Z/18\Z,+\right)\).
    \end{q}
\end{exo}

\begin{exo}
    Soit \(G\) = \(\langle(1\ 2\ \dots\ 11\ 12)\rangle\subseteq\mathfrak{S}_{12}\).
    Déterminer les générateurs et les sous-groupes de \(G\).
\end{exo}

\begin{exo}
    Soient \(x,y\) deux éléments d'ordre fini d'un groupe tels que \(xy=yx\). On note
    \(n\) et \(m\) leurs ordres respectifs.
    \begin{q}{1}
        Démontrer que si \(n\land m = 1\) alors \(xy\) est d'ordre \(mn\).
        \boxans{On commence par remarquer que \((xy)^{nm}=(x^n)^m(y^m)^n=e\) donc
        \(o(xy) \mid mn\). On note maintenant \(k=o(xy)\) alors \(((xy)^k)^m=(x^m)^k=e\)
        donc \(n\mid mk\) ainsi par le lemme de \textsc{Gauss} \(n\mid k\) et par symétrie
        du raisonnement \(m \mid k\) ainsi \(k = mn\).}
    \end{q}
    \begin{q}{2}
        Démontrer que \(xy\) est d'ordre fini et que \(\frac{mn}{d^2} \mid k \mid \frac{mn}{d}\)
        en notant \(k\) l'ordre de \(xy\) et \(d=n\land m\).
        \boxans{D'après la question précédente \(k\) est fini et divise \(nm\).
        \begin{enumerate}
            \itt \((xy)^{mn/d}= (x^n)^{m/d}(y^m)^{n/d} = e\) donc \(k \mid \frac{nm}{d}\).
            \itt \(o((xy)^d) \mid k\) donc \(\frac{nm}{d^2}\mid k\).
        \end{enumerate}}
    \end{q}
    \begin{q}{3}
        On suppose que \(m=24\) et \(n=18\). Démontrer que \(G\) contient un
        élément d'ordre \(72\)
        \boxans{Si \(x\) est d'ordre \(24\) et \(y\) est d'ordre \(18\) on a alors
        \(x^3\) est d'ordre \(8\) et \(y^2\) est d'ordre \(9\) ce qui d'après la question
        \(1\) donne que l'ordre de \(x^3y^2\) est \(72\).}
    \end{q}
\end{exo}

\begin{exo}
    Retour rapide sur les matrices.
    \begin{q}{1}
        Exhiber un élément d'ordre \(3\) dans \(\glnr\).
    \end{q}
    \begin{q}{2}
        Démontrer qu'un tel élément est toujours dans \(\text{SL}_n(R)\)
    \end{q}
\end{exo}

\begin{exo}
    Soient \(G\) un groupe fini d'ordre noté \(n\) et \(m\) un entier premier à \(n\).
    Démontrer que pour tout \(a\in G\) l'équation \(x^m=a\) admet une unique solution.
\end{exo}

\begin{exo}
    Soit \(n\in\N^*\)
    \begin{q}{1}
        Quels sont les générateurs de \(\Z/n\Z\) ? On note \(\varphi(n)\) ce nombre.
    \end{q}
    \begin{q}{1}
        Démontrer l'égalité \(\ds\sum_{d\mid n} \varphi(d) = n\).
    \end{q}
\end{exo}

\begin{exo}
    Montrer que \(\textbf{Aut}\left(\Z/7\Z\right)\) est cyclique et déterminer
    les morphismes de groupes de \(\left(\Z/4\Z,+\right)\) dans \(\textbf{Aut}\left(\Z/7\Z\right)\).
\end{exo}
\end{document}