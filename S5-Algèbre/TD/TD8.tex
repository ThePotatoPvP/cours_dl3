\documentclass{report}
\usepackage{../../exercices}

\begin{document}

\begin{center}
    \huge{\textbf{TD8: Anneau \(\Z/n\Z\) et Anneaux de polynômes}}
\end{center}

\begin{exo}
    Montrer que les groupes \(G=\Z/24\Z\times\Z/5\Z\) et \(H=\Z/8\Z\times\Z/15\Z\)
    sont isomorphes. Expliciter les isomorphismes inverses l'un de l'autre
    entre ces deux groupes.
    \boxans{
        Soit \(\phi\colon G\to H\colon (a,b)\mapsto (a \mod 8, b + 5*(a \div 8))\).
        C'est par construction un isomorphisme.
    }
\end{exo}

\begin{exo}
    Trouver tous les ordres des éléments des groupes
    \begin{enumerate}
        \itt \(\left(\Z/12\Z, +\right)\)
        \boxans{
            Les éléments de \(\Z/12\Z\) ont les ordres suivants :
            \begin{itemize}
                \item Ordre 1 : \(0\)
                \item Ordre 2 : \(6\), \(4\), \(8\), \(10\)
                \item Ordre 3 : \(\text{Aucun}\)
                \item Ordre 4 : \(3\), \(9\)
                \item Ordre 6 : \(2\), \(5\), \(7\), \(11\)
            \end{itemize}
        }
        \itt \((\left(\Z/12\Z\right)^{\times}, \cdot)\)
        \boxans{
            Les éléments inversibles de \(\Z/12\Z\) ont les ordres suivants :
            \begin{itemize}
                \item Ordre 1 : \(1\)
                \item Ordre 2 : \(5\), \(7\), \(11\)
                \item Ordre 4 : \(\text{Aucun}\)
                \item Ordre 6 : \(1\), \(5\), \(7\), \(11\)
            \end{itemize}
        }
    \end{enumerate}
\end{exo}

\begin{exo}
    \begin{q}{1}
        Les groupes \((\left(\Z/21\Z\right)^\times,\cdot)\) et
        \(\left(\Z/12\Z, +\right)\) sont-ils isomorphes ?
        \boxans{
            Non, car \((\left(\Z/21\Z\right)^\times,\cdot)\) n'est pas cyclique.
        }
    \end{q}
    \begin{q}{2}
        Les groupes \((\left(\Z/13\Z\right)^\times,\cdot)\) et
        \(\left(\Z/12\Z, +\right)\) sont-ils isomorphes ?
        \boxans{
            Oui, car \((\left(\Z/13\Z\right)^\times,\cdot)\) et \(\left(\Z/12\Z, +\right)\)
            sont tous deux cycliques d'ordre \(12\).
        }
    \end{q}
    \begin{q}{3}
        Montrer que les groupes \((\left(\Z/26\Z\right)^\times,\cdot)\)
        et \(\left(\Z/12\Z, +\right)\) sont isomorphes.
        \boxans{
            Oui, car \((\left(\Z/26\Z\right)^\times,\cdot)\) est isomorphe à
            \((\left(\Z/13\Z\right)^\times,\cdot)\), et par la question précédente,
            \((\left(\Z/13\Z\right)^\times,\cdot)\) est isomorphe à \(\left(\Z/12\Z,
            +\right)\).
        }
    \end{q}
\end{exo}

\begin{exo}
    Soit \(n\in\Z\). Montrer que l'application
    \[\Phi\colon\Z[X]\Longrightarrow\left(\Z/n\Z\right)[X],
    \sum_{k=0}^da_kX^k\mapsto\sum_{k=0}^d\bar{a_k}X^k\]
    est un morphisme d'anneau surjectif dont on déterminera le noyau.
    \boxans{
        Pour montrer que \(\Phi\) est un morphisme d'anneau surjectif, il suffit de vérifier
        que \(\Phi(P+Q) = \Phi(P) + \Phi(Q)\) et \(\Phi(P\cdot Q) = \Phi(P)\cdot \Phi(Q)\) pour
        tous \(P, Q\in\Z[X]\). Ces propriétés découlent directement des opérations dans \(\Z/n\Z\).
        Pour le noyau, \(\ker(\Phi) = \{P\in\Z[X] \mid P \equiv 0 \pmod{n}\}\), c'est-à-dire,
        les polynômes dont tous les coefficients sont multiples de \(n\).
    }
\end{exo}

\begin{exo}
    Soit \(\alpha\in\C\).
    \begin{q}{1}
        Montrer que l'application \(\varphi\colon\Q[X]\to\C,T\mapsto T(\alpha)\)
        est un morphisme d'anneaux.
        \boxans{
            Pour montrer que \(\varphi\) est un morphisme d'anneaux, il faut vérifier
            que \(\varphi(T_1 + T_2) = \varphi(T_1) + \varphi(T_2)\) et \(\varphi(T_1\cdot T_2) = \varphi(T_1)\cdot \varphi(T_2)\)
            pour tous \(T_1, T_2\in\Q[X]\). Ces propriétés découlent directement de l'arithmétique
            dans \(\C\).
        }
    \end{q}
    \begin{q}{2}
        On suppose que \(\varphi\) n'est pas injective. Montrer qu'il existe \(P\in\Q[X]\)
        irréductible sur \(\Q\) tel que \(\Ker(\varphi)=(P)\).
        \boxans{
            Puisque \(\varphi\) n'est pas injective, \(\Ker(\varphi)\neq\{0\}\). Soit \(P\in\Ker(\varphi)\)
            un polynôme non nul de degré minimal. Montrons que \(P\) est irréductible sur \(\Q\).
            Si \(P = QR\) pour certains \(Q, R\in\Q[X]\), alors \(\varphi(Q)\cdot \varphi(R) = 0\),
            ce qui implique que \(\varphi(Q) = 0\) ou \(\varphi(R) = 0\) (sinon, \(\alpha\) serait une
            racine commune à \(Q\) et \(R\)). Supposons sans perte de généralité que \(\varphi(Q) = 0\).
            Puisque \(Q\) est de degré strictement inférieur à \(P\), cela contredit le choix de \(P\)
            comme étant de degré minimal.
        }
    \end{q}
    \begin{q}{3}
        Soit \(Q\in\Q[X]\) un polynôme irréductible sur \(\Q\) tel que \(\alpha\) est
        racine de \(Q\). Montrer que \(P\) et \(Q\) sont associés.
        \boxans{
            Comme \(\Ker(\varphi) = (P)\), cela signifie que \(\varphi\) est injective sur \(\Q[X]/(P)\),
            l'anneau quotient par l'idéal engendré par \(P\). En d'autres termes, \(\Q[X]/(P)\) est isomorphe
            à l'image de \(\varphi\), qui est isomorphe à \(\Q[X]\). Cela implique que \(\Q[X]/(P)\) est
            isomorphe à \(\Q[X]\), et donc, \(\Q[X]/(P)\) est un corps. Ainsi, \(P\) est irréductible
            sur \(\Q\) et par conséquent, \(P\) et \(Q\) sont associés et à \(\pi_\alpha\).
        }
    \end{q}
\end{exo}


\begin{exo}
    Montrer que l'anneau \(\R[X]/(X^2+X+1)\R[X]\) est un corps isomorphe à \(\C\).
    \boxans{
        On pose \(\phi : \R[X]/(X^2+X+1)\R[X] \to \C\) défini par \(\phi([P]) = P(j)\)
        pour tout \(P \in \R[X]\). Montrons maintenant que \(\phi\) est un isomorphisme.

        \itt Injectivité : Soit \(P \in \R[X]\) tel que \(\phi([P]) = 0\). Cela
        signifie que \(P\left(j\right) = 0\), et donc \(P\) est divisible par
        \(X^2+X+1\). Ainsi, \([P] = [0]\), montrant que \(\phi\) est injectif.

        \itt Surjectivité : Soit \(z\in\C\) alors \(Q[X]=\frac{X-X^2}{\sqrt{3}}
        \Im(z)+\Re(z)\) vérifie \(\phi(Q)=z\).

        Enfin, le polynôme \(\pi_j\) est irréductible sur \(\R[X]\) donc le quotient
        de départ est un corps. Conséquemment \(\phi\) est un isomorphisme de notre
        corps de départ vers \(\C\) ce qui conclut la rédaction.
    }
\end{exo}

\begin{exo}
    Montrer que pour tout corps \(K\), l'anneau de polynômes \(K[X]\)
    a une infinité de polynômes unitaires irréductibles.
    \boxans{
        Supposons par l'absurde que \(K[X]\) a un nombre fini de polynômes
        unitaires irréductibles. Considérons le polynôme \(q(X) = (p_1 p_2 \ldots p_n) + 1\),
        où \(\{p_1, p_2, \ldots, p_n\}\) sont les polynômes irréductibles de \(K[X]\).
        Ce polynôme est unitaire et irréductible, ce qui contredit notre hypothèse.
        Ainsi, \(K[X]\) doit avoir une infinité de polynômes unitaires irréductibles.
    }
\end{exo}

\begin{exo}
    Soit \(K\) un corps.
    \begin{q}{1}
        Montrer qu'un polynôme \(P\) de degré 2 ou 3 dans \(K[X]\)
        est irréductible si et seulement s'il n'a pas de zéro dans \(K\).
        \boxans{
            Soit \(P\) un polynôme de degré 2 ou 3 dans \(K[X]\). Supposons
            que \(P\) est réductible. Cela signifie qu'il peut être factorisé
            comme \(P = Q \cdot R\) où \(Q\) et \(R\) sont des polynômes non constants
            de degrés respectifs \(1\) et \(2\) (dans le cas \(P\) de degré \(3\))
            ou \(1\) et \(1\) (dans le cas \(P\) de degré \(2\)).
            Si \(P\) a une racine dans \(K\), cette racine est un zéro de \(Q\)
            ou \(R\), et donc \(P\) a un zéro dans \(K\). Ainsi, si \(P\) est réductible,
            il a un zéro dans \(K\).

            Réciproquement, supposons que \(P\) n'ait pas de zéro dans \(K\).
            Dans ce cas, \(P\) n'est pas divisible par un polynôme linéaire,
            ce qui signifie qu'il est irréductible.
        }
    \end{q}
    \begin{q}{2}
        En déduire tous les polynômes irréductibles de degré \(2\) dans \(\Z/2\Z\).
        \boxans{
            Dans \(\Z/2\Z\), les polynômes de degré 2 sont \(X^2\), \(X^2 + X\),
            et \(X^2 + 1\). Appliquons la condition de la question 1 :
            \begin{enumerate}
                \itt Si \(X^2\) et \(X^2+X\) admettent 0 en racine et sont donc réductibles.
                \itt \(X^2 + 1\) n'a pas de \(0\) dans \(\Z/2\Z\) et est donc irréductible.
            \end{enumerate}
        }
    \end{q}
\end{exo}



\begin{exo}
    Déterminer les polynômes irréductibles de
    \begin{enumerate}
        \itt \(\C[X]\)
        \boxans{
            Les polynômes irréductibles dans \(\C[X]\) sont les polynômes de degré 1
            car \(\C\) est algébriquement clos comme le démontre indirectment le théorème
            de \(\textsc{Liouville}\).
       }
        \itt \(\R[X]\)
        \boxans{
            Les polynômes irréductibles dans \(\R[X]\) sont les polynômes de degré 1 et les polynômes quadratiques irréductibles.
            \begin{enumerate}
                \item Les polynômes de degré 1 : Tout polynôme de degré 1 est irréductible dans \(\R[X]\).
                \item Les polynômes quadratiques irréductibles : Un polynôme quadratique \(ax^2 + bx + c\) est irréductible dans \(\R[X]\) s'il n'a pas de racines réelles.
                Les polynômes quadratiques irréductibles peuvent être de deux types :
                \begin{enumerate}
                    \itt Si \(b^2 - 4ac < 0\), alors le polynôme n'a pas de racines réelles et est irréductible.
                    \itt Si \(b^2 - 4ac > 0\), alors le polynôme peut être décomposé en \(a(x - x_1)(x - x_2)\), où \(x_1\) et \(x_2\) sont les racines réelles distinctes.
                \end{enumerate}
            \end{enumerate}
        }
    \end{enumerate}
    On a ici utilisé les résultats de l'exercice précédent afin de gagner un temps précieux.
\end{exo}


\begin{exo}
    Soient \(f,g\in \Q[X]\). Supposons que \(f\) soit irréductible et qu'il existe
    \(\alpha\in \C\) tel que \(f(\alpha)=g(\alpha)=0\). Montrer que \(f\) divise \(g\).
    \boxans{
        Puisque \(f\) est irréductible et \(f(\alpha) = 0\), \(f=\pi_\alpha\).
        Par conséquent, l'idéal \((f)\) n'est autre que \(\pi_\alpha\Q[X]\).

        Puisque \(g(\alpha) = 0\), cela signifie que \(g\in\pi_\alpha\Q[X]\).
        En d'autres termes, il existe un polynôme \(h\) tel que \(g = hf\),
        montrant ainsi que \(f\) divise \(g\).
    }
\end{exo}

\begin{exo}
    On dahak avec les polynômes.
    \begin{q}{1}
        Montrer que l'anneau \(\Q[X]/\left(X^3-X+2\right)\Q[X]\) est un corps.
        \boxans{
            Le polynôme \(X^3-X+2\) est irréductible dans \(\Q[X]\) (source tkt) donc
            d'après le théorème fondamental l'anneau quotient est bien un corps.
        }
    \end{q}
    On note \(x\) l'image de \(X\) par la surjection canonique.
    \begin{q}{2}
        Calculer l'inverse de \(x\).
        \boxans{On remarque que \(x\times \left(-\frac{1}{2}(X^2-1)\right)=-\frac 12
        (X^3-X) \equiv 1\) et donc \(x^{-1}= \cl\left(-\frac{1}{2}(X^2-1)\right)\).}
    \end{q}
    \begin{q}{3}
        Montrer que \(x^2+x+1\) est inversible et déterminer son inverse.
    \end{q}
\end{exo}


\end{document}