\documentclass{report}
\usepackage{../../exercices}

\begin{document}

\begin{center}
    \huge{\textbf{TD8: Anneau \(\Z/n\Z\) et Anneaux de polynômes}}
\end{center}

\begin{exo}
    Montrer que les groupes \(\Z/24\Z\times\Z/5\Z\) et \(\Z/8\Z\times\Z/15\Z\)
    sont isomorphes. Expliciter les isomorphismes inverses l'un de l'autres
    entre ces deux groupes.
\end{exo}

\begin{exo}
    Trouver tous les ordres des éléments des groupes
    \begin{enumerate}
        \itt \(\left(\Z/12\Z, +\right)\)
        \itt \((\left(\Z/12\Z\right)^{\times}, \cdot)\)
    \end{enumerate}
\end{exo}

\begin{exo}
    \begin{q}{1}
        Les groupes \((\left(\Z/21\Z\right)^\times,\cdot)\) et
        \(\left(\Z/12\Z, +\right)\) sont-ils isomorphes ?
    \end{q}
    \begin{q}{2}
        Les groupes \((\left(\Z/13\Z\right)^\times,\cdot)\) et
        \(\left(\Z/12\Z, +\right)\) sont-ils isomorphes ?
    \end{q}
    \begin{q}{3}
        Montrer que les groupes \((\left(\Z/6\Z\right)^\times,\cdot)\)
        et \(\left(\Z/12\Z, +\right)\) sont isomorphes.
    \end{q}
\end{exo}

\begin{exo}
    Soit \(n\in\Z\). Montrer que l'application
    \[\Phi\colon\Z[X]\Longrightarrow\left(\Z/n\Z\right)[X],
    \sum_{k=0}^da_kX^k\mapsto\sum_{k=0}^d\bar{a_k}X^k\]
    est un morphisme d'anneau surjectif dont on déterminera le noyau.
\end{exo}

\begin{exo}
    Soit \(\alpha\in\C\)
    \begin{q}{1}
        Montrer que l'application \(\varphi\colon\Q[X]\to\C,T\mapsto T(\alpha)\)
        est un morphisme d'anneaux
    \end{q}
    \begin{q}{2}
        On suppose que \(\varphi\) n'est pas injective. Montrer qu'il existe \(P\in\Q[X]\)
        irréductible sur \(Q\) tel que \(\Ker(\varphi)=(P)\).
    \end{q}
    \begin{q}{3}
        Soit \(Q\in\Q[X]\) un polynôme irréductible sur \(\Q\) tel que \(\alpha\) est
        racine de \(Q\). Montrer que \(P\) et \(Q\) sont associés.
    \end{q}
\end{exo}

\begin{exo}
    Montrer que l'anneau \(\R[X]/\left(X^2+X+1\right)\R[X]\) est un corps isomorphe à \(\C\).
\end{exo}

\begin{exo}
    Montrer que pour tout corps \(K\), l'anneau de polynômes \(K[X]\)
    a une infinité de polynômes unitaires irréductibles.
    \boxans{Soit \(\alpha\in K\) la famille \(P_n=(X-\alpha)^n\) est libre
    et est une famille infinie de polynnômes irréductibles.}
\end{exo}

\begin{exo}
    Soit \(K\) un corps.
    \begin{q}{1}
        Montrer qu'un polynôme \(P\) de degré 2 ou 3 dans \(K[x]\)
        est irréductible si et seulement si il n'a pas de zéro dans \(K\).
    \end{q}
    \begin{q}{2}
        En déduire tous les polynômes irréductibles de degré \(2\) dans \(\Z/2\Z\).
    \end{q}
\end{exo}

\begin{exo}
    Déterminer les polynômes irréductibles de
    \begin{enumerate}
        \itt \(\C[X]\)
        \itt \(\R[X]\)
    \end{enumerate}
\end{exo}

\begin{exo}
    Soient \(f,g\in \Q[X]\). Supposons que \(f\) soit irréductible et qu'il existe
    \(\alpha\in \C\) tel que \(f(\alpha)=g(\alpha)=0\). Montrer que \(f\) divise \(g\).
\end{exo}

\begin{exo}
    Montrer que l'anneau \(\Q[X]/\left(X^3-X+2\right)\Q[X]\) est un corps. On note
    \(x\) l'image de \(X\) par la surjection canonique.
    \begin{q}{1}
        Calculer l'inverse de \(x\).
    \end{q}
    \begin{q}{2}
        Montrer que \(x^2+x+1\) est inversible et déterminer son inverse.
    \end{q}
\end{exo}
\end{document}