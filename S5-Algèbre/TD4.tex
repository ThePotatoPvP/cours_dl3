\documentclass{report}
\usepackage{../../exercices}

\begin{document}

\begin{center}
    \huge{\textbf{TD4: Théorème de Lagrange, sous-groupes distingués,
    groupe quotient}}
\end{center}

\begin{exo}
    Soit \(p\) un nombre premier et \(G\) un groupe d'ordre \(p\). Montrer que \(G\) est cyclique.
    \boxans{Soit \(g\in G\) alors d'après le théormème de \textsc{Lagrange} on
    a \(o(g)\mid p\) donc soit \(o(g)=1\) aucquel cas \(g=e\) soit \(o(g)=p\)
    aucquel cas \(\langle g\rangle\subset G\) et les deux ensembles sont de
    même cardinal, donc \(g\) engendre \(G\) qui est donc cyclique.}
\end{exo}

\begin{exo}
    Soit \(f\colon G\to H\) un morphisme de groupes finis et \(K\) un sous-groupe
    de \(G\)
    \begin{q}{1}
        Montrer que l'ordre de \(f(K)\) divise les ordres de \(K\) et de \(H\)
        \boxans{D'après le premier théorème d'isomorphisme, pour un groupe \(K\) fini
        on a \(|K|=|\Ker(f)|\times|\Im(f)|\) ce qui donne la permière partie de la réponse.
        Enfin on montre que l'ordre de \(f(K)\) divise celui de \(H\) par le théorème de
        \textsc{Lagrange} car c'est un sous-groupe.}
    \end{q}
    \begin{q}{2}
        Montrer que si l'ordre de \(K\) est premier à l'ordre de \(H\) alors
        \(K \subset \Ker(f)\)
        \boxans{Dans ce cas \(|f(K)|\) divise deux nombres premiers entre eux,
        donc \(|f(K)|=1\) et \(e\in f(K)\) donc \(f(K)=\{e\}\)}
    \end{q}
\end{exo}

\begin{exo}
    Soient \(G\) et \(G'\) deux groupes finis. On suppose que les ordres de ces groupes
    sont premiers entre eux. Déterminer tous les morphismes de groupes de \(G\) dans \(G'\)
    \boxans{Soit \(f\in\textbf{Hom}(G,G')\), \(|f(G)|\) divise \(G'\) or
    \(|f(G)|=\frac{|G|}{|\Ker(f)|}\) ainsi \(f(G)\) a un cardinal qui divise deux
    nombres premiers entre eux et donc \(f(G)=e\). La synthèse est immédiate.}
\end{exo}

\begin{exo}
    Montrer que si \(H\) et \(K\) sont deux sous-groupes distinguées d'un groupe \(G\)
    alors leur intersection est distinguée.
    \boxans{On se passera de démontrer que \(H\cap K\) est bien un groupe, montrons qu'il est
    normal. Soient \(x\in G\) et \(h\in H\cap K\) alors en particulier \(h\in H\) donc
    \(xhx^{-1}\). On a ainsi bien que \(H\cap K\) est un sous groupe normal de \(G\).}
\end{exo}

\begin{exo}
    Montrer que si \(N\) est distingué dans \(G\) et \(H\) un sous groupe quelconque de
    \(G\) alors \(N\cap H\) est distingué dans \(H\).
    \boxans{L'énoncé nous indique que \(\forall x\in G, \forall n\in n, nxn^{-1} \in G\), ce qui
    reste vrai si on réduit les ensembles car les quantificateurs sont universels, ainsi on a bien
    \(\forall h\in H, \forall y\in H\cap N, yhy^{-1}\in H\) par stabilité de \(H\).}
\end{exo}

\begin{exo}
    Soient deux sous-groupes distingués \(H\) et \(K\) de \(G\) tels que \(H\cap K=\{e\}\)
    Montrer que \(\forall x\in H,\forall y\in K, xy=yx\).
    \boxans{Soit \(h\in H, k\in K\) on a pour tout \(x\in G\colon xhx^{-1}\in H,
    xkx^{-1}\in K\), ainsi \(kh^{-1}k^{-1}\in H\) et donc \(hkh^{-1}k^{-1}\in H\) par stabilité,
    de la même façon \(hkh^{-1}\in K\) et donc \(hkh^{-1}k^{-1} \in K\) finalement
    \(hkh^{-1}k^{-1}\in K\cap H = \{e\}\) donc \(hk = kh\).}
\end{exo}

\begin{exo}
    Soient les éléments \(a = (1,2,3,4)\) et \(b = (1,2) \circ (3,4)\) dans \(\mathfrak{S}_4\)
    et \(G\) le sous-groupe de \(\mathfrak{S}_4\) engendré par \(a\) et \(b\). Soit
    \(H = \langle b, a^2 \rangle\) et \(K = \langle b \rangle\).
    \begin{q}{1}
        Montrer que \(H\) est distingué dans \(G\).
    \end{q}
    \begin{q}{2}
        Montrer que \(K\) est distingué dans \(H\).
    \end{q}
    \begin{q}{3}
        Montrer que \(K\) n'est pas distingué dans \(G\).
    \end{q}
\end{exo}


\begin{exo}
    Soit \(G=\left\{\begin{pmatrix}a&b\\0&d\end{pmatrix}\mid a,d\in\R^*, b\in\R
    \right\}\subset \text{GL}_2(\R)\).

    \begin{q}{1}
        Montrer que \(G\) est un sous-groupe de GL\(_2(\R)\).
        \boxans{
            Soit \(A\) et \(B\) deux matrices dans \(G\), c'est-à-dire
            \(A=\begin{pmatrix}a&b\\0&d\end{pmatrix}\) et
            \(B=\begin{pmatrix}a'&b'\\0&d'\end{pmatrix}\) avec \(a,d,a',d'\in\R^*\)
            et \(b,b'\in\R\). Alors, on a
            \[
                AB = \begin{pmatrix}aa'&ab'+bd'\\0&dd'\end{pmatrix},
            \]
            et \(AB\) appartient également à \(G\). De plus, l'inverse de \(A\) est
            \[
                A^{-1} = \begin{pmatrix}1/a&-b/(ad)\\0&1/d\end{pmatrix},
            \]
            ce qui montre que \(A^{-1}\) appartient à \(G\). Ainsi, \(G\) est un
            sous-groupe de GL\(_2(\R)\).
        }
    \end{q}

    \begin{q}{2}
        Dans chacun des cas suivants, décider si \(H\subset G\) est un sous-groupe
        normal ou non.
        \begin{q}{a}
            \(H=\{M \in G \mid a = 1\}\)
            \boxans{
                \(H\) est un sous-groupe normal. Soit \(A\in G\) et \(B\in H\). On a
                \(ABA^{-1} = \begin{pmatrix}1&\frac{-b+ab'+bd'}{d}\\0&1\end{pmatrix}\in H\).
            }
        \end{q}
        \begin{q}{b}
            \(H=\{M \in G \mid b = 0\}\)
            \boxans{
                \(H\) est un sous-groupe normal. Soit \(A\in G\) et \(B\in H\). On
                a \(ABA^{-1}=\begin{pmatrix}a&0\\0&d\end{pmatrix}\in H\).
            }
        \end{q}
        \begin{q}{c}
            \(H=\{M \in G \mid d = a\}\)
            \boxans{
                \(H\) est pas un sous-groupe normal. Soit \(A\in G\) et \(B\in H\). On
                a \(ABA^{-1}=\begin{pmatrix}d'&\frac{ab'}{d}\\0&d'\end{pmatrix}\in H\).
            }
        \end{q}
    \end{q}

    \begin{q}{3}
        Dans la question précédente, si \(H\triangleleft G\) est normal, déterminer
        le quotient \(G/H\) à isomorphisme près.
        \boxans{
            Si \(H\) est normal, alors \(G/H\) est isomorphe à \(\R^*\) sous
            l'isomorphisme donné par la projection canonique.
        }
    \end{q}
\end{exo}


\begin{exo}
    On considère le groupe \(\mathfrak{S}_3\) des permutations de 3 éléments.
    \begin{q}{1}
        Déterminer les sous-groupes de \(\mathfrak{S}_3\).
        \boxans{D'après le théorème de \textsc{Lagrange}, un sous groupe de
        \(\mathfrak{S}_3\) est de cardinal \(1, 2, 3\) ou \(6\)
        \begin{enumerate}
            \itt 1 : \(\{\id\}\)
            \itt 2 : \(\langle (1\ 2)\rangle,\langle (1\ 3)\rangle,\langle (2\ 3)\rangle\) sont les seuls car \(H\)
            est cyclique et n'a que \(3\) éléments d'ordre \(2\).
            \itt 3 : \(\langle (1\ 2\ 3)\rangle=\langle(1\ 3\ 2)\rangle=\mathfrak{A}_3\)
            \itt 6 : \(\mathfrak{S}_3\)
        \end{enumerate}}
    \end{q}
    \begin{q}{2}
        Parmi ces sous-groupes, lesquels sont distingués ?
        \boxans{Le groupe de cardinal \(1\) et le groupe entier sont trivialement
        distingués. Pour montrer que \(\mathfrak{A}_3\) l'est montrons un lemme :
        \medskip
        \begin{lemme}
            Soit \(G\) un groupe, si \(H\vartriangleleft G\) vérifie \(|G/H|=2\)
            alors \(H\) est distingué dans \(G\).
            \begin{proof}
                Soit \(x\in G\) on veut \(xH=Hx\). Si \(x\in H\) le résultat est immédiat,
                sinon \(G/H\) est une partition de \(G\) ainsi \(G=H\sqcup xH = H \sqcup Hx\)
                et on a bien l'égalité souhaitée.
            \end{proof}
        \end{lemme}
        Ainsi comme \(|\mathfrak{S}_n/\mathfrak{A}_n|=2\) alors \(\mathfrak{A}_3\) est normal.
        Les autre sous groupes ne sont pas distingués, ce qui se vérifie de façon bourrine.}
    \end{q}
    \begin{q}{3}
        Donner la liste des groupes quotients de \(\mathfrak{S}_3\).
        \boxans{Les groupes quotients obtenus sont \(\mathfrak{S}_3, \{\id\}\) et \(\{id, (1\ 2)\}\)}
    \end{q}
\end{exo}

\begin{exo}
    On considère le groupe quotient \(\left(\Q/\Z,+\right)\).
    \begin{q}{1}
        Montrer que l'application \(\varphi : \left(\R,+\right)\to\left(\C,\times\right)\)
        définie par \(\varphi(x)=e^{2i\pi x}\) est un morphisme de groupes.
        \boxans{
            Soient \(x, y \in \R\). Alors,
            \[\varphi(x + y) = e^{2i\pi (x + y)} = e^{2i\pi x} \cdot e^{2i\pi y} = \varphi(x) \cdot \varphi(y).\]
            Donc, \(\varphi\) est un morphisme de groupes.
        }
    \end{q}
    \begin{q}{2}
        Calculer son noyau et son image, et en déduire que \(\R/\Z\simeq \U\).
        \boxans{
            Le noyau de \(\varphi\) est \(\Z\) et son image est \(\U\). Par le premier théorème d'isomorphisme, \(\R/\Z\simeq \U\).
        }
    \end{q}
    \begin{q}{3}
        On note \(\phi\) la restriction de \(\varphi\) à \(\Q\). Montrer que l'image
        de \(\phi\) est égale à \(\cup_{n\in\N} \U_n = \U_\N\).
        \boxans{
            Pour tout \(n \in \N\), \(\phi(\frac{1}{n}) = e^{2i\pi/n}\) et \(\phi(\frac{-1}{n}) = e^{-2i\pi/n}\), donc l'image de \(\phi\) est \(\cup_{n\in\N} \U_n\).
        }
    \end{q}
    \begin{q}{4}
        En déduire que \(\Q/\Z\simeq \U_\N\).
        \boxans{
            Par le premier théorème d'isomorphisme, \(\Q/\Z \simeq \U_\N\).
        }
    \end{q}
\end{exo}

\begin{exo}
    TODO
\end{exo}

\begin{exo}
    Soit \(G\) un groupe et \(Z(G)\) son centre.
    \begin{q}{1}
        Montrer que tout sous-groupe de \(Z(G)\) est un sous-groupe distingué
        de \(G\).
        \boxans{
            Soit \(H\) un sous-groupe de \(Z(G)\). Soit \(g \in G\) et \(h \in H\).
            On a \(ghg^{-1} =h\in H\) car \(h\) commute avec \(g\) et donc \(H\) est
            normal dans \(G\).
        }
    \end{q}
    \begin{q}{2}
        Soit \(H\) un sous-groupe de \(Z(G)\). On suppose que \(G/H\)
        est cyclique. Montrer que \(G\) est abélien.
        \boxans{
            Soit \(xH\) et \(yH\) deux éléments de \(G/H\), où \(x, y \in G\). Puisque
            \(G/H\) est cyclique, il existe un entier \(n\) tel que \(yH = x^nH\).
            Ainsi \(G = \{x^nh \mid n\in\Z,h\in H\}\). et donc tout commute bien.
        }
    \end{q}
\end{exo}


\begin{exo}
    Soit \(G\) un groupe et \(H\) un sous-groupe normal d'indice \(n \in \N\).
    \begin{q}{1}
        Montrer que pour tout \(a \in G\), on a \(a^n \in H\).
        \boxans{
            Soit \(a \in G\). Puisque \(H\) est d'indice \(n\), le groupe quotient
            \(G/H\) a ordre \(n\). Par le théorème de Lagrange, l'ordre de \(aH\)
            dans \(G/H\) divise l'ordre de \(G/H\), c'est-à-dire \(n\). Ainsi,
            \((aH)^n = a^nH = H\), ce qui implique que \(a^n \in H\).
        }
    \end{q}
    \begin{q}{2}
        Donner un contre-exemple avec un \(H\) non distingué.
        \boxans{
            Soit \(G = S_3\), le groupe symétrique d'ordre \(3\), et \(H = \langle
            (1,2) \rangle\), le sous-groupe engendré par la permutation \((1,2)\). On a \(|G : H| = 3\). Cependant, \(H\) n'est pas distingué dans \(G\), car il n'est pas invariant par conjugaison.
        }
    \end{q}
\end{exo}


\begin{exo}
    Soient \(G\) un groupe fini et \(H\triangleleft G\) d'ordre \(n\)
    et d'indice \(m\). On suppose que \(n\) et \(m\) sont premiers entre
    eux. Montrer que \(H\) est l'unique sous-groupe de \(G\) d'ordre \(n\).
    \boxans{
        D'après l'exercice précédent, pour tout \(a\in G\) on a \(a^n\in H\).
        Soit \(K\) un sous-groupe de \(G\) d'ordre \(n\). Puisque \(H\) est normal
        dans \(G\), le groupe quotient \(G/H\) est bien défini. On a \(|G/H| = m\) et
        \(|K| = n\). Comme \(n\) et \(m\) sont premiers entre eux, il existe un unique
        sous-groupe d'ordre \(n\) dans \(G/H\), qui est \(H/H\). En utilisant le
        correspondant du troisième théorème d'isomorphisme, on conclut que \(K = H\).
    }
\end{exo}

\begin{exo}
    Soit \(G\) un groupe. On appelle groupe dérivé de \(G\) et on note \(D(G)\)
    le sous-groupe de \(G\) engendré par \(\left\{ xyx^{-1}y^{-1} \mid x,y\in G \right\}\).
    \begin{q}{1}
        Montrer que \(D(G)\) est distingué dans \(G\) et que \(G/D(G)\) est abélien.
        \boxans{
            Soit \(g \in G\) et \(h \in D(G)\). On a \(ghg^{-1}h^{-1} = (g^{-1})^{-1}
            hg^{-1}h^{-1} \in D\), donc \(gHg^{-1} \subseteq H\). Puisque cela est
            vrai pour tout \(g \in G\), on conclut que \(D(G)\) est distingué dans \(G\).
            Considérons le groupe quotient \(G/D(G)\). Soit \(gD(G)\) et \(hD(G)\) deux éléments de ce groupe. On a
            \[(gD(G))(hD(G)) = ghD(G) = (hgD(G))^{-1} = (hD(G))(gD(G))\]
            ce qui montre que \(G/D(G)\) est abélien.
        }
    \end{q}
\end{exo}

\end{document}