\documentclass{report}
\usepackage{../exercices}

\begin{document}

\begin{center}
    \huge{\textbf{TD5: Ordre d'un élément, groupes cycliques}}
\end{center}

\begin{exo}
    Soient \(m,n\in\N\backslash\{0\}\).
    \begin{q}{1}
        Démontrer l'équivalence \(\U_m\subseteq \U_n \Leftrightarrow m\mid n\).
        \boxans{
            \(\U_m \subseteq \U_n\) si et seulement si \(e^{2\pi/m}\in\U_n\)
            or d'après le théorème de \textsc{Lagrange} celà se produit
            si et seulement si \(m\mid n\).
        }
    \end{q}
    \begin{q}{2}
        Montrer que \(\U_n\) est cyclique.
        \boxans{
            L'élément \(\omega=e^{2\pi/n}\) est clairement générateur.
        }
    \end{q}
    \begin{q}{3}
        Décrire les sous-groupes de \(\U_n\).
        \boxans{
            Les sous-groupes de \(\U_n\) correspondent exactement aux \(\U_d\) avec
            \(d\mid n\) par isomorphisme avec les \(\Z/n\Z\). D'après la question \(1\)
            ce sont les seuls et ils sont clairement générés par \(\omega\).
        }
    \end{q}
    \begin{q}{4}
        Donner l'ordre des éléments de \(\U_n\).
        \boxans{
            Les éléments de \(\U_n\) sont de la forme \(e^{2k\pi i/n}\), où \(k\) est un entier.
            L'ordre d'un tel élément est le plus petit entier positif \(m\) tel que \((e^{2k\pi i/n})^m = 1\),
            c'est-à-dire \(me^{2k\pi i/n} = 2\pi i l\) pour un certain entier \(l\). Cela se produit si et
            seulement si \(n\) divise \(km\). Ainsi, l'ordre de \(e^{2k\pi i/n}\) dans \(\U_n\) est \(n\) si
            \(k\land n=1\) et \(\frac nk\) sinon.
        }
    \end{q}
\end{exo}

\begin{exo}
    Démontrer que des éléments conjugués d'un groupe sont de même ordre. La réciproque
    est-elle vraie ?
    \boxans{Soit \(G\) un groupe, on pose \(a\) et \(b\) conjugués, il existe
    \(g\in G\) tel que \(a=gbg^{-1}\) alors si \(b\) est d'ordre \(n\in N\) on a
    \(a^n = g\star e\star g^{-1}=e\) donc \(o(a)\mid o(b)\). Aussi en écrivant \(b=g^{-1}ag^{-1}\) on a
    \(o(b)\mid o(a)\) donc \(o(b)=o(a)\). La réciproque est fausse car \(\Z_3\) est
    abélien et compte \(3\) éléments d'ordre \(3\).}
\end{exo}

\begin{exo}
    Soit \(G\) un groupe et \(x\in G\) un élément d'ordre \(n\). Démontrer
    que \(x^m\) est d'ordre \(\frac{n}{\gcd(m,n)}\).
    \boxans{Soit \(G\) un groupe et \(x\in G\) d'ordre \(n\in\N\). Soit \(m\in \N\) on considère
    \(d = n \land m\) alors \((x^m)^{n/d} = (x^n)^{m/d} = e\) donc \(o(x^m) \mid \frac nd\).
    Soit \(p\in\N\) tel que \((x^m)^p=e\) alors \(n\mid np\) donc \(\frac nd \mid \frac md \times p\)
    donc d'après le lemme de \textsc{Gauss} \(\frac nd\mid p\) et donc \(o(x^m)=\frac nd\).}
\end{exo}

\begin{exo}
    Soit \(f\colon G\to H\) un homomorphisme de groupes. Soit \(x\in G\) d'ordre
    fini noté \(n\).
    \begin{q}{1}
        Démontrer que \(f(x)\) est d'ordre fini, et que son ordre divise \(n\).
        \boxans{Notons \(m=o(f(x))\), on a \(f(x)^m = 1 = f(x^n) = f(x)^n\). On peut ensuite
        effectuer la division euclidienne de \(n\) par \(m\) pour obtenir \(n=mq+r\), montrons
        que \(r=0\). On suppose par l'absurde \(r>0\) alors \(f(x)^n = f(x)^r\) or \(f(x)^n=1\) et \(r<m\)
        implique \(f(x)^r\neq 1\) ce qui est absurde, ainsi \(r=0\) et \(m\mid n\).}
    \end{q}
    \begin{q}{2}
        Démontrer que \(f(x)\) est d'ordre exactement \(n\) si \(\Ker(f)\) est
        d'ordre premier à \(n\).
        \boxans{On note \(k\) l'ordre de \(\Ker(f)\) et \(m\mid n\) l'ordre de \(f(x)\), on note
        \(d = \frac nm\in\N\). On sait que \(f(x)^m = (f(x)^m)^d=1\) ainsi \(\langle x^m\rangle \subseteq \Ker(f)\).
        La théorème de \textsc{Langrange} affirme alors que \(d \mid k\) or \(d\mid n\) et
        \(k\land n = 1\) donc \(d= 1\) et finalement \(m=n\)}
    \end{q}
\end{exo}

\begin{exo}
    Montrer que les groupes suivants ne sont pas isomorphes.
    \begin{q}{1}
        \(\left(\Z/3\Z\times\Z/3\Z,+\right)\) et \(\left(\Z/9\Z,+\right)\).
        \boxans{Le premier n'est pas monogène.}
    \end{q}
    \begin{q}{2}
        \(\left(\Z/3\Z\times\Z/8\Z,+\right)\) et \(\left(\Z/4\Z\times\Z/6\Z,+\right)\).
        \boxans{
            Supposons par l'absurde qu'il existe un isomorphisme
            \(\phi: \Z/3\Z\times\Z/8\Z \to \Z/4\Z\times\Z/6\Z\).
            Alors, considérons les images de \((1,0)\) et \((0,1)\) par \(\phi\).
            Ces images doivent générer le groupe \(\Z/4\Z\times\Z/6\Z\), mais
            c'est impossible car l'ordre du premier générateur divise \(4\) et
            l'ordre du second divise \(6\), et il n'y a pas d'élément d'ordre commun à \(4\) et \(6\).
            Ainsi, \(\Z/3\Z\times\Z/8\Z\) et \(\Z/4\Z\times\Z/6\Z\) ne sont pas isomorphes.
        }
    \end{q}
\end{exo}

\begin{exo}
    Soient \(a\) et \(b\) deux entiers naturels strictement plus grands que \(1\).
    Montrer que le nombre de morphismes de groupes de \(\Z/a\Z\) vers \(\Z/b\Z\)
    est \(\gcd(a,b)\).
    \boxans{
        Supposons que \(d = \gcd(a, b)\). Considérons
        l'application \(\phi: \Z/a\Z \to \Z/b\Z\) définie par \(\phi(1) = b/d\).
        On construit alors \(\Phi\colon \Z/d\Z\to \hom(\Z/a\Z,\Z/b\Z)\) quoi est un
        clair isomorphisme tel que \(\Phi(k) = k \phi\).
    }
\end{exo}

\begin{exo}
    Travail non pertinent sur de la multiplication par 3.
    \begin{q}{1}
        Montrer que les groupes \(\left(3\Z/18\Z,+\right)\) et \(\left(\Z/6\Z,+\right)\)
        sont isomorphes.
        \boxans{L'isomorphisme \(\phi\colon a\to \frac a3\) suffit à la démonstration.}
    \end{q}
    \begin{q}{2}
        Décrire les sous-groupes, l'ordre des éléments et les générateurs de
        \(\left(3\Z/18\Z,+\right)\).
        \boxans{
            Cf. travail sur les \(\Z/n\Z\) dans le TD précédent.
        }
    \end{q}
\end{exo}


\begin{exo}
    Soit \(G\) = \(\langle(1\ 2\ \dots\ 11\ 12)\rangle\subseteq\mathfrak{S}_{12}\).
    Déterminer les générateurs et les sous-groupes de \(G\).
    \boxans{Tous les éléments sont générateurs, le groupe est cyclique et isomorphe à \(\Z/12\Z\).}
\end{exo}

\begin{exo}
    Soient \(x,y\) deux éléments d'ordre fini d'un groupe tels que \(xy=yx\). On note
    \(n\) et \(m\) leurs ordres respectifs.
    \begin{q}{1}
        Démontrer que si \(n\land m = 1\) alors \(xy\) est d'ordre \(mn\).
        \boxans{On commence par remarquer que \((xy)^{nm}=(x^n)^m(y^m)^n=e\) donc
        \(o(xy) \mid mn\). On note maintenant \(k=o(xy)\) alors \(((xy)^k)^m=(x^m)^k=e\)
        donc \(n\mid mk\) ainsi par le lemme de \textsc{Gauss} \(n\mid k\) et par symétrie
        du raisonnement \(m \mid k\) ainsi \(k = mn\).}
    \end{q}
    \begin{q}{2}
        Démontrer que \(xy\) est d'ordre fini et que \(\frac{mn}{d^2} \mid k \mid \frac{mn}{d}\)
        en notant \(k\) l'ordre de \(xy\) et \(d=n\land m\).
        \boxans{D'après la question précédente \(k\) est fini et divise \(nm\).
        \begin{enumerate}
            \itt \((xy)^{mn/d}= (x^n)^{m/d}(y^m)^{n/d} = e\) donc \(k \mid \frac{nm}{d}\).
            \itt \(o((xy)^d) \mid k\) donc \(\frac{nm}{d^2}\mid k\).
        \end{enumerate}}
    \end{q}
    \begin{q}{3}
        On suppose que \(m=24\) et \(n=18\). Démontrer que \(G\) contient un
        élément d'ordre \(72\)
        \boxans{Si \(x\) est d'ordre \(24\) et \(y\) est d'ordre \(18\) on a alors
        \(x^3\) est d'ordre \(8\) et \(y^2\) est d'ordre \(9\) ce qui d'après la question
        \(1\) donne que l'ordre de \(x^3y^2\) est \(72\).}
    \end{q}
\end{exo}

\begin{exo}
    Retour rapide sur les matrices.
    \begin{q}{1}
        Exhiber un élément d'ordre \(3\) dans \(\glnr\).
        \boxans{
            La matrice suivante est d'ordre \(3\) dans \(\glnr\).
            \[
                A = \begin{pmatrix}
                    0 & 1 & 0 \\
                    0 & 0 & 1 \\
                    1 & 0 & 0
                \end{pmatrix}
            \]
        }
    \end{q}
    \begin{q}{2}
        Démontrer qu'un tel élément est toujours dans \(\text{SL}_n(\R)\).
        \boxans{
            On a \(\det(I)=1\) or \(I=A^3\) donc \(\det(A^3)=\det(A)^3=1\)
            Ainsi \(\det(A)\) est un réél, racine triple de l'unité, donc \(\det(A)=1\).
            ce qui signifie \(A\in\text{SL}_n(\R)\).
        }
    \end{q}
\end{exo}


\begin{exo}
    Soient \(G\) un groupe fini d'ordre noté \(n\) et \(m\) un entier premier à \(n\).
    Démontrer que pour tout \(a\in G\) l'équation \(x^m=a\) admet une unique solution.
    \boxans{
        Comme \(m\) est premier à \(n\), l'élément \(m\) est inversible modulo \(n\).
        Soit \(b\) l'inverse de \(m\) modulo \(n\). Ainsi, l'équation \(x^m = a\) peut
        être écrite comme \(x = a^b\), et cette équation admet une unique solution dans
        \(G\) puisque l'élément \(b\) est unique dans \(\mathbb{Z}/n\mathbb{Z}\) qui est cyclique.
    }
\end{exo}


\begin{exo}
    Soit \(n\in\N^*\)
    \begin{q}{1}
        Quels sont les générateurs de \(\Z/n\Z\) ? On note \(\varphi(n)\) ce nombre.
        \boxans{Ce sont les classes des éléments inférieurs à \(n\) et permier à \(n\).}
    \end{q}
    \begin{q}{1}
        Démontrer l'égalité \(\ds\sum_{d\mid n} \varphi(d) = n\).
        \boxans{D'après le théorème chinois on a
        \[(\Z/n\Z)^*\sim \prod_{d\mid n} (\Z/d\Z)^* \]
        Ce qui donne le résultat attendu sur un plateau d'argent}
    \end{q}
\end{exo}

\begin{exo} Un peu de fun
    \begin{q}{1}
        Montrer que \(\textbf{Aut}\left(\Z/7\Z\right)\) est cyclique
        \boxans{
            L'automorphisme de \textsc{Frobenius} \(\phi : x \mapsto x^3\) génère \(\textbf{Aut}\left(\Z/7\Z\right)\),
            car \(\phi\) a un ordre maximal (6) dans le groupe des automorphismes.
        }
    \end{q}
    \begin{q}{2}
        Déterminer les morphismes de groupes de \(\left(\Z/4\Z,+\right)\) dans \(\textbf{Aut}\left(\Z/7\Z\right)\).
        \boxans{
            Pour déterminer les morphismes de groupes de \(\left(\Z/4\Z,+\right)\) dans \(\textbf{Aut}\left(\Z/7\Z\right)\),
            nous devons trouver les éléments de \(\textbf{Aut}\left(\Z/7\Z\right)\) d'ordre diviseur de 4.

            L'ordre de \(\phi\) est 6, donc les éléments d'ordre 4 sont générés par \(\phi^2\).
            Ainsi, les morphismes sont donnés par :
            \[\begin{aligned}
                \psi_1 : & \quad 0 \mapsto \text{id}, \quad 1 \mapsto \phi^2, \quad 2 \mapsto \phi^4, \quad 3 \mapsto \phi^6 \\
                \psi_2 : & \quad 0 \mapsto \text{id}, \quad 1 \mapsto \phi^4, \quad 2 \mapsto \phi^1, \quad 3 \mapsto \phi^5 \\
                \psi_3 : & \quad 0 \mapsto \text{id}, \quad 1 \mapsto \phi^6, \quad 2 \mapsto \phi^5, \quad 3 \mapsto \phi^4 \\
            \end{aligned}
            \]
        }
    \end{q}
\end{exo}
\end{document}