\documentclass[french]{report}
\usepackage{../exercices}

\usepackage{causets}

\begin{document}

\begin{center}
    \huge{\textbf{S6- Logique - TD2}}
\end{center}

\begin{exo}
    Montrer que les trois axiomes de \textsc{Peano} sont indépendants : pour cela il
    faut construire pour chacun des axiomes un \textit{contre-modèle}, c'est à dire
    un ensemble \(\eta\) conetnant un élément distingué, appelé \(0\), et sur lequel
    on défini une fonction \(s:\eta\to\eta\) de façon que pour \(\left(\eta,0,s\right)\),
    l'axiome dont on veut montrer qu'il est indépendant n'est pas vérifié, mais les
    deux autres le sont.
    \boxans{On procède en trois cas
    \begin{enumerate}
        \itt En prenant \(s\) le prédécesseur sur tout autre que \(0\) on a l'indépendance
        du premier axiome.
        \itt En prenant \(s\) constante on a l'indépendance du deuxième axiome.
        \itt En prenant \(s: x\mapsto 2+x\) et \(P:\) "est pair" on a
        l'indépendance du troisième axiome.
    \end{enumerate}}
\end{exo}

\begin{exo}
    Démontrer que chacun des deux axiomes, le successeur d'un entier est non nul, et
    l'injectivité du successeur, sont bien nécessaires pour montrer l'existence de
    la fonction définie par rédcurrence (\textit{Indication: utilisez les contre-modèles
    de l'exercice 1}).
    \boxans{Si on reprend les contre-modèles au dessus, pour le coup du successeur non
    nul, ça revient par l'absurde à forcer \(h^n(a)=a\) pour un \(n\) dépendant du modèle.
    Pour le coup de l'injectivité ça force \(h\) a ne pas être injective non plus, ou à boucler.}
\end{exo}

\begin{exo}
    Soit \(A\) un ensemble, \(E\) un ensemble non vide, trois fonctions \(g\colon A\to E,
    h\colon (E\times A)\to E\) et \(\sigma\colon A\to A\). Montrer qu'alors il existe
    une unique fonction \(f\colon\left(\N\times A\right)\to E\) vérifiant:
    \[f(0,y)=g(y);\quad f(s(x),y) = h(f(x,\sigma(y)),y)\]
    \boxans{Cette démonstration est presque celle dans le cours, on se passera de la
    recopier ici.}
\end{exo}

\begin{exo}
    Démontrer ces propriétés par récurrence :
    \begin{q}{1}
        \(x+(y+z)=(x+y)+z\)
        \boxans{Par récurrence sur \(z\), on initialise, on a d'un côté \(x+(y+0)=x+y=
        (x+y)+0\). Pour la récurence, \(x+(y+s(z))=x+s(y+z)=s(x+y+z)=(x+y)+s(z)\).}
    \end{q}
    \begin{q}{2}
        \(0+x=x\)
        \boxans{Par récurrence immédiate avec \(0+s(x)=s(0+x)=s(x)\) par hypothèse.}
    \end{q}
    \begin{q}{3}
        \(s(x)+y=s(x+y)\)
    \end{q}
    \begin{q}{4}
        \(x+y=y+x\)
    \end{q}
    \begin{q}{5}
        \(x\cdot(y+z)=x\cdot y+x\cdot z\)
        \boxans{Par récurrence, \(x\cdot(y+s(z))=x\cdot s(y+z)=x\cdot(y+z)+x=x\cdot
        y+x\cdot z+x=x\cdot y+x\cdot s(z)\)}
    \end{q}
    \begin{q}{6}
        \(x\cdot(y\cdot z)=(x\cdot y)\cdot z\)
    \end{q}
    \begin{q}{7}
        \(0\cdot x=0\)
    \end{q}
    \begin{q}{8}
        \(s(x)\cdot y=x\cdot y+y\)
    \end{q}
    \begin{q}{9}
        \(x\cdot y=y\cdot x\)
    \end{q}
\end{exo}

\begin{exo}
    Montrer les propriétés usuelles suivantes de la fonction puissance pour tous les
    entiers naturels \(x,y\) et \(z\) :
    \begin{q}{1}
        \(0^{x+1}=0\)
    \end{q}
    \begin{q}{2}
        \(1^x=1\)
    \end{q}
    \begin{q}{3}
        \(x^1=x\)
        \boxans{\(x^1 = x^0\cdot x = 1\cdot x =x\)}
    \end{q}
    \begin{q}{4}
        \(\left(x\cdot y\right)^z=x^z\cdot y^z\)
    \end{q}
    \begin{q}{5}
        \(x^{y+z}=x^y\cdot x^z\)
    \end{q}
    \begin{q}{6}
        \(\left(x^y\right)^z=x^{y\cdot z}\)
    \end{q}
\end{exo}

\begin{exo}
    Montrer, à partir des axiomes de Peano ou des propriétés déjà démontrées, la
    régularité de l'addition, à savoir que pour tous les entiers naturels \(x,z,z'\):
    \[z+x=z'+x\Rightarrow z=z'\quad x+z=x+z'\Rightarrow z=z'\]
\end{exo}

\begin{exo}
    Montrer, à partir des axiomes de Peano ou des propriétés déjà démontrées,
    que pour tous les entiers naturels \(x,y\) :
    \[x+y=0\Rightarrow\left(x=0\land y=0\right)\quad x\cdot
    y=0\Rightarrow\left(x=0\lor y=0\right)\]
\end{exo}
\end{document}