\documentclass[french]{report}
\usepackage{../exercices}

\usepackage{causets}

\begin{document}

\begin{center}
    \huge{\textbf{S6- Logique - TD1}}
\end{center}

\begin{exo}
    Dans un ensemble ordonné \(\left(A, \leq_A\right)\) on dit que \(y\) est successeur
    de \(x\) noté \(x\prec y\), quand
    \[x<y\ \land\ \forall z\in A, (x\leq z\leq y)\Rightarrow(z=x\lor z=y)\]
    Montrer que si \(\left(A, \leq_A\right)\) est fini alors la relation \(\leq\) est la
    clôture transitive de la relation successeur. C'est-à-dire qu'étant donné \(x,y\in A\),
    si \(x\leq y\) alors \(x=y\) ou il existe une suite finie \(x=x_0\prec x_1\prec\dots
    \prec x_n=y\).
    \boxans{On suppose \(A\) fini de cardinal au moins \(2\), on se donne deux
    éléments de \(A\) que l'on suppose distincts. On établit alors une disjonction de cas
    si \(x\prec y\) on a le résultat voulu, sinon, il existe \(z\) tel que
    \(x\leq z\leq y\) et on relance recursivement sur \((x,z)\) et \((z,y)\). L'algorithme
    de création de chaine termine car \(A\) est fini, on obtient le résultat voulu. Le fait
    que la clôture transitive est la relation \(\leq\) est immédiat.}
\end{exo}

\begin{exo}
    Pour représenter un ordre fini, il suffit de représenter le graphe de la relation
    successeur associée d'après la question précédente. On représente habituellement
    ces graphes de façon que si \(x<y\), \(y\) soit placé au dessus de \(x\). Un tel
    graphe est appelé diagramme de \textsc{Hasse}. Décrire à l'aide d'un graphe tous les
    ordres (à isomorphisme près) à 1, 2, et 3 éléments. En déduire tous les ordres à
    \(4\) éléments possédant un plus petit élément.
    \boxans{Les ordres à 1,2,3 éléments sont ci-contre : \causet{1}{} | \causet{1,2}{1/2} | \causet{2,1}{} | \causet{1,2,3}{1/2,2/3}
    | \causet{1,3,2}{1/2,1/3} | \causet{2,1,3}{2/3,1/3} | \causet{3,1,2}{1/2} | \causet{3,2,1}{} ce qui donne que les ordre à
    \(4\) éléments possédant un maximum sont:\[
        \causet{1,2,3,4}{1/2,2/3,3/4}\quad\causet{1,4,3,2}{1/2,1/3,1/4}\quad
        \causet{1,4,3,2}{1/2,1/4,2/3,4/3}\quad
        \causet{1,4,7,2}{1/2,1/4,4/7} \text{ et le Y}
    \]}
\end{exo}

\begin{exo}
    Soit \(\left(A, \leq\right)\) un ensemble ordonné. Montrer que l'ordre \(\leq\)
    si et seulement si \(\not\leq\) est une relation d'ordre strict.
\end{exo}

\begin{exo}
    Soient \(\left(A,\leq_A\right)\) et \(\left(B,\leq_B\right)\) deux ensemble ordonnés.
    \begin{q}{1}
        Montrer que si \(\left(A,\leq\right)\) est totalement ordonné, toute injection
        (bijection) croissante de \(\left(A,\leq\right)\) dans \(\left(B,\leq\right)\)
        est un plongement d'ordre.
        \boxans{Par définition d'une fonction croissante, l'implication est immédiate.
        Soient \(f\) un plongement d'ordreet \(x,y\) vérifiant \(f(x)\leq f(y)\), si
        \(x=y\) c'est bon, sinon si \(x\geq y\) la fonction n'est pas croissante, c'est absurde,
        donc \(x\leq y\).}
    \end{q}
    \begin{q}{2}
        Montrer que l'on peut trouver \(\left(B,\leq_B\right)\) tel que la propriété
        ci-dessus est fausse dès que \(\left(A,\leq\right)\) n'est pas totalement
        ordonné.
        \boxans{Soit \(\left(A,\leq\right)\) qui n'est pas totalement ordonné, on cherche
        à créer une injection croissante qui n'est pas un plongement. Soient \(x,y\)
        des éléments non comparables de \(A\). On pose l'injection identité modifiée
        en ce que \(f(x)=y\) et \(f(y)=x\). C'est bien une injection croissante}
    \end{q}
    \begin{q}{3}
        Montrer que le théorème de \textsc{Cantor-Schröder-Bernstein} pour les ordres
        totaux est faux: Il existe deux ordres totaux \(\left(A,\leq_A\right)\) et
        \(\left(B,\leq_B\right)\) tels que \(A\) se plonge dans \(B\), \(B\) se plonge
        dans \(A\) mais \(A\) et \(B\) ne sont pas isomorphes.
        \boxans{}
    \end{q}
\end{exo}

\begin{exo}
    Soient deux ensembles ordonnés \(\left(A,\leq_A\right)\) et \(\left(B,\leq_B\right)\)
    supposés disjoints, il existe plusieurs façons de définit un ordre sur \(A\cup B\)
    la réunion disjointe de \(A\) et \(B\), la plus simple étant de prendre
    la réunion des deux relations d'ordres qui est encore un ordre.

    La somme linéaire \(\left(A,\leq_A\right)\oplus\left(B,\leq_B\right)\) et l'ensemble
    prdonné \(\left(A\cup B,\leq_{A\cup B}\right)\) qui prolonge les deux ordres sur \(A\)
    et sur \(B\) et place tous les éléments de \(B\) après ceux de \(A\).
    \[x\leq_{A\cup B} y \Leftrightarrow \left(x\in A\land y\in B\right)
    \lor\left(x\in A\land y\in A\land x\leq_A y\right)\lor
    \left(x\in B\land y\in B\land x\leq_B y\right)\]
    \begin{q}{1}
        Montrer que si \(A\) et \(B\) sont totalement ordonnés, leur somme l'est aussi.
        \boxans{Soient \(x,y\in A\cup B\), si les deux sont dans \(A\) (respectivement \(B\)) alors
        ils sont comparables par hypothèse, sinon, l'un est dans \(A\) et l'autre dans \(B\),
        ils sont alors bien comparables d'après le premier cas de la définition de \(\leq_{A\cup B}\).}
    \end{q}
    \begin{q}{2}
        Montrer que si \(A\) et \(B\) sont bien ordonnés, la somme aussi.
        \boxans{Soit \(X\) une partie non vide de \(A\cup B\) alors le minimum
        de \(X\cap A\) convient si cet ensemble est non vide, s'il l'est on prend
        le minimum de \(X\cap B = X\)}
    \end{q}
\end{exo}
\end{document}