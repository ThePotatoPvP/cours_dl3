\documentclass[french]{report}
\usepackage{../exercices}

\usepackage{causets}

\begin{document}

\begin{center}
    \huge{\textbf{S6- Logique - TD3}}
\end{center}

\begin{exo}
    Trouver à chaque fois une fonction bijective :
    \begin{q}{1}
        De \(\N\) dans \(\N^*\)
        \boxans{\(\varphi\colon x\mapsto x+1\) convient.}
    \end{q}
    \begin{q}{2}
        De \(\N\) dans \(2\N\) et de \(\Z\) dans \(\N\).
    \end{q}
    \begin{q}{3}
        De \((-\frac\pi2,\frac\pi2)\) dans \(\R\) puis de \((-1,1)\) dans \(\R\).
    \end{q}
\end{exo}

\begin{exo}
    Le but de cet exercice est de montrer que \(\R\) n'est pas dénombrable, sans utiliser
    le théorème de \textsc{Cantor}.
    \begin{q}{1}
        Soit \(\textsc{Dec}:=\{0,1,\dots,9\}^\N\) l'ensemble de toutes les suites
        de chiffres. Soit \(F:\textrm{Dec}\to\text{Dec}\), définie par \(F((a_n))=(b_n)\)
        où \(b_n = 2 - \delta_{1,a_n}\).
        Soit \(\mathcal{A}:=\{\bar{a}^k\}_{k\in\N}\subset\text{Dec},\bar{a}^k=
        \left(a^{(k)}_n\right)\). Montrer que \(F((a^{(n)}_n)_{n\in\N})\notin\mathcal{A}\).
    \end{q}
\end{exo}

\begin{exo}
    Montrer que si \(E\) est infini et \(A\) est une partie finie ou dénombrable de \(E\)
    telle que \(E\backslash A\) est infini alors \(E\) a le même cardinal
    que \(E\backslash A\).
\end{exo}

\begin{exo}
    Sur \(\pow\)
    \begin{q}{1}
        Rappeler pourquoi \(\pow\) et \(\{0,1\}^N\) ont le même cardinal
        \boxans{La fonction \(\varphi : P\mapsto \ind_P\) forme une bijection entre
        ces deux ensembles, qui sont donc de même cardinal d'après le théorème de
        \textsc{Cantor-Bernstein}.}
    \end{q}
    \begin{q}{2}
        Montrer que l'ensemble des parties finies de \(\N\) est dénombrable.
        \boxans{Soit \(P\) une partie finie de \(\N\) de cardinal \(n\), soit
        \(p\) le \(n\)-ième nombre premier. Il y a moins de \(\N^n\) ensembles de même
        cardinal, ce qui est dénombrable, on note \(k\) l'indice maximum atteint par \(P\)
        dans une suite croissante pour un ordre fixé, alors \(\varphi\colon P\mapsto p^k\)
        forme une injection des parties finies de \(\N\) dans \(\N\) ce qui démontre le résultat.}
    \end{q}
    \begin{q}{3}
        En déduire que l'ensemble des parties infinies de \(\N\) est de même cardinal
        que \(\pow\).
        \boxans{On se rapelle que \(\pow\) n'est pas dénombrable, donc en retirant
        une quantité dénombrable d'éléments il en reste une quantité non dénombrable.}
    \end{q}
    \begin{q}{4}
        En se servant de la numérotation binaire, établir une bijection entre \([0,1)\)
        et l'ensemble des parties infinies de \(\N\).
        \boxans{On peut prendre la concaténation des écritures en base \(9\) de chaque
        nombre de l'ensemble, séparés par un \(9\) pour avoir une injection des
        parties infinies de \(\pow\) dans \([0,1)\)}
    \end{q}
    \begin{q}{5}
        En déduire que \(\pow\) et \(\R\) ont le même cardinal.
    \end{q}
\end{exo}

\end{document}