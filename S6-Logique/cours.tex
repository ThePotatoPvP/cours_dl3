\documentclass{report}
\usepackage{../exercices}

\begin{document}

\begin{center}
    \huge{\textbf{Cours de Logique}}
\end{center}

\boxans{
    \begin{enumerate}
        \itt Sylvy Anscombe
        \itt sylvy.anscombe@inij-prog.fr
    \end{enumerate}
}

\section*{Sommaire à titre indicatif}

\boxans{
    \begin{enumerate}
        \itt Ordres linéaires
        \itt Bons ordres
        \itt Définitions inductives
        \itt Arithmétique de Péano
        \itt -----
        \itt Démonstration par récurrence
        \itt L'ordre sur \(\N\)
        \itt-----
        \itt Notion de cardinalité, équipotence
        \itt Ensemble dénombrables de \textsc{Dedekind}
        \itt Diagonale de \textsc{Kantor}
        \itt Théorème de \textsc{Kantor}-\textsc{Burnstein}- ?
        \itt-----
        \itt Lemme de \textsc{Zorn} et axiome du choix
        \itt-----
        \itt Calcul propositionnel (syntaxe, sémantique, valuations)
        \itt Formes normales (disjonctive, conjonctive) et systèmes complets
        \itt Calcul de prédicat (logique du premier ordre)
    \end{enumerate}
}

% \tableofcontents

\chapter{Les axiomes des Ordres}

\bdef{Relation d'ordre}{
    On dit que la relation \(\preceq\) est une relation d'ordre sur un ensemble \(E\) si
    elle est \begin{enumerate}
        \itt \textbf{Reflexive} \(\colon \forall x\colon x\preceq x\)
        \itt \textbf{Antisymmétrique} \(\colon \forall x\forall y\colon (x\preceq y\land y\preceq x) \Rightarrow x=y\)
        \itt \textbf{Transitive} \(\colon\forall x\forall y\forall z\colon (x\preceq y\land y\preceq z) \Rightarrow x\preceq z\)
    \end{enumerate}
}
\begin{examples}
    Les relations \(\leq, \geq\) sur les nombres, l'ordre lexicographique ...
\end{examples}

\bdef{Ensemble ordonné}{
    Un ensemble ordonné \((E, \preceq)\) est un corps \textsc{Gaussien}
    munit d'une relation d'ordre.
}
\begin{examples}
    \((\N, \leq)\), \((\Q, \leq)\), \((\R, \leq)\), \((\Z, \leq)\), \((\Sigma^*, \preceq_\alpha)\).
\end{examples}

\bdef{Ordre total}{
    Une relation d'ordre est totale (ou linéaire) sur \(E\) si
    \(\forall x\forall y \colon x\preceq y\lor y\preceq x\).
}
\begin{examples}
    \((\N, \leq)\) est un ordre total, \((\N, \mid)\) non.
\end{examples}

\bdef{Morphisme d'ordre}{
    Un morphisme d'ordre \(\varphi\) est une application de \((A,\preceq_A)\) dans
    \((B,\preceq_B)\) vérifiant \(\forall x\forall y\in A\colon (x\preceq_A y)
    \Rightarrow(\varphi(x)\preceq_B \varphi(y))\). On peu aussi parler de morphismes
    d'ordres bijectifs.
}

\brem{
    Un isomorphisme d'ordre est une bijection croissante dont la réciproque
    est croissante.
}

\bdef{Bon ordre}{
    Un ordre \(\preceq\) est dit bon si tout sous-ensemble de \((E, \preceq)\)
    admet un plus petit élément. Un bon ordre est forcément total.
}
\begin{examples}
    \((\N, \leq)\) admet un bon ordre,
    \((\Z, \leq)\) n'admet pas de bon ordre,
    \(([0,1],\leq)\) non plus.
\end{examples}

\begin{proposition}
    Soit \((A, \preceq_A)\) un bon ordre et soit \(B\subseteq A\), \((B, \preceq_{A\mid B})\)
    est bien ordonné.
    \begin{proof}
        Soit \(C\subseteq B\) alors \(C\subseteq A\) donc \(C\) admet un minimum pour
        \(\preceq_A\), qui est aussi trivialement minimum pour \(\preceq_{A\mid B}\)
        donc \(B\) est bien ordonné.
    \end{proof}
\end{proposition}

\begin{exo}
    Attention à ne pas confondre isomorphisme et isomorphisme d'ordre.
    \begin{q}{1}
        Soit \(A = \{\frac{1}{n} \mid n\in\N\}\) est-il bien ordonné ?
        \boxans{Non, car \(A\) conttient une suite infinie décroissante.}
    \end{q}
    \begin{q}{2}
        Qu'en est-il de \(B = -A\) ?
        \boxans{L'ensemble \(B\) est bien ordonné, il admet un morphisme d'ordre trivial vers \(\N\).}
    \end{q}
\end{exo}

\begin{exo}
    Existe-t-il une suite infinie décroissante dans \(\pow\) ?
    \boxans{Oui, en posant \(X_0=\N\) et \(X_{n+1}=X_n\backslash\{\min(X_n)\}\) on
    construit une telle suite.}
\end{exo}

\begin{proposition}
    Un ensemble non vide avec un ordre total est bien ordonné si et seulement si il
    vérifie la propriété de récurrence bien fondée :
    \(\forall J\subseteq E\forall y(\forall z<y\colon z\in J\Rightarrow y\in J)\Rightarrow J=E\)
    qui se comprend comme : si tous les prec de y sont dans J, y aussi, c'est le concept
    de récurrence.
\end{proposition}

\begin{exo}
    On suppose \(X\subset \N\) pour lequel \(|\) est un ordre total, \(X\) est-il
    bien ordonné ?
    \boxans{\(X\) est alors uniquement constitué de produits de facteurs premiers
    dont tous les facteur croient de la même façon d'un élément à l'autre, l'ordre
    lexicographique y est donc total est bon.}
\end{exo}

\chapter{L'arithmétique}
\setcounter{exo}{0}

On veut une définition explicite des entiers, par exemple
\(0=\emptyset, 1 =\ \mid, 2 =\ \mid\mid\dots\) ici l'opérateur successeur n'est
autre que \(\text{succ}(x)=x\mid\). On cherche donc à passer par une définition
implicite par une axiomatisation.

\subsection*{Définition inductive}
On se place dans un univers avec deux symboles primitifs : \(0\) et \(succ\).
L'ensemble des entiers naturels est alors vu comme le plus petit ensemble (au
sens de l'inclusion) \(\N\) qui contient \(0\) et est stable par \(succ\). C'est
à dire, si \(x\in\N\) alors \(succ(x)\in\N\).


Appelons \(\text{Cl}(A)\) la propriété :l'ensemble \(A\) est clos par successeur :
\(0\in A\land \forall x\in A, s(x)\in A\). Si chacun des ensembles d'une familles \((A_i)\)
vérifie la propriété \(\text{Cl}(A_i)\) alors leur intersection également. Ainsi
\[\N = \bigcap_{\text{Cl}(A)=\top,A\in\textrm{Obj}(\textbf{Set})} A\]

\subsection*{Axiomes de Peano}

Les axiomes pour \(\left(\N, s, 0\right)\) sont :
\begin{enumerate}
    \itt \(\forall x\in\N, s(x)\neq 0\)
    \itt \(\forall x,y\in\N, s(x)=s(y)\Rightarrow x=y\)
    \itt Pour toute propriété \(P\) \textbf{bien définie} sur \(\N\) on a \(P(0)\land
    \left(\forall y\in\N \left(P(y)\Rightarrow P(s(y))\right)\right)\Rightarrow\forall x\in\N,P(x)\)
\end{enumerate}

\brem{Toute propriété du premier ordre est \textbf{bien définie}}

\begin{lemme}
    Tout entier est soit \(0\) soit un successeur :
    \(\forall x\in\N, x=0\lor (\exists y\in\N, x=s(y))\).
\end{lemme}
\begin{proof}
    La propriété du lemme est du premier ordre, elle est vraie en \(0\), elle est
    donc vraie.
\end{proof}

\begin{exo}
    Montrer que les \(3\) axiomes de Peano sont indépendants. C'est-à-dire avec un
    symbole \(0\) et une fonction \(s\) créer un ensemble respectant exactement
    \(2\) axiomes
    \boxans{
        Avec \(N=\{0,s(0)\}\), \(\forall x\in N, s(x)=s(0)\) et la propriété
        \(P(\cdot)=1s(0)\) on a bien exactement deux axiomes de vérifiés.}
\end{exo}

\subsection*{Définition par récurrence}

On peut définit \(\leq\) par \textit{il existe une suite} \(x\prec x_1\prec\dots\prec x_n\)
entre les éléments, ce n'est pas une défintition du premier ordre. On peu définit la
somme de la façon suivante \(\forall a\in\N, (a+0\colon=a)\land(\forall b\in\N a+s(b)=s(a+b))\).

\btheo{Dedekind}{
    Soit \(E\) un ensemble (non vide), \(a\in E\) et une fonction \(h\colon E\to E\)
    alors il existe une unique fonction \(f\colon\N\to E\) vérifiant \(f(0)=a\) et
    \(f(s(x)) = h(f(x))\).

    \bproof{
        L'unicité se démontre par récurrence. Soit \(g\) une fonction vers les mêmes
        équations, on a \(f(0)=a=g(0)\) et \(\forall x\in\N (f(x)=g(x))\Rightarrow
        f(s(x))=g(s(x))\) donc \(g=f\) par récurrence.

        Pour l'existence on a besoin d'une
        propriété de clôture, de travail sur les relations entre éléments de \(\N\) et
        de \(E\) c'est-à-dire des sous ensembles de \(E\times \N\). Cette propriété
        est inspirée par la définition de \(f\) pour les sous-ensembles
        \(R\subseteq \N\times E\). Cl\((R):=\left(0,a\right)\in R\land\left(
        \forall n\in\N, \forall y\in E, (n,y)\in E\Rightarrow (s(n),h(y))\in R\right)\).
        Si \(R\) est le graphe d'une application \(f\) on retrouve les équation du théorème.
        L'ensemble des \(R \subseteq \N\times E\) vérifiant Cl\((R)\) est non vide car
        \(\N\times E\) satisfait la propriété de clôture, donc on peut poser
        \(G = \bigcap_{\text{Cl}(R)} R\). Il reste à montrer que \(G\) est le graphe
        d'une application. On sait que Cl\((G)\) est vraie donc tout élément \(n\)
        possède une image par la relation de \(G\) par récurrence. La minimalité de \(G\)
        assure l'unicité de l'image.
    }
}

Le théorème précédent nous permet par exemple de définir par récurrence la fonction
\(f\colon\N\to\N\) donnée par \(f\colon x\mapsto 2x\) ainsi que plein de belles petites
choses comme la famille des fonctions d'\textsc{Ackermann} par exemple.


\bdef{Opérations sur les entiers}{
    On peut maintenant définir les opérations usuelles sur les entiers comme le théorème
    de \textsc{Dedekind} s'étend au fonction à plusieurs paramètres, on a :
    \begin{enumerate}
        \itt La somme : \(\quad\quad\quad\quad x+0=0;\quad x+s(y)=s(x+y)\)
        \itt La multiplication : \(\quad x\cdot0=0;\quad\ x\cdot s(y)=x\cdot y+x\)
        \itt La puissance : \(\quad\quad\quad x^0=1;\quad\ \ \ x^{s(y)}=x^y\cdot x\)
    \end{enumerate}
}

\end{document}