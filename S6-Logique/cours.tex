\documentclass{report}
\usepackage{../exercices}

\begin{document}

\begin{center}
    \huge{\textbf{Cours de Logique}}
\end{center}

\boxans{
    \begin{enumerate}
        \itt Sylvy Anscombe
        \itt sylvy.anscombe@inij-prog.fr
    \end{enumerate}
}

\section*{Sommaire à titre indicatif}

\boxans{
    \begin{enumerate}
        \itt Ordres linéaires
        \itt Bons ordres
        \itt Définitions inductives
        \itt Arithmétique de Péano
        \itt -----
        \itt Démonstration par récurrence
        \itt L'ordre sur \(\N\)
        \itt-----
        \itt Notion de cardinalité, équipotence
        \itt Ensemble dénombrables de \textsc{Dedekind}
        \itt Diagonale de \textsc{Kantor}
        \itt Théorème de \textsc{Kantor}-\textsc{Burnstein}- ?
        \itt-----
        \itt Lemme de \textsc{Zorn} et axiome du choix
        \itt-----
        \itt Calcul propositionnel (syntaxe, sémantique, valuations)
        \itt Formes normales (disjonctive, conjonctive) et systèmes complets
        \itt Calcul de prédicat (logique du premier ordre)
    \end{enumerate}
}

% \tableofcontents

\chapter{Les axiomes des Ordres}

\bdef{Relation d'ordre}{
    On dit que la relation \(\preceq\) est une relation d'ordre sur un ensemble \(E\) si
    elle est \begin{enumerate}
        \itt \textbf{Reflexive} \(\colon \forall x\colon x\preceq x\)
        \itt \textbf{Antisymmétrique} \(\colon \forall x\forall y\colon (x\preceq y\land y\preceq x) \Rightarrow x=y\)
        \itt \textbf{Transitive} \(\colon\forall x\forall y\forall z\colon (x\preceq y\land y\preceq z) \Rightarrow x\preceq z\)
    \end{enumerate}
}
\begin{examples}
    Les relations \(\leq, \geq\) sur les nombres, l'ordre lexicographique ...
\end{examples}

\bdef{Ensemble ordonné}{
    Un ensemble ordonné \((E, \preceq)\) est un corps \textsc{Gaussien}
    munit d'une relation d'ordre.
}
\begin{examples}
    \((\N, \leq)\), \((\Q, \leq)\), \((\R, \leq)\), \((\Z, \leq)\), \((\Sigma^*, \preceq_\alpha)\).
\end{examples}

\bdef{Ordre total}{
    Une relation d'ordre est totale (ou linéaire) sur \(E\) si
    \(\forall x\forall y \colon x\preceq y\lor y\preceq x\).
}
\begin{examples}
    \((\N, \leq)\) est un ordre total, \((\N, \mid)\) non.
\end{examples}

\bdef{Morphisme d'ordre}{
    Un morphisme d'ordre \(\varphi\) est une application de \((A,\preceq_A)\) dans
    \((B,\preceq_B)\) vérifiant \(\forall x\forall y\in A\colon (x\preceq_A y)
    \Rightarrow(\varphi(x)\preceq_B \varphi(y))\). On peu aussi parler de morphismes
    d'ordres bijectifs.
}

\brem{
    Un isomorphisme d'ordre est une bijection croissante dont la réciproque
    est croissante.
}

\bdef{Bon ordre}{
    Un ordre \(\preceq\) est dit bon si tout sous-ensemble de \((E, \preceq)\)
    admet un plus petit élément. Un bon ordre est forcément total.
}
\begin{examples}
    \((\N, \leq)\) admet un bon ordre,
    \((\N, \leq)\) n'admet pas de bon ordre
\end{examples}

\begin{proposition}
    Soit \((A, \preceq_A)\) un bon ordre et soit \(B\subseteq A\), \((B, \preceq_{A\mid B})\)
    est bien ordonné.
    \begin{proof}
        Soit \(C\subseteq B\) alors \(C\subseteq A\) donc \(C\) admet un minimum pour
        \(\preceq_A\), qui est aussi trivialement minimum pour \(\preceq_{A\mid B}\)
        donc \(B\) est bien ordonné.
    \end{proof}
\end{proposition}

\begin{exo}
    Attention à ne pas confondre isomorphisme et isomorphisme d'ordre.
    \begin{q}{1}
        Soit \(A = \{\frac{1}{n} \mid n\in\N\}\) est-il bien ordonné ?
        \boxans{Non, car \(A\) conttient une suite infinie décroissante.}
    \end{q}
    \begin{q}{2}
        Qu'en est-il de \(B = -A\) ?
        \boxans{L'ensemble \(B\) est bien ordonné, il admet un morphisme d'ordre trivial vers \(\N\).}
    \end{q}
\end{exo}

\begin{proposition}
    Un ensemble non vide avec un ordre total est bien ordonné si et seulement si il
    vérifie la propriété de récurrence bien fondée :
    \(\forall J\subseteq E\forall y(\forall z<y\colon z\in J\Rightarrow y\in J)\Rightarrow J=E\)
    qui se comprend comme : si tous les prec de y sont dans J, y aussi, c'est le concept
    de récurrence.
\end{proposition}

\begin{exo}
    On suppose \(X\subset \N\) pour lequel \(|\) est un ordre total, \(X\) est-il
    bien ordonné ?
    \boxans{\(X\) est alors uniquement constitué de produits de facteurs premiers
    dont tous les facteur croient de la même façon d'un élément à l'autre, l'ordre
    lexicographique y est donc total est bon.}
\end{exo}

\end{document}