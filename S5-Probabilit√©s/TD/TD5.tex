\documentclass{report}
\usepackage{../../exercices}

\begin{document}
\begin{center}
    \Huge{\textbf{PR5-L3 Maths - TD 5}}
\end{center}
\bigskip

\begin{exo}
    Déterminer la loi des variables aléatoires suivantes en utilisant
    la formule de transfert.
    \begin{q}{1}
        Si \(X\sim\)Cauchy, déterminer la loi de \(Y=1/X\).
        \boxans{\(X\) est de densité \(f_X(x)=\frac{1}{\pi (1+x^2)}\).
        Soit \(h\) continue bornée, on souhaite calculer \(\esp(h(Y))=\esp(h(1/X))\)
        Soit \(g\colon x\mapsto h(\frac{1}{x})\) lorsque celà a du sens.
        \[\esp[h(Y)]=\int_\R g(x)f_X(x) = \int_{\R_-} h\left(\frac{1}{x}\right)f_X(x)
        +\int_{\R_+}h\left(\frac{1}{x}\right)f_X(x)=\int_{\R}h(x)\frac{1}{x^2}f_X\left(\frac{1}{x}\right)\]
        On pose maintenant \(f_Z\colon x\mapsto \frac{1}{x^2}f_X\left(\frac{1}{x}\right)\)
        qui est bien une densité car elle est à valeurs positives et d'intégrale neutre multiplicativement.
        Considérons une variable continue \(Z\) de densité \(f_Z\) on a pour tout fonction
        continue bornée \(\esp[h(Y)]=\esp[h(Z)]\) donc \(Y\) est continue de densité \(f_Z\), par
        analyse synthèse. Dans notre cas précis, les deux variables ont la même loi.}
    \end{q}
    \begin{q}{2}
        Si \(X\sim\mathcal{N}(0,1)\), déterminer la loi de \(Y=e^X\).
        \boxans{\(X\) est de densité \(f_X(x)=\frac{1}{\sqrt{2\pi}}e^{-x^2/2}\)
        Soit \(h\) continue bornée, on souhaite calculer \(\esp([h(Y)])=\)
        \[\int_\R h(e^x)f_X(x)\D x=\int_{\R_+} h(t)f_X(\ln(t))\frac{\D t}{t}\]
        On pose maintenant \(f_Y\colon t\mapsto\frac{f_X(\ln(t))}{t}\) pour \(t>0\),
        la fonction est bien positive, d'intégrale neutre multiplicativement, c'est
        donc une densité. Ainsi \(Y\) est de densité \(f_Y\).}
    \end{q}
\end{exo}

\begin{exo}
    On définit la loi de \textsc{Laplace} de paramètres \(\mu \in\R, b>0\) comme la loi
    correspondant àa la densité \[f(x)\colon=\frac{1}{2b}\exp\left(-\frac{|x-\mu|}{b}\right)\]
    On pensera à observer que si \(X\sim\LL(\mu,b)\) et \(Y\sim\LL(0,1)\) alors
    \(X/b\) suit la loi de \(Y+\mu\), on se contentera donc de travailler avec \(X\sim\LL(0,1)\)
    \begin{q}{1}
        Calculer \(\esp[X]\).
        \boxans{\(f_X = \frac{1}{2}\exp(-|x|)\).
        La densité étant paire, l'espérance est \(0\). \(\esp[\LL(b(Y+\mu))]=b\mu\)}
    \end{q}
    \begin{q}{2}
        Calculer var\((X)\).
        \boxans{On utilise la formule \(\var(X) = E[X^2]-E[X]^2 = \int_\R x^2e^{-|x|} - 0 = 4\).
        \(\var(\LL(b(Y+\mu)))=4b^2\) }
    \end{q}
    \begin{q}{3}
        Calculer \(\esp[e^{tX}]\) pour tout \(t\in\R\).
    \end{q}
    \begin{q}{4}
        Expliciter la loi de la variable aléatoire \(\sqrt{|X-\mu|b}\).
    \end{q}
\end{exo}

\begin{exo}
    La loi du vecteur aléatoire \(X=\left(X_1,X_2\right)\) est uniforme
    sur le disque unité, \(f_X = \frac{1}{\pi}\mathds{1}_{B(0,1)}\)
    \begin{q}{1}
        Expliciter la loi de \(X_1\) et celle de \(X_2\).
        \boxans{Soit \(x\in\R\) calculons \(F_{X_1}=\proba(X_1\leq x)=\proba(X\in
        (-\infty, x]\times\R)\), ainsi \(F_{X-1}\) a une allure exponentielle
        entre \(-1\) et \(0\) puis \(1-\exp\) entre \(0\) et \(-1\)}
    \end{q}
    \begin{q}{2}
        Calculer \(\cov(X_1, X_2)\).
    \end{q}
    \begin{q}{3}
        Montrer que \(X_1\) et \(X_2\) ne sont pas indépendantes.
    \end{q}
\end{exo}

\begin{exo}
    \(X\) et \(Y\) sont deux variables indépendantes \(X\sim\text{Bar}(1/2)\) et
    \(Y\sim U(0,1)\). Donner la loi de \(X+Y\).
\end{exo}

\begin{exo}
    \(X\) et \(Y\) sont deux variables indépendantes et \(X\sim Y\sim \mathcal{N}(0,1)\).
    Donner la loi de \(X/Y\).
\end{exo}

\begin{exo}
    \(X\) et \(Y\) sont deux variables indépendates et \(X\sim Y\sim U(0,a), a>0\).
    Donner la loi de
    \begin{enumerate}
        \itt \(X+Y\)
        \boxans{
            On a les densités marginales :
            \[ f_X(x) = f_Y(x) = \begin{cases}
                            \frac{1}{a}, & \text{si } 0 \leq x \leq a, \\
                            0, & \text{sinon.}
                        \end{cases}
            \]

            La convolution est donnée par :
            \[ f_{X+Y}(z) = \int_{-\infty}^{\infty} f_X(x) \cdot f_Y(z-x) \D x \]

            Pour \(0 \leq z \leq a\), l'intervalle d'intégration est \([0, z]\) et
            \( f_{X+Y}(z) = \frac{1}{a^2} \int_{0}^{z} \D x = \frac{z}{a^2}\)

            Pour \(a \leq z \leq 2a\), l'intervalle d'intégration est \([z-a, a]\) et
            \( f_{X+Y}(z) = \frac{1}{a^2} \int_{z-a}^{a} \D x = \frac{1}{a^2} \cdot (2a - z) \)
        }
        \itt \(X-Y\)
        \boxans{
            On se contentera de remarquer que \(X-Y\sim X+Y-a\).
        }
        \itt \(XY\)
        \boxans{
            La convolution est donnée par :
            \[f_{XY}(z)=\int_{-\infty}^{\infty} f_X\left(\frac{z}{y}\right)\cdot f_Y(y)\frac{1}{|y|}\D y\]
        }
        \itt \(X/Y\)
        \boxans{
            La convolution est donnée par :
            \[f_{X/Y}(z)=\int_{-\infty}^{\infty} f_X(zy) \cdot f_Y(y)|y|\D y\]
        }
    \end{enumerate}
\end{exo}

\begin{exo}
    \(X\) et \(Y\) sont deux variables indépendantes et \(X\sim Y\sim \text{Exp}(a), a>0\).
    Donner la loi de
    \begin{enumerate}
        \itt \(X+Y\)
        \itt \(X-Y\)
        \itt \(|X-Y|\)
        \itt \(X/Y\)
        \itt \(X/(X+Y)\)
    \end{enumerate}
    Montrer aussi que \(\max(X,Y)\sim X+ Y/2\).
\end{exo}

\begin{exo}
    Soient \(X\) et \(X\) deux variables aléatoires de loi Exp\((1)\) et
    \(Z = (X-Y)/(X+Y)\).
    \begin{q}{1}
        Donner la fonction de répartition de \(Z\) et sa densité si elle existe.
    \end{q}
    \begin{q}{2}
        Donner la loi du vecteur \((X,Z)\). Vérifier ensuite le calcul de la loi
        de \(Z\) en calculant la densité marginale.
    \end{q}
\end{exo}

\begin{exo}
    Soit \(a>0\) et \(\lambda>0\) on appelle \textit{Loi de Pareto} de paramètres \(a\)
    et \(\lambda\) la loi de densité donnée par \(f(x)=\frac{a\lambda^a}{x^{a+1}}
    \mathds{1}_{x\geq\lambda}\) pour \(x\in\R\). Soient \(X\) et \(Y\) deux variables
    aléatoires indépendantes Pareto(1,1). On pose \(S=XY\) et \(T=X/Y\).
    \begin{q}{1}
        Calculer la densité du couple \((S,T)\).
    \end{q}
    \begin{q}{2}
        Calculer les densités de \(S\) et \(T\)
    \end{q}
    \begin{q}{3}
        \(S\) et \(T\) sont-elle indépendantes ?
    \end{q}
\end{exo}

\begin{exo}
    Soient \(X, Y\) deux variables gaussiennes normales centrées réduites indépendantes.
    Soit \(\varphi\) la bijection qui à tout point de \(U=\R^2\backslash\{(x,0)\mid x\leq 0\}\)
    associe ses coordonnées polaires \((r, \theta)\) alors \(\varphi^{-1}\) est
    continuellement différentiable. On définit le couple aléatoire \((R,\Theta) = \varphi{-1}(X,Y)\).
    Donner la densité du vecteur aléatoire \((R,\Theta)\).
\end{exo}
\end{document}