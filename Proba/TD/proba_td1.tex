\documentclass{report}
\usepackage[T1]{fontenc}
\usepackage[utf8x]{inputenc}
\usepackage{lmodern}
\usepackage{amsmath,amsthm,amsfonts,amssymb}
\usepackage{graphicx}
\usepackage[shortlabels]{enumitem}
\usepackage{xcolor}
\usepackage{accents}
\usepackage{titlesec}
\usepackage{etoolbox}
\usepackage{bookmark}
\usepackage{stmaryrd}
\usepackage{fancyhdr}
\usepackage[margin=25mm]{geometry}
\usepackage[french]{babel}
\usepackage{changepage}
\usepackage{listings}
\usepackage[most,breakable,listings]{tcolorbox}

\date{\today}
\author{Corentin Sallin \thanks{Cours de L3}}
\title{Calcul Différentiel et topologie}


\renewcommand{\theenumi}{\Alph{enumi}}

\newcommand{\N}{\mathbb{N}}
\newcommand{\Z}{\mathbb{Z}}
\newcommand{\R}{\mathbb{R}}
\newcommand{\Q}{\mathbb{Q}}
\newcommand{\C}{\mathbb{C}}
\newcommand{\K}{\mathbb{K}}
\renewcommand{\L}{\mathcal{L}}
\newcommand{\M}{\mathcal{M}}
\newcommand{\T}{\mathcal{T}}
\newcommand{\Diag}{\mathcal{D}}
\newcommand{\curlyv}{\mathcal{V}}

\newcommand{\esp}{\mathbb{E}}
\newcommand{\proba}{\mathbb{P}}


\newcommand{\CC}{\mathcal{C}}
\newcommand{\Int}{\mathrm{Int}}
\newcommand{\id}{\mathrm{id}}
\newcommand{\eps}{\varepsilon}
\newcommand{\mnr}{\mathcal{M}_n(\R)}
\newcommand{\mnk}{\mathcal{M}_n(\K)}
\newcommand{\glnr}{\mathrm{GL}_n(\R)}
\newcommand{\glnk}{\mathrm{GL}_n(\K)}

\newcommand{\D}{\mathop{}\!\mathrm{d}}
\newcommand{\ds}{\displaystyle}
\newcommand*{\ensemble}[3][]{#1\{ #2 \mid #3 #1\}}
\newcommand{\Vvert}{\vert\kern-0.25ex\vert\kern-0.25ex\vert}
\newcommand{\tnorm}[1]{\Vvert #1 \Vvert}

\DeclareMathOperator{\Ker}{Ker}
\DeclareMathOperator{\Isom}{Isom}
\DeclareMathOperator{\Tr}{Tr}

\pagestyle{fancy}
\fancyhf{}
\fancyhead[L]{Corentin Sallin}
\fancyhead[R]{}
\fancyfoot[L]{\leftmark}
\fancyfoot[R]{Page : \thepage}

\renewcommand{\headrulewidth}{2pt}
\renewcommand{\footrulewidth}{1pt}


\theoremstyle{definition}
\newtheorem{exo}{Exercice}
%\newtheorem{q}{\(\quad\)}[exo]

\newtheorem{theorem}{Théorème}[section]
\newtheorem{lemma}[theorem]{Lemme}
\newtheorem{definition}[theorem]{Définition}
\newtheorem{lemme}[theorem]{Lemme}
\newtheorem*{corollary}{Corollaire}

\newtheorem{proposition}{Proposition}[theorem]
\newtheorem{rk}[proposition]{Remarque}
\newtheorem*{example}{Exemple}

\definecolor{main1white} {RGB}{121, 129, 134}
\definecolor{main1white2} {RGB}{204, 204, 204}

\newenvironment{q}[1]{
    \begin{adjustwidth}{1cm}{}
    \textbf{#1} : }{
    \end{adjustwidth}}

\titleformat{\chapter}[display]
    {\normalfont\bfseries}{}{0pt}{\Huge}

\patchcmd{\chapter}{\thispagestyle{plain}}{\thispagestyle{fancy}}{}{}

\newcommand{\boxans}[1]{
    \begin{tcolorbox}[
            breakable,
            enhanced,
            interior style      = {
                left color      = main1white2!65!gray!8,
                middle color    = main1white2!50!gray!7,
                right color     = main1white2!35!gray!6
            },
            %borderline north    = {.3pt}{0pt}{main1white!10},
            %borderline south    = {.3pt}{0pt}{main1white!10},
            frame hidden,
            borderline west     = {2pt}{0pt}{main1white!30},
            sharp corners       = downhill,
            arc                 = 0 cm,
            boxrule             = 0 cm,
            %nobeforeafter,
            %before={},
            %nobeforeSTYLE,
            %noafterSTYLE,
            %after=\par\nointerlineskip
            %source=remy
        ]
        #1
    \end{tcolorbox}
}

\begin{document}

\begin{center}
    \Huge{\textbf{PR5-Feuille de TD 1.}}
\end{center}
\bigskip
mails des teachers : daures@lspm.paris et le plus important collin@lspm.paris

\begin{exo}
    On lance un dé rouge et un dé vert. Donner explicitement un espace probabilisé pour étudier cette experience aléatoire et l'utiliser pour répondre aux questions
    suivantes :
    \boxans{On utilisera comme espace probabilisé \(\Z_6 \times \Z_6\) avec chaque élément d'une probabilité répondant à une loi uniforme.}
    \begin{q}{(1)}
        Quelle est la probabilité que le nombre sur le dé rouge soit plus strictement plus grand que le nombre sur le dé vert?
        \boxans{On a 36 tirages possibles, parmi lesquels 6 où les tirage des dés sont égaux. Parmis les 30 tirages restants,
        pour la moitié d'entre eux le tirage du dé rouge est strictement supérieur au tirage du dé vert, pour l'autre moitié c'est l'inverse, la probabilité
        recherchée est alors \(\frac{15}{36}\)}
    \end{q}
    \begin{q}{(2)}
        Quelle est la probabilité que les deux nombres diffèrent de au plus 1 ?
        \boxans{On a 6 tirages possibles avec aucune différence, puis 5 tirages possibles avec le dé rouge supérieur au vert, et 5 dans l'autre sens.
        Ce qui donne en tout une probabilité de \(\frac{16}{36}\)}
    \end{q}
    \begin{q}{(3)}
        Quelle est la probabilité que le maximum des deux nombres soit supérieur ou égal à 5 ?
        \boxans{On a \(4^2\) façons distinctes d'avoir un maximum strictement inférieur à 5,
        ainsi la probabilité d'avoir un maximum supérieur à 5 est \(\frac{20}{36}\)}
    \end{q}
\end{exo}

\begin{exo}
    Si on lance une pièce 5 fois, quelle est la probabilité que le nombre de faces
    soit pair ? On écrira explicitement l'espace probabilisé associé à cette experience.

    \boxans{L'espace probabilisé est défini ici par \(\Omega=\{0,1\}^2\) avec \(\proba=\mathcal{U}(\Omega)\) et la tribu \(\mathcal{F}=\mathcal{P}(\Omega)\).
    On remarque ensuite que la probabilité d'avoir un nombre pair de face est par définition la même que celle d'avoir un nombre impair de piles, or ces
    deux évènements sont équiprobables ce qui donne que la probabilité recherchée est \(\frac12\)}
\end{exo}

\begin{exo}
    Posons \(\Omega=\{a,b,c\}\) muni de la tribu de l'ensemble de ses parties. On donne \(\proba(\{a,b\}) = 0,7\) et \(\proba(\{b,c\}) = 0,6\). Calculer
    les probabilités de \(\{a\}, \{b\}\) et \(\{c\}\).
    \boxans{La relation \(\proba(A)=1-\proba(\bar{A})\) on a \(\proba(\{a\}) = 1-\proba(\{b,c\})= 0,4\) et de la même façon \(\proba(\{c\})=0,3\) finalement
    le système complet d'évènements permet d'obtenir \(proba(\{b\}) = 1 - \proba(\{a\}) - \proba(\{c\}) = 0,3\).}
\end{exo}

\begin{exo}
    Supposons que \(A\) et \(B\) soient deux évènements disjoints avec \(\proba(A)=0,4\) et \(\proba(B)=0,5\) que vaut \(\proba(\bar{A}\cap\bar{B})\) ?
    \boxans{Les deux évènements étant disjoints on a, en passant au complémentaire, \(\proba(\bar{A}\cap\bar{B}) = 1- \proba(A\cup B)=1-\proba(A)-\proba(B)=0,1\)}
\end{exo}

\begin{exo}
    Soit \(\left(\Omega, \mathcal{A}, \proba\right)\) un espace probabilisé. Soit \(A,\ B\) des évènements tels que \(\proba(A)=\frac34\) et \(\proba(B)=\frac13\).
    \begin{q}{(1)}
        Montrer que \(\frac{1}{12} \leq \proba(A \cap B) \leq \frac13\)
        \boxans{On a \(\proba(A)+\proba(B) = 1 + \frac{1}{12}\) or \(\proba(A\cup B)\) doit être au plus \(1\) ce qui force \(\proba(A \cap B) \geq \frac{1}{12}.\)
        Pour ce qui est de l'autre borne on constate que \((A\cap B) \subset B\) donc \(\proba(A\cap B)\leq \proba(B) \leq \frac13\)}
    \end{q}
    \begin{q}{(2)}
        Soit \(\Omega = {1,...,12}\) muni de la tribu de l'ensemble de ses parties et de la probabilité uniforme. Donner des exemples de tels \(A\) et \(B\)
        et tels que les bornes ci-dessus soient atteintes.
        \boxans{
            Dans les deux cas l'expérience consiste à tirer un élément au hasard.
            \begin{enumerate}
                \item[Borne inf] \(A\) : "L'élément vaut au plus 9" et \(B\) : "L'élément vaut au moins 9"
                \item[Borne sup] \(A\) : "L'élément vaut au plus 9" et \(B\) : "L'élément vaut au plus 4"
            \end{enumerate}
        }
    \end{q}
    \begin{q}{(3)}
        Donner les bornes correspondantes pour \(\proba(A\cup B)\)
        \boxans{En utilisant l'exemple au dessus, ou un raisonnement semblable à celui de la première question, on trouve \(\frac34\leq\proba(A\cup B)\leq 1\)}
    \end{q}
\end{exo}

\begin{exo}
    Soit \(\Omega = \{1,2,3,4,5,6\}\) on définit des ensembles de parties
    \begin{enumerate}
        \item[(1)] : \(\mathcal{A}_1 = \{\{1,2\}, \{3,4,5,6\}, \{3,4\}, \{1,2,5,6\}, \Omega, \emptyset\}\)
        \item[(2)] : \(\mathcal{A}_2 = \{\{1,2,3\}, \{4,5,6\}, \Omega, \emptyset\}\)
    \end{enumerate}
    \begin{q}{(1)}
        \(\mathcal{A}_1\) et \(\mathcal{A}_2\) sont-elles des tribus ? Justifier votre réponse.
        \boxans{\(\mathcal{A}_1\) n'est pas stable par union finie, en effet \(\{3,4\}\cup\{1,2\} = \{1,2,3,4\}\not\in\mathcal{A_1}\).
        D'autre part on a bien \(\emptyset\in \mathcal{A}_2\) ainsi que la stabilité évidente par réunion et passage au complémentaire, on a ainsi une tribu.}
    \end{q}
    \begin{q}{(2)}
        On définit l'application \(\proba:\mathcal{A}_2\rightarrow [0,1]\) par \(\proba(\{1,2,3\})=\frac14\), \(\proba(\{4,5,6\})=\frac23\),
        \(\proba(\emptyset)=0\) et \(\proba(\Omega)=1\). L'application \(\proba\) est elle une probabilité ?
        \boxans{La \(\sigma\)-additivité voudrait \(\proba(\{1,2,3\})+\proba(\{4,5,6\})=1\) ce qui n'est pas le cas, \(\proba\) n'est donc pas une probabilité.}
    \end{q}
    \begin{q}{(3)}
        Donner un ensemble de parties différent de l'ensemble des parties de \(\Omega\) et des exemples ci-dessus et qui soit une tribu de \(\Omega\).
        Définir ensuite une probabilité sur la tribu.
        \boxans{On pose par exemple \(\mathcal{A}_3 = \mathcal{A}_1 \cup \{1,2,3,4\} \cup \{5,6\}\) avec
        \(\proba : A \mapsto \frac{|A|}{6}\) pour \(A\in\mathcal{A}_3\)}
    \end{q}
\end{exo}

\begin{exo}
    TODO
\end{exo}

\begin{exo}
    Soit \(\left(\mathcal{F}_\alpha\right)_{\alpha\in A}\) une famille quelconque de tribus sur un ensemble \(\Omega\).
    Montrer que \(\mathcal{H}=\cap_{\alpha\in A}\mathcal{F}_\alpha\) est aussi une tribu.
    \boxans{
        \begin{enumerate}
            \item Les \(\mathcal{F}_\alpha\) étant des tribus on a \(\emptyset\in\mathcal{F}_\alpha\) pour tout \(\alpha\in A\) donc \(\emptyset\in\mathcal{H}\)
            \item Soit \(H\in\mathcal{H}\) par définition \(H\in\mathcal{F_\alpha}\) pour tout \(\alpha\in A\) donc \(\bar{H}\in \mathcal{F}_\alpha\Rightarrow \bar{H}\in\mathcal{H}\)
            \item Soit \(\left(H_i\right)_{i\in\N}\) suite d'éléments de \(\mathcal{H}\), ils sont dans tous les \(\mathcal{F}_\alpha\) ainsi leur union aussi et elle est donc dans \(\mathcal{H}\).
        \end{enumerate}
    }
\end{exo}

\end{document}