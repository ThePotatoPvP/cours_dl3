\documentclass{report}
\usepackage{../../exercices}

\begin{document}
\begin{center}
    \Huge{\textbf{PR5-L3 Maths - TD 2}}
\end{center}
\bigskip

\begin{exo}
    \(\left(\Omega, \mathcal{F},P\right)\) est un espace probabilisé et on choisit
    \(B\in\mathcal{F}\) tel que \(P(B)>0\). Montrer que l'application \(A\mapsto
    P(A\mid B)\) est aussi une probabilité sur l'espace mesurable \(\left(\Omega,
    \mathcal{F}\right)\)
    \boxans{L'application est évidemment positive et \(\sigma\)-additive, d'autre
    par \(B\subset\Omega\) donc par croissance de \(P\), \(P(\Omega\mid B)=1\).
    L'application définie est finalement bien une probabilité.}
\end{exo}

\begin{exo}
    Montrer que \(P(A\mid B)= \frac{P(B\mid A)P(A)}{P(B)}\) si \(P(A)P(B)>0\).
    \boxans{Utilisons un raisonnement combinatoire, pour piocher un élément dans
    \(A\cap B\) on peut d'abord vérifier si l'élément piocher est dans \(A\),
    ce qui donne une probabilité de \(P(A)P(B\mid A)\), ou bien l'inverse qui
    donne la probabilité \(P(B)P(A\mid B)\). L'évènement étant le même, la probabilité
    doit l'être aussi, donc \(P(B)P(A\mid B)=P(A)P(B\mid A)\)}
\end{exo}

\begin{exo}
    Montrer que si on a des évènements \(\left(A_i\right)_{i\in\leq n}\) tels
    que \(P(A_1\cap...\cap A_{n-1})>0\) on a aussi :
    \[P\left(\cap^n A_i\right) = P(A_1)P(A_2\mid A_1)\dots
    P(A_n \mid A_1\cap\dots\cap A_{n-1})\]
    \boxans{On utilise ici encore un raisonnement combinatoire, la probabilité qu'un
    élément tiré au hasard soit dans l'intersection des \(A_n\) est la même que celle qu'il soit
    dans chacun des \(A_n\) par défintion de l'intersection, en les vérifiant
    dans l'ordre on a la formule de l'énoncé.}
\end{exo}

\begin{exo}
    \textit{Deux exercices de base sur l'indépendance}
    \begin{q}{1}
        Nous lançons deux pièces équilibrées : \(A\) = \{première pièce est pile\},
        \(B\) = \{deuxième pièce est pile\},\(C\)= \{les deux pièces ont donné le même
        issu\}. Montrer que \(A\) et \(B\) sont indépendants. Pareil pour \(A\) et
        \(C\), et pour \(B\) et \(C\). Peut on dire que \(A\), \(B\) et \(C\) sont
        indépendants ? Et si les deux pièces sont indentiques mais biaisées ?
        \boxans{On se convainc à l'aide d'un arbre de probabilités par exemple que
        \(P(A)=\frac12=P(B)\) et que \(P(A\cap B)=\frac14\) ce qui donne bien l'indépendance
        des deux évènements. Aussi \(P(C)=\frac12\) en regardant la liste des issues possibles
        et en constatant qu'elles sont équiprobables.
        On a alors \(P(A\cap C)=P(A)\times P(C)\) et avec un raisonnement similaire
        l'indépendance entre \(B\) et \(C\). On ne peut par contre pas dire que les
        trois évènements sont indépendants pour autant car toute pair d'évènement
        implique le troisième. Les raisonnement se faisant uniquement avec la probabilité
        de faire pile, si les deux pièces sont identiques mais biaisées on obtient le même résultat.}
    \end{q}
    \begin{q}{2}
        Nous lançons deux dés équilibrés. Montrer que les évènements \{la somme
        des dés vaut 7\} et \{l'issu du premier dé est j\} sont indépendants.
        \boxans{Il y a \(36\) issus possibles parmi lesquelles \(6\) donnent une somme
        égale à \(7\), ainsi \(P(A)=\frac16\). Les dés étant équilibrés,
        \(P(\delta_1=j)=\frac16\). Les deux dés étant indépendants, on a
        \(P(A\cap \delta_1=j)=P(\delta_1=j\cap \delta_2=7-j)=P(\delta_1=j)P(\delta_2=7-j)=\frac{1}{36}\)
        les deux évènements sont donc bien indépendants.}
    \end{q}
\end{exo}

\begin{exo}
    \textit{Deux questions sur l'indépendance}
    \begin{q}{1}
        Soit \(A\) un évènement indépendant de lui même, que vaut \(P(A)\) ?
        \boxans{Soit \(A\) indépendant de lui même \(P(A)=P(A\cap A)=P(A)^2\) donc
        \(P(A)\in\{0,1\}\)}
    \end{q}
    \begin{q}{2}
        Montrer que \(A_1,A_2,A_3\) sont indépendants si et seulement si
        \(\sigma(A_1), \sigma(A_2)\) et \(\sigma(A_3)\) le sont. Nous rappelons
        que les tribus \(\mathcal{F}_1,\dots,\mathcal{F}_n\) sont indépendantes si
        pour tout choix de \(A_j\) dans \(\mathcal{F}_j\) on a \(P(\cap A_j)=
        \prod P(A_j)\).
    \end{q}
\end{exo}

\begin{exo}
    Soit \(\left(A_n\right)_{n\in\N}\) une suite d'évènements de probabilité
    nulle sur un espace de probabilités. Soit \(A\) un évènement de probabilité
    \(\frac13\) sur le même espace. Montrer que \(A\cup (\cup A_i)\) est un évènement
    et calculer sa probabilité.
    \boxans{Les \(A_i\) étant tous de probabilité nulle on a d'après la formule de \textsc{Poincarré} \(P(A\cup (\cup A_i))
    =P(A)+0-0=\frac13\)}
\end{exo}

\begin{exo}
    Soit \(F\) une fonction de répartition. Montrer que \(F\) peut avoir une infinité
    de points de discontinuité mais que le nombre de ces points est au plus dénombrable.
    \boxans{La fonction de répartition d'une loi discrète a des discontinuités en chaque
    évènement, ainsi la loi de répartition d'une loi de \textsc{Poisson} par exemple
    possède une infinité de points de discontinuité. Cependant une fonction de répartition
    a au plus un nombre dénombrable de discontinuités, n effet supposons \(f\) ayant
    un intervalle de discontinuité, alors \(f\) n'est pas continue à droite sur cet
    intervalle et n'est donc pas une fonction de répartition.}
\end{exo}

\begin{exo}
    Soit \(F\) la fonction de répartition d'une loi sur \(\R\) donnée par
    \[F(x)=\frac14\mathds{1}_{\R^+}(x)+
            \frac12\mathds{1}_{[1,+\infty)}(x)+
            \frac14\mathds{1}_{[2,+\infty)}\]
    Trouver la probabilité des ensembles suivants:
    \begin{enumerate}
        \itt \( (-\frac12,\frac12) \rightarrow F(\frac12-)-F(-\frac12) = F(\frac12-) = \frac14\)
        \itt \( (-2,\frac32) \rightarrow F(\frac32-) - F(-2) = F(\frac32-) = \frac34\)
        \itt \((\frac23,\frac52) \rightarrow F(\frac52-)-F(\frac23)= 1 - \frac14 = \frac34\)
        \itt \([0,2) \rightarrow F(2-)-F(0) = \frac34 - 0 = \frac34\)
        \itt \((3,+\infty) \rightarrow 1 - F(3) = 1-1=0\)
    \end{enumerate}
\end{exo}

\begin{exo}
    Soit \(F\) la fonction définie pour tout \(x\in\R\) par
    \[F(x)=\sum_{i=1}^\infty \frac{1}{2^i}\mathds{1}_{[\frac{1}{i},+\infty)}(x)\]
    Montrer qu'il s'agit d'une fonction de répartition d'une loi sur \(\R\) puis
    trouver la probabilité des ensembles suivants :
    \boxans{La fonction est trivialement croissante, continue par morceaux, admet
    une limite nulle en \(-\infty\) et \(1\) en \(+\infty\) c'est donc bien une
    fonction de répartition.}
    \begin{enumerate}
        \itt \([1,+\infty) \rightarrow 1-F(1-) = \frac12\)
        \itt \([\frac{1}{10},+\infty)\rightarrow 1 - F(\frac{1}{10}-)= 1- \frac{1}{2^{10}}\)
        \itt \(\left\{ 0 \right\}\rightarrow F(0)-F(0-)=0-0=0\)
        \itt \([0,\frac12)\rightarrow F(\frac12-) - F(0-)= \frac14 - 0 = \frac14\)
        \itt \(\R_-^*\rightarrow 1\)
        \itt \(\R_+^*\rightarrow 0\)
    \end{enumerate}
\end{exo}

\begin{exo}
    Soit \(F\) la fonction définie pour tout \(x\in\R\) par
    \[F(x)=\frac{\min(x,1)}{3}\mathds{1}_{\R_+}
    + \sum_{i=0}^\infty \frac{1}{3*2^i}\mathds{1}_{[i,+\infty)}(x)\]
    \begin{q}{1}
        Montrer qu'il s'agit de la fonction de répartition d'une probabilité.
        \boxans{On commence par observer que \(F(x)=0\) pour \(x\in\R_-\) donc
        \(\lim_{-\infty}F(x)=0\). D'autre part la somme tend vers \(\frac23\) pour \(x\to\infty\)
        ce qui donne \(\lim_{+\infty}=1\). La fraction est croissante sur \([0,1]\) puis
        constante. La somme est croissante sur \(\R\) ainsi \(F\) est bien croissante.
        La continuité par morceaux s'obtient que la somme  d'indicatrice est dénombrable.}
    \end{q}
    \begin{q}{2}
        Trouver la probabilité des évènements suivants :
        \begin{enumerate}
            \itt \([1,+\infty)\rightarrow1-F(1-)=1-\frac23=\frac13\)
            \itt \((0,1)\rightarrow F(1-)-F(0)=\frac23-\frac13=\frac13\)
            \itt \(\{0\}\rightarrow F(0)-F(0-)=\frac13-0=\frac13\)
            \itt \(\{1,2,3\}\rightarrow F(3)-F(3-)+F(2)-F(2-)+F(1)-F(1-)=
            F(3)-F(1-)=\frac{11}{12} - \frac23 =\frac{1}{4}\)
            \itt \([n,+\infty)\) pour \(n=0,1,2\dots\)
            \itt \(\R_-^*\rightarrow 0\) car \(F\) est nulle.
            \itt \(\{1,\dots,n\}\) pour \(n=1,2,\dots\)
            \itt \(\N\rightarrow 1\)
        \end{enumerate}
    \end{q}
\end{exo}
\end{document}