\documentclass{report}
\usepackage{../exercices}

\lstset{language=OCaml}

\begin{document}

\begin{center}
    \huge{\textbf{Examen 11 janvier 2024}}
\end{center}

Les fonction demandées doivent êtres rédigées en style fonctionnel pur : ni réferences,
ni tableaux, ni boucles \textbf{for} ou \textbf{while}, ni champs mutables. chaque question
ci-dessous peut utiliser les fonctions prédéfinies et/ou les fonciton des questions
précédentes. À titre indicatif, toutes les ofnctions demandées peuvent s'écrire en
moins de 10 lignes.

\begin{exo}[Expressions, types, valeurs]
    Pour chacune des expressions suivantes, indiquer si OCaml l'accepte et donnez dans
    ce cas le \textit{type} et la \textit{valeur} calculés par OCaml. Si une valeur fonctionelle
    est présente dans le résultat, la noter <fun> sans détailler plus. Si OCaml
    signale une erreur, la décrire succintement.
    \begin{q}{1}
        \camlline{(2, 3, (let x = 2*2 in x), 2+3*4+5*2);;}
        \boxans{\camlline{> : int * int * int * int = (2, 3, 4, 24)}}
    \end{q}
    \begin{q}{2}
        \camlline{let _ = 3*3 in x;;}
        \boxans{\camlline{> Error: Unbound value x}}
    \end{q}
    \begin{q}{3}
        \camlline{try String.sub "exam" 0 5 with Not_found -> "pas trouve";;}
        \boxans{\camlline{> Exception: Invalid_argument "String.sub / Bytes.sub".}}
    \end{q}
    \begin{q}{4}
        \camlline{List.find (fun x -> x mod 3 = 0) [3;33;333;3333];;}
        \boxans{\camlline{> : int = 3}}
    \end{q}
    \begin{q}{5}
        \camlline{(List.init 6 (fun x -> x * 2))@[12;14];;}
        \boxans{\camlline{> : int list = [0; 2; 4; 6; 8; 10; 12; 14]}}
    \end{q}
    \begin{q}{6}
        \camlline{List.map (fun x -> not x) [true;false;false;true];;}
        \boxans{\camlline{> : bool list = [false; true; true; false]}}
    \end{q}
    \begin{q}{7}
        \camlline{List.find (fun x -> if (x mod 3) = 0 then true) [66;666;6];;}
        \boxans{\camlline{> Error: This variant expression is expected to have type unit
        because it is in the result of a conditional with no else branch}}
    \end{q}
    \begin{q}{8}
        \camlline{(fun f -> f true false) (fun a b -> not (a && b));;}
        \boxans{\camlline{> : bool = true}}
    \end{q}
    \begin{q}{9}
        \camlline{(fun f -> f true) (fun a b -> not (a && b)) false;;}
        \boxans{\camlline{> : bool = true}}
    \end{q}
    \begin{q}{10}
        \camlline{(fun f -> f true) (fun a b -> not (a && b));;}
        \boxans{\camlline{> : bool -> bool = <fun>}}
    \end{q}
\end{exo}
\clearpage

\begin{exo}[Récursivité terminale]
    Voici un programme OCaml.
    \begin{lstlisting}
type 'a option = None | Some of 'a

let rec max_liste l = match l with
    | [] -> None
    | h::r -> match max_liste r with
            | None -> Some h
            | Some m -> Some (max h m)
;;\end{lstlisting}

    \begin{q}{1}
        Quel est le type de la fonction \camlline{max_liste} ?
        \boxans{La fonction \camlline{max_liste} est du type : \camlline{int list -> int option}.}
    \end{q}
    \begin{q}{2}
        Donner la résultat obtenu par la fonction \camlline{max_liste} sur la liste [4;2;8;5],
        puis expliquer en une phrase la procédure générale employée par la fonction
        \camlline{max\_liste} pour parvenir à son résultat final.
    \end{q}
    \begin{q}{3}
        POurquoi la fonction \camlline{max_liste} n'est pas récursive terminale ? Justifier.
    \end{q}
    \begin{q}{4}
        Écrire une fonction \camlline{max_liste_opt}, fournissant les mêmes résultats que
        \camlline{max_liste}, tout en étant exclusivement composée d'appels récursifs terminaux.
    \end{q}
    \begin{q}{5}
        Écrire une fonction \camlline{flatten_opt: 'a list list -> 'a list} ayant le
        même comportement que \camlline{List.flatten} tout en étant exclusivement composée
        d'appels récursifs terminaux.
    \end{q}
\end{exo}

\begin{exo}[Arbres]
    On considère des arbres binaires non vides dont tous les noeuds portent une
    étiquette. Les feuilles sont spécifiées par le constructeur \camlline{F} et le
    noeuds interne sont spécifiés par le constructeur \camlline{I}. Le type Ocaml qu'on
    utilisera pour spécifier ces arbres est le suivant :
    \begin{lstlisting}
type 'a btree =
    | F of 'a
    | I of 'a * 'a btree * 'a btree\end{lstlisting}
    Dans cet exercice nous spécialiserons le type polymorphe au dessus au cas où
    les étiquettes sont des valeurs de type \camlline{string}. Voici trois arbres
    différents :

    Et voici leur représentation en Ocaml (de gauche à droite) :
    \begin{enumerate}
        \itt \camlline{let ab1 = I ("a", F "a", F"a")}
        \itt \camlline{let ab2 = I ("", F "0", I ("1", F "01", F "11"))}
        \itt \camlline{let ab3 = I ("", I ("0", F "00", F "10"), I ("1", F "01", F "11"))}
    \end{enumerate}
    \begin{q}{1}
        Écrire une fonction \camlline{liste_feuilles: 'a btree -> 'a list}, qui renvoie
        la liste d'étiquettes de toutes les \textit{feuilles} d'un arbre. Par exemple
        \camlline{liste_feuilles ab2 = ["0"; "01"; "11"]}.
        \begin{lstlisting}
let rec liste_feuilles t = match t with
    | F a -> [a]
    | I (_, b, c) -> (liste_feuilles b)@(liste_feuilles c)
;;      \end{lstlisting}
    \end{q}
    \begin{q}{2}
        Un arbre est dit \textbf{parfait} lorsque toutes ses feuilles sont au même
        niveau. Écrire une fonction \camlline{cons_parfait: int -> 'a -> 'a btree},
        telle que \camlline{cons_parfait n e} construise un arbre parfait de
        hateur \camlline{n} où tous les noeuds portent la même étiquette \camlline{e}.
        Ainsi par exemple \camlline{cons_parfait 1 "a"} renvoie l'arbre \camlline{ab1}.
        \begin{lstlisting}
let rec cons_parfait n e = match n with
    | 0 -> F e
    | _ -> I(e, cons_parfait (n-1) e, cons_parfait (n-1) e)
;;      \end{lstlisting}
    \end{q}
    \begin{q}{3}
        Écrire une fonction \camlline{eten: 'a btree -> 'a -> int -> 'a btree} telle
        qu'un appel à \camlline{eten a e n} étend l'arbre \camlline{a} en remplaçant
        \textit{chacune de ses feuilles} étiquetées par \camlline{e} par un nouvel
        arbre parfait de niveau \camlline{n}, dont tous les noeuds auront la même
        étiquette \camlline{e}.
        \begin{lstlisting}
let rec eten a e n = match a with
    | F m when m = e -> cons_parfait n e
    | F m -> F m
    | I(b, c, d) -> I(b, eten c e n, eten d e n)
;;      \end{lstlisting}
    \end{q}
    \begin{q}{4}
        On s'intéresse maintenant aux arbres dont les étiquettes sont des \textbf{chaînes
        de caractères} ne contenant que 0 et 1. On veut modéliser un arbre \camlline{a}
        dont l'étiquette de chaque noeud encode sa \textbf{position relative} dans
        \camlline{a}, c'est à dire le chemion qu'il faut suivre depuis la racine de \camlline{a}
        pour atteindre ce noeud, ce chemin étant lu \textit{de la droite vers la gauche}. Ainsi,
        l'étiquette de chaque noeud est une chaîne de caractères commençant par 0 ou 1,
        suivi d'une chaîne de caractères représentant l'étiquette de son parent. Par exemple
        si la racine de l'arbre \camlline{a} est étiquetée "01", alors le noeuds étiqueté
        "1001", indique que, depuis la racine, il faut d'abord se diriger vers le fils à gauche
        de la racine, et puis ensuite vers le fils à droite de ce dernier. Plus formellement :
        \begin{enumerate}
            \item[(F)] Un arbre \camlline{a} dont sa racine est une feuille vérifie la propriété \textbf{BE}.
            \item[(N)] Un arbre \camlline{a} dont sa racine est un noeud interne étiqueté par
            \camlline{e} vérifie la propriété \textbf{BE} si
            \begin{enumerate}
                \itt le sous-arbre gauche G de \camlline{a} vérifie la propriété \textbf{BE} et
                l'étiquette de la racine de G est obtenue en concaténant "0" à gauche de
                l'étiquette \camlline{e}
                \itt le sous-arbre droit D de \camlline{a} vérifie la propriété \textbf{BE} et
                l'étiquette de la racine de D est obtenue en concaténant "1" à gauche de
                l'étiquette \camlline{e}
            \end{enumerate}
        \end{enumerate}
        \begin{q}{4.1}
            Écrire une fonction \camlline{verif_arbre: string btree -> bool}, telle que
            \camlline{verif_tree a} renvoie \camlline{true} si et seulement si l'arbre
            vérifie la propriété \textbf{BE}. Par exemple \camlline{verif_artbre ab1}
            renvoie \camlline{false} mais l'appel \camlline{verif_line ab2} renvoie
            \camlline{true}, ainsi que l'appel de la fonction \camlline{verif_arbre} sur
            chaque sous-arbre de \camlline{ab2}.
        \end{q}
        \begin{q}{4.2}
            Écrire une fonction \camlline{cons_arbre_be: string -> int -> string btree},
            telle qu'un appel \camlline{cons_arbre_be e n} construit un arbre parfait
            vérifiant la propriété \textbf{BE} dont l'étiquette de sa racine est
            \camlline{e} et sa hauteur est \camlline{n}. Par exemple
            \camlline{cons_arbre_be "" 2} renvoie l'arbre \camlline{ab3}.
        \end{q}
        \begin{q}{4.3}
            On cherche à écrire \camlline{eten_be_opt: string btree -> string -> int -> string btree}
            , qui est une variante de la fonction \camlline{eten} de la question
            \textbf{3}, où maintenant l'arbre passé en argumentainsi que l'arbre résultat
            véirfieront la propriété \textbf{BE}. En conséquence, les \textit{nouvelles}
            étiquettes ajutées par la fonction \camlline{eten_be_opt} seront toutes différentes.
            De plus, on veut que la nouvelle fonction soit optimale, dans le sens qu'elle ne visitera
            que les noeuds de l'arbre se trouvant entre la racine et la feuille à atteindre.
            Ainsi par exemple, l'appel \camlline{eten_be_opt ab2 "0" 1} renvoie l'arbre
            \camlline{ab3} et visite uniquement les noeuds étiquetés par \camlline{""} et
            \camlline{"0"} tandis que \camlline{eten_be_opt "01" 1} visite uniquement les noeuds
            étiquetés par \camlline{""}, \camlline{"1"} et \camlline{"01"} et ajoute deux
            feuilles, étiquetées respectivement par \camlline{"001"} et \camlline{"101"}.
        \end{q}
    \end{q}
\end{exo}

\begin{exo}
    On considère à présent des arbres étiquetés \textit{d'arité variable}, dans lesquels
    chaque noeud peut avoir un nombre quelconque de fils. Ces arbres, ainsi que les
    ensembles d'arbres - les forêts - seront représentés à l'aide des deux types
    mutuellement récursifs :
    \begin{lstlisting}
type 'a tree =
    | N of 'a * 'a forest
and 'a forest =
    | V
    | A of 'a tree * 'a forest
;;  \end{lstlisting}
    Le constructeur \camlline{N} permet de spécifier l'étiquette d'un noeud ainsi que
    l'ensemble de ses fils, sous la forme d'une forêt. Le constructeur \camlline{V}
    représente la forêt vide, le constructeur \camlline{A} permet d'ajouter un nouvel
    arbre à une forêt d'ajà construite. Voici un exemple où les étiquettes sont de type
    \camlline{float}:
    \begin{lstlisting}
let ar_1 = N(1.2, A(N(3.44,V), A(N(2.15, V), V)));;\end{lstlisting}
    \begin{q}{1}
        Appelons \camlline{ar_2} la représentation en OCaml de l'arbre suivant, étiqueté par
        des valeurs de type \camlline{int list}. À partir de la définition du type
        \camlline{type 'a tree}, écrire la définition de cet arbre.
    \end{q}
    \begin{q}{2}
        Écrire une fonction \camlline{elem: 'a tree -> ('a -> bool) -> 'a list}, telle
        que \camlline{elem a prop} renvoie la liste de toutes les étiquettes de l'arbre
        \camlline{a} vérifiant la propriété \camlline{prop}. Ainsi par exemple, si
        \camlline{contient_3: int list -> bool} est une fonction qui renvoie \camlline{true}
        si et seulement si la liste passé en argument contient au moins un \camlline{3},
        alors \camlline{elem ar_2 contient_3 = [[2;3];[3];[3;3]]}.
        \begin{lstlisting}
let rec elem a prop = match a with
    | N(e, f) when prop e -> [e]@(_elem f prop)
    | N(_, f) -> (_elem f prop)
and rec _elem f prop = match f with
    | V -> []
    | A(a, f') -> (elem a prop)@(_elem f' prop)
;;      \end{lstlisting}
    \end{q}
    \begin{q}{3}
        Écrire une fonction \camlline{search: 'a tree -> 'a -> bool}, telle que
        \camlline{search a p} renvoie \camlline{true} si et seulement si l'arbre
        \camlline{a} contient un noeur étiqueté par \camlline{p}. Par exemple
        \camlline{search ar_3 [5;0]} renvoie \camlline{false}, où \camlline{ar_2}
        est l'arbre construit à la question \textbf{1}.
        \begin{lstlisting}
let search a p = (List.length (elem a (fun h -> h == p))) > 0;;\end{lstlisting}
    \end{q}
\end{exo}

\end{document}