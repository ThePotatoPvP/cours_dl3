\documentclass[french]{report}
\usepackage{../exercices}

\begin{document}
% augeri@lpsm.paris
\begin{center}
    \huge{\textbf{S6- ISF - TD1}}
\end{center}

\section*{Théorie des ensembles}

\begin{exo}
    Soit \((A_i)_{i\in I}\) une famille de parties d'un ensemble \(E\). Montrer que
    \begin{q}{1}
        \(\left(\cup_{i\in I}A_i\right)^c = \cap_{i\in I} A_i^c\)
        \boxans{\(x \in \left(\cup A_i\right)^c \Leftrightarrow x \not\in \cup A_i \Leftrightarrow \forall i,x\not\in A_I
        \Leftrightarrow \forall i, x\in A_i^c \Leftrightarrow x \in \cap A_i^c\)}
    \end{q}
    \begin{q}{2}
        \(\left(\cap_{i\in I}A_i\right)^c = \cup_{i\in I} A_i^c\)
        \boxans{\(x\in\left(\cap A_i\right)^c \Leftrightarrow x \not\in\cap A_i
        \Leftrightarrow\exists i, x\not\in A_i \Leftrightarrow \exists i, x\in A_i^c
        \Leftrightarrow x\in \cup A_i^c\)}
    \end{q}
\end{exo}

\begin{exo}
    Soit \(X=\{1,2,3,4,5,6\}, A=\{1,2,3,4\}, B=\{2,3,4,5,6\}, C=\{2,4,6\}\)
    \begin{q}{1}
        Déterminer \(A\cup B\), \(A\cap B\) et \(\left(A\cap B\right)\cup C\).
        \boxans{\(A\cup B = X\), \(A\cap B = \{2,3,4\}\),
        \((A\cap B)\cup C=\{2,3,4,6\}\)}
    \end{q}
    \begin{q}{2}
        On note \(D=\{x\in X\mid 2x\in A\}\). Déterminer \(D\).
        \boxans{\(D=\{1,2\}\)}
    \end{q}
    \begin{q}{3}
        On note maintenant \(\mathcal{E}=\{A,B\}, \mathcal{F}=\{B,C\}, \mathcal{G}=\{C,D\}.\)
        Déterminer \(\mathcal{E}\cup\mathcal{G}\) et \(\mathcal{E}\cap\mathcal{G}\)
        \boxans{\(\mathcal{E}\cup\mathcal{G} = \{A,B,C,D\}\) et
        \(\mathcal{E}\cap\mathcal{G}=\emptyset\)}
    \end{q}
\end{exo}

\section*{Dénombrabilité}

\begin{exo}
    Soit \(\pow(\N)\) l'ensemble des parties de \(\N\). Le but de cet exercice est de
    montrer que \(\pow(\N)\) n'est pas dénombrable. Soit \(\varphi\) une application
    \(\varphi\colon N\to\pow(\N)\) on pose \(U=\{n\in\N\mid n\not\in \varphi(n)\}\)
    \begin{q}{1}
        Montrer par l'absurde qu'il n'existe pas d'entier \(k\in\N\) vérifiant \(\varphi(k)=U\).
        \boxans{On suppose par l'absure qu'il existe un tel entier qu'on appelle \(k\) alors
        si \(k\in U\Rightarrow k\in \varphi(k)\Rightarrow k\not\in U\) et si
        \(k\not\in U\Rightarrow k\not\in \varphi(k)\Rightarrow k\in U\) ce qui est absurde.}
    \end{q}
    \begin{q}{2}
        En déduire que \(\pow(\N)\) n'est pas dénombrable.
        \boxans{On a montré ci-dessus qu'il n'existe pas de bijcetion entre \(\N\)
        et \(\pow(\N)\), qui n'est donc pas dénombrable. Merci \textsc{Cantor} pour
        la Sevrannité de ton poulet.}
    \end{q}
\end{exo}

\section*{Tribu}

\begin{exo}
    Soit \(E\) un ensemble.
    \begin{q}{1}
        Montrer que l'ensemble des parties finies de \(E\) est une tribu si et seulement
        si \(E\) est fini.
        \boxans{Si \(E\) est fini, alors l'ensemble de ses parties est trivialement
        une tribu. Si \(E\) est infini l'ensemble de ses parties finies n'est pas
        stable par passage au complémentaire.}
    \end{q}
    \begin{q}{2}
        Montrer que l'ensemble des parties \(A\subseteq E\) telle que \(A\) ou \(A^c\)
        est au plus dénombrable est une tribu. On la note \(\CC\).
        \boxans{\(\CC\) est stable par passage au complémentaire, par définition. \(\CC\)
        contient trivialent \(\emptyset\). Une union dénombrables d'éléments dénombrables est dénombrable,
        pareil pour une intersection avec au moins 1 élément dénombrable, donc \(\CC\) est bien une tribu.}
    \end{q}
    \begin{q}{3}
        Montrer que \(\CC\) est la tribu engendrée par \(\{\{x\}\mid x\in E\}\)
        \boxans{Les singletons étant dénombrables ils sont bien danc \(\CC\), montrons qu'ils
        l'engendre. Chaque élément \(A\in\CC\) vérifie \(A\) ou \(A^c\) dénombrable
        donc \(A\) ou \(A^c\) s'obtient comme union dénombrable de singleton, et donc
        \(A\) est bien dans la tribu engendrée par les singletons, ce qui suffit à conclure.}
    \end{q}
\end{exo}

\begin{exo}
    On note \(\bor\) la tribu engendrée par les ouverts de \(\R\). Pour tour réél
    \(\lambda\) et toute partie \(A\subseteq \R\) on pose
    \(A+\lambda=\{a+\lambda\mid a\in A\}\) et \(\bor+\lambda=\{A+\lambda\mid A\in\bor\}\).
    \begin{q}{1}
        Montrer que pour tout \(\lambda\in\R\) on a \(\bor\subseteq \bor+\lambda\)
        \boxans{Soit \(A\in\bor\) on note \(I_0\dots I_N\) lune famille dénombrable de segments
        l'engendrant, la translation ne changeant pas la topologie et \(R\) étant isomorphe
        à la longue droite, ces même segments sont dans \(\bor+\lambda\) et donc \(A\) aussi.}
    \end{q}
    \begin{q}{2}
        En déduire que pour tout \(\lambda\in\R\) on a \(\bor-\lambda\subseteq\bor\)
        \boxans{Par \(\alpha\)-renommage et la question précédente c'est vrai.}
    \end{q}
    \begin{q}{3}
        Conclure que \(\bor=\bor+\lambda\) pour tout \(\lambda\in\R\).
        \boxans{Les deux \(\lambda\) des questions précédentes étant indépendants,
        on a \(\bor-(-\lambda) \subseteq \bor \subseteq \bor+\lambda\)}
    \end{q}
\end{exo}

\begin{exo}
    On note\begin{enumerate}
        \itt \(\bor\) la tribu engendrée par les ouverts de \(\R\)
        \itt \(\CC_1\) la tribu engendrée par les fermés de \(\R\).
        \itt \(\CC_2\) la tribu engendrée par \(\{(a,b)\mid a,b\in\R, a<b\}\)
        \itt \(\CC_3\) la tribu engendrée par \(\{[a,b]\mid a,b\in\R, a<b\}\)
        \itt \(\CC_4\) la tribu engendrée par \(\{(a,b]\mid a,b\in\R, a<b\}\)
        \itt \(\CC_5\) la tribu engendrée par \(\{[a,b)\mid a,b\in\R, a<b\}\)
    \end{enumerate}
    \begin{q}{1}
        Montrer que \(\CC_2\subseteq \bor\) et \(\CC_3\subseteq \CC_1\).
        \boxans{Les ensembles qui engendrent \(\CC_2\) sont des ouverts, donc en particulier
        ils sont aussi parmi les ensembles qui engendrent \(\bor\) donc \(\CC_2\subseteq\bor\).
        On fait un raisonnement similaire avec les fermés pour avoir \(\CC_3\subseteq\CC_1\).}
    \end{q}
    \begin{q}{2}
        Montrer que \(\bor=\CC_1\).
        \boxans{Une tribu étant stable par passage au complémentaire, \(\bor\) qui est engendré
        par tous les ouverts l'est aussi par tous les fermés, donc \(\bor=\CC_1\).}
    \end{q}
    \begin{q}{3}
        Soit \(\mathcal{O}\) un ouvert. Montrer que \(\mathcal{O}=\bigcup_{(a,b)\in\Q^2
        \mid a<b\text{ et } (a,b)\subseteq\mathcal{O}} (a,b)\). En déduire que
        \(\bor = \CC_2\) puis que \(\bor=\CC_3\).
        \boxans{Par densité de \(\Q\) dans \(\R\) on peut prendre, sans perte de généralité
        \(a,b\in\Q\). Ainsi l'union contient un élément maximal, et le résultat est trivial.
        Comme \(\Q\) est dénombrable, \(\mathcal{O}\) est une union dénombrable d'ouvert, ainsi
        les engendreurs de \(\bor\) sont dans \(\CC_2\) donc \(\bor=\CC_2\). Par passage au
        complémentaire, le même raisonnement donc \(\bor=\CC_3\).}
    \end{q}
    \begin{q}{4}
        Montrer que \(\bigcap_{n\geq1}(a-\frac{1}{n},a+\frac{1}{n}]=\{a\}=
        \bigcap_{n\geq1}[a-\frac{1}{n},a+\frac{1}{n})\) pour tout \(a\in\R\). En
        déduire que \(\bor=\CC_4\) et que \(\bor=\CC_5\).
        \boxans{Tous les éléments des intersection contiennent \(a\), montrons qu'ils ne
        contiennent rien d'autre. Soit \(b\in\R, b\neq a\) on note \(\eps = \lceil|b-a|^{-1}\rceil \)
        et à partir de l'indice \(\eps\) l'élément \(b\) n'est plus dans les intersections,
        qui valent donc \(\{a\}\). Ainsi \(\CC_4\) et \(\CC_5\) engendrent toute
        partie dénombrable de \(\R\). On a \(\CC_4\subseteq\bor\) et \(\CC_5\subseteq\bor\)
        trivialement car les engendreurs peuvent s'écrire comme union de deux segments, l'un
        ouvert et l'autre fermés, qui sont dans \(\bor\) par définition Pour les inclusions
        réciproque on peut modifier la formule de l'énoncé pour ne faire bouger qu'une borne
        et ainsi engendrer les intervalles fermés, donc \(\CC_1=\bor\).}
    \end{q}
\end{exo}

\begin{exo}
    Soient \(X,Y\) deux ensembles, et \(f\colon X\to Y\) une application.
    On se donne \(A,B\subseteq X\) et \(E,F\subseteq Y\).
    \begin{q}{1}
        Rappeler les définitions de \(f(A)\) et \(f^{-1}(E)\).
        \boxans{\(f(A)=\{f(a)\mid a\in A\}\) et
        \(f^{-1}(E) = \{x\in X\mid f(x)\in E\}\)}
    \end{q}
    \begin{q}{2}
        Parmi les propositions suivantes, déterminer celles qui sont vraies et les
        prouver. Pour celles qui sont fausses, fournir un contre exemple.
        \begin{q}{a}
            \(f(A\cap B) = f(A)\cap f(B)\).
            \boxans{Cette affirmation et fausse, supposons \(x,y\in A\times B\) avec
            \(f(x)=f(y), \forall r\not\in\{x,y\}, f(r)\neq f(x)\) alors l'élément
            \(f(x)\) est dans l'ensemble de droite mais pas celui de gauche.}
        \end{q}
        \begin{q}{b}
            \(f(A\cup B) = f(A)\cup f(B)\).
            \boxans{Cette affirmation est vraie.
            \(y\in f(A\cup B) \Leftrightarrow \exists x\in A\cup B\mid f(x)=y
            \Leftrightarrow (\exists x\in A\mid f(x)=y)\lor(\exists x\in B\mid f(x)=y)
            \Leftrightarrow (y\in f(A))\lor(y\in f(B))\Leftrightarrow y\in f(A)\cup f(B)\)}
        \end{q}
        \begin{q}{c}
            \(f^{-1}(E\cap F)=f^{-1}(E)\cap f^{-1}(F)\)
            \boxans{Cette affirmation est vraie.
            \(x\in f^{-1}(E\cap F)\Leftrightarrow f(x)\in E\cap F
            \Leftrightarrow (f(x)\in E)\land (f(x)\in F)\Leftrightarrow
            (x\in f^{-1}(E))\land(x\in f^{-1}(F))\Leftrightarrow x\in
            f^{-1}(E)\cap f^{-1}(F)\)}
        \end{q}
        \begin{q}{d}
            \(f^{-1}(E\cup F)=f^{-1}(E)\cup f^{-1}(F)\)
            \boxans{Cette affirmation est vraie.
            \(x\in f^{-1}(E\cup F)\Leftrightarrow f(x)\in E\cup F
            \Leftrightarrow (f(x)\in E)\lor (f(x)\in F)\Leftrightarrow x\in
            f^{-1}(E)\cup f^{-1}(F)\)}
        \end{q}
    \end{q}
\end{exo}

\begin{exo}
    Soit \(E\) un ensemble.
    \begin{q}{1}
        Soient \(\left(F, \mathcal{B}\right)\) un espace mesurable et
        \(f\colon E\to F\) une application. Démontrer que \(f^{-1}(\mathcal{B})
        \colon\{f^{-1}(B)\mid B\in\mathcal{B}\}\) est une tribu sur \(E\).
        \boxans{D'après l'exercice précédent, \(f^{-1}(\mathcal{B})\) est stable par
        union et intersection, contient l'ensemble vide comme image réciproque de lui même.
        L'image réciproque est un morphisme préservant aussi le passage au complémentaire,
        donc c'est bien une tribu.}
    \end{q}
    \begin{q}{2}
        Déterminer toutes les tribus finies sur \(E\). Lesquelles ont
        \(0,1,2,\dots,8\) éléments ?
        \boxans{D'après la question précédente, on peut associer toute tribu à l'ensemble
        des parties de \(\llbracket1,n\rrbracket\) avec le bon \(n\).}
    \end{q}
\end{exo}

\begin{exo}
    Soient \(\mathcal{A}_1\) et \(\mathcal{A}_2\) deux tribus sur un ensemble \(E\).
    Montrer que la tribu engendrée par \(\mathcal{A}_1\cup\mathcal{A}_2\) coïncide
    avec la tribu engendrée par \(\{A_1\cap A_2\mid A_1\in\mathcal{A}_1, A_2\in\mathcal{A}_2\}\)
    \boxans{On note \(\CC\) la seconde tribu, on veut montrer
    \(\CC=\sigma(\mathcal{A}_1\cup\mathcal{A}_2)=\CC'\). On commence par remarquer que
    pour tout \(A_1\in\mathcal{A}_1,A_2\in\mathcal{A}_2\) on a \(A_1\cap A_2\in\CC'\)
    par la stabilité de la tribu, donc \(\CC\subseteq\CC'\). Dans l'autre sens on a bien
    \(\forall A_1\in\mathcal{A}_1,A_1=A_1\cap E\in \CC\) par symétrie on a ensuite
    \(\mathcal{A}_1\cup\mathcal{A}_2\in\CC\) ce qui suffit.
    }
\end{exo}

\begin{exo}
    On définit sur \(\N\), pour chaque \(n\geq 0\), la tribu \(\mathcal{F}_n=
    \sigma\left(\{0\},\{1\},\dots,\{n\}\right)\).
    \begin{q}{1}
        Montrer que la suite de tribus \(\left(\mathcal{F}_n,n\in\N\right)\) est
        croissante, mais que \(\bigcup \mathcal{F}_n\) n'est pas une tribu.
    \end{q}
\end{exo}

\begin{exo}
    Soit \(\mathcal{S}\) l'ensemble des boréliens de \(\R\) symétrique par
    rapport à \(0\), montrer que c'est une tribu. \[\mathcal{S}=\{A\in\bor\mid \forall x\in A, -x\in A\}\].
    \boxans{Cet ensemble contient clairement \(\emptyset\) et est clairement
    stable par union dénombrable et passage au complémentaire, c'est donc une tribu.}
\end{exo}

\section*{Mesures}

\begin{exo}
    Soient \(\left(X,\mathcal{A},\mu\right)\) un espace mesuré et \(S\in \mathcal{A}\).
    On définit une application \(\mu_{\mid S}\colon \mathcal{A}\to \R_+\colon A\mapsto \mu(S\cap A)\).
    Montrer que c'est une mesure.
\end{exo}

\section*{Applications mesurables}

\begin{exo}
    Calculer toutes les applications mesurables de \(\left(E, \mathcal{A}\right)\)
    ver \(\left(F, \mathcal{B}\right)\).
    \begin{q}{1}
        \(\mathcal{B} = \{\emptyset, F\}\)
    \end{q}
    \begin{q}{2}
        \(\mathcal{A}=\pow[E]\)
    \end{q}
    \begin{q}{3}
        \(\mathcal{A} = \{\emptyset, E\}\) et \(\mathcal{B}\) contient les singletons.
    \end{q}
\end{exo}

\begin{exo}
    Montrer que l'application \(f\colon\R\to\R\) définie par
    \(f(x)=\begin{cases}
        x+1&\text{si }x\geq 0\\
        -x&\text{sinon}
    \end{cases}\) est borélienne.
\end{exo}

\begin{exo}
    Soit \(f\colon\R\to\R\). Montrer que:
    \begin{q}{1}
        Si \(f\) est continue alors \(f\) est borélienne.
    \end{q}
    \begin{q}{2}
        Si \(f\) = \(\ind_\Q\) alors:
        \begin{q}{a}
            \(f\) est borélienne
        \end{q}
        \begin{q}{b}
            pour tout \(x\in\R\), \(f\) n'est pas continue en \(x\).
        \end{q}
    \end{q}
    \begin{q}{3}
        Si \(f\) est monotone alors \(f\) est borélienne.
    \end{q}
    \begin{q}{4}
        Si \(f\) est dérivable alors \(f'\) est borélienne.
    \end{q}
\end{exo}

\begin{exo}
    Soient \(\left(E,\mathcal{A}\right),(F,\mathcal{B})\) deux espaces mesurables.
    On note \(A\otimes B\) la tribu produit, celle sur \(E\times F\) engendrée par
    \(\{A\times B, A\in\mathcal{A},B\in\mathcal{B}\}\). Soient \(p_E\) et \(p_F\)
    les projections canoniques.
    \begin{q}{1}
        Montrer que \(p_E\) et \(p_F\) sont mesurables.
    \end{q}
    \begin{q}{2}
        Montrer que \(A\otimes B\) est la plus petite tribu pour laquelle
        \(p_E\) et \(p_F\) sont mesurables.
    \end{q}
    \begin{q}{3}
        Démontrer que \(\bor[\R^2] = \bor\otimes\bor\).
    \end{q}
\end{exo}

\begin{exo}
    Soient \(\left(E,\mathcal{A}\right)\) un espace mesurable, \(F\) un ensemble
    et \(f\colon E\to F\) une application. On note
    \(\mathcal{B}=\{B\in\pow[F]|f^{-1}(B)\in\mathcal{A}\}\)
\end{exo}

\section*{Classes monotones}

\begin{exo}
    Soit \(E\) un ensemble et \(\CC\in\pow[E]\), calculer la classe monotone
    engendrée par \(\CC\) lorsque :
    \begin{q}{1}
        \(\CC=\{A\}\) où \(A\subseteq E\).
    \end{q}
    \begin{q}{2}
        \(\CC=\{A,B\}\) où \(A,B\subseteq E\).
    \end{q}
    \begin{q}{3}
        \(\CC=\{B\subseteq E\mid A\subseteq B\}\) où \(A\subseteq E\).
    \end{q}
    Donner un exemple de classe monotone qui n'est pas une tribu.
\end{exo}

\begin{exo}
    Montrer que la fonction de répartition caractérise la loi d'une variable aléatoire
    réelle.
\end{exo}

\end{document}