\documentclass{report}
\usepackage{../../exercices}

\begin{document}
\begin{center}
    \Huge{\textbf{PR5-L3 Maths - TD 6}}
\end{center}
\bigskip

\begin{exo}
    On considère un jeu de pile ou face non biaisé, chaque tirage
    \(X_n\) suit une loi de bernoulli de paramètre \(1/2\). Et on pose
    \(S_n=X_1+\dots+X_n\). On considère que l'on a gagné au temps \(n\)
    s'il on a au moins \(2n/3\) piles.

    \begin{q}{1}
        Calculer \(\esp[\exp(tS_n)]\) pour \(t\in \R\) et en déduire qu'il existe \(c>0\)
        tel que \(\proba(S_n\geq 2n/3)\leq \exp(-cn)\).
    \end{q}
    \begin{q}{2}
        Montrer que l'estimation obtenue dans \(1\) implqie que dans ce jeu on ne gagne qu'un
        nombre fini de fois.
    \end{q}
    \begin{q}{3}
        Montrer que l'on peut obtenir le résultat du point \(2\) en utilisant
        la loi des grands nombres et sans utiliser le premier résultat.
    \end{q}
\end{exo}

\begin{exo}
    Dans l'espace canonique nous considérons la suite de variables aléatoires \(X_n\)
    définies par \(X_n(\omega)=n\mathds{1}_{(0,1/n)}(\omega)\).
    \begin{q}{1}
        Montrer qu'on a \(\lim X_n = 0\) en probabilité.
    \end{q}
    \begin{q}{2}
        Montrer qu'on a \(\lim X_n = 0\) preque sûrement.
    \end{q}
    \begin{q}{3}
        Montrer qu'on a \(\lim \esp[X_n]\neq 0\).
    \end{q}
    \begin{q}{4}
        Montrer que \((X_n)\) ne converge pas dans \(\mathbb{L}^1\).
    \end{q}
    \begin{q}{5}
        Montrer que \((\sqrt{X_n})\) converge dans \(\mathbb{L}^1\).
    \end{q}
\end{exo}

\begin{exo}
    Soit \((X_n)\) une suite IID et on suppose que \(X_1\) n'est pas trivial, i.e. \(\proba(X=x)\leq 1\).
    pour tout \(x\in\R\).
    \begin{q}{1}
        La suite \((X_n)\) converge-t-elle en loi ?
    \end{q}
    \begin{q}{2}
        Et presque sûrement ?
    \end{q}
    \begin{q}{3}
        Et en probabilité ?
    \end{q}
\end{exo}

\begin{exo}
    \((X_j)\) est une suite de variables aléatoires (aucune hypothèse d'indépendance).
    On pose \(Z(\omega)\colon=\sum_{j=1}^\infty 2^{-j}X_j(\omega)\) si cette série converge.
    \begin{q}{1}
        Montrer que s'il existe une constante \(C\) telle que \(\proba(|X_j \leq C) = 1\)
        pour tout \(j\) alors \(Z\) est bien définie presque sûrement.
        \boxans{On a presque sûrement l'inégalité suivante, assurant que \(Z\) est bien définie.
        \[ \left|\sum_{j>0} 2^{-j}X_j\right|\leq\sum_{j>0} 2^{-j}|X_j|\leq\sum_{j>0} 2^{-j}C = 2C\]}
    \end{q}
    \begin{q}{2}
        Montrer que si \(X_j\sim \textsc{Cauchy}\) pour tout \(j\) alors \(Z\) est
        bien définie presque sûrement.
    \end{q}
\end{exo}

\begin{exo}
    \((X_j)\) est une suite de variables aléatoires dans \(\mathbb{L}^2\) avec \(\cov(X_j,X_k)=0\) si \(j\neq k\)
    et on pose \(S_n = X_1+\dots+X_n\). De plus on suppose \(\esp[X_j]=\mu\) et \(\esp[X_j^2]=\sigma^2\) pour tout \(j\).
    Montrer que \(\lim S_{n^2}/n^2 = \mu\) presque sûrement.
\end{exo}

\begin{exo}
    \((X_j)\) est une suis IID avec \(X_1\in\mathbb{L}^1\) et \(S_n=X_1+\dots+X_n\).
    \begin{q}{1}
        Sur le même espace de probabilité il y a aussi une autre suite de variables
        aléatoires \((T_n)\) avec \(\proba(T_n\in\N)=1\) pour tout \(n\) et telle que
        \(\lim T_n=\infty\) presque sûrement. Calculer, si elle existe, la limite presque
        sûrement de la suite \((S_{T_n}/T_n)\).
    \end{q}
    \begin{q}{2}
        Calculer, si elle existe, aussi la limite presque sûrement de la suite
        \[\left(\frac{X_1X_2+X_2X_3+\dots+X_{n-1}X_n}{n}\right)_n\]
    \end{q}
\end{exo}

\begin{exo}
    Soit \((U_n)\) une suite IID \(U([0,1])\). On pose \(Z_n=n\min(U_1, \dots U_n)\)
    pour \(n\in\N\). Montrer que \((Z_n)\) converge en loi vers Exp(\(1\)).
\end{exo}
\end{document}