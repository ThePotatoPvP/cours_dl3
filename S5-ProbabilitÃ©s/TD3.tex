\documentclass{report}
\usepackage{../exercices}

\begin{document}
\begin{center}
    \Huge{\textbf{PR5-L3 Maths - TD 3}}
\end{center}
\bigskip

\textbf{Rappel} : On dit que \(\left( \Omega,\mathcal{F}, P\right)\) est
canonique si \(\Omega=(0,1),\mathcal{F}=\{B\cap\Omega;B\in\mathcal{B}(\R)\}
\) et \(P(A)=\lambda_1(A)\) pour tout \(A\in\mathcal{F}\) et \(\lambda_1\)
est la mesure de \textsc{Lebesgue} sur \(\left(\R,\mathcal{B}(\R)\right)\)

\begin{exo}
    Soit \(X\) une variable aléatoire discrète dont la loi \(P_X\) est
    donnée par \[P_X=\frac14\delta_{-1}+\frac18\delta_0+\frac18\delta1
    +\frac12\delta_2\]
    \begin{q}{1}
        Dessiner la fonction de répartition de \(X\).
    \end{q}
    \begin{q}{2}
        Soit \(Y\) la variable aléatoire définie par \(Y=X^2\). Donner
        la distribution de \(Y\) et sa fonction de répartition.
        \boxans{On trouve la distribution suivante en mettant les indices de \textsc{Dirac}
        au carré : \[P_Y=\frac18\delta_0+\frac38\delta1
        +\frac12\delta_2\]}
    \end{q}
    \begin{q}{3}
        Si \(\left(\Omega, \mathcal{F},P\right)\) est canonique, trouver
        \(A_1, A_2\) et \(A_3\in\mathcal{F}\) tel qu'en posant \(X_1(\omega)
        =-\mathds{1}_{A_1}(\omega)+\mathds{1}_{A_2}(\omega)+ 2\mathds{1}_{A_3}
        (\omega)\) on ait \(X_1\sim X\).
        \boxans{On choisit par exemple \(A_1=(0,\frac14], A_2=(\frac38,\frac12]\)
        et \(A_3=(\frac12, 1)\) donnant bien \(X_1\sim X\)}
    \end{q}
    \begin{q}{4}
        Vérifier que l'on a bien \(X_1^2\sim Y\).
        \boxans{D'après la formule de transfert, on a bien l'égalité souhaitée.}
    \end{q}
    \begin{q}{5}
        On considère maintenant \(\Omega=\{1,2,3,4\}\), \(\FF=\PP(\Omega)\)
        et \(P\) une probabilité telle que \(P(\{1\})=\frac14\) et
        \(P({2})=P({3})= \frac18\). Montrer que si on pose \(X_2(\omega)=\omega - 2\)
        on a \(X_2\sim X\).
        \boxans{Bah effectivenent, si on ouvre les yeux on voit que c'est \(X\) décalé de 2.}
    \end{q}
    \begin{q}{6}
        Peut on calculer \(P(X=X_1)\) ? et \(P(X=X_2)\) et \(P(X=Y)\) ?
        \boxans{Seul \(P(X=Y)\) a du sens, les autres variables ne vivent pas
        dans le même univers probabilisé.}
    \end{q}
\end{exo}

\begin{exo}
    Soit \(\left(\Omega,\FF, P\right)\) l'espace cannonique. Nous considérons
    les variables aléatoires \[X=\ind_{(\frac12, 1)}\quad\text{ et }\quad
    Y=\ind_{(\frac14,\frac12)} + \ind_{(\frac34, 1)}\] Montrer que \(X\sim Y
    \sim B(\frac12)\) et qu'elles sont indépendantes.
    \boxans{On a \(P(X=1)=P(Y=1)=P(X=0)=P(Y=0)=\frac12\) donc
    \(X\sim Y\sim B(\frac12)\). Aussi \(P(X\cap Y) = P((\frac34, 1))=
    \frac14=P(X)P(Y)\) les variables sont donc indépendantes.}
\end{exo}

\begin{exo}
    Soit \(\left(\Omega,\FF, P\right)\) l'espace cannonique. Nous considérons
    les variables aléatoires \[X(\omega):=\omega\quad Y(\omega)=
    \omega\ind_{(0,\frac12]}(\omega)+\left(\frac32-\omega\right)
    \ind_{(\frac12, 1)}(\omega)\]
    \begin{q}{1}
        Montrer que \(X\sim Y\).

    \end{q}
    \begin{q}{2}
        Calculer \(P(X=Y)\)
    \end{q}
    \begin{q}{3}
        \(X\) et \(Y\) sont elles indépendantes ?
    \end{q}
\end{exo}


\begin{proof}[Inégalité de Markov]
    Soit \(a\in\R^+\), on considère la fonction \(a\mathds{1}_{X\geq a}\)
    qui est inférieure à \(X\) sur \(\R^+\) ainsi par la croissance de l'espérance :
    \(E[X]\geq E[a\mathds{1}_{X\geq a}] = a\esp[\mathds{1}_{X\geq a}] = a\proba(X\geq a)\)
\end{proof}
\end{document}