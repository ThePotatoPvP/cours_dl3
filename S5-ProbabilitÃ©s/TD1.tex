\documentclass{report}
\usepackage{../../exercices}


\begin{document}

\begin{center}
    \Huge{\textbf{PR5-L3 Maths - TD 1.}}
\end{center}
\bigskip
mails des teachers : daures@lspm.paris et le plus important collin@lspm.paris

\begin{exo}
    On lance un dé rouge et un dé vert. Donner explicitement un espace probabilisé pour étudier cette experience aléatoire et l'utiliser pour répondre aux questions
    suivantes :
    \boxans{On utilisera comme espace probabilisé \(\Z_6 \times \Z_6\) avec chaque élément d'une probabilité répondant à une loi uniforme.}
    \begin{q}{(1)}
        Quelle est la probabilité que le nombre sur le dé rouge soit plus strictement plus grand que le nombre sur le dé vert?
        \boxans{On a 36 tirages possibles, parmi lesquels 6 où les tirage des dés sont égaux. Parmis les 30 tirages restants,
        pour la moitié d'entre eux le tirage du dé rouge est strictement supérieur au tirage du dé vert, pour l'autre moitié c'est l'inverse, la probabilité
        recherchée est alors \(\frac{15}{36}\)}
    \end{q}
    \begin{q}{(2)}
        Quelle est la probabilité que les deux nombres diffèrent de au plus 1 ?
        \boxans{On a 6 tirages possibles avec aucune différence, puis 5 tirages possibles avec le dé rouge supérieur au vert, et 5 dans l'autre sens.
        Ce qui donne en tout une probabilité de \(\frac{16}{36}\)}
    \end{q}
    \begin{q}{(3)}
        Quelle est la probabilité que le maximum des deux nombres soit supérieur ou égal à 5 ?
        \boxans{On a \(4^2\) façons distinctes d'avoir un maximum strictement inférieur à 5,
        ainsi la probabilité d'avoir un maximum supérieur à 5 est \(\frac{20}{36}\)}
    \end{q}
\end{exo}

\begin{exo}
    Si on lance une pièce 5 fois, quelle est la probabilité que le nombre de faces
    soit pair ? On écrira explicitement l'espace probabilisé associé à cette experience.

    \boxans{L'espace probabilisé est défini ici par \(\Omega=\{0,1\}^2\) avec \(\proba=\mathcal{U}(\Omega)\) et la tribu \(\mathcal{F}=\mathcal{P}(\Omega)\).
    On remarque ensuite que la probabilité d'avoir un nombre pair de face est par définition la même que celle d'avoir un nombre impair de piles, or ces
    deux évènements sont équiprobables ce qui donne que la probabilité recherchée est \(\frac12\)}
\end{exo}

\begin{exo}
    Posons \(\Omega=\{a,b,c\}\) muni de la tribu de l'ensemble de ses parties. On donne \(\proba(\{a,b\}) = 0,7\) et \(\proba(\{b,c\}) = 0,6\). Calculer
    les probabilités de \(\{a\}, \{b\}\) et \(\{c\}\).
    \boxans{La relation \(\proba(A)=1-\proba(\bar{A})\) on a \(\proba(\{a\}) = 1-\proba(\{b,c\})= 0,4\) et de la même façon \(\proba(\{c\})=0,3\) finalement
    le système complet d'évènements permet d'obtenir \(proba(\{b\}) = 1 - \proba(\{a\}) - \proba(\{c\}) = 0,3\).}
\end{exo}

\begin{exo}
    Supposons que \(A\) et \(B\) soient deux évènements disjoints avec \(\proba(A)=0,4\) et \(\proba(B)=0,5\) que vaut \(\proba(\bar{A}\cap\bar{B})\) ?
    \boxans{Les deux évènements étant disjoints on a, en passant au complémentaire, \(\proba(\bar{A}\cap\bar{B}) = 1- \proba(A\cup B)=1-\proba(A)-\proba(B)=0,1\)}
\end{exo}

\begin{exo}
    Soit \(\left(\Omega, \mathcal{A}, \proba\right)\) un espace probabilisé. Soit \(A,\ B\) des évènements tels que \(\proba(A)=\frac34\) et \(\proba(B)=\frac13\).
    \begin{q}{(1)}
        Montrer que \(\frac{1}{12} \leq \proba(A \cap B) \leq \frac13\)
        \boxans{On a \(\proba(A)+\proba(B) = 1 + \frac{1}{12}\) or \(\proba(A\cup B)\) doit être au plus \(1\) ce qui force \(\proba(A \cap B) \geq \frac{1}{12}.\)
        Pour ce qui est de l'autre borne on constate que \((A\cap B) \subset B\) donc \(\proba(A\cap B)\leq \proba(B) \leq \frac13\)}
    \end{q}
    \begin{q}{(2)}
        Soit \(\Omega = {1,...,12}\) muni de la tribu de l'ensemble de ses parties et de la probabilité uniforme. Donner des exemples de tels \(A\) et \(B\)
        et tels que les bornes ci-dessus soient atteintes.
        \boxans{
            Dans les deux cas l'expérience consiste à tirer un élément au hasard.
            \begin{enumerate}
                \item[Borne inf] \(A\) : "L'élément vaut au plus 9" et \(B\) : "L'élément vaut au moins 9"
                \item[Borne sup] \(A\) : "L'élément vaut au plus 9" et \(B\) : "L'élément vaut au plus 4"
            \end{enumerate}
        }
    \end{q}
    \begin{q}{(3)}
        Donner les bornes correspondantes pour \(\proba(A\cup B)\)
        \boxans{En utilisant l'exemple au dessus, ou un raisonnement semblable à celui de la première question, on trouve \(\frac34\leq\proba(A\cup B)\leq 1\)}
    \end{q}
\end{exo}

\begin{exo}
    Soit \(\Omega = \{1,2,3,4,5,6\}\) on définit des ensembles de parties
    \begin{enumerate}
        \item[(1)] : \(\mathcal{A}_1 = \{\{1,2\}, \{3,4,5,6\}, \{3,4\}, \{1,2,5,6\}, \Omega, \emptyset\}\)
        \item[(2)] : \(\mathcal{A}_2 = \{\{1,2,3\}, \{4,5,6\}, \Omega, \emptyset\}\)
    \end{enumerate}
    \begin{q}{(1)}
        \(\mathcal{A}_1\) et \(\mathcal{A}_2\) sont-elles des tribus ? Justifier votre réponse.
        \boxans{\(\mathcal{A}_1\) n'est pas stable par union finie, en effet \(\{3,4\}\cup\{1,2\} = \{1,2,3,4\}\not\in\mathcal{A_1}\).
        D'autre part on a bien \(\emptyset\in \mathcal{A}_2\) ainsi que la stabilité évidente par réunion et passage au complémentaire, on a ainsi une tribu.}
    \end{q}
    \begin{q}{(2)}
        On définit l'application \(\proba:\mathcal{A}_2\rightarrow [0,1]\) par \(\proba(\{1,2,3\})=\frac14\), \(\proba(\{4,5,6\})=\frac23\),
        \(\proba(\emptyset)=0\) et \(\proba(\Omega)=1\). L'application \(\proba\) est elle une probabilité ?
        \boxans{La \(\sigma\)-additivité voudrait \(\proba(\{1,2,3\})+\proba(\{4,5,6\})=1\) ce qui n'est pas le cas, \(\proba\) n'est donc pas une probabilité.}
    \end{q}
    \begin{q}{(3)}
        Donner un ensemble de parties différent de l'ensemble des parties de \(\Omega\) et des exemples ci-dessus et qui soit une tribu de \(\Omega\).
        Définir ensuite une probabilité sur la tribu.
        \boxans{On pose par exemple \(\mathcal{A}_3 = \mathcal{A}_1 \cup \{1,2,3,4\} \cup \{5,6\}\) avec
        \(\proba : A \mapsto \frac{|A|}{6}\) pour \(A\in\mathcal{A}_3\)}
    \end{q}
\end{exo}

\begin{exo}
    Soit \(\Omega\) un ensemble fini. Montrer que la tribu
    \(\mathcal{P}(\Omega)=2^\Omega\) de toutes les parties est également finie.
    Par contre, montrer que si \(\Omega\) n'est pas fini alors \(\mathcal{P}(\Omega)\)
    n'est pas dénombrable.
    \boxans{Par combinatoire, chaque élément peut ou non être dans chaque ensemble. Pour la seconde
    partie Cantor nous assure qu'un ensemble n'admet de bijection avec ses parties, voici une
    courte démonstration : on suppose \(f\) une surjection de \(E\) dans \(\mathcal{P}(E)\)
    on considère \(V = \{x\mid x\not\in f(x)\}\) une partie de \(E\), si \(x\in V\)
    alors \(f(x)\neq V\) et si \(x\not\in V\) alors \(f(x)\neq V\) sans quoi \(x\in V\) bref ça colle pas.
    On a créé un ensemble qui n'a pas d'image, ça contredit la surjectivité de \(f\) donc pas de bijection.}
\end{exo}

\begin{exo}
    Soit \(\left(\mathcal{F}_\alpha\right)_{\alpha\in A}\) une famille quelconque de tribus sur un ensemble \(\Omega\).
    Montrer que \(\mathcal{H}=\cap_{\alpha\in A}\mathcal{F}_\alpha\) est aussi une tribu.
    \boxans{
        \begin{enumerate}
            \itt Les \(\mathcal{F}_\alpha\) étant des tribus on a \(\emptyset\in\mathcal{F}_\alpha\) pour tout \(\alpha\in A\) donc \(\emptyset\in\mathcal{H}\)
            \itt Soit \(H\in\mathcal{H}\) par définition \(H\in\mathcal{F_\alpha}\) pour tout \(\alpha\in A\) donc \(\bar{H}\in \mathcal{F}_\alpha\Rightarrow \bar{H}\in\mathcal{H}\)
            \itt Soit \(\left(H_i\right)_{i\in\N}\) suite d'éléments de \(\mathcal{H}\), ils sont dans tous les \(\mathcal{F}_\alpha\) ainsi leur union aussi et elle est donc dans \(\mathcal{H}\).
        \end{enumerate}
    }
\end{exo}

\begin{exo}
    Nous rappelons que les ensembles boreliens \(\mathcal{B}(\R)\) est la plus
    petite tribu contenue dans \(\mathcal{P}(\R)\) qui contient les ouverts de \(\R\).
    \begin{q}{1}
        Montrer que \(\mathcal{B}(\R)=\sigma\left( ]a,b[; a,b\in\R, a<b\right)\)
        \boxans{Soit \(U\) un ouvert de \(\R\) alors chaque partie connexe de \(U\)
        est trivialement un intervalle ainsi on a bien l'égalité par axiome d'extensionnalité
        ainsi que l'axiomatique des tribus. L'inclusion réciproque est triviale.}
    \end{q}
    \begin{q}{2}
        Montrer que \(\mathcal{B}(\R)=\sigma\left( ]a,b[; a,b\in\Q\cap ]a,b[, a<b\right)\)
        \boxans{Par densité de \(\Q\) on prend \(a_n, b_n\in\Q\) tels que \(]a_n,b_n[\rightarrow]a,b[\)
        La tribu étant stable par union dénombrable on a l'inclusion dans la tribu du dessus,
        la réciproque est triviale, on a donc l'égalité.}
    \end{q}
    \begin{q}{3}
        Montrer que \(\mathcal{B}(\R)=\sigma\left( [a,b]; a,b\in\R, a<b\right)\)
        \boxans{Soit \([a_n, b_n]\subsetneq[a,b]\) dans la tribu tendant vers \([a,b]\) alors
        en utilisant la stabilité par union dénombrable \(]a,b[\) est dans la tribu,
        on se ramène à la question 1.}
    \end{q}
\end{exo}

\begin{exo}
    Soit \(\left(A_n\right)_{n\geq1}\) une suite de parties qui appartiennent
    à une tribu \(\mathcal{A}\) d'un ensemble \(\Omega\). Montrer que :
    \begin{enumerate}
        \itt \(\ds \lim\inf A_n := \cup_{n=1}^\infty\cap_{k\geq n} A_k \in \mathcal{A}\)
        et que \(\lim\inf 1_{A_n} = 1_{\lim\inf A_n}\)
        \itt \(\ds \lim\inf A_n := \cap_{n=1}^\infty\cup_{k\geq n} A_k \in \mathcal{A}\)
        et que \(\lim\sup 1_{A_n} = 1_{\lim\sup A_n}\)
    \end{enumerate}
    \boxans{\(\mathcal{A}\) est une tribu, c'est donc un ensemble stable par union
    et intersection dénombrable ainsi les deux limites sont bien dans \(\mathcal{A}\).
    La limite supérieure d'une suite dans \(\{0,1\}^\N\) ne peut être 1 que si
    la suite contient une infinité de \(1\) (de même pour la limite inférieure et 0
    ) on a donc le résultat espéré sur les limites.}
    Montrer aussi que \(\lim\inf A_n\subset\lim\sup A_n\).
    \boxans{Soit \(x\in\lim\inf A_n\) alors par définition il existe \(n_0\in\N\)
    tel que pour tout \(n\geq n_0\) on ait \(x\in A_n\) ainsi pour tout
    \(n\in\N, x\in\cup_{k\geq n} A_k\) et donc \(x\in\lim\sup A_n\).}
\end{exo}

\begin{exo}
    Soit \(\mathcal{A}\) une tribu sur \(\Omega\) et \(B\subset\Omega\). Montrer que
    \(\mathcal{A}_B=\{A\cap B\mid A\in\mathcal{A}\}\) est une tribu sur
    l'ensemble \(B\). On l'appelle \(\sigma\)-algèbre trace de \(\mathcal{A}\) sur \(B\).
    \boxans{L'ensemble vide appartient bien à \(\mathcal{A}_B\), la stabilité
    par complémentaire et union sont triviales et découlent de celles dans \(\mathcal{A}\).}
\end{exo}

\begin{exo}
    Soient \(\left(A_i\right)_{i\in\N}\) des évennements d'une tribu d'un espace
    probabilisé, démontrer par récurrence la formule de poincarré :
    \[\proba\left(\cup_{i=1}^nA_i\right)=\sum_i\proba(A_i)-\sum_{i<j}\proba(A_i\cap A_j)
    + \dots + (-1)^{n+1}\proba(A_1\cap A_2 \cap\dots\cap A_n)\]
\end{exo}

\begin{exo}
    Soient \(\left(A_i\right)_{i\in\N}\) des évennements d'une tribu montrer:
    \[ \proba(\cup_{i=1}^nA_i)\leq \sum_{i=1}^n\proba(A_i)\]
    \boxans{Récurrence triviale avec \(\proba(\cup^n A_i) \leq \proba(\cup^{n-1} A_i) + \proba(A_n)\)
    car \(\proba(\cup A_i \cap A)\geq 0\). Pour faire tendre \(n\) vers l'infini on majore la somme
    par une série, la proba reste la même (passer par limite de l'union croissante
    pour le faire proprement.)}
\end{exo}

\begin{exo}
    Démontrer les inégalités de \textsc{Bonferroni} suivantes.
    \[\proba(\cup A_i)\geq \sum_i \proba(A_i)-\sum_{i<j}\proba(A_i\cap A_j)\quad
    \proba(\cup A_i)\leq\sum_i \proba(A_i)-\sum_{i<j}\proba(A_i\cap A_j)+
    \sum_{i<j<k}\proba(A_i\cap A_j\cap A_k)\]
\end{exo}

\begin{exo}
    Démontrer l'inégalité suivante : \(\proba\left(\cap A_i\right)\geq\sum\proba(A_i)-n+1\)
    \boxans{\[\proba (\cap A_i) = 1 - \proba (\overline{\cap A_i}) = 1 - \proba(\cap \bar{A_i})
    \geq 1 - \sum \proba (\bar{A_i}) = 1 - n + \sum \proba(A_i)\]}
\end{exo}
\end{document}