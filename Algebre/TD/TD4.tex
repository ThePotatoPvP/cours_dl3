\documentclass{report}
\usepackage{../../exercices}

\begin{document}

\begin{center}
    \huge{\textbf{Théorème de Lagrange, sous-groupes distingués,
    groupe quotien}}
\end{center}

\begin{exo}
    Soit \(p\) un nombre premier et \(G\) un groupe d'ordre \(p\). Montrer que \(G\) est cyclique.
\end{exo}

\begin{exo}
    Soit \(f\colon G\to H\) un morphisme de groupes finis et \(K\) un sous-groupe
    de \(G\)
    \begin{q}{1}
        Montrer que l'ordre de \(f(K)\) divise les ordres de \(K\) et de \(H\)
    \end{q}
    \begin{q}{2}
        Montrer que si l'ordre de \(K\) est premier à l'ordre de \(H\) alors
        \(K \subset \Ker(f)\)
    \end{q}
\end{exo}

\begin{exo}
    Soient \(G\) et \(G'\) deux groupes finis. On suppose que les ordres de ces groupes
    sont premiers entre eux. Déterminer tous les morphismes de groupes de \(G\) dans \(G'\)
\end{exo}

\begin{exo}
    Montrer que si \(H\) et \(K\) sont deux sous-groupes distinguées d'un groupe \(G\)
    alors leur intersection est distinguée.
\end{exo}

\begin{exo}
    Montrer que si \(N\) est distingué dans \(G\) et \(H\) un sous groupe quelconque de
    \(G\) alors \(N\cap H\) est distingué dans \(H\).
\end{exo}

\begin{exo}
    Soient deux sous-groupes distingués \(H\) et \(K\) de \(G\) tels que \(H\cap K=\{e\}\)
    Montrer que \(\forall x\in H,\forall y\in K, xy=yx\)
\end{exo}

\begin{exo}
    Soient les éléments \(a=(1,2,3,4)\) et \(b=(2,2)\circ(3,4)\) dans \(\mathfrak{S}_4\)
    et \(G\) le sous groupe de \(\mathfrak{S}_4\) engendré par \(a\) et \(b\). Soit
    \(H=\langle b,a^2\rangle\) et \(K=\langle b\rangle\). Montrer que \(H\) est un sous-groupe
    distingué dans \(G\), que \(K\) est un sous groupe distingué dans \(H\), mais que
    \(K\) n'est pas un sous-groupe distingué dans \(G\).
\end{exo}
\end{document}