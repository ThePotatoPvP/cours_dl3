\documentclass{report}
\usepackage{../../exercices}

\begin{document}

\begin{center}
    \huge{\textbf{Groupes, sous-groupes}}
\end{center}

\begin{exo}
    Parmi les couples suivants, lesqules sont des groupes ?
    \begin{q}{1}
        \(\left(\N, \max\right)\)
        \boxans{Ce n'est pas un groupe, en effet le neutre est \(0\) mais il
        n'y a pas d'inverse pour les autres entiers naturels.}
    \end{q}
    \begin{q}{2}
        \(\left(\R\backslash\{-1\},\star\right)\) où \(x\star y = xy+x+y\).
        \boxans{On remarque que \(0\) est l'élément neutre, aussi l'ensemble
        est stable par l'opération \(\star\) et chaque \(x\in\R\backslash\{-1\}\)
        admet \(\frac{1}{x+1}-1\) comme inverse, on a bien un groupe. L'associativité
        se vérifie aussi au brouillon.}
    \end{q}
    \begin{q}{3}
        \textit{tableau à recopier}
        \boxans{On observe que \(d\) est le neutre et que \(c\) n'a pas d'inverse,
        l'ensemble n'est donc pas un groupe.}
    \end{q}
    \begin{q}{4}
        \(\left(\R, \star\right)\) où \(x\star y=x+y-2\) pour tous \(x,y\in\R\).
        \boxans{Ici \(2\) est l'élément neutre, chaque réel \(x\) admet \(4-x\)
        comme inverse, l'associativité provient directement de celle sur l'addition.}
    \end{q}
\end{exo}

\begin{exo}
    Soit un groupe \(\left(G, \times\right)\). Pour tous \(g, x \in G\) on note
    \(^gx = g\times x\times g^{-1}\). Soient \(g,x,y\in G\) démontrer que
    \(^g(x\times y) = (^gx)\times(^gy)\) et \(^g(x^{-1})=(^gx)^{-1}\).
    \boxans{\((^gx)\times(^gy)=g\times x\times g^{-1}\times g\times y\times g^{-1}
    =g\times x\times y\times g^{-1}=^g(x\times y)\)\newline
    \(\left(^gx\right)^{-1}= \left(g\times x\times g^{-1}\right)^{-1}=g \times (g\times x)^{-1}
    =g\times x^{-1}\times g^{-1}=^g(x^{-1})\)}
\end{exo}

\begin{exo}
    Soit un groupe \(\left(G,\times\right)\).
    \begin{q}{1}
        Démontrer que si \(g^2\times h^2=(g\times h)^2\) pour tout \(g,h\in G\)
        alors \(G\) est abélien.
        \boxans{On a l'égalité entre \((g\times h)^2 = g\times h\times g\times h\)
        et \(g\times g\times h\times h\). En composant à gauche par \(g^{-1}\) et
        à droite par \(h^{-1}\) on a la commutativité.}
    \end{q}
    \begin{q}{2}
        Démontrer que si \(g^2=e\) pour tout \(g\in G\) alors \(G\) est abélien.
        \boxans{Soient \(g,h\in G\) on sait que \(\left(gh\right)^{-1}=h^{-1}g^{-1}\)
        or par hypothèse \((gh)^2=e\) donc en composant par \((gh)^{-1}\) à gauche
        on a \(gh = hg\) car \(h^{_1}=h\) et \(g^{-1}=g\), on a la commutativité.}
    \end{q}
\end{exo}

\begin{exo}
    Soit un groupe \(\left(G,\times\right), s,t\in G\) on suppose que \(s^2=t^2=e\)
    et \(s\times t\times s = t\times s \times t\). Soit \(n\in\N^*\) et
    \(a_1\dots a_n\in\{s,t\}\). Démontrer que \(\prod a_k \in
    \{e,s,t,s\times t\times s, s\times t, t\times s\}\)
    \boxans{Montrons que tout produit de \(s\) et \(t\) de \(4\) éléments peut
    s'écrire comme un produit d'au plus trois élément. On commence par remarquer
    que si \(s\) ou \(t\) est présent deux fois d'affilé, on obtient un \(e\) ce
    qui donne bien un produit de \(3\) termes ou moins. Sinon, on a un produit de
    la forme \(s\times t\times s\times t\) or la deuxième égalité de l'énoncé donne
    que le produit vaut \(s\times s\times t\times s\) et on retombe sur le cas précédent.
    Alors par récurrence, tout produit de \(s\) et \(t\) peut s'écrire
    comme produit d'au plus trois termes.}
\end{exo}

\begin{exo}
    Dans quel(s) cas \(H\) est-il un sous groupe de \(\left(G,\times\right)\)
    \begin{q}{1}
        \(G=\textrm{GL}_2(\R)\) et \(H = \left\{
            \begin{pmatrix} 1 & 0 \\ 0 & 1 \end{pmatrix},
            \begin{pmatrix} 1 & 0 \\ 0 & -1 \end{pmatrix},
            \begin{pmatrix} -1 & 0 \\ 0 & 1 \end{pmatrix},
            \begin{pmatrix} -1 & 0 \\ 0 & -1 \end{pmatrix}\right\}\)
        \boxans{On a ici un cas d'école, \(H\) est le sous groupe des quaternions.}
    \end{q}
    \begin{q}{2}
        \(G=\Q\) et étant donné \(a\in\Z^*, H=\left\{ x\in\Q\mid\exists n\in\N,a^nx\in\Z \right\}\)
        \boxans{\(H\) n'est pas un sous groupe de \(\Q\) en effet \(a+1\)
        ne possède pas d'inverse dans \(H\) car \(a\land a+1=1\). On a ici traité le cas
        où l'on considére le groupe multiplicatif, en voyant \(G\) comme un
        groupe additif \(H\) est bien un sous groupe.}
    \end{q}
    \begin{q}{3}
        \(G=\glnr\) et \(H=\glnr\cap M_n(\Z)\)
        \boxans{Ce n'est pas un groupe, en effet \(\det(A^{-1})=\det(A)^{-1}\) donc toute
        matrice inversible à coefficients dans \(\Z\) et à déterminant non unitaire
        n'aura pas son inverse dans \(H\).}
    \end{q}
    \begin{q}{4}
        \(G=\textrm{GL}_2(\R)\) et \(H=O_2(\Z)\)
        \boxans{On a bien \(I_2\in H\), l'inverse est bien dans \(H\) d'après
        la question précédente et l'associativité provient de celle dans \(G\),
        on a donc bien un sous groupe.}
    \end{q}
    \begin{q}{5}
        \(G=\Z\) et \(H=\cup_{n=1}^{100} 2^n\Z\) et.
        \boxans{Chaque groupe est inclus dans le précédent ainsi \(H=2\Z\)
        qui est bien un sous groupe de \(\Z\).}
    \end{q}
\end{exo}

\begin{exo} On se propose de s'échauffer avec des sous-groupes de \(\Z\).
    \begin{q}{1}
        Justifier que \(2\Z\cap3\Z\) est un sous groupe de \(\Z\). Faire de même
        pour \(4\Z\cap6\Z\cap8\Z\).
        \boxans{L'intersection de sous-groupes étant un sous groupes, on a le résultat
        souhaité pour les deux groupes qui sont respectivement \(6\Z\) et \(24\Z\).}
    \end{q}
    \begin{q}{2}
        Soient \(a,b\in\Z\) on note \(a\Z+b\Z\) l'ensemble \(\left\{ au+bv
        | u,v\in\Z \right\}\).
        \begin{q}{a}
            Démontrer que \(a\Z+b\Z=\langle a,b\rangle\).
            \boxans{\(\langle a,b\rangle\) est le groupe généré par \(a\)
            et \(b\) qui est donc inclus dans \(a\Z+b\Z\). Or la stabilité du groupe
            impose l'inclusion réciproque et donc l'égalité.}
        \end{q}
        \begin{q}{b}
            Démontrer que \(\langle a,b\rangle=\gcd(a,b)\Z\).
            \boxans{D'après le théorème de \textsc{Bézout} le minimum non nul de
            l'ensemble \(a\Z+b\Z\) est le pgcd de \(a\) et \(b\). Ainsi
            \(\langle a,b\rangle\) est un sous groupe de \(\Z\) d'élément minimal
            \(\gcd(a,b)\) donc on a \(\langle a,b\rangle=\gcd(a,b)\Z\)}
        \end{q}
    \end{q}
\end{exo}

\begin{exo}
    On note \(\U_n\) l'ensemble des racines \(n\)-ièmes de l'unité, et \(\U\) le cercle unité.
    \begin{q}{1}
        Démontrer que \(\U\) est un sous groupe de \(\left(C,\times\right)\)
        \boxans{On a bien \(1\in\U\), montrons ensuite la stabilité par opération
        interne et passage à l'inverse. Soit \(z,z'\in\U\) on a \(|zz'|=|z||z'|=1\)
        donc \(zz'\in\U\)}
    \end{q}
    \begin{q}{2}
        Soit \(n\in\N^*\) décrire \(\U_n\) à l'aide de l'écriture exponentielle
        des nombres complexes et démontrer que c'est un sous groupe de
        \(\left(C,\times\right)\).
    \end{q}
    \begin{q}{3}
        Décrire \(\U_6\), quels sont ses sous groupes ?
    \end{q}
    \begin{q}{4}
        Soit \(n\in\N_0\) démontrer que \(U_n\) est monogène.
    \end{q}
    \begin{q}{5}
        Démontrer que \(\cup_{n\in\N^*}\U_n\) est un sous groupe de \(\C^*\).
    \end{q}
    \begin{q}{6}
        Démontrer que \(\cup_{n\in\N^*}\U_n\) est infini et que tous ses sous groupes
        monogènes sont finis.
    \end{q}
\end{exo}

\begin{exo} \textit{(Initiation au groupe dihédral)}
    Soit \(n\) un entier tel que \(\geq 2\). On note \(R\) et \(S\) les matrices
    respectives \(\begin{pmatrix}
        \cos(\frac{2\pi}{n}) & -\sin(\frac{2\pi}{n}) \\
        \sin(\frac{2\pi}{n}) & \cos(\frac{2\pi}{n})
    \end{pmatrix}\) et \(\begin{pmatrix}1 & 0 \\ 0 & -1\end{pmatrix}\)
    du groupe \(\left(O_2(\R), \times\right)\)
    \begin{q}{1}
        Démontrer que \(S^2=R^n=I_2\) et que \(RS=SR^{-1}\)
    \end{q}
    \begin{q}{2}
        Décrire \(\langle R,S\rangle\) et démontrer que \(|\langle R,S\rangle|=2n\)
    \end{q}
\end{exo}

\begin{exo}
    Soit \(V\) un \(\R\)-ev, les sous-espaces vectoriels de \(V\) sont ils tous des
    sous-groupes de \(\left(V,+\right)\) ? Les sous groupes de \(\left(V,+\right)\)
    sont ils tous des sous-espaces vectoriels de \(V\)?
    \boxans{Oui, un sev est un groupe, par définition. Tout groupe de \((V,+)\) non nul
    est soit un réseau soit un sev.}
\end{exo}

\begin{exo}
    Soient \(H\) et \(K\) deux sous-groupes d'un groupe \(\left(G,\star\right)\).
    Démontrer \(H\cup K<G\Leftrightarrow (H\subset K \lor K\subset H)\).
    \boxans{Supposons que \(H \cup K\) est un sous groupe de \(G\) et que
    \(H\not\subseteq K\). Alors on exhibe \(h\in H\backslash K\). Alors pour tout
    \(k\in K, h\star k\in H \cup K\) par stabilité, or ces éléments ne sont pas
    dans \(K\) sans quoi \(h\in K\) s'obtient de la stabilité par passage
    à l'inverse, ce qui donne que tous les \(hk\) sont dans \(H\) et donc \(K\subseteq H\).
    Ce qui donne le résultat voulu avec un raisonnement par symétrie.}
\end{exo}

\begin{exo}
    Soit \(\left(G,\star\right)\) un groupe, on note \(Z(G)\) son centre
    \(\{z\in G\mid \forall g\in G : z\star g = g\star z\}\).
    \begin{q}{1}
        Démontrer que \(Z(G)\) est un sous-groupe abélien.
        \boxans{Si \(g\) commute alors \(g^{-1}\) commute (ce qu'on peut remarquer
        avec des choses de la forme \((gh)^{-1}\)), c'est aussi stable par \(\star\)
        c'est juste du poussage de craie, enfin l'élément neutre commute par définition,
        ce qui donne bien que \(Z(G)\) est un sous groupe abélien.}
    \end{q}
    \begin{q}{2}
        Déterminer \(Z(\textrm{GL}_2(\R))\)
        \boxans{Ce sont les homothéties.}
    \end{q}
\end{exo}

\begin{exo}
    Soit \(\left(G,\star\right)\) un groupe fini.
    \begin{q}{1}
        On suppose que \(G\) est d'ordre pair. Démontrer que le cardinal
        de l'ensemble \(\{g\in G\mid g=g^{-1}\}\) est pair et non nul.
    \end{q}
    \begin{q}{2}
        On suppose que \(g\mapsto g^2\) est un isomorphisme de groupe. Démontrer que
        \(G\) est d'ordre impair.
        \boxans{Si l'application est bijective alors en particulier elle est injective
        ainsi \(1_G\) possède un unique antécédent, donc par contraposée de la question
        précédente, \(G\) est d'ordre impair.}
    \end{q}
\end{exo}

\begin{exo}
    Soit \(H\) un sous-groupe de \(\left(\R,+\right)\). On suppose que \(H\) n'est pas dense.
    \begin{q}{1}
        Démontrer que \(H\) possède un plus petit élément strictement positif \(x_0\).
        \boxans{Si \(H\) n'est pas dense alors tous ses éléments sont au plus à une distance \(\mu\)
        non nulle les uns des autres donc on a une bijection avec \(\N\) pour
        les positifs, donc on peut distinguer un minium \(x_0\)}
    \end{q}
    \begin{q}{2}
        Démontrer que \(H=x_0\Z\)
        \boxans{Par division euclidienne on a \(H \subset x_0\Z\) l'autre sens est immédiat.}
    \end{q}
\end{exo}

\begin{exo}
    \textbf{(IL FAUDRA PASSER AU TABLEAU)}
    Soient \(a,b\in\R^\times\).
    \begin{q}{1}
        Démontrer que \(H=a\Z+b\Z\) est un sous-groupe de \(\left(\R,+\right)\)
        \boxans{On a bien la présence de l'élément neutre \(0\) dans \(H\). On
        s'assure ensuite de la stabilité. Soient \(x=an+bm\) et \(x'=an'+bm'\)
        avec \((n,n',m,m')\in\Z\) alors par commutativité de l'addition dans \(\R\)
        et stabilité de \(\Z\) par l'addition, \(x+x'=a(n+n')+b(m+m')\in H\).
        Aussi tout élément \(x\) admet \(-x\) comme inverse trivial.}
    \end{q}
    \begin{q}{2}
        Démontrer que \(H\) est dense dans \(\R\) si et seulement si
        \(\frac{a}{b}\not\in\Q\)
        \boxans{\begin{enumerate}
            \itt Supposons par contraposée que \(\frac{a}{b}\in\Q\) alors il existe
            \(\left(p,q\right)\in\Z\times\N^*\) tel que \(\frac{a}{b}=\frac{p}{q}\)
            et \(a\land q=1\). Soit \(u=an+bm\in a\Z+b\Z, u= b(\frac{p}{q}n + m) = \frac{b}{q}(pn+qm)\)
            ainsi \(a\Z+b\Z\subset \frac{b}{q}\Z\) donc \(H\) n'est pas dense.
            \itt Supposons que \(\frac{a}{b}\not\in\Q\) alors par densité de \(\Q\)
            dans \(\R\) on exhibe une suite \(\left(\frac{\alpha_n}{\beta_n}\right)_{n\in\N}\)
            de rationnels telle que \(\frac{\beta_n}{\alpha_n}\to \frac{a}{b}\).
            Alors \(\beta_n b - \alpha_n a\to 0\) et on a la densité de \(H\).
        \end{enumerate}}
    \end{q}
\end{exo}

\begin{exo}
    Soit un groupe \(\left(G,\star\right)\). On suppose qu'il existe
    des éléments de \(G\) notés \(s,i,j\) et \(k\) tels que \(G=\langle
    i,j,k\rangle\) et tels que \(i^2=j^2=k^2=i\star j\star k=s, s^2=e,
    s\neq e\) et \(s\in Z(G)\).
    \begin{q}{1}
        Démontrer que \(|G|=8\). Décrire \(G\) et la table de \(\star\)
    \end{q}
    \begin{q}{2}
        \(G\) est-il abélien ?
    \end{q}
    \begin{q}{3}
        Quels sont ses sous-groupes ?
    \end{q}
    \begin{q}{4}
        Démontrer que le sous-groupe de \(\mathrm{GL}_2(\C)\) engendré
        par les matrices \(\begin{pmatrix} i&0\\0&-i\end{pmatrix}\begin{pmatrix}
        0&1\\-1&0\end{pmatrix}\begin{pmatrix} 0&i\\i&0\end{pmatrix}\) vérifie
        les hypothèses de l'énoncé.
    \end{q}
\end{exo}

\end{document}