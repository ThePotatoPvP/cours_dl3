\documentclass{report}
\usepackage{../../exercices}

\begin{document}

\begin{center}
    \huge{\textbf{Relations d'équivalence, Ensembles quotient}}
\end{center}
\bigskip
mails des teachers :

\begin{exo}
    Soit \(E=\N\times\N\). On définit la relation\(\mathcal{R}\) sur \(E\) par
    \[\left(a,b\right)\mathcal{R}\left(a',b'\right)\Leftrightarrow a+b'=a'+b\]
    \begin{q}{1}
        Montrer que \(\mathcal{R}\) est une relation d'équivalence.
        \boxans{Soient \(\left(a,b\right), (a',b'), \left(\alpha, \beta\right)\in E\)
        \begin{enumerate}
            \itt \(a+b = a+b\) par réflexivité de l'égalité donc \(\mathcal{R}\) est réflexive.
            \itt De même, la symétrie de l'égalité donne imédiatement la symétrie de \(\mathcal{R}\).
            \itt Supposons \((a,b)\mathcal{R}(a',b')\) et \((a',b')\mathcal{R}(\alpha,\beta)\) alors
            \[a+b'=a'+b \quad \text{et} \quad a'+\beta=\alpha+b'\]
            en sommant les deux équations on a \(a+\beta=\alpha+b\) et
            donc \((a,b)\mathcal{R}(\alpha, \beta)\) et ainsi la transitivité de
            \(\mathcal{R}\) qui est donc bien une relation d'équivalence.
        \end{enumerate}}
    \end{q}
    \begin{q}{2}
        Soit \(\left(a,b\right)\in E\). Déterminer cl\(\left(a,b\right)\).
        \boxans{On remaque que cl\((a,b)\) est l'ensemble des couples dont
        les deux éléments sont à distance constante, soit \(\{(n\Z, b-a+n\Z)|n\in\Z\}\cap E\)}
    \end{q}
    \begin{q}{3}
        Démontrer qu'il existe une application de \(E/\mathcal{R}\) dans \(\Z\)
        telle que, pour tout \(\left(m,n\right)\in E\) l'image de la classe d'équivalence
        par cette relation soit \(m-n\).
        \boxans{Avec la remarque de la question précédente on a bien que l'image ne dépend pas du représentant donc l'application est bien définie.}
    \end{q}
    \begin{q}{4}
        Démontrer que l'application obtenue à la question précédente est bijective.
        \boxans{On construit la réciproque \(\varphi : n \mapsto \{(n +k\Z, k\Z)|k\in \N\}\) qui délivre la bijectivité.}
    \end{q}
\end{exo}

\begin{exo}
    Soit \(E=\C\). On définit la relation \(\mathcal{R}\) sur \(E\) par
    \[z\mathcal{R}z'\Leftrightarrow|z|=|z'|\]
    \begin{q}{1}
        Montrer que \(\mathcal{R}\) est une relation d'équivalence.
        \boxans{Soient \(z, z', \zeta \in E\)
        \begin{enumerate}
            \itt On a trivialement \(|z|=|z|\) donc la relation \(\mathcal{R}\) est réflexive.
            \itt La symétrie provient de la symétrie de l'égalité.
            \itt Supposons \(z\mathcal{R}z'\) et \(z'\mathcal{R}\zeta\) alors \(|z|=|z'|=|\zeta|\).
            La transitivité s'obtient de l'égalité et la relation est bien d'équivalence.
        \end{enumerate}}
    \end{q}
    \begin{q}{2}
        Soit \(z\in E\). Décrire la classe d'équivalence de \(z\) à l'aide de l'écriture
        exponentielle des complexes, la représenter sur un dessin.
        \boxans{La classe d'équivalence de \(z\in E\) est cl\((z)=\{|z|e^{i\theta}\mid\theta\in\R\}\). On se passera de la dessiner, c'est le cercle d'origine 0 et de rayon \(|z|\)}
    \end{q}
    \begin{q}{3}
        Démontrer qu'il existe une application de \(E/\mathcal{R}\) dans \(\R^+\) telle
        que pour tout \(z\in E\) l'image de la classe d'équivalence de \(z\) par cette application est \(|z|\).
        \boxans{Chaque représentant d'une même classe ayant le même module, l'application est bien définie.}
    \end{q}
    \begin{q}{4}
        Démontrer que l'application obtenue à la question précédente est bijective.
        \boxans{En utilisant la deuxième question on peut construire une réciproque : \(\varphi : |z| \mapsto \{|z|e^{i\theta}\mid\theta\in\R\}\)}
    \end{q}
\end{exo}

\begin{exo}
    Soit \(E=\R[X]\) et \(B\in E\). On définit la relation \(\mathcal{R}\) sur \(E\) par
    \[P\mathcal{R}Q \Leftrightarrow B\mid\left(P-Q\right)\]
    \begin{q}{1}
        Montrer que \(\mathcal{R}\) est une relation d'équivalence.
        \boxans{Deux polynômes sont en relation si le reste de leur division euclidienne
        par \(B\) est le même, l'égalité étant une relation d'équivalence, \(\mathcal{R}\)
        l'est aussi.}
    \end{q}
    \begin{q}{2}
        On suppose \(B\) non nul. Démontrer qu'il existe une application
        de \(\R[X]/\mathcal{R}\) dans \(\R_{\deg(B)-1}[X]\) telle que, pour tout
        \(A\in\R[X]\), l'image de la classe d'équivalence de \(A\) par cette application
        soit le reste de la division euclidienne de \(A\) par \(B\).
        \boxans{Deux polynômes en relation ayant le même reste, une telle application
        est bien définie.}
    \end{q}
    \begin{q}{3}
        Démontrer que l'application obtenue à la question précédente est bijective.
        \boxans{Pour démontrer la bijectivité on construit une réciproque \(\varphi : \R[X]/\mathcal{R} \to \R[X] :
        P \mapsto P + B\R[X]\).}
    \end{q}
\end{exo}

\begin{exo}
    Pour cet exercice on pose :
    \begin{enumerate}
        \itt \(\varphi\) est l'application \(t\mapsto e^{2i\pi t}\)
        \itt \(\mathcal{C} = \{g\in\CC^0\left([0,1],\R\right)\mid g(0)=0, g(1)=2\pi\Z\}\)
        \itt \(\mathcal{L}=\{f\in\CC^0\left([0,1],\mathbb{U}\right)\mid f(0)=f(1)=1\}\)
    \end{enumerate}
    \begin{q}{1}
        On note \(\mathcal{R}\) la relation binaire sur \(\R\) définie par \(x\mathcal{R}y\Leftrightarrow x-y\in\Z\).
        Démontrer que \(\mathcal{R}\) est une relation d'équivalence et que
        \(\varphi\) induit une application bijective de \(\R/\mathcal{R}\) sur \(\mathbb{U}\).
        \boxans{\(\left(\Z,-\right)\) étant trivialement un monoïde, \(\mathcal{R}\) est
        une relation d'équivalence. Des représentants de \(\R/\mathcal{R}\) sont par exemple \([0,1[\)
        ce qui donne bien que \(\varphi\) induit une application bijective comme demandé.}
    \end{q}
    \begin{q}{2}
        On note \(\equiv\) la relation binaire sur \(\CC\) telle que pour tout \(g_0, g_1\in\CC\) on a \(g_0\equiv g_1\)
        si et seulement si il existe une application continue \(h: [0,1] \times [0,1] \to \R\) telle que
        \(h(s,-)\in\CC\) pour tout \(s\in[0,1]\) et \(h(0,\cdot)=g_0\) et
        \(h(1,\cdot)=g_1\). Démontrer que \(\equiv\) est une relation d'équivalence
        \boxans{\(\equiv\) est réfléxive en prenant \(h(s,\cdot)=g\) par exemple. Elle est
        aussi symétrique en effet si \(g_0\mathcal{R}g_1\) avec une fonction \(h\) on pose \(h'\)
        tel que \(h'(s,\cdot)=h(1-s,\cdot)\) donne bien \(g_1\mathcal{R}g_0\).
        La transitivité repose sur l'idée de poser \(h''(s,\cdot)=h(2s,\cdot)\) pour \(s\leq \frac12\)
        et \(h''(s, \cdot) = h(2s, \cdot)\). On connecte les deux chemins, \(g_0\) et \(g_2\) sont dans la
        même composante connexe, on a le résultat voulu.}
    \end{q}
    \begin{q}{3}
        On note \(\sim\) la relation binaire sur \(\mathcal{L}\) telle que
        pour tout \(f_0,f_1\in\mathcal{L}\) on a \(f_0\sim f_1\) si et seulement si
        il existe une application \(h\) définit de façon semblable à au dessus.
        Démontrer que \(\sim\) est une relation d'équivalence.
        \boxans{Le raisonnement est exactement le même qu'à la question précédente où
        on construit des arcs de connexité.}
    \end{q}
    \begin{q}{4}
        Démontrer que l'application \(g\mapsto \varphi\circ g\) induit
        une application de \(\CC/\equiv\) dans \(\mathcal{L}/\sim\)
        \boxans{Soient \(g_0, g_1\in\CC\) tel que \(g_0\equiv g_1\) montrons
        \(\varphi \circ g_0 \sim \varphi \circ g_1\). On sait par hypothèse qu'il
        existe \(h(s,\cdot)\in\CC\) vérifiant \(h(0,\cdot)=g_0\) et \(h(1,\cdot)=g_1\)
        On pose alors \(\eta = \varphi \circ h\) qui est continue comme composée de fonction
        continue et vérifie bien \(\eta(0)=\eta(1)=1\). L'application induite
        est alors bien défnie.}
    \end{q}
\end{exo}

\begin{exo}
    On note \(f\) l'application \(\R\to\R\), \(x\mapsto 2x^3+3x^2\). On note \(\mathcal{R}\)
    la relation d'équivalence sur \(\R\) telle que \(x\mathcal{R}y\Leftrightarrow f(x)=f(y)\).
    \begin{q}{1}
        Pour tout \(x\in\R\). Déterminer le cardinal de la classe d'équivalence de \(x\).
        \boxans{Soit \(x\in\R\) on a \(|\cl(x)|=3\) pour \(x\in[0,1]\) et \(|\cl(x)|=1\) sinon.}
    \end{q}
    \begin{q}{2}
        Déterminer un système de représentants.
        \boxans{Un système de représentant pour \(\mathcal{R}\) est \((-\infty,-\frac12) \cup [0, +\infty)\)}
    \end{q}
\end{exo}

\begin{exo}
    Soit \(E\) un ensemble. Soit \(A\in\mathcal{P}(E)\). On note \(\mathcal{R}\) la relation binaire de \(\mathcal{P}(E)\)
    telle que : \(X\mathcal{R}Y \Leftrightarrow X\cap A = Y\cap A\).
    \begin{q}{1}
        Démontrer que \(\mathcal{R}\) est une relation d'équivalence.
        \boxans{Par symétrie des opérations ensemblistes, \(\mathcal{R}\) est une relation d'équivalence}
    \end{q}
    \begin{q}{2}
        Démontrer que \(\mathcal{P}(A)\) est un sysème de représentants
        \boxans{On remarque que chaque élément de \(\mathcal{P}(A)\) est dans une classe
        différente, en effet si \(X\mathcal{R}Y\) alors les deux on la même partie de \(A\) en commun.
        Supposons maintenant un élément \(X\) de \(\mathcal{P}(E)\) montrons qu'il est dans la même
        classe qu'un représentant de \(\mathcal{P}(A)\), on pose \(Y = X\cap A\) mézalor
        \(Y\in \mathcal{P}(A)\) donc on a directement \(X\mathcal{R}Y\) ce qui livre le résultat escompté.}
    \end{q}
\end{exo}

\end{document}