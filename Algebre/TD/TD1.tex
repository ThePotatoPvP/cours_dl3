\documentclass{report}
\usepackage[T1]{fontenc}
\usepackage[utf8x]{inputenc}
\usepackage{lmodern}
\usepackage{amsmath,amsthm,amsfonts,amssymb}
\usepackage{graphicx}
\usepackage[shortlabels]{enumitem}
\usepackage{xcolor}
\usepackage{accents}
\usepackage{titlesec}
\usepackage{etoolbox}
\usepackage{bookmark}
\usepackage{stmaryrd}
\usepackage{fancyhdr}
\usepackage[margin=25mm]{geometry}
\usepackage[frenchb]{babel}
\usepackage{changepage}
\usepackage{listings}
\usepackage[most,breakable,listings]{tcolorbox}

\date{\today}
\author{Corentin Sallin \thanks{Cours de L3}}
\title{Calcul Différentiel et topologie}


\renewcommand{\theenumi}{\Alph{enumi}}

\newcommand{\N}{\mathbb{N}}
\newcommand{\Z}{\mathbb{Z}}
\newcommand{\R}{\mathbb{R}}
\newcommand{\Q}{\mathbb{Q}}
\newcommand{\C}{\mathbb{C}}
\newcommand{\K}{\mathbb{K}}
\renewcommand{\L}{\mathcal{L}}
\newcommand{\M}{\mathcal{M}}
\newcommand{\T}{\mathcal{T}}
\newcommand{\Diag}{\mathcal{D}}
\newcommand{\curlyv}{\mathcal{V}}

\newcommand{\esp}{\mathbb{E}}
\newcommand{\proba}{\mathbb{P}}


\newcommand{\CC}{\mathcal{C}}
\newcommand{\Int}{\mathrm{Int}}
\newcommand{\id}{\mathrm{id}}
\newcommand{\eps}{\varepsilon}
\newcommand{\mnr}{\mathcal{M}_n(\R)}
\newcommand{\mnk}{\mathcal{M}_n(\K)}
\newcommand{\glnr}{\mathrm{GL}_n(\R)}
\newcommand{\glnk}{\mathrm{GL}_n(\K)}

\newcommand{\D}{\mathop{}\!\mathrm{d}}
\newcommand{\ds}{\displaystyle}
\newcommand*{\ensemble}[3][]{#1\{ #2 \mid #3 #1\}}
\newcommand{\Vvert}{\vert\kern-0.25ex\vert\kern-0.25ex\vert}
\newcommand{\tnorm}[1]{\Vvert #1 \Vvert}

\DeclareMathOperator{\Ker}{Ker}
\DeclareMathOperator{\Isom}{Isom}
\DeclareMathOperator{\Tr}{Tr}

\pagestyle{fancy}
\fancyhf{}
\fancyhead[L]{Corentin Sallin}
\fancyhead[R]{}
\fancyfoot[L]{\leftmark}
\fancyfoot[R]{Page : \thepage}

\renewcommand{\headrulewidth}{2pt}
\renewcommand{\footrulewidth}{1pt}


\theoremstyle{definition}
\newtheorem{exo}{Exercice}
%\newtheorem{q}{\(\quad\)}[exo]

\newtheorem{theorem}{Théorème}[section]
\newtheorem{lemma}[theorem]{Lemme}
\newtheorem{definition}[theorem]{Définition}
\newtheorem{lemme}[theorem]{Lemme}
\newtheorem*{corollary}{Corollaire}

\newtheorem{proposition}{Proposition}[theorem]
\newtheorem{rk}[proposition]{Remarque}
\newtheorem*{example}{Exemple}

\definecolor{main1white} {RGB}{121, 129, 134}
\definecolor{main1white2} {RGB}{204, 204, 204}

\newenvironment{q}[1]{
    \begin{adjustwidth}{1cm}{}
    \textbf{#1} : }{
    \end{adjustwidth}}

\titleformat{\chapter}[display]
    {\normalfont\bfseries}{}{0pt}{\Huge}

\patchcmd{\chapter}{\thispagestyle{plain}}{\thispagestyle{fancy}}{}{}

\newcommand{\boxans}[1]{
    \begin{tcolorbox}[
            breakable,
            enhanced,
            interior style      = {
                left color      = main1white2!65!gray!8,
                middle color    = main1white2!50!gray!7,
                right color     = main1white2!35!gray!6
            },
            %borderline north    = {.3pt}{0pt}{main1white!10},
            %borderline south    = {.3pt}{0pt}{main1white!10},
            frame hidden,
            borderline west     = {2pt}{0pt}{main1white!30},
            sharp corners       = downhill,
            arc                 = 0 cm,
            boxrule             = 0 cm,
            %nobeforeafter,
            %before={},
            %nobeforeSTYLE,
            %noafterSTYLE,
            %after=\par\nointerlineskip
            %source=remy
        ]
        #1
    \end{tcolorbox}
}

\newcommand{\itt}{\item[\(\triangleright\)]}

\begin{document}

\begin{center}
    \huge{\textbf{Relations d'équivalence, Ensembles quotient}}
\end{center}
\bigskip
mails des teachers :

\begin{exo}
    Soit \(E=\N\times\N\). On définit la relation\(\mathcal{R}\) sur \(E\) par
    \[\left(a,b\right)\mathcal{R}\left(a',b'\right)\Leftrightarrow a+b'=a'+b\]
    \begin{q}{1}
        Montrer que \(\mathcal{R}\) est une relation d'équivalence.
        \boxans{Soient \(\left(a,b\right), (a',b'), \left(\alpha, \beta\right)\in E\)
        \begin{enumerate}
            \itt \(a+b = a+b\) par réflexivité de l'égalité donc \(\mathcal{R}\) est réflexive.
            \itt De même, la symétrie de l'égalité donne imédiatement la symétrie de \(\mathcal{R}\).
            \itt Supposons \((a,b)\mathcal{R}(a',b')\) et \((a',b')\mathcal{R}(\alpha,\beta)\) alors
            \[a+b'=a'+b \quad \text{et} \quad a'+\beta=\alpha+b'\]
            en sommant les deux équations on a \(a+\beta=\alpha+b\) et
            donc \((a,b)\mathcal{R}(\alpha, \beta)\) et ainsi la transitivité de
            \(\mathcal{R}\) qui est donc bien une relation d'équivalence.
        \end{enumerate}}
    \end{q}
    \begin{q}{2}
        Soit \(\left(a,b\right)\in E\). Déterminer cl\(\left(a,b\right)\).
        \boxans{On remaque que cl\((a,b)\) est l'ensemble des couples dont
        les deux éléments sont à distance constante, soit \(\{(n\Z, b-a+n\Z)|n\in\Z\}\cap E\)}
    \end{q}
    \begin{q}{3}
        Démontrer qu'il existe une application de \(E/\mathcal{R}\) dans \(\Z\)
        telle que, pour tout \(\left(m,n\right)\in E\) l'image de la classe d'équivalence
        par cette relation soit \(m-n\).
        \boxans{Avec la remarque de la question précédente on a bien que l'image ne dépend pas du représentant donc l'application est bien définie.}
    \end{q}
    \begin{q}{4}
        Démontrer que l'application obtenue à la question précédente est bijective.
        \boxans{On construit la réciproque \(\varphi : n \mapsto \{(n +k\Z, k\Z)|k\in \N\}\) qui délivre la bijectivité.}
    \end{q}
\end{exo}

\begin{exo}
    Soit \(E=\C\). On définit la relation \(\mathcal{R}\) sur \(E\) par
    \[z\mathcal{R}z'\Leftrightarrow|z|=|z'|\]
    \begin{q}{1}
        Montrer que \(\mathcal{R}\) est une relation d'équivalence.
        \boxans{Soient \(z, z', \zeta \in E\)
        \begin{enumerate}
            \itt On a trivialement \(|z|=|z|\) donc la relation \(\mathcal{R}\) est réflexive.
            \itt La symétrie provient de la symétrie de l'égalité.
            \itt Supposons \(z\mathcal{R}z'\) et \(z'\mathcal{R}\zeta\) alors \(|z|=|z'|=|\zeta|\).
            La transitivité s'obtient de l'égalité et la relation est bien d'équivalence.
        \end{enumerate}}
    \end{q}
    \begin{q}{2}
        Soit \(z\in E\). Décrire la classe d'équivalence de \(z\) à l'aide de l'écriture
        exponentielle des complexes, la représenter sur un dessin.
        \boxans{La classe d'équivalence de \(z\in E\) est cl\((z)=\{|z|e^{i\theta}\mid\theta\in\R\}\). On se passera de la dessiner, c'est le cercle d'origine 0 et de rayon \(|z|\)}
    \end{q}
    \begin{q}{3}
        Démontrer qu'il existe une application de \(E/\mathcal{R}\) dans \(\R^+\) telle
        que pour tout \(z\in E\) l'image de la classe d'équivalence de \(z\) par cette application est \(|z|\).
        \boxans{Chaque représentant d'une même classe ayant le même module, l'application est bien définie.}
    \end{q}
    \begin{q}{4}
        Démontrer que l'application obtenue à la question précédente est bijective.
        \boxans{En utilisant la deuxième question on peut construire une réciproque : \(\varphi : |z| \mapsto \{|z|e^{i\theta}\mid\theta\in\R\}\)}
    \end{q}
\end{exo}

\begin{exo}
    Soit \(E=\R[X]\) et \(B\in E\). On définit la relation \(\mathcal{R}\) sur \(E\) par
    \[P\mathcal{R}Q \Leftrightarrow B\mid\left(P-Q\right)\]
    \begin{q}{1}
        Montrer que \(\mathcal{R}\) est une relation d'équivalence.
    \end{q}
    \begin{q}{2}
        On suppose \(B\) non nul. Démontrer qu'il existe une application
        de \(\R[X]/\mathcal{R}\) dans \(\R_{\deg(B)-1}[X]\) telle que, pour tout
        \(A\in\R[X]\), l'image de la classe d'équivalence de \(A\) par cette application
        soit le reste de la division euclidienne de \(A\) par \(B\).
    \end{q}
    \begin{q}{3}
        Démontrer que l'application obtenue à la question précédente est bijective.
    \end{q}
\end{exo}

\begin{exo}
    Pour cet exercice on pose :
    \begin{enumerate}
        \itt \(\varphi\) est l'application \(t\mapsto e^{2i\pi t}\)
        \itt \(\mathcal{C} = \{g\in\CC^0\left([0,1],\R\right)\mid g(0)=0, g(1)=2\pi\Z\}\)
        \itt \(\mathcal{L}=\{f\in\CC^0\left([0,1],\mathbb{U}\right)\mid f(0)=f(1)=1\}\)
    \end{enumerate}
    \begin{q}{1}
        On note \(\mathcal{R}\) la relation binaire sur \(\R\) définie par \(x\mathcal{R}y\Leftrightarrow x-y\in\Z\).
        Démontrer que \(\mathcal{R}\) est une relation d'équivalence et que
        \(\varphi\) induit une application bijective de \(\R/\mathcal{R}\) sur \(\mathbb{U}\).
    \end{q}
    \begin{q}{2}
        On note \(\equiv\) la relation binaire sur \(\CC\) telle que pour tout \(g_0, g_1\in\CC\) on a \(g_0\equiv g_1\)
        si et seulement si il existe une application continue \(h: [0,1] \times [0,1] \to \R\) telle que
        \(h(s,-)\in\CC\) pour tout \(s\in[0,1]\) et \(h(0,\cdot)=g_0\) et
        \(h(1,\cdot)=g_1\). Démontrer que \(\equiv\) est une relation d'équivalence
    \end{q}
    \begin{q}{3}
        On note \(\sim\) la relation binaire sur \(\mathcal{L}\) telle que
        pour tout \(f_0,f_1\in\mathcal{L}\) on a \(f_0\sim f_1\) si et seulement si
        il existe une applciation \(h\) définit de façon semblable à au dessus.
        Démontrer que \(\sim\) est une relation d'équivalence.
    \end{q}
    \begin{q}
        Démontrer que l'application \(g\mapsto \varphi\circ g\) induit
        une application de \(\CC/\equiv\) dans \(\mathcal{L}/\sim\)
    \end{q}
\end{exo}

\begin{exo}
    On note \(f\) l'application \(\R\to\R\), \(x\mapsto 2x^3+3x^2\). On note \(\mathcal{R}\)
    la relation d'équivalence sur \(\R\) telle que \(x\mathcal{R}y\Leftrightarrow f(x)=f(y)\).
    \begin{q}{1}
        Pour tout \(x\in\R\). Déterminer le cardinal de la classe d'équivalence de \(x\).
    \end{q}
    \begin{q}{2}
        Déterminer un système de représentants.
    \end{q}
\end{exo}

\begin{exo}
    Soit \(E\) un ensemble. Soit \(A\in\mathcal{P}(E)\). On note \(\mathcal{R}\) la relation binaire de \(\mathcal{P}(E)\)
    telle que : \(X\mathcal{R}Y \Leftrightarrow X\cap A = Y\cap A\).
\end{exo}

\end{document}