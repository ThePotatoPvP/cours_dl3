\documentclass[french]{report}
\usepackage{../exercices}


\usepackage{causets}

\begin{document}

\begin{center}
    \huge{\textbf{S6- Maths Discrètes - TD3}}
\end{center}

\begin{exo}
    Le graphe \(k\)-parti complet \(K_{n_1,\dots,n_k}\) est le graphe dont l'ensemble
    des sommets de \(V\) est partitionné en ensembles \(V_1,\dots,V_k\) de sorte que
    \(|V_i|=n_i\) et tout sommet de \(V_i\) est connecté à tous les sommets
    de \(V_j\) pour tout \(j\neq i\). Déterminer l'ensemble des uples d'entiers
    pour lequels \(K_{n_1,\dots,n_k}\) est planaire.
\end{exo}

\begin{exo}
    Considérer un dessin de \(K_n\) dans le plan, montrer qu'au moins
    \(\frac15\binom{n}{4}\) paires d'arêts doivent se croiser.
\end{exo}

\begin{exo}
    Des bornes ...
    \begin{q}{1}
        Un graphe planaire est maximal si l'ajout d'une arête ne parmet plus de le
        dessiner sans croisements. On admettra ici que dans un graphe planaire maximal
        avec au moins 3 sommets toutes les régions sont délimitées par exactement 3
        arêtes. Montrer que dans un tel graphe \(G=(V,E)\) on a \(|E|=3|V|-6\).
        En déduire que dans un graphe planaire on a \(|E|\leq 3|V|-6\).
    \end{q}
    \begin{q}{2}
        Pour les graphes planaires sans triangles on peut montrer que \(|E|\leq 2|V|-4\).
        Montrer que cette borne est la meilleure possible.
    \end{q}
\end{exo}

\begin{exo}
    On dit qu'un graphe est planaire extérieur si on peut le déssiner de sorte qu'il
    existe une face dont la frontière contient tous les sommets du graphe. Montrer qu'un
    graphe planaire extérieur est toujours \(3\)-coloriable.
\end{exo}

\begin{exo}
    Soit \(G\) un graphe planaire dont tous le sommets sont de degré pair. Montrer que
    la carte obtenue à partir d'un dessin planaire de \(G\) est \(2\)-coloriable.
\end{exo}

\begin{exo}
    Montrer qu'un graphe planaire où chaque sommet est de degré au moins \(5\) doit avoir
    au moins \(12\) sommets.
\end{exo}

\begin{exo}
    Encore des bornes youpi
    \begin{q}{1}
        Montrer qu'un graphe planaire à \(n\) sommets a au plus \(2n-4\) faces.
    \end{q}
    \begin{q}{2}
        Montrer qu'un graphe planaire sans triangle à \(n\) sommets a au plus \(n-2\)
        faces
    \end{q}
\end{exo}

\begin{exo}
    On pousse un peu cette fois.
    \begin{q}{1}
        Donner un exemple de graphe planaire dont tous les sommets sont de degré \(4\)
    \end{q}
    \begin{q}{2}
        Dessiner le graphe suivant de manière planaire (sans arêtes qui se croisent).
    \end{q}
    \begin{q}{3}
        Rappeler la formule d'\textsc{Euler} reliant le nombre de sommets, d'arêtes
        et de face d'un graphe topologique planaire connexe.
    \end{q}
\end{exo}

\end{document}