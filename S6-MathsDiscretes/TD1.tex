\documentclass[french]{report}
\usepackage{../exercices}

\begin{document}

\begin{center}
    \huge{\textbf{S6- Maths Discrètes - TD1}}
\end{center}

\begin{exo}
    Soit \(X\) un ensemble. Trouver toutes les relations sur \(X\)
    qui sont à la fois des relations d'ordre et des relations d'équivalence.
    \boxans{Une telle relation doit être symétrique et anti-symétrique, si deux éléments
    sont comparables alors ils sont égaux, il n'existe donc que la relation qui lie chaque
    élément à lui même.}
\end{exo}

\begin{exo}
    Existe-t-il une partie stricte de \(\N\) en bijection avec \(\N\) ?
    \boxans{Oui, \(2\N\) est en bijection triviale avec \(\N\).}
\end{exo}

\begin{exo} Sur les relations
    \begin{q}{1}
        Combien de relations y a-t-il sur un ensemble de cardinal \(n\) ?
        \boxans{Chaque paire d'éléments peut être en relation où non, il y en a
        \(2\) puissance le nombre de paires, soit \(2^{\frac{n(n-1)}{2}}\)}
    \end{q}
    \begin{q}{2}
        Combien de relations d'équivalences y a-t-il sur un ensemble de cardinal \(5\) ?
        \boxans{Il y en a exactement le \(5\)-ième nombre de Bell, soit 52.}
    \end{q}
    \begin{q}{3}
        Et de relations d'ordre sur un ensemble de cardinal \(4\) ?
    \end{q}
\end{exo}

\begin{exo}
    Soit \(X\) un ensemble fini. Montrer qu'une fonction de \(X\) dans \(X\) est
    bijective si et seulement si elle est injective ou surjective.
\end{exo}

\begin{exo}
    Soit \(n\in\N^*\) et \(A=\{1,\dots,n\}\). Déterminer le nombre de fonctions \(f\)
    de \(A\to A\) telles que \(f(x)\leq x\) pour tout \(x\in A\). Combien sont injectives ?
    \boxans{On a \(k\) choix pour \(f(k)\) donc en tout celà fait \(\prod_{1}^n k = n!\).
    Si maintenant on veut que \(f\) soit injective, \(1\) ne peut en tous les cas qu'avoir
    \(1\) en image, ainsi \(2\) pourrait avoir \(1\) ou \(2\) mais est contraint \(2\) lorsqu'on
    force l'injectvité, par récurrence immédiate il n'y a que l'identité.}
\end{exo}

\begin{exo}
    Un tournoi d'échecs comporte 1025 participants. À chaque tour, un tirage au sort
    est organisé et les joueurs s'affrontent deux à deux; les perdants sont éliminés,
    jusqu'à ce qu'il ne reste qu'un vaincoeur. Lorsqu'il y a un nombre impair de joueurs,
    celui qui se retrouve tout seul est automatiquement qualifié, combien y a-t-il
    de parties jouées au total ?
    \boxans{Chaque partie élminie une personne, il doit rester 1 personne à la fin,
    il faut en éliminer 1024 donc 1024 parties.}
\end{exo}

\begin{exo}
    Soit \(X\) un ensemble à \(n\) éléments. Quel est le nombre de paires \(X_1,X_2\)
    telles que \(X_1\subseteq X, X_2\subseteq X\) et \(X_1\cap X_2=\emptyset\).
    \boxans{On raisonne de façon combinatoire, on a \(n+1\) choix pour le cardinal
    de \(X_1\), noté \(x_1\), une fois cela fixé on choisit ces éléments de
    \(\binom{n}{x_1}\) manières différentes. Une fois \(X_1\) fixé de cardinal \(x_1\),
    \(X_2\) peut être n'importe quel sous ensemble de \(X\backslash X_1\) il y a donc
    \(2^{n-x_1}\) possibilités. Finalement le nombre total de paires est
    \(\sum_{x_1=0}^n \binom{n}{x_1} 2^{n-x_1}\) ce qui vaut \(3^n\) d'après la formule
    du binôme de \textsc{Newton}. On aurait pu arriver à ce résultat plus facilement,
    chacun des \(n\) éléments est soit dans \(X_1\) soit dans \(X_2\) soit aucun.}
\end{exo}

\begin{exo}
    Montrer par une preuve de combinatoire que pour tout \(n\in\N^*\) :
    \[\sum_{i=0}^n\binom{n}{i}^2 = \binom{2n}{n}\]
    \boxans{Supposons une urne contenant \(2n\) boules, choisir \(n\) boules
    dedans revient séparer en deux urnes de \(n\) boules, choisir \(i\) boules dans la
    première puis \(n-i\) boules dans la seconde, ainsi par symmétrie des
    coefficients binomiaux, on a la réponse attendue.}
\end{exo}

\begin{exo}
    Supposons que dans un ensemble de \(x\) personnes, chacune parle soit le latin,
    soit l'anglais, soit le toucan. En tout six personnes parlent l'anglais, six
    le toucan et sept le latin. Quatre parlent l'anglais et le toucan, trois parlent
    le toucan et le latin et deux parlent le latin et l'anglais. Une seule personne parle
    toutes les langues, que vaut \(x\).
    \boxans{On a un trilingue, 1 bilingue anglais-latin, 2 bilingues toucan-latin,
    3 bilingues anglais-toucan, et cinq personnes culturellement fermées, \(x=11\).}
\end{exo}

\begin{exo}
    Combien y a-t-il de nombres entre \(1\) et \(2012\) qui ne sont pas multiples de
    \(2,3\) ou \(5\).
    \boxans{Il y a \(1006\) nombre divisibles par \(2\), \(670\) divisibles par \(3\),
    \(402\) divisibles par \(5\), \(335\) divisibles par \(6\), \(201\) divisibles par
    \(10\), \(134\) divisibles par \(15\) et \(67\) divisibles par \(30\). Donc en utilisant le crible de
    \textsc{Poincarré} on trouve notre réponse :\(2012-1006-670-402+335+201+134-67=537\).}
\end{exo}

\begin{exo}
    Encore du binôme.
    \begin{q}{1}
        Montrer que  pour tout entier \(n\geq 1\), le nombre de sous-ensembles de
        cardinal pair de tout ensemble à \(n\) éléments est \(2^{n-1}\).
        \boxans{Dans le cas où \(n\) est impair, choisir un sous-ensemble
        de cardinal pair est la même chose que choisir son complémentaire, qui est lui
        de taille impaire, on a bien \(2^{n-1}\) sous-ensembles de cardinal pair.
        Soit \(X\) à \(n\) éléments et \(x\not\in X\) les parties de \(X\cup\{x\}\)
        sont les parties de \(X\) union les parties de \(X\cup \{x\}\) contenant \(x\).
        Les sous-ensembles de \(X\) de cardinal pair sont dupliqués en sous-ensembles
        de \(X\cup\{x\}\) contenant \(\{x\}\) de cardinal impair, la formule se déduit.}
    \end{q}
    \begin{q}{2}
        En déduire un preuve de l'égalité \(\sum_{i=0}^n (-1)^{i}\binom{n}{i}=0\).
        \boxans{On peut décomposer en : nombres de sous-ensembles de cardinal pair
        moins nombre de sous-ensembles de cardinal impair, ce qui vaut \(0\) d'après
        la question précédente.}
    \end{q}
    \begin{q}{3}
        Connaissez (nb) vous une autre preuve ?
        \boxans{On peut remarquer un binôme de \textsc{Newton} pour le calcul de
        \((1-1)^n\) qui vaut bien 0.}
    \end{q}
\end{exo}

\begin{exo}
    Soient \(n,k\in\N^*\).
    \begin{q}{1}
        Combien y a-t-il d'uplets \(\left(x_1,\dots,x_k\right)\in\N^k\) tels que \(\sum x_i = n\).
        \boxans{Ce qu'on cherche revient à compter le nombre de façons de placer
        \(k-1\) séparateurs sur une ligne de \(n\) boules, ce qui revient à dire qu'on
        compte les façons de placer \(n\) boules parmi les \(n+k-1\) symboles,
        \(B_{n,k}=\binom{n+k-1}{n}\).}
    \end{q}
    \begin{q}{2}
        Même question pour \(\N^*\) et \(\Z\).
        \boxans{Dans \(\Z\) on a trivialement un nombre infini d'uplets pour \(k\geq2\)
        et sinon \(k\). Pour \(\N^*\) celà revient à compter les façons de placer
        \(k-1\) séparateurs parmi les \(n-1\) trous entre les boules, donc \(\binom{n-1}{k-1}\).}
    \end{q}
\end{exo}

\begin{exo}
    Soient \(n,k\in\N^*\). Combien y a-t-il d'uplets \(\left(x_1,\dots,x_k\right)\in\N^k\)
    tels que \(1\leq x_1\leq\dots\leq x_k\leq n\).
    \boxans{Celà revient à placer \(n-1\) séparateurs sur une ligne de \(k\) boules
    donc parmi \(n+k-1\) symboles, finalement le résultat est \(\binom{n+k-1}{n-1}\)}
\end{exo}

\begin{exo}
    Combien y a-t-il d'anagrammes du mot \textsc{Locomotive} ne contenant pas
    deux \textsc{O} consécutifs.
    \boxans{On retire les \textsc{O} pour avoir un mot de \(7\) lettres qui admet donc
    \(7!\) permuttations, à se mot il y a \(\binom{8}{3}\) façon de placer des \textsc{O}
    sans que deux \textsc{O} ne se touchent, la réponse est \(7!\binom{8}{3}\).}
\end{exo}

\begin{exo}
    Pour tout entier naturel \(n\geq 1\), soit \(a_n\) le nombre de manières de recouvrir
    un damier de dimension \(2\times n\) avec des pièces de dimension \(1\times 2\).
    Dire comment calculer récursivement \(a_n\).
    \boxans{On a la formule suivante \(a_n = a_{n-1} + a_{n-2}\). On retrouve \textsc{Fibonnacci}.}
\end{exo}

\begin{exo} Sur les sommes
    \begin{q}{1}
        Soit \(n,k\in\N\) tels que \(k\leq n\). Montrer par récurrence sur \(n\) la formule
        \[\binom{k}{k}+\binom{k+1}{k}+\binom{k+2}{k}+\dots+\binom{n}{k}=\binom{n+1}{k+1}\]
        \boxans{En \(n=k\) l'initialisation est évidente \(1=1\). On procède ensuite par
        récurrence immédiate avec la formule de \textsc{Pascal} :
        \(\binom{n+2}{k+1}=\binom{n+1}{k+1}+\binom{n+1}{k}\) comme argument majoritaire.}
    \end{q}
    \begin{q}{2}
        En déduire la valeur de \(\sum_{i=0}^n i^2\) et \(\sum_{i=0}^n i^3\).
        \boxans{On déduit de la question précédente les formules :
        \[\sum_{i=0}^ni^2=\frac{n(n+1)(2n+1)}{6}\quad\textrm{et}\quad
        \sum_{i=0}^ni^3=\left(\frac{n(n+1)}{2}\right)^2\]}
    \end{q}
\end{exo}

\begin{exo}
    Avec des quantificateurs qui fonctionnent, démontrer la formule suivante
    \[\binom{n}{k_1,\dots,k_m}=\binom{n}{k_1-1,\dots,k_m}+\dots+\binom{n}{k_1,\dots,k_m-1}\]
    \boxans{En posant le calcul, on met tout au bon dénominateur, ça se simplifie bien.}
\end{exo}

\begin{exo}
    Soit \(n\in\N^*\). De combien de manière peut-on placer \(n\) tours sur un échiquier
    de taille \(n\times n\) de sorte que toute case vide se trouve sur la trajectoire
    d'au moins une tour.
    \boxans{Chaque tour doit couvrir une colonne et une ligne à elle seule, en posant les
    tours de la colonne de gauche vers celle de droite on à \(n\) choix de ligne pour la
    première puis \(n-1\) pour la seconde et ainsi de suite, finalement on a \(n!\)
    solutions au problème.}
\end{exo}
\end{document}