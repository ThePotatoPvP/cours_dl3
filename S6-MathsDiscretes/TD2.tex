\documentclass[french]{report}
\usepackage{../exercices}


\usepackage{causets}

\begin{document}

\begin{center}
    \huge{\textbf{S6- Maths Discrètes - TD2}}
\end{center}

\begin{exo}
    Pour chacune des affirmations suivantes, dire si elle est vraie ou fausse
    et justifier la réponse.
    \begin{q}{1}
        \(G\) et \(H\) sont isomorphes ssi pour toute application \(f:V(G)\to V(H)\)
        et pour tous les sommets \(u,v\in V(G)\), on a \(\{u,v\}\in E(G)
        \Leftrightarrow\{f(u),f(v)\}\in E(H)\)
        \boxans{Soit \(G\) et \(H\) isomorphes, on peut toujours prendre une application
        qui renvoie tous les sommets de \(G\) sur un même sommet de \(H\) ainsi si \(G\)
        a une arête l'implication est fausse.}
    \end{q}
    \begin{q}{2}
        \(G\) et \(H\) sont isomorphes ssi il existe une bijection \(f:E(G)\to E(H)\).
        \boxans{Soit \(f\) une telle bijection, alors les deux graphes ont le même
        nombre d'arêtes mais rien de plus, ce n'est pas assez pour conclure à l'isomorphisme.}
    \end{q}
    \begin{q}{3}
        S'il existe une bijection \(f:V(G)\to V(H)\) telle que tout sommet \(u\in V(G)\)
        a le même degré que \(f(u)\) alors \(G\) et \(H\) sont isomorphes
        \boxans{C'est faux, voir dessin. Avec \(6\) sommets, deux triangles et un héxagone par exemple.}
    \end{q}
    \begin{q}{4}
        Si \(G\) et \(H\) sont isomorphes, alors il existe une bijection \(f:V(G)\to V(H)\)
        telle que tout sommet \(u\in V(G)\) a le même degré que \(f(u)\).
        \boxans{Si \(G\) et \(H\) sont isomorphe, par définition il y a autant d'arêtes
        de la forme \(\{u,x\}\) dans \(G\) que d'arêtes \(\{f(u),y\}\) dans \(H\) donc oui.}
    \end{q}
    \begin{q}{5}
        Si \(G\) et \(H\) sont isomorphes, alors il existe une bijection \(f:E(G)\to E(H)\).
        \boxans{Oui, d'après la définition d'isomorphisme de graphe \(G\) et \(H\)
        doivent avoir le même nombre d'arêtes, ce qui suffit.}
    \end{q}
    \begin{q}{6}
        Deux graphes \(G\) et \(H\) sont isomorphes ss'il existe une application \(f:V(G)\to V(H)\)
        telle que pour tous sommets \(u,v\in V(G)\) on a \(\{u,v\}\in E(G)\Leftrightarrow
        \{f(u),f(v)\}\in E(H)\).
        \boxans{La réciproque est fausse, tous les graphes sans arêtes ne sont pas
        isomorphes au graphe à \(1\) sommet.}
    \end{q}
    \begin{q}{7}
        Tout graphe sur \(n\) sommets est isomorphe à un graphe dont l'ensemble des
        sommets est \(\llbracket 1,n\rrbracket\).
        \boxans{Oui, c'est le concept d'\(\alpha\)-renommage.}
    \end{q}
    \begin{q}{8}
        Tout graphe sur \(n\geq 1\) sommets est isomorphe à une infinité de graphes.
        \boxans{Oui, on peut \(\alpha\)-renommer le sommet autant de fois qu'on le souhaite.}
    \end{q}
\end{exo}

\begin{exo}
    Soit \(G=(V,E)\) un graphe, on note \(\bar{G}\) son complémentaire.
    \begin{q}{1}
        Montrer que si \(G\) est non connexe alors \(\bar{G}\) est connexe.
        \boxans{Supposons \(G_1,G_2\) deux parties connexes de \(G\) distinctes
        alors tous les sommets de \(G_1\) sont reliés à chacun des sommets de \(G_2\)
        ainsi les deux ensembles de points sont reliés l'un à l'autre et \(\bar{G}\) contient
        un graphe biparti avec un couplage couvrant assurant que les points de \(G_1\)
        et ceux de \(G_2\) forment une seule partie connexe dans \(\bar{G}\), ce qui
        suffit à la démonstration.}
    \end{q}
    \begin{q}{2}
        Peut-on montrer la réciproque ?
        \boxans{Non, il est possible que \(G\) et \(\bar{G}\) soient connexes, on peut
        par exemple prendre \(G\) à \(4\) sommets de la forme \(\bigsqcup\)}
    \end{q}
\end{exo}

\begin{exo}
    Montrer que si \(G\) est un graphe dont tous les sommets sont de degré supérieur
    à \(d\in\N\) alors \(G\) contient un chemin élémentaire de longueur \(d\).
    \boxans{Soit \(s\) un sommet de \(G\), par hypothèse il est de degré au moins
    \(d\), on part le long d'une arrête pour arriver sur un sommet dont au moins \(d-1\)
    arrêtes sont disponibles pour repartir, on en choisit une et on arrive sur un sommet
    duquel on peut prendre au moins \(d-2\) sommets pour repartir et de proche
    en proche on construit un chemin de longueur \(d\).}
\end{exo}

\begin{exo}
    Soit \(G=(V,E)\) un graphe et \(|V|=n\). Soit \(A_G\) sa matrice d'adjacence.
    \begin{q}{1}
        Donner un exemple de graphe connexe tel que, pour tout \(k\geq 1\), certains
        coefficients de \(A_G^k\) sont nuls.
        \boxans{On peut prendre la droite, chaque sommet relié au suivant, ainsi \(A_G\)
        est le nilpotent cyclique usuel, le triangle du bas reste nul.}
    \end{q}
    \begin{q}{2}
        Montrer que \(G\) est connexe si et seulement si tous les coefficients
        de la matrice \(\left(I_n+A_G\right)^{n-1}\) sont non nuls.
        \boxans{Cette matrice représente le nombre de chemins distincts entre deux sommets, \(0\) s'il
        n'existe pas de chemin. On ajoute l'identité pour s'autoriser à boucler au cas
        où on a un chemin plus court que \(n\). Si \(G\) est connexe alors les \(n\) sommets
        sont reliés par des chemins d'au plus \(n-1\) arêtes et inversement.}
    \end{q}
\end{exo}

\begin{exo}
    Un graphe est dit \textit{asymétrique} si son seul automorphisme est l'identité.
    \begin{q}{1}
        Donner un exemple d'un graphe asymétrique ayant au moins 2 sommets.
        \boxans{Le graphe à 6 sommets représentant une maison avec une barre horizontale
        qui sort du toit vers le côté semble convenir.
        \[\causet{2,1,5,4,3,6}{1/3,2/4,1/2,3/4,3/6,4/6,4/5}\]}
    \end{q}
    \begin{q}{2}
        Montrer qu'un graphe asymétrique doit avoir au moins \(6\) sommets.
        \boxans{On peut procéder au case pas cas en remarquant qu'un graphe est asymétrique
        si et seulement si toutes ses parties connexes le sont (l'image d'une partie
        connexe par un automorphisme étant une partie connexe) pour éliminer beaucoup de cas
        à traiter.}
    \end{q}
\end{exo}

\begin{exo}
    Montrer que deux graphes \(G\) et \(G'\) sont isomorphes s'il existe une
    matrice de permutation \(P\) telle que \(A_{G'}=PA_GP^t\)
\end{exo}

\begin{exo}
    Soit \(G=\left(V_G,E_G\right)\) et \(H=\left(V_H,E_H\right)\) deux graphes.
    Un homomorphisme de \(G\) dans \(H\) est une application \(f:V_G\to V_H\)
    telle que pour toute arête \({u,v}\) dans \(E_G\), l'arête \(\{f(u),f(v)\}\)
    appartient à \(E_H\).
    \begin{q}{1}
        Quelle est la différence avec la définition de \(G\) isomorphe à \(H\).
        \boxans{Ici \(f\) n'a besoin d'être ni monomorphe ni épi, on ne peut donc
        pas espérer de réciproque.}
    \end{q}
    \begin{q}{2}
        Caractériser les graphes \(G\) tels qu'il existe un homomorphime de \(G\)
        dans \(K_1\) puis \(K_2\) et \(K_3\).
        \boxans{\(G\) doit avoir chacune de ses parties connexes isomorphe à un sous
        grahe du \(k\) en question.}
    \end{q}
\end{exo}

\begin{exo}
    Quel est le nombre maximum d'arêtes (exprimé en fonction du nombre \(n\) de sommets)
    \begin{q}{1}
        d'un graphe ?
        \boxans{La borne est \(\binom{n}{2}\) c'est le cas d'un graphe complet.}
    \end{q}
    \begin{q}{2}
        d'un graphe avec \(k\leq n\) composantes connexes.
        \boxans{Si on regarde le triangle de \textsc{Pascal} on observe qu'il vaut
        mieux avoir une grande partie connexe que plusieurs petites, donc la borne
        est \(\binom{n-k+1}{2}\)}
    \end{q}
\end{exo}

\begin{exo}
    Décrire l'ensemble des couples \(\left(n,k\right)\) tels qu'il existe un graphe
    à \(n\) sommets dont tous sont de degré \(k\) (un tel graphe est dit \(k\)-régulier).
    \boxans{Un graphe \(k\) régulier à \(n\) sommets possède \(k\times n\) extrémités
    d'arêtes, il faut donc que \(k\times n\) soit pair. Aussi \(k\) est compris
    entre \(0\) et \(n-1\) parce qu'on se restreint aux graphes \textit{toddler friendly}
    Ces conditions sont nécessires et suffisantes, pour tous les \(k\) pairs, on peut les
    atteindre en faisant des cycles partiels pour monter de \(2\) en \(2\). Sinon, si \(k\)
    est impair, alors \(n\) est pair, on relie les sommets par paire pour les avoir de
    degré \(1\) puis on monte de \(2\) en \(2\) avec des sous-cycles.}
\end{exo}

\begin{exo}
    Donner un exemple de graphe \(3\)-régulier asymétrique.
\end{exo}

\begin{exo}
    On appelle \textit{promenade} dans un graphe \(G\) une suite \(u_0, e_1, u_1,\dots
    e_n,u_n\) où les \(u_i\) sont des sommets de \(G\) et les \(e_i\) des arêtes et
    pour tout \(i\in\llbracket1,n\rrbracket e_i=\{u_{i-1},u_i\}\). Caractériser les
    graphes possédant une promenade passant exactement une fois par chaque arête et
    au moins une fois par chaque sommet.
\end{exo}

\begin{exo}
    Soit \(G\) un graphe à \(n\) sommets, chacun étant de degré au moins
    \(\left(n-1\right)/2\). Montrer que \(G\) est connexe.
\end{exo}


\begin{exo}
    On rapelle qu'un cycle hamiltonien dans un graphe est un cycle passant exactement
    une fois par chaque sommet du graphe. Soit \(G\) un graphe a \(n\) sommets. On
    considère l'opération suivante : pour deux sommets non reliés \(u\) et \(v\) tels
    que la somme des degrés soit au moins \(n\), ajouter l'arête \(\{u,v\}\). Soit
    \(G'\) le graphe obtenu après cette opération.
    \begin{q}{1}
        Montrer que si \(G\) a un cycle hamiltonien alors \(G'\) aussi.
        \boxans{\(G\) étant un sous graphe couvrant de \(G'\) (par bon sens) tout cycle hamiltonien
        de \(G\) est aussi un cycle hamiltonien de \(G'\).}
    \end{q}
    \begin{q}{2}
        Montrer la réciproque, c'est-à-dire que si \(G'\) a un cycle hamiltonien
        alors \(G\) en avait un aussi.
        \boxans{Supposons \(G'\) hamiltonien, on exhibe \(C\) un cycle hamiltonien de \(G'\),
        si \(\{u,v\}\notin C\) alors on a rien à faire. Dans le cas contraire, de par les
        degrés on sait que \(u\) et \(v\) ont au moins deux voisins communs, notés \(w\)
        et \(\omega\) et qu'au moins \(n/2\) sommet sont reliés soit à \(u\) soit à \(v\). Ainsi
        quelque soit le cycle, on aura toujorus dedans deux sommet qui se suivent, l'un relié à \(u\)
        et l'autre à \(v\), donc tavu ça passe.}
    \end{q}
    Soit \(G\) un graphe, on répète l'opération d'ajout d'arêtes décrite en début
    d'exercice tant que c'est possible. Soit \([G]\) le graphe obtenu.
    \begin{q}{3}
        À chaque étape, il pourrait y avoir plusieurs choix possibles de couples
        de sommets \(u,v\) satisfaisant les conditions. Montrer que le graphe
        obtenu à la fin ne dépend pas de l'ordre.
        \boxans{Quelque soit l'ordre dans lequel on s'y prend, les degré de paires ne peuvent
        qu'augmenter et donc condire à plus d'arêtes, on ne diminue jamais les possibilités
        de rajouter des arêtes, donc pas de raison que ça bloque.}
    \end{q}
    \begin{q}{4}
        Montrer que \(G\) a un cycle hamiltonien ssi \([G]\) en a un.
        \boxans{On a démontrer dans les deux premières question que chaque étape de la
        transformation de \(G\) préserve le caractère hamiltonien, on a bien
        l'équivalence.}
    \end{q}
    \begin{q}{5}
        Soit \(G\) un graphe à \(n\) sommets, chacun de degré au moins \(n/2\). Montrer
        que \(G\) a un cycle hamiltonien.
        \boxans{Si chacun des sommets est de degré au moins \(n/2\) alors pour toute paire
        de sommet le degré est supérieur à \(n\) et donc \([G]\) est le graphe complet, qui
        a bien un cycle hamiltonien, d'après la question précédente, \(G\) en a alors un aussi.}
    \end{q}
    \begin{q}{6}
        Soit \(G\) un graphe à \(n\) sommets et ayant au moins \(\frac12(n-1)(n-2)+2\)
        arêtes. Montrer que \(G\) a un cycle hamiltonien.
        \boxans{Ce graphe est à \(n-3\) arêtes près d'être le graphe complet. Donc tout
        sommet est au moins de degré \(2\). Le degré d'une paire est au moins \(2(n-1) - (n-3 - 1) = n\)
        donc les sommet sont reliés dans \([G]\) qui est donc complet. Par le même raisonnement
        qu'au dessus, \(G\) a un cycle hamiltonien.}
    \end{q}
    \begin{q}{7}
        Donner un graphe à \(6\) sommets avec le plus petit nombre possible d'arêts tel
        que \([G]\) soit le graphe complet \(K_6\). Pouvez vous montrer que c'est le plus
        petit nombre d'arêtes possible ?
        \boxans{D'après la question précédente c'est possible avec \(12\) peu importe
        comment on les place, si on opti avec \(8\) c'est possible. Avec \(7\) c'est impossible,
        \(6\) arêtes au minium sont prises dans le but de faire un grpahe connexe, en plaçant
        la dernière on a deux sommet de degré \(3\) reliés entre eux, le reste de degré \(2\) donc
        on peut pas démarrer, c'est ded.}
    \end{q}
    \begin{q}{8}
        Le fait que \(G\) ait un cycle hamiltonien ssi \([G]\) en a un permet-il
        d'obtenir un algorithme efficace pour déterminer si un graphe donné a un
        cycle hamiltonien ?
        \boxans{Bah non, c'est NP-complet.}
    \end{q}
\end{exo}

\begin{exo}
    On définit le \textit{graphe triangulaire} \(T_n\) comme étant le graphe \(G\)
    dont l'ensemble des sommets \(V(G)\) est \(\binom{\llbracket1,n\rrbracket}{2}\)
    c'est à dire l'ensemble des sous-ensembles de \(\llbracket1,n\rrbracket\) à \(2\)
    éléments, et dont l'ensemble des arêtes \(E(G)\) est \(\{\{X,Y\}\mid|X\cap Y|=1\}\),
    c'est à dire qu'il y a une arête entre \(X\) et \(Y\) ssi ces deux sous-ensembles
    de \(\llbracket1,n\rrbracket\) ont exactement un élément en commun.
    \begin{q}{1}
        Dessiner les graphes \(T_3, T_4,T-5\).
    \end{q}
    \begin{q}{2}
        Montrer que \(T_n\) est \(T_{2n-4}\)-régulier.
    \end{q}
    \begin{q}{3}
        Soit \(X,Y\) deux sommets de \(T_n\) tels que \(\{X,Y\}\in E(T_n)\). Montrer
        que le nombre de sommets de \(T_n\) reliés à la fois à \(X\) et \(Y\) vaut
        \(n-2\).
    \end{q}
\end{exo}

\end{document}