\documentclass[french]{report}
\usepackage{../exercices}


\usepackage{causets}

\begin{document}

\begin{center}
    \huge{\textbf{S6- Maths Discrètes - TD3}}
\end{center}

\begin{exo}
    Pour chaque description ci-dessous, dessiner un graphe satisfaisant les conditions
    ou montrer qu'il n'en existe pas :
    ...
\end{exo}

\begin{exo}
    Montrer qu'un arbre ayant un sommet de degré \(k\) contient au moins \(k\)
    feuilles.
    \boxans{\emph{On procèdera par induction structurelle}}
\end{exo}

\begin{exo}
    TODO
\end{exo}

\begin{exo}
    Montrer qu'un graphe ayant \(n\geq 3\) sommets est un arbre si et seulement
    si il n'est pas isomorphe à \(K_n\) et si l'ajout d'une arête crée toujours
    exactement un cycle.
\end{exo}

\begin{exo}
    Un graphe connexe est appelé \emph{unicyclique} s'il contient exactement
    un cycle. Montrer que pour tout graphe \(G\) les propriétés suivantes
    sont équivalentes:
    \begin{enumerate}
        \item \(G\) est unicyclique
        \item Il existe une arête \(e\) de \(G-e\) soit un arbre
        \item \(G\) est connexe et a autant de sommets que d'arêtes
        \item \(G\) est connexe et l'ensemble de toutes les arêtes qui ne sont pas
        des isthmes constitue un cycle.
    \end{enumerate}
\end{exo}

\begin{exo}
    Quel est le nombre d'arbres couvrants du graphe suivant ?
\end{exo}

\begin{exo}
    Montrer que si \(T\) est un arbre couvrant d'un graphe \(G\) alors pour toute
    arête \(e\in E(G)\backslash E(T)\) il existe une arête \(e'\in E(T)\) telle que
    \(T+e-e'\) est aussi un arbre couvrant de \(G\).
\end{exo}

\begin{exo}
    foo
    \begin{q}{1}
        Soit \(G\) un graphe connexe et \(e\) une arête de \(G\). Existe-t-il toujours
        un arbre couvrant de \(G\) contenant l'arête \(e\) ?
    \end{q}
    \begin{q}{2}
        Soit \(G\) un graphe connexe. Caractériser les sous-ensembles \(F\) des arêtes
        de \(G\) tels qu'il existe un arbre couvrant de \(G\) contenant \(F\).
    \end{q}
\end{exo}

\begin{exo}
    TODO
\end{exo}

\begin{exo}
    Soit \(G\) un graphe, avec \(n\) sommets et \(p\) composantes connexes, et \(D\)
    la matrixe d'incidence de \(G\). Montrer que le rang de \(D\) est \(n-p\).
\end{exo}

\begin{exo}
    Un couplage parfait dans un graphe \(G\) est un sous-ensemble \(C\) des arêtes
    de \(G\) tel que tout sommet de \(G\) appartient a exactement à une arête de \(C\).
    \begin{q}{1}
        Quel est le nombre de couplages parfaits de \(K_n\)
    \end{q}
    \begin{q}{2}
        Soit \(C\) un couplage parfait de \(K_n\) combient y a-t-il d'arbres couvrants
        de \(K_n\) contenant \(C\).
    \end{q}
\end{exo}

\begin{exo}
    Soit \(G\) un graphe et \(k\geq 1\) un entier, on définit le graphe \(G^{(k)}\)
    ayant les sommets de \(G\) et tels que deux sommets sont connectés si et seulement
    si leur distance dans \(G\) est au plus \(k\).
    \begin{q}{1}
        Montrer que pour tout arbre \(T\) ayant au moins \(3\) sommets, le
        graphe \(T^{(3)}\) a un cycle hamiltonien.
    \end{q}
    \begin{q}{2}
        Montrer que pour tout graphe connexe \(G\) ayant au moins \(3\) sommets,
        \(G^{(3)}\) a un cycle hamiltonien.
    \end{q}
    \begin{q}{3}
        Trouver un graphe connexe \(G\) avec au moins \(3\) sommets tel que
        \(G^{(2)}\) n'a pas de cycle hamiltonien.
    \end{q}
\end{exo}

\begin{exo}
    Montrer que le nombre d'arbres couvrants du graphe complet \(K_n\) est \(n^{n-2}\).
\end{exo}

\begin{exo}
    Montrer que le nombre d'arbes couvrants du graphe biparti complet \(K_{n,m}\) est
    \(m^{n-1}n^{m-1}\).
\end{exo}

\begin{exo}[]
    Soit \(e\) une arête de \(K_n\), le graphe complet à \(n\) sommets. Calculer
    le nombre d'arbres couvrants de \(K_n-e\).
    \boxans{Tout arbre couvrant contenant \(e\) peut être découpé en un sous arbre
    couvrant un sous graphe d'un côté de \(e\) et un autre de l'autre, ainsi on a
    \(\sum_{k=1}^{n-1}\binom{n-2}{k-1} k^{k-2} (n-k)^{n-k-2}\) arbres couvrants de \(K_n\) qui contiennent
    \(e\) et donc \(n^{n-2}\) moins ce nombre qui sont valides pour \(K_n-e\).}
\end{exo}

\begin{exo}
    Soit \(G\) un graphe non orienté, on définit \(\lambda(G)=\min\{|F|
    \mid F\subseteq E(G) \textrm{et (V, E-F) pas connexe}\}\)
    \begin{q}{1}
        Calculer \(\lambda(G)\) pour les graphes usuels, \(P_n, C_n, K_n\).
        \boxans{On a \(\lambda(P_n)=1, \lambda(C_n)=2\) et \(\lambda(K_n)=(n-1)\)}
    \end{q}
    \begin{q}{2}
        Donner un graphe \(G\), sans sommet de degré \(1\) tel que
        \(\lambda(G)=1\).
        \boxans{En reliant deux \(K_3\) par une arête on obtient un graphe qui convient.}
    \end{q}
    \begin{q}{3}
        Soit \(n\geq 2\), quel est le nombre maximum d'arêtes que \(G\)
        à \(n\) sommets peut avoir tel que \(\lambda(G)=1\).
        \boxans{On peut avoir un graphe complet à \(n-1\) arêtes et un sommet relié
        pas une arête au graphe complet, ce qui donne \(1 +\binom{n-1}{2}\). On comprend
        à la construction que c'est le maximum.}
    \end{q}
    \begin{q}{3}
        Montrer que si \(G\) contient 2 arbres couvrants \(T_1=(V_1,E_1)\) et
        \(T_2=(V_2,E_2)\) tels que \(E_1\cap E_2 = \emptyset\) alors \(\lambda(G)\geq 2\).
        \boxans{Entre toute paire de sommets, on a au moins deux chemins différents, un en
        suivant chaque arbre. Et pour que le graphe ne soit plus connexe il faut qu'il
        existe une paire de sommets sans chemin, donc briser deux chemins, donc briser une arête
        le long de chacun, soit au moins \(2\) arêtes.}
    \end{q}
\end{exo}

\end{document}