\documentclass{report}
\usepackage{../../exercices}

\begin{document}
\begin{center}
    \Huge{\textbf{TD 6. Inversion locale - Fonctions implicites}}
\end{center}
\bigskip

\begin{exo}
    Soit \(f:\R\to\R\) la fonction définie par
        \(f(x)=x+x^2\sin\left(\frac{\pi}{x}\right);\ f(0)=0\)
    \begin{q}{1}
        Montrer que \(f\) est dérivable sur \(\R\) et que \(f'(0)=1\)
        \boxans{
            \[\lim_{h\to0} \frac{f(0)-f(h)}{h}= \lim_{h\to 0} 1+h\sin\frac{\pi}{h} = 1\]
            car la fonction \(\sin\) est bornée. Finalement la fonction est dérivable en tout point
            de \(\R^*\) comme composition de fonction dérivables, et dérivable sur \(\R\) entier
            avec \(f'(0)=1\).
        }
    \end{q}
    \begin{q}{2}
        Existe-t-il un voisinage de \(0\) sur lequel \(f\) est injective ?
        \boxans{Non, il n'existe pas de tel voisinage. Supposons par l'absurde
        qu'un tel voisinage existe, notons \(2\eps\) sa longueur. Alors pour tout
        \(k\in2\N\) vérifiant \(k^{-1}<\eps\) on a \(f(k^{-1})=0\) et donc une infinité
        d'annulations, \(f\) n'est pas injective sur l'intervalle, c'est absurde.}
    \end{q}
\end{exo}

\begin{exo}
    On considère \(f\colon\R_2[X]\to\R_1[X]\) définie comme
    \(a+bX+cX^2\mapsto a+b+(a^2+b^2-c^2)X\)
    \begin{q}{1}
        Montrer que \(f\) est \(\CC^1\) et calculer sa différentielle.
        \boxans{
            \(f\) est \(\CC^1\) en tant que polynôme des coefficients du polynôme de
            départ. Maintenant, calculons la différentielle de \(f\). Soit
            \(h = h_1X^2 + h_2X + h_3 \in \R_2[X]\), alors la différentielle de \(f\)
            en \(a+bX+cX^2\) appliquée à \(h\) est donnée par :
            \[
                df_{a+bX+cX^2}(h) = \frac{\partial f}{\partial a}(a+bX+cX^2)h_1 +
                \frac{\partial f}{\partial b}(a+bX+cX^2)h_2 + \frac{\partial f}{\partial c}(a+bX+cX^2)h_3.
            \]
            En substituant les dérivées partielles, on obtient :
            \[df_{a+bX+cX^2}(h) = (1 + 2aX)h_1 + (1 + 2bX)h_2 - 2cXh_3.\]
        }
    \end{q}
    \begin{q}{2}
        L'application \(f\) est-elle un difféomorphisme local ?
        \boxans{
            Pour montrer que \(f\) est un difféomorphisme local, nous devons vérifier
            que sa différentielle est un isomorphisme au voisinage de chaque point
            de son domaine. Calculons la matrice jacobienne de \(f\):
            \[
                J_f(a+bX+cX^2) =
                \begin{bmatrix}
                    1 & 2a & 0 \\
                    0 & 1 & 2b \\
                    0 & 0 & -2c
                \end{bmatrix}
            \]
            Le déterminant de cette matrice est \(-2c\), qui n'est jamais nul lorsque
            \(c\) est non nul. Ainsi, la matrice jacobienne est inversible pour tout
            \(a+bX+cX^2\) avec \(c\neq 0\), ce qui signifie que \(f\) est un
            difféomorphisme local sur ces points.
        }
    \end{q}
\end{exo}

\begin{exo}
    On considère \(f\colon\R^2\to\R^2\) définie comme \((x,y)\mapsto(x+y,xy)\).
    \begin{q}{1}
        Montrer que \(f\) est \(\CC^1\) et calculer sa jacobienne.
        \boxans{
            Pour montrer que \(f\) est \(\CC^1\), il suffit de vérifier que ses
            composantes sont \(\CC^1\). Les composantes de \(f\) sont \(f_1(x, y)
            = x + y\) et \(f_2(x, y) = xy\), qui sont clairement \(\CC^1\).
            Calculons maintenant la jacobienne de \(f\):
            \[
                J_f(x, y) =
                \begin{bmatrix}
                    \frac{\partial f_1}{\partial x} & \frac{\partial f_1}{\partial y} \\
                    \frac{\partial f_2}{\partial x} & \frac{\partial f_2}{\partial y}
                \end{bmatrix}
                =
                \begin{bmatrix}
                    1 & 1 \\
                    y & x
                \end{bmatrix}.
            \]
        }
    \end{q}
    \begin{q}{2}
        Déterminer l'ensemble \(U\) des points pour lesquels \(df_{(x,y)}\) est inversible.
        \boxans{
            La jacobienne \(J_f(x, y)\) est inversible si et seulement si son
            déterminant est non nul. Ainsi, \(U\) est l'ensemble des points tels que
            \(1 \cdot x - 1 \cdot y \neq 0\), c'est-à-dire \(x \neq y\).
        }
    \end{q}
    \begin{q}{3}
        Expliquer à quel titre \(f_{|U}\) n'est pas un difféomorphisme.
        \boxans{
            Bien que \(f_{|U}\) soit injective, elle n'est pas surjective. En effet,
            \(f_{|U}\) n'atteint pas les points de la forme \((a, 0)\) pour tout
            \(a\in\R\), car \(f_2(x, y) = xy\) ne peut pas être nul pour \(y \neq 0\). Ainsi, \(f_{|U}\) n'est pas un difféomorphisme, car elle ne réalise pas une bijection sur son image.
        }
    \end{q}
    \begin{q}{4}
        Soit \(U'=\{(x,y)\in\R^2\mid y<x\}\). Montrer que \(f_{|U'}\) est un
        difféomorphisme.
        \boxans{
            La restriction \(f_{|U'}\) est une bijection, car chaque ligne de \(U'\)
            est envoyée sur une ligne de \(\R^2\) par \(f_{|U'}\). De plus, la
            jacobienne de \(f_{|U'}\) est inversible sur \(U'\), car \(J_f(x, y)\)
            est inversible pour \(y < x\). Par conséquent, \(f_{|U'}\) est un
            difféomorphisme.
        }
    \end{q}
    \begin{q}{5}
        Déterminer \(V'\colon=f(U')\) et le représenter.
        \boxans{TODO}
    \end{q}
\end{exo}


\begin{exo}
    Soient \(a,b\) deux réels et soit \(f\colon\R^2\to\R^2\) définie comme
    \((x,y)\mapsto(x+a\sin y, y+ b\sin x)\)
    \begin{q}{1}
        Montrer que \(f\) est un difféomorphisme local ssi \(|ab|<1\)
    \end{q}
    On suppose dans la suite de l'exercice \(|ab|<1\).
    \begin{q}{2}
        Montrer que \(V\colon=f(\R^2)\) est un ouvert de \(\R^2\)
    \end{q}
    \begin{q}{3}
        Montrer que \(f\) est un difféomorphisme de \(\R^2\) sur \(V\).
    \end{q}
    \begin{q}{4}
        Soit \(((x_n,y_n))_n\) une suite de \(\R^2\). Supposons que la suite
        \((u_n,v_n)=f(x_n,y_n)\) converge vers \((u,v)\) dans \(\R^2\).
        \begin{q}{a}
            Montrer que la suite \((x_n,y_n)\) est bornée.
        \end{q}
        \begin{q}{b}
            Montrer que si deux sous-suites extraites de \(((x_n,y_n))_n\) convergent
            alors elles ont la même limite.
        \end{q}
        \begin{q}{c}
            En déduire que la suite \(((x_n,y_n))_n\) converge.
        \end{q}
    \end{q}
    \begin{q}{5}
        Montrer que \(f\) est un difféomorphisme de \(\R^2\) sur lui même.
    \end{q}
\end{exo}

\begin{exo}
    On considère \(f:\mathcal{M}_2(\R)\to\mathcal{M}_2(\R)\) définie par
    \(A\mapsto A^2\). On admet que \(f\) est \(\CC^1\), on pose \(I=I_2\) et
    \(J=\begin{pmatrix}1&0\\0&-1\end{pmatrix}\) et on observe \(f(J)=f(I)=I\).
    \begin{q}{1}
        Calculer les applications \(df_I\) et \(df_J\). Sont-elle injectives ?
    \end{q}
    \begin{q}{2}
        Montrer qu'il n'existe pas de couple \((U,V)\) d'ouverts et d'application
        différentiable \(g:U\to V\) tels que \(I\in U\), \(J\in V\), \(g(I)=J\) et \(f\circ g=\id\)
    \end{q}
    \begin{q}{3}
        Montrer que si dans la question précédente avec \(I\in V\) et \(g(I)=I\) en
        oubliant \(J\)alors on peut trouver le couple d'ouverts et l'application.
    \end{q}
\end{exo}

\begin{exo}
    On considère le sous-ensemble \(\Gamma\) de \(\R^2\) défini par:
    \(x^3+y^3-3xy-1=0\).
    \begin{q}{1}
        Montrer que \(\Gamma\) est au voisinade de \((x,y)=(1,0)\) le graphe d'une
        fonction implicite \(y=h(x)\) telle que \(h(1)=0\).
    \end{q}
    \begin{q}{2}
        Montrer que la corbe \(\Gamma\) a une tangente au point \((1,0)\) dont on déterminera
        l'équation.
    \end{q}
    \begin{q}{3}
        Donner un développement limite à l'ordre \(3\) de \(h\) au voisinage de \(1\).
        Dessiner \(\Gamma\) au voisinage de \((1,0)\).
    \end{q}
\end{exo}

\begin{exo}
    Soit \(f\colon\R^3\to\R^2\) définie par \((x,y,z)\mapsto(x^2-y^2+z^2-1,xyz-1)\).
    \begin{q}{1}
        Montrer qu'il existe un intervalle ouvert \(I\) contenant \(x=1\) et une
        application \(\varphi;I\to\R^2\) de classe \(\CC^k\) pour tout \(k\in\N\) tels
        que \(\varphi(1)=1\) et \(f(x,\varphi(x))=(0,0)\) pour tout \(x\in I\).
    \end{q}
    \begin{q}{2}
        Plus généralement, soit \((x_0,y_0,z_0)\in\R^3\) un point d'annulation
        de \(f\). Montrer qu'il existe un intervalle ouvert \(I_{x_0}\) contenant \(x_0\)
        et une application \(\varphi_{x_0}\colon I_{x_0}\to\R^2\) de classe \(\CC^k\) pour
        tout \(k\in\N\), tels que \(\varphi(x_0)=(y_0,z_0)\) et \(f(x,\varphi_{x_0}(x)=(0,0))\)
        pour tout \(x\in I_{x_0}\).
    \end{q}
\end{exo}

\begin{exo}
    \begin{q}{1}
        Montrer qu'il existe un ouvert \(\Omega\) de \(\R^2\) contenant \((0,0)\) et
        des applications \(y_1,y_1\colon \Omega\to\R\) de classe \(\CC^\infty\) vérifiant:
    \end{q}
\end{exo}

\end{document}