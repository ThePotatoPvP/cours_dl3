\documentclass{report}
\usepackage{../exercices}

\begin{document}
\begin{center}
    \Huge{\textbf{TD 2. Normes et Boules}}
\end{center}
\bigskip

\begin{exo}
    Soit \(E\) un evn et \(a\in E\).
    \begin{q}{1}
        Montrer que le singleton \(\{a\}\) est un fermé de \(E\)
        \boxans{Il existe une unique suite dans \(\{a\}^\N\) qui est
        constante et de limite \(a\), ainsi par caractérisation
        séquencielle on a bien un fermé de \(E\).}
    \end{q}
    \begin{q}{2}
        Montrer que, pour toit \(r>0\), la sphère \(S_r(a)=
        \{x\in E\mid \|x-a\|=r\}\) est un fermé de \(E\).
        \boxans{Il s'agit de l'image réciproque du singleton
        \(\{r\}\) de \(R\) par l'application continue qu'est la norme,
        c'est donc un fermé.}
    \end{q}
    \begin{q}{3}
        Montrer que pour tour \(0<r<R\), l'ensemble \(C_r=
        \{x\in E\mid r<\|x-a\|<R\}\) est un ouvert de \(E\). Quelle
        est son adhérence ?
        \boxans{C'est un ouvert comme \(N^{-1}((r,R))\), son adhérence
        est l'ensemble avec des inégalités larges.}
    \end{q}
\end{exo}

\begin{exo}
    Soit \(\left(E,\|\cdot\|\right)\) un env, montrer que la norme est
    une application continue.
    \boxans{La norme est continue car elle est \(1-lipchtizienne\).}
\end{exo}

\begin{exo}
    Pour chacun des ensembles suivants, déterminer s'il est ouvert ou fermé
    au aucun des deux.
    \begin{enumerate}
        \itt \(A=\{(x,y)\in\R^2\mid y\geq x^2, 0\leq y\leq1\}\)
        \boxans{C'est un fermé car toutes les inégalités sont larges.}
        \itt \(B=\{(x,y)\in\R^2\mid y\geq x^2, 0\leq y<1\}\)
        \boxans{Ouvert d'un côté et fermé de l'autre donc aucun des deux.}
        \itt \(C=\{(x,y)\in\R^2\mid x^2-2x+y^2=0\}\cup\{(x,0);x\in[2;3]\}\)
        \boxans{C'est un fermé comme union de deux fermés (des variétés de dim 1)}
        \itt \(D=\{(x,y)\in\R^2\mid x^2-y^2<1, -1<y<1\}\)
        \boxans{C'est un ouvert comme intersection finie d'ouverts car les
        inégalités sont strictes.}
    \end{enumerate}
\end{exo}

\begin{exo}
    Soit \(f:\R\to\R\) une application continue. Montrer que le graphe
    \(\{(t,f(t))\mid t\in\R\}\) de \(f\) est un fermé de \(\R^2\). Quel
    est son intérieur ?
    \boxans{La continuité indique que le graphe est une variété de dimension
    1 qui a donc un intérieur nulle car l'espace ambient est de dimension 2.}
\end{exo}

\begin{exo}
    On munit l'espace vectoriel \(\CC([0,1], \R)\) de deux normes, la
    norme uniforme \(\|\cdot\|_\infty\) et la norme \(\|\cdot\|_1\) qui
    sont respectivement définies par.
    \[\forall f\in\CC([0,1],\R) \quad
    \|f\|_\infty = \sup_{[0,1]}|f| \quad
    \|f\|_1 = \int_0^1|f(t)|\D t\]
    On pose \(F\colon\CC([0,1],\R)\to\CC([0,1],\R)\) l'application définie
    par \(F(f) = f^2\)
    \begin{q}{1}
        Montrer que \(F\) est continue pour la norme \(\|\cdot\|_\infty\)
    \end{q}
    \begin{q}{2}
        Montrer que \(F\) n'est pas continue pour la norme \(\|\cdot\|_1\). On
        pourra s'intéresser à la suite de fonctions \(f_n(t) = \sqrt{n}e^{-nt}\)  
    \end{q}
\end{exo}

\begin{exo}
    Étudier la limite en \((0,0)\) des fonction suivantes :
    \begin{align*}
        f_1(x,y)=\frac{xy}{x^2+y^2} \quad
        f_2(x,y)=\frac{xy^2}{x^2+y^2} \\
        f_3(x,y)=\frac{x^2y}{x^4+y^2} \quad
        f_4(x,y)=\frac{x^3y}{x^4+y^2}
    \end{align*}
    \boxans{
        \begin{enumerate}
            \itt \(f_1\) n'admet pas de limite en \((0,0)\) car la limite selon
            la droite \(x=0\) et celle selon \(x=y\) sont distinctes.
            \itt TODO
            \itt TODO
            \itt TODO
        \end{enumerate}
    }
\end{exo}

\begin{exo}
    Soit \(\left(E,\|\cdot\|\right)\) un espace vectoriel normé.
    \begin{q}{1}
        Montrer que l'intérieur de tout sphère est vide.
        \boxans{Toute sphére peut s'écrire \(N^{-1}(r)\) avec un certain \(r\)
        et \(N\colon e \mapsto \|e-a\|\) avec un certain \(a\). Ce qui fournit le
        résultat de part les résultats de l'exercice 2.}
    \end{q}
    \begin{q}{2}
        Montrer que l'intérieur de tout sous espace vectoriel
        propre de \(E\) est vide.
    \end{q}
\end{exo}

\begin{exo}
    Montrer que le premier quadrant \(Q\) de \(\R^2\) est une partie ouverte de
    \(\R^2\colon\)\[Q=\left\{ \left(x,y\right)\in\R^2\mid x>0 \land y>0 \right\}\]
\end{exo}

\begin{exo}
    Soit \(\left(E, \|\cdot\|\right)\) un espace vectoriel normé et soit \(V\) un
    sous espace vectoriel de \(E\). Montrer que \(\bar{V}\) est un sous espace vectoriel
    de \(E\).
\end{exo}

\begin{exo}
    Pour chacun des sous-ensembles suivants de \(\R^2\) déterminer s'il est ouvert
    ou fermé, ou ni ouvert ni fermé.
    \begin{q}{1}
        \(A=\left\{ \left(x,y\right)\in\R^2\mid 0<|x-y|<1\right\} \)
    \end{q}
    \begin{q}{2}
        \(B=\left\{ \left(x,y\right)\in\R^2\mid |x|<1\land |y|<1\right\} \)
    \end{q}
    \begin{q}{3}
        \(C=\left\{ \left(x,y\right)\in\R^2\mid 0\leq x\leq y\right\} \)
    \end{q}
    \begin{q}{4}
        \(D=\left\{ \left(x,y\right)\in\R^2\mid x^2+y^2\leq 4\right\} \)
    \end{q}
\end{exo}

\begin{exo}
    Soient \(\left(E,\|\cdot\|\right)\) et \(\left(F,\|\cdot\|\right)\)
    deux espaces vectoriels normés et \(f,g\colon E\to F\) deux
    applications continues. Soit \(A\) une partie de \(E\) sur laquelle
    \(f\) et \(g\) coincident. Montrer qu'elles coincident sur \(\bar{A}\).
\end{exo}

\begin{exo}
    On considère l'application \(f:\R^2\to\R\) définie par
    \[f(x,y)=\frac{x^2y^2}{(x^2+y^2)^\alpha}\quad\text{quand ça a
    du sens, sinon } 0\] où \(\alpha\in\R\). Pour quelles valeurs de
    \(\alpha\) a-t-on \(f\) continue ?
\end{exo}

\begin{exo}
    On considère l'application \(f\colon\R^2\to\R\) définie par
    \[f(x,y)=\frac{x^3y}{x^6+(y-x^2)^2}\quad\text{quand ça a du sens,
    sinon }0\]
    \begin{q}{1}
        Montrer que la restriction de \(f\) à toute droite \(D\)
        passant par \((0,0)\) est continue et bornée.
    \end{q}
    \begin{q}{2}
        Montrer que \(f\) est continue en dehors de \((0,0)\).
    \end{q}
    \begin{q}{3}
        L'application \(f\) est-elle continue en \((0,0)\) ?
    \end{q}
\end{exo}

\begin{exo}
    Prolonger par continuité en \((0,0)\), lorsque c'est possible,
    les fonctions suivantes:
    \begin{q}{1}
        \(f_1(x,y)=\frac{|x+y|}{\sqrt{x^2+y^2}}\)
    \end{q}
    \begin{q}{2}
        \(f_2(x,y)=\frac{x^3-y^3}{x^2+y^2}\)
    \end{q}
    \begin{q}{3}
        \(f_3(x,y)=\frac{x^2y}{x^4+y^2}\)
    \end{q}
    \begin{q}{4}
        \(f_4(x,y)=\frac{x^3y}{x^4+y^2}\)
    \end{q}
\end{exo}

\begin{exo}
    Trouver des fonctions continues \(f\colon\R\to\R\) et des parties \(A\) de \(\R\)
    telles que
    \begin{q}{1}
        \(A\) est ouvert et \(f(A)\) n'est pas ouvert.
    \end{q}
    \begin{q}{1}
        \(A\) est fermé et \(f(A)\) n'est pas fermé.
    \end{q}
\end{exo}

\begin{exo}
    Soient \(A,B\) deux parties d'un espace vectoriel normé \(\left(E,\|\cdot\|\right)\)
    Montrer que
    \begin{q}{1}
        Si \(A\subseteq B\) alors \(A^\circ\subseteq B^\circ\)
    \end{q}
    \begin{q}{2}
        \(\left(A\cap B\right)^\circ=A^\circ\cap B^\circ\)
    \end{q}
    \begin{q}{3}
        \(\left(A\cup B\right)^\circ \supseteq A^\circ \cup B^\circ\). Donner un exemple où l'inclusion est stricte.
    \end{q}
\end{exo}

\begin{exo}
    Soient \(A,B\) deux parties d'un espace vectoriel normé \(\left(E,\|\cdot\|\right)\).\\
    On définit \(A+B=\left\{ z\in E\mid \exists (x,y)\in A\times B, z=x+y \right\}\).
    \begin{q}{1}
        Montrer que si \(A\) est ouvert, alors \(A+B\) est ouvert.
    \end{q}
    \begin{q}{2}
        Montrer que \(A=\left\{ \left(x,y\right)\in\R^2\mid xy=1 \right\}\) et
        \(B=\{0\}\times\R\) sont fermées, mais que \(A+B\)\\ ne l'est pas.
    \end{q}
\end{exo}

\begin{exo}
    Soit \(\left(E, \|\cdot\|\right)\) un espace vectoriel normé et notons \(B(0,1)\)
    la boule unité ouverte de \(E\). Montrer que l'application \(h\colon x\mapsto \frac{x}{1+\|x\|}\)
    est un homéomorphisme.
\end{exo}

\begin{exo}
    Soit \(C\) une partie convexe d'un espace vectoriel normé \(E\).
    \begin{q}{1}
        Montrer que l'adhérence de \(C\) est convexe.
    \end{q}
    \begin{q}{2}
        Montrer que l'intérieur de \(C\) est convexe.
    \end{q}
\end{exo}

\begin{exo}
    Soit \(E=\CC([0,1],\R)\) et \(F=\left\{ f\in E\mid f(0)=0 \right\}\)
    \begin{q}{1}
        Montrer que \(F\) est un sous-espace vectoriel de \(E\).
        \boxans{La fonction nulle est bien dans \(F\), qui est stable par combinaison
        linéaire, c'est donc bien un sous-espace vectoriel de \(E\).}
    \end{q}
    \begin{q}{2}
        Montrer que \(F\) est fermé dans \(\left(E, \|\cdot\|_\infty\right)\)
        \boxans{Soit \(g\in E\backslash F\) on note \(\alpha\neq 0\) la valeur de \(g\)
        en 0, montrons que \(f\) ne peut être limite d'une suis d'éléments de \(F\).
        Pour toute fonction \(f\in F\), \(\|g-f\|_\infty\geq g(0)-f(0) = \alpha>0\)
        Ainsi aucune suite de \(F\) ne peut se rapprocher de \(g\) à une distance inférieure
        à \(\alpha\) et donc \(F\) est fermé.}
    \end{q}
    \begin{q}{3}
        Montrer que \(F\) est dense dans \(\left(E, \|\cdot\|_1\right)\)
        \boxans{Soit \(g\in E\), construisons une suite d'éléments de \(F\) qui tend
        vers \(g\) pour la norme \(\|\cdot\|_1\). On pose, pour tout \(n\in\N\) la fonction
        \(f_n : x \mapsto (1-e^{-nx})g\) qui est bien dans \(F\). On remarque que pour tout
        \(n\in\N\) on a \(f_n\leq g\) et que \(\lim_{n\to\infty}f_n=g\) ainsi par convergence
        dominée \(\|f_n\|_1\to\|g\|_1\) ce qui donne la convergence pour la norme \(\|\cdot\|_1\)}
    \end{q}
\end{exo}
\end{document}