\documentclass{report}
\usepackage{../exercices}

\begin{document}
\begin{center}
    \Huge{\textbf{TD 3. App. Linéaires et Matrices}}
\end{center}
\bigskip

\begin{exo}
    Pour chacune des applications \(f_i\) de \(\R^2\) dans \(\R\) ci-dessous,
    montrer que \(f_i\) est différentiable et en calculer la différentielle, le
    gradient et la jacobienne en tous points.
    \begin{enumerate}
        \itt \(f_1(x,y) = xy+2\cos(x)\cos(y)\)
        \boxans{
            \(\frac{\partial f}{\partial x}=y-2\sin(x)\cos(y)
            \quad \frac{\partial f}{\partial y}=x-2\cos(x)\sin(y)\)
            Ainsi la Jacobienne est \(\left(\frac{\partial f}{\partial x}
            \ \frac{\partial f}{\partial y}\right)\)
            Et la différentielle est \(Df(h,k) = Jac_f \cdot (h,k)\)
        }
        \itt \(f_2(x,y) = x^2+y^2\)
        \boxans{La Jacobienne est \(\left(2x\ 2y\right)\)}
        \itt \(f_3(x,y) = e^x\sin(y)\)
        \boxans{La Jacobienne est \(\left(e^x\sin(y)\ e^x\cos(y)\right)\)}
        \itt \(f_4(x,y) = (x^2+y^2)\exp(-xy)\)
    \end{enumerate}
    \boxans{Toutes les fonction qui apparaissent étant \(\CC^\infty\) les fonctions
    de l'exercice sont différentiables.}
\end{exo}

\begin{exo}
    Soit \(f\) l'application de \(\R^2\) dans \(\R\) telle que :
    \[\forall (x,y)\in\R^2, f(x,y)=\mathds{1}_{\R^*}(x^2+y^2)\sin\left(\frac{1}{\sqrt{x^2+y^2}}\right)\]
    \begin{q}{1}
        Prouver que \(f\) est différentiable sur \(\R^2\).
        \boxans{En tout point non nul, les dérivées partielles de la fonctions sont
        \(\CC^1\) comme composition de fonctions \(\CC^1\) usuelles et donc \(f\)
        est différentiable. Pour l'étude en \((0,0)\) on a
        \(-x^2-y^2\leq f \leq x^2+y^2\) sur \(\R^2\) ainsi d'après le théorème
        des gendarmes, \(f\) est de limite nulle en \(0\) et y est donc différentiable,
        le développement de Taylor donne la différentielle, la fonction nulle.}
    \end{q}
    \begin{q}{2}
        Calculer les dérivées partielles d'ordre 1 de \(f\) et sa jacobienne.
        \boxans{Le soin de le faire et de souhaiter mourir est laissé au lecteur.}
    \end{q}
    \begin{q}{3}
        Montrer que les dérivées partielles sont discontinues en \((0,0)\), conclure.
        \boxans{On conclut que \(f\) est \(\CC^1\) mais pas différentiable.}
    \end{q}
\end{exo}

\begin{exo}
    Soit \(f:\R^2\to\R\) telle que \(f(x,y)=\frac{x^3-y^3}{x^2+y^2}\)
    \begin{q}{1}
        Établie que \(f\) possède une dérivée partielle en tout point de \(\R^2\)
        et selon toute direction.
        \boxans{En tout point non nul, l'énoncé est vrai car \(f\) est \(\CC^\infty\)
        en \((0,0)\) la dérivée partielle selon la direction \((u,v)\) est \(f(u,v)\)
        qui n'est pas linéaire en \(t\) donc c'est pas différentiable.}
    \end{q}
    \begin{q}{2}
        \(f\) est elle différentiable sur \(\R^2\) ? On traitera \((0,0)\) à part.
        \boxans{\(f\) est différentiable en tout point non nul comme composée de fonctions
        \(\CC^\infty\) et non différentiable en \((0,0)\)}
    \end{q}
\end{exo}

\begin{exo}
    Soit \(f\) une application différentiable de \(\R^2\) dans \(\R\).
    \begin{q}{1}
        Soient \(\gamma_1\) et \(\gamma_2\) des fonctions dérivables de \(\R\)
        dans \(\R\). Montrer que l'application \(F\) de \(\R\) dans \(\R\) telle
        que, pour tout \(t\in\R, F(t)=f(\gamma_1(t), \gamma_2(t))\) est dérivable
        et que \[\forall t\in\R, F'(t)=
        \frac{\partial f}{\partial x}(\gamma_1(t), \gamma_2(t))\gamma_1'(t) +
        \frac{\partial f}{\partial y}(\gamma_1(t), \gamma_2(t))\gamma_2'(t)\]
        \boxans{
            \[
                F' = \frac{\partial f}{\partial \gamma_1}+\frac{\partial f}{\partial \gamma_2}
                = \frac{\partial \gamma_1}{\partial t}\frac{\partial f}{\partial x}
                + \frac{\partial \gamma_2}{\partial t}\frac{\partial f}{\partial y}
            \]
        }
    \end{q}
    \begin{q}{2}
        Soient \(g_1\) et \(g_2\) deux fonctions de classe \(\CC^1\) de \(\R^2\)
        dans \(\R\). Montrer que l'application \(\Phi:\R^2\to\R\) définie comme
        \(\Phi(u,v)=f(g_1(u,v),g_2(u,v))\) est différentiable et que
        \[\forall (u,v)\in\R^2 \begin{cases}
        \ds\frac{\partial\Phi}{\partial u}(u,v)=\frac{\partial f}{\partial x}(
            g_1(u,v),g_2(u,v))\frac{\partial g_1}{\partial u}(u,v)
            +\frac{\partial f}{\partial y}(
            g_1(u,v),g_2(u,v))\frac{\partial g_2}{\partial u}(u,v)
        \\
        \ds\frac{\partial\Phi}{\partial v}(u,v)=\frac{\partial f}{\partial x}(
            g_1(u,v),g_2(u,v))\frac{\partial g_1}{\partial v}(u,v)
            +\frac{\partial f}{\partial y}(
            g_1(u,v),g_2(u,v))\frac{\partial g_2}{\partial v}(u,v)
        \end{cases}\]
        \boxans{
            \[
                \frac{\partial\Phi}{\partial u}= \frac{\partial f}{\partial g_1|_v}
                + \frac{\partial f}{\partial g_2|_v} =
                \frac{\partial g_1}{\partial u}\frac{\partial f}{\partial x}
                +\frac{\partial g_2}{\partial u}\frac{\partial f}{\partial y}
            \]
        }
    \end{q}
    \begin{q}{3}
        En utilisant la question précédente, calculer les dérivées partielles de
        \(\Phi(x,y)=f(xy, x^2+y^2)\) selon les dérivées partielles de \(f\)
        \boxans{
            \[
                \frac{\partial\Phi}{\partial x}(x,y)=
                y\frac{\partial f}{\partial x}(xy, x^2+y^2) +
                2x\frac{\partial f}{\partial y}(xy, x^2+y^2)
                \quad\quad \frac{\partial\Phi}{\partial x}(x,y) =
                \frac{\partial\Phi}{\partial y}(y,x)
            \]
        }
    \end{q}
\end{exo}

\begin{exo}
    On munit \(\R^n\) de la norme euclidienne usuelle. Soit \(f\colon\R^n\to\R\)
    l'application norme définie par \(f(x)=\|x\|\). Montrer que \(f\)
    est de classe \(\CC^1\) sur \(\R^n\backslash\{0\}\) mais pas différentiable en 0.
    \boxans{\(f\) étant 1-lipchitzienne donc elle est \(\CC^\infty\) en tout
    point qui n'est pas le neutre additif. Soit \(v\) une direction, montrons que \(f\) n'est pas
    différentiable en le neutre pour cette direction.
    \(\lim_{t\to 0}\frac{\|tv\| - \|0\|}{t} = \pm\|v\|\) selon le signe de
    \(t\) donc \(f\) n'y est pas différentiable.}
\end{exo}

\begin{exo}
    Soit \(n\) un entier et \(E=\R_n[X]\).
    \begin{q}{1}
        Montrer que \(f_m:E\to\R\colon P\mapsto \int_0^1 P^m\) est différentiable
        et déterminer sa différentielle pour tout \(P\) et pour tout \(m\).
        \boxans{L'application \(\varphi\colon t\mapsto t\int_0^1 X^m\) est linéaire donc
        différentiable, ainsi par linéarité de l'intégrale, \(f\) est différentiable, sa
        différentielle est \(\sum \frac{1}{k}N^k\cdot\nabla\) où \(N\) est la nilpotente cyclique
        usuelle.}
    \end{q}
    \begin{q}{2}
        Montrer que \(g\colon P\mapsto P'\) est différentiable et donner sa différentielle.
        \boxans{L'application est linéaire en dimension finie, elle est donc continue.
        Elle est différentiable car différentiable sur sa base canonique comme polynôme.
        Sa différentielle est \(\sum kN^k\cdot\nabla\)}
    \end{q}
\end{exo}

\begin{exo}
    On fixe \(n\in\N^*\) et on munit \(\mathcal{M}_n(\R)\) de la norme subordonnée
    associée à la norme euclidienne sur \(\R^n\).
    \begin{q}{1}
        Montrer que pour tout \(p\in\N, f_p\colon \mathcal{M}_n(\R)\ni
        A\mapsto f_p(A)=A^p\) est différentiable et calculer sa différentielle
        \(Df_p(A)\) ainsi que sa norme \(\Vvert Df_p(A)\Vvert\). Prouver que
        la série \(\sum \frac{1}{k!}D_{f_k}(A)\) est normalement convergente.
    \end{q}
    \begin{q}{2}
        En déduire que l'exponentielle est différentiable.
    \end{q}
    \begin{q}{3}
        Calculer \(D(\exp)(A)\)
        \begin{q}{a}
            lorsque \(A=0\)
        \end{q}
        \begin{q}{b}
            lorsque \(A=I_n\)
        \end{q}
    \end{q}
\end{exo}

\begin{exo}
    TODO
\end{exo}

\begin{exo}
    Soient \(\left(E_1,\|\cdot\|_1\right), \left(E_2,\|\cdot\|_2\right),
    \left(F,\|\cdot\|_F\right)\) des \(\R\)-evn et \(E=E_1\times E_2\) le
    \(\R\)-ev muni de la norme \(\|\cdot\|_E\colon(x_1,x_2)\mapsto\|x_1\|_1+
    \|x_2\|_2\). On considère une application bilinéaire \(B\) de \(E\) dans \(F\).
    \begin{q}{1}
        Montrer que \(B\) est continue ssi il existe \(C\in\R^+\) telle que
        \(\|B(x_1,x_2)\|_F\leq C\|(x_1,x_2)\|_E\) pour tout couple d'éléments.
    \end{q}
    On suppose maintenant \(B\) continue.
    \begin{q}{2}
        Montrer que \(B\) est \(\CC^1\) et que pour tout \(x,y\in E\)
        \[DB(x)(y)=B(y_1,x_2)+B(x_1,y_2)\]
    \end{q}
    \begin{q}{3}
        En déduire que \(B\) est \(\CC^2\), de différentielle seconde
        constante. Justifier alors que \(B\) est \(\CC^\infty\).
    \end{q}
    \begin{q}{4}
        Montrer que le produit de matrices dans \(\mnr\) est une application
        différentiable, calculer sa différentielle et sa différentielle seconde.
    \end{q}
\end{exo}

\begin{exo}
    Soient \(n\in\N^*, U\) un ouvert borné non vide de \(\R^n,\ f\) une application
    de \(\bar{U}\) dans \(\R\), continue sur \(U\) et différentiable sur \(U\).
    On suppose \(f\) constante sur la frontière de \(U\). Montrer qu'il existe \(a\in U\)
    tel que la différentielle de \(f\) en \(a\) est identiquement nulle.
    \boxans{Si \(f\) est constante alors tout point de \(U\) convient, sinon
    la différentielle change de signe selon au moins une direction, on réitère
    la recherche sur son orthogonale, éventuellement en dimension \(1\) on a une différentielle
    nulle selon toutes les directions, et donc on a notre point \(a\)}
\end{exo}

\begin{exo}
    Cet exercice propose une meilleure visualisation des concepts en travaillant
    la différentiabilité en dimension 1.
    \begin{q}{1}
        Soient \(I\) un intervalle ouvert non vide de \(\R\) et \(f\) une application
        de \(I\) dans \(\R\). Montrer que \(f\) est dérivable sur \(a\in I\) ssi
        \(f\) est différentiable en \(a\). Si tel est le cas, préciser le lien entre
        le nombre dérivé \(f'(a)\) et la différentielle \(df(a)\).
    \end{q}
    \begin{q}{2}
        On considère l'application valeur absolue sous le nom \(g\colon\R\to\R\).
        \begin{q}{a}
            Montrer que \(g\) est dérivable sur \(R_+^*\) et \(R_-^*\).
        \end{q}
        \begin{q}{b}
            Montrer que \(g\) n'est pas dérivable en \(0\).
            \boxans{Cf. exercice 5}
        \end{q}
    \end{q}
\end{exo}

test \(\displaystyle\int_\R f(x)\D x\).
\end{document}