\documentclass{report}
\usepackage{../../exercices}

\begin{document}
\begin{center}
    \Huge{\textbf{TD 2. Normes et Boules}}
\end{center}
\bigskip

\begin{exo}
    Soit \(E\) un evn et \(a\in E\).
    \begin{q}{1}
        Montrer que le singleton \(\{a\}\) est un fermé de \(E\)
        \boxans{Il existe une unique suite dans \(\{a\}^\N\) qui est
        constante et de limite \(a\), ainsi par caractérisation
        séquencielle on a bien un fermé de \(E\).}
    \end{q}
    \begin{q}{2}
        Montrer que, pour toit \(r>0\), la sphère \(S_r(a)=
        \{x\in E\mid \|x-a\|=r\}\) est un fermé de \(E\).
        \boxans{Il s'agit de l'image réciproque du singleton
        \(\{r\}\) de \(R\) par l'application continue qu'est la norme,
        c'est donc un fermé.}
    \end{q}
    \begin{q}{3}
        Montrer que pour tour \(0<r<R\), l'ensemble \(C_r=
        \{x\in E\mid r<\|x-a\|<R\}\) est un ouvert de \(E\). Quelle
        est son adhérence ?
        \boxans{C'est un ouvert comme \(N^{-1}((r,R))\), son adhérence
        est l'ensemble avec des inégalités larges.}
    \end{q}
\end{exo}

\begin{exo}
    Soit \(\left(E,\|\cdot\|\right)\) un env, montrer que la norme est
    une application continue.
    \boxans{La norme est continue car elle est \(1-lipchtizienne\).}
\end{exo}

\begin{exo}
    Pour chacun des ensembles suivants, déterminer s'il est ouvert ou fermé
    au aucun des deux.
    \begin{enumerate}
        \itt \(A=\{(x,y)\in\R^2\mid y\geq x^2, 0\leq y\leq1\}\)
        \boxans{C'est un fermé car toutes les inégalités sont larges.}
        \itt \(B=\{(x,y)\in\R^2\mid y\geq x^2, 0\leq y<1\}\)
        \boxans{Ouvert d'un côté et fermé de l'autre donc aucun des deux.}
        \itt \(C=\{(x,y)\in\R^2\mid x^2-2x+y^2=0\}\cup\{(x,0);x\in[2;3]\}\)
        \boxans{C'est un fermé comme union de deux fermés (des variétés de dim 1)}
        \itt \(D=\{(x,y)\in\R^2\mid x^2-y^2<1, -1<y<1\}\)
        \boxans{C'est un ouvert comme intersection finie d'ouverts car les
        inégalités sont strictes.}
    \end{enumerate}
\end{exo}

\begin{exo}
    Soit \(f:\R\to\R\) une application continue. Montrer que le graphe
    \(\{(t,f(t))\mid t\in\R\}\) de \(f\) est un fermé de \(\R^2\). Quel
    est son intérieur ?
    \boxans{La continuité indique que le graphe est une variété de dimension
    1 qui a donc un intérieur nulle car l'espace ambient est de dimension 2.}
\end{exo}

\begin{exo}
    TODO
\end{exo}

\begin{exo}
    Étudier la limite en \((0,0)\) des fonction suivantes :
    \begin{align*}
        f_1(x,y)=\frac{xy}{x^2+y^2} \quad
        f_2(x,y)=\frac{xy^2}{x^2+y^2} \\
        f_3(x,y)=\frac{x^2y}{x^4+y^2} \quad
        f_4(x,y)=\frac{x^3y}{x^4+y^2}
    \end{align*}
    \boxans{
        \begin{enumerate}
            \itt \(f_1\) n'admet pas de limite en \((0,0)\) car la limite selon
            la droite \(x=0\) et celle selon \(x=y\) sont distinctes.
            \itt TODO
            \itt TODO
            \itt TODO
        \end{enumerate}
    }
\end{exo}

\begin{exo}
    Soit \(\left(E,\|\cdot\|\right)\) un espace vectoriel normé.
    \begin{enumerate}
        \itt Montrer que l'intérieur de tout sphère est vide.
        \boxans{Toute sphére peut s'écrire \(N^{-1}(r)\) avec un certain \(r\)
        et \(N\colon e \mapsto \|e-a\|\) avec un certain \(a\). Ce qui fournit le
        résultat de part les résultats de l'exercice 2.}
        \itt Montrer que l'intérieur de tout sous espace vectoriel
        propre de \(E\) est vide.
    \end{enumerate}
\end{exo}

\end{document}