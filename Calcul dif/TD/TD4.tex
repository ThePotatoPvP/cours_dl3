\documentclass{report}
\usepackage{../../exercices}

\begin{document}
\begin{center}
    \Huge{\textbf{TD 3. App. Linéaires et Matrices}}
\end{center}
\bigskip

\begin{exo}
    Pour chacune des applications \(f_i\) de \(\R^2\) dans \(\R\) ci-dessous,
    montrer que \(f_i\) est différentiable et en calculer la différentielle, le
    gradient et la jacobienne en tous points.
    \begin{enumerate}
        \itt \(f_1(x,y) = xy+2\cos(x)\cos(y)\)
        \boxans{
            \(\frac{\partial f}{\partial x}=y-2\sin(x)\cos(y)
            \quad \frac{\partial f}{\partial y}=x-2\cos(x)\sin(y)\)
            Ainsi la Jacobienne est \(\left(\frac{\partial f}{\partial x}
            \ \frac{\partial f}{\partial y}\right)\)
            Et la différentielle est \(Df(h,k) = Jac_f \cdot (h,k)\)
        }
        \itt \(f_2(x,y) = x^2+y^2\)
        \boxans{La Jacobienne est \(\left(2x\ 2y\right)\)}
        \itt \(f_3(x,y) = e^x\sin(y)\)
        \boxans{La Jacobienne est \(\left(e^x\sin(y)\ e^x\cos(y)\right)\)}
        \itt \(f_4(x,y) = (x^2+y^2)\exp(-xy)\)
    \end{enumerate}
    \boxans{Toutes les fonction qui apparaissent étant \(\CC^\infty\) les fonctions
    de l'exercice sont différentiables.}
\end{exo}

\begin{exo}
    Soit \(f\) l'application de \(\R^2\) dans \(\R\) telle que :
    \[\forall (x,y)\in\R^2, f(x,y)=\mathds{1}_{\R^*}(x^2+y^2)\sin\left(\frac{1}{\sqrt{x^2+y^2}}\right)\]
    \begin{q}{1}
        Prouver que \(f\) est différentiable sur \(\R^2\).
        \boxans{En tout point non nul, les dérivées partielles de la fonctions sont
        \(\CC^1\) comme composition de fonctions \(\CC^1\) usuelles et donc \(f\)
        est différentiable. Pour l'étude en \((0,0)\) on a
        \(-x^2-y^2\leq f \leq x^2+y^2\) sur \(\R^2\) ainsi d'après le théorème
        des gendarmes, \(f\) est de limite nulle en \(0\) et y est donc différentiable,
        le développement de Taylor donne la différentielle, la fonction nulle.}
    \end{q}
    \begin{q}{2}
        Calculer les dérivées partielles d'ordre 1 de \(f\) et sa jacobienne.
        \boxans{Le soin de le faire et de souhaiter mourir est laissé au lecteur.}
    \end{q}
    \begin{q}{3}
        Montrer que les dérivées partielles sont discontinues en \((0,0)\), conclure.
        \boxans{On conclut que \(f\) est \(\CC^1\) mais pas différentiable.}
    \end{q}
\end{exo}

\begin{exo}
    Soit \(f:\R^2\to\R\) telle que \(f(x,y)=\frac{x^3-y^3}{x^2+y^2}\)
    \begin{q}{1}
        Établie que \(f\) possède une dérivée partielle en tout point de \(\R^2\)
        et selon toute direction.
        \boxans{En tout point non nul, l'énoncé est vrai car \(f\) est \(\CC^\infty\)
        en \((0,0)\) la dérivée partielle selon la direction \((u,v)\) est \(f(u,v)\)
        qui n'est pas linéaire en \(t\) donc c'est pas différentiable.}
    \end{q}
    \begin{q}{2}
        \(f\) est elle différentiable sur \(\R^2\) ? On traitera \((0,0)\) à part.
        \boxans{\(f\) est différentiable en tout point non nul comme composée de fonctions
        \(\CC^\infty\) et non différentiable en \((0,0)\)}
    \end{q}
\end{exo}

\begin{exo}
    Soit \(f\) une application différentiable de \(\R^2\) dans \(\R\).
    \begin{q}{1}
        Soient \(\gamma_1\) et \(\gamma_2\) des fonctions dérivables de \(\R\)
        dans \(\R\). Montrer que l'application \(F\) de \(\R\) dans \(\R\) telle
        que, pour tout \(t\in\R, F(t)=f(\gamma_1(t), \gamma_2(t))\) est dérivable
        et que \[\forall t\in\R, F'(t)=
        \frac{\partial f}{\partial x}(\gamma_1(t), \gamma_2(t))\gamma_1'(t) +
        \frac{\partial f}{\partial y}(\gamma_1(t), \gamma_2(t))\gamma_2'(t)\]
        \boxans{
            \[
                F' = \frac{\partial f}{\partial \gamma_1}+\frac{\partial f}{\partial \gamma_2}
                = \frac{\partial \gamma_1}{\partial t}\frac{\partial f}{\partial x}
                + \frac{\partial \gamma_2}{\partial t}\frac{\partial f}{\partial y}
            \]
        }
    \end{q}
    \begin{q}{2}
        Soient \(g_1\) et \(g_2\) deux fonctions de classe \(\CC^1\) de \(\R^2\)
        dans \(\R\). Montrer que l'application \(\Phi:\R^2\to\R\) définie comme
        \(\Phi(u,v)=f(g_1(u,v),g_2(u,v))\) est différentiable et que
        \[\forall (u,v)\in\R^2 \begin{cases}
        \ds\frac{\partial\Phi}{\partial u}(u,v)=\frac{\partial f}{\partial x}(
            g_1(u,v),g_2(u,v))\frac{\partial g_1}{\partial u}(u,v)
            +\frac{\partial f}{\partial y}(
            g_1(u,v),g_2(u,v))\frac{\partial g_2}{\partial u}(u,v)
        \\
        \ds\frac{\partial\Phi}{\partial v}(u,v)=\frac{\partial f}{\partial x}(
            g_1(u,v),g_2(u,v))\frac{\partial g_1}{\partial v}(u,v)
            +\frac{\partial f}{\partial y}(
            g_1(u,v),g_2(u,v))\frac{\partial g_2}{\partial v}(u,v)
        \end{cases}\]
        \boxans{
            \[
                \frac{\partial\Phi}{\partial u}= \frac{\partial f}{\partial g_1|_v}
                + \frac{\partial f}{\partial g_2|_v} =
                \frac{\partial g_1}{\partial u}\frac{\partial f}{\partial x}
                +\frac{\partial g_2}{\partial u}\frac{\partial f}{\partial y}
            \]
        }
    \end{q}
    \begin{q}{3}
        En utilisant la question précédente, calculer les dérivées partielles de
        \(\Phi(x,y)=f(xy, x^2+y^2)\) selon les dérivées partielles de \(f\)
        \boxans{
            \[
                \frac{\partial\Phi}{\partial x}(x,y)=
                y\frac{\partial f}{\partial x}(xy, x^2+y^2) +
                2x\frac{\partial f}{\partial y}(xy, x^2+y^2)
                \quad\quad \frac{\partial\Phi}{\partial x}(x,y) =
                \frac{\partial\Phi}{\partial y}(y,x)
            \]
        }
    \end{q}
\end{exo}

\begin{exo}
    On munit \(\R^n\) de la norme euclidienne usuelle. Soit \(f\colon\R^2\to\R\)
    l'application norme définie par \(f(x)=\|x\|\). Montrer que \(f\)
    est de classe \(\CC^1\) sur \(\R^n\backslash\{0\}\) mais pas différentiable en 0.
    \boxans{\(f\) étant l'idendité à un scalaire près, elle est \(\CC^\infty\) en tout
    point qui n'est pas le neutre additif. Soit \(v\) une direction, montrons que \(f\) n'est pas
    différentiable en le neutre pour cette direction.
    \(\lim_{t\to 0}\frac{\|tv\| - \|0\|}{t} = \|v\|\) qui n'est pas
    linéaire en \(t\) donc \(f\) n'y est pas différentiable.}
\end{exo}
\end{document}