\documentclass{report}
\usepackage{../../exercices}

\begin{document}
\begin{center}
    \Huge{\textbf{TD 3. App. Linéaires et Matrices}}
\end{center}
\bigskip

\begin{exo}
    Soit \(T\) une forme linéaire sur \(R^n\).
    \begin{q}{1}
        Montrer qu'il existe un unique vecteur \((a_i)\) de \(\R^n\) tel que
        \[\forall x\in\R^n, T(x)=\sum a_ix_i\]
    \end{q}
    \begin{q}{2}
        Justifier que \(T\) est conitnue puis déterminer en fonction des \(a_i\)
        la norme d'opérateur \(\tnorm{T}_p\) de \(T\) subornnée à
        la norme \(\|\cdot\|_p\) : \[\tnorm{T}_p=\sup_{x\in R^n
        ,\|x\|_p=1}|T(x)|\].
    \end{q}
\end{exo}

\begin{exo}
    Soit \(n\in\N^*\). On munit \(\mnr\) de la norme subordonnée à la norme
    \(\|\cdot\|_p\) sur \(\R^n\) :
    \[\forall A\in\mnr, \tnorm{A}_p := \sup{x\in R^n,\|x\|_p=1} \|Ax\|_p\].
    \begin{q}{1}
        Montrer que \(\tnorm{A}_1=\max_{1\leq j\leq n} \sum |a_{i,j}|\)
    \end{q}
    \begin{q}{2}
        Montrer que \(\tnorm{A}_\infty=\max_{1\leq i\leq n} \sum |a_{i,j}|\)
    \end{q}
    \begin{q}{3}
        On rappelle que la matrice symétrique \(^tAA\) est diagonalisable
        sur \(\R\) de valeurs propres positives, on note \(\lambda_A\) sa
        plus grande valeur propre.
        \begin{q}{a}
            Montrer que \(\tnorm{A}_2=\sqrt{\lambda_A}\).
        \end{q}
        \begin{q}{b}
            Exprimer \(\tnorm{A}_2\) en fonction des valeurs propres de
            \(A\) dans le cas où celle-ci est symétrique.
        \end{q}
    \end{q}
    \begin{q}{4}
        \textit{Application :} Soit \(A=\begin{pmatrix}0&1\\0&0\end{pmatrix}\)
        . Calculer \(\tnorm{A}_2\) puis \(\max_{\sigma(A)}|\lambda|\)
    \end{q}
\end{exo}

\begin{exo}
    Soit \(n\geq 2\).
    \begin{q}{1}
        Vérifier que \(\|\cdot\|_1\) et \(\|\cdot\|_\infty\) sont des normes
        sur \(\mnr\). Sont-elles des normes d'algèbre ?
    \end{q}
    \begin{q}{2}
        Montrer que pour tout norme \(N\) sur \(\mnr\) il existe une
        constante \(k>0\) telle que \(kN\) soit une norme d'algèbre.
    \end{q}
    \begin{q}{3}
        Toute norme d'algèbre sur \(\mnr\) est-elle une norme subordonnée ?
        \boxans{Toute norme sur \(\mnr\) est une norme sur \(E=R^n\) qui est
        subordonnée d'une norme sur le dual donc oui.}
    \end{q}
\end{exo}

\begin{exo}
    Étudier la continuité et, lorsque c'est le cas, déterminer la norme des
    applications linéaires suivantes : \textit{(elles sont linéaire en dimension finie donc continues)}
    \begin{enumerate}
        \itt \(X\mapsto X^2\)
        \itt \(X \mapsto AX\)
        \itt \(X\mapsto X^2\)
        \itt \((X,Y)\mapsto XY\)
    \end{enumerate}
\end{exo}

\begin{exo}
    On munit \(E=\C[X]\) de la norme \(\|P\|_1 = \sum |a_k|\). Étudier la continuité
    et si possible la norme des application linéaires suivantes :
    \[ \varphi_1\colon P \mapsto P(x_0) \quad \varphi_2\colon P \mapsto \int_0^1 P
    \quad \varphi_3\colon P \mapsto P'\]
\end{exo}

\end{document}