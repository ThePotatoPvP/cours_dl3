\documentclass{report}
\usepackage{../../exercices}

\begin{document}
\begin{center}
    \Huge{\textbf{TD 1. Normes et Boules}}
\end{center}
\bigskip
\begin{exo}
    Soient \(n \in\N^*\), pour tout \(x=(x_k)_{1\leq i\leq n} \in \R^n\) on définit:
    \[\|x\|_1 = \sum_{i=0}^n |x_i| \quad \|x\|_2=\left(\sum_{i=0}^n |x_i|^2\right)^\frac{1}{2} \quad \|x\|_\infty = \sup_{1\leq i \leq n}\left|x_i\right|\]

    \begin{q}{1}
        Démontrer que \(\|\cdot\|_1, \|\cdot\|_2, \|\cdot\|_\infty\) sont des normes sur \(\R^n\).
        \boxans{
            \begin{enumerate}[leftmargin=1cm]
                \item[S] \begin{enumerate}
                    \itt Supposons \(\|x\|_1=0\) alors pour tout \(i\) on a \(|x_i|=0\Rightarrow x_i=0\)
                    \itt Supposons \(\|x\|_2=0\) alors pour tout \(i\) on a \(|x_i|=0\Rightarrow x_i=0\)
                    \itt Supposons \(\|x\|_\infty=0\) alors pour tout \(i\) on a \(|x_i|=0\Rightarrow x_i=0\)
                \end{enumerate}
                \item[H] \begin{enumerate}
                    \itt \(\|\lambda x\|_1 = \sum \|\lambda x_i\| = |\lambda|\sum\|x_i\| = |\lambda|\|x\|_1\)
                    \itt \(\|\lambda x\|_2 = \sum \left(\|\lambda x_i\|^2\right)^\frac{1}{2} = |\lambda|\left(\sum\|x_i\|^2\right)^\frac{1}{2} = |\lambda|\|x\|_1\)
                    \itt \(\|\lambda x\|_\infty =  \sup_{1\leq i \leq n}\left|\lambda x_i\right| = |\lambda|\|x\|\)
                \end{enumerate}
                \item[IT] Les inégalités triangulaires s'obtiennent trivialement du fait que \(|\cdot|\) est une norme.
            \end{enumerate}
        }
    \end{q}
    \begin{q}{2}
        Décrire géométriquement la boule fermée de centre \(0\) et de rayon \(1\) pour chaque norme.
        \boxans{
            \begin{enumerate}
                \itt Pour \(\|\cdot\|_1\) la boule a une forme d'octaèdre.
                \itt Pour \(\|\cdot\|_2\) la boule a une forme de boule usuelle.
                \itt Pour \(\|\cdot\|_\infty\) la boule a une forme d'hypercube.
            \end{enumerate}
        }
    \end{q}
    \begin{q}{3}
        Soit \(x\in\R^n\), prouver les inégalités :
        \[\|x\|_2\leq\|x\|_1\leq\sqrt{n}\|x\|_2\quad\|x\|_\infty\leq\|x\|_1\leq n\|x\|_\infty\quad\|x\|_\infty\leq \|x\|_2\leq\sqrt{n}\|x\|\]
        \boxans{
            \begin{enumerate}
                \itt En regardant les boules, l'inclusions donne l'inégalité de gauche, la plus grande distance entre les boules se situant au milieu des arêtes on a la deuxième inégalité pour \(x=\left(\frac{1}{n}\right)_k\).
                \itt En regardant encore une fois les boules, l'inégalité de gauche est évidente, celle de droite s'obtient pareillement au milieu des arêtes de la boules incluses pour le vecteur explicité plus haut.
                \itt Le dernier encadrement provient directement des deux premiers.
            \end{enumerate}
        }
    \end{q}
    \begin{q}{4}
        Prouver grâce à des exemples bien choisis que les inégalités ci-dessus sont optimales.
        \boxans{Déjà repondu à la question précédente.}
    \end{q}
\end{exo}

\begin{exo}
    TODO
\end{exo}

\begin{exo}
    On cherche dans cet exercice à démontrer des résultats classiques sur \(\K^n\)
    \begin{q}{1}
        Soit \(\left(p,q\right)\) un couple de réels conjugués positifs.
        \begin{q}{a}
            Prouver l'inégalité de \textsc{Young} : \(ab\leq \frac{1}{p}a^p + \frac{1}{q}b^q\)
            \boxans{\(\ln\left(ab\right)=\frac{1}{p}\ln\left(a^p\right)+\frac{1}{q}\ln\left(b^q\right)\leq\ln\left(\frac{a^p}{p}+\frac{b^q}{q}\right)\)
            où la concavité du logarithme donne l'inégalité et la croissance donne le résultat souhaité.}
        \end{q}
        \begin{q}{b}
            Soit \(\left(x,y\right)\in\K^n\), démontrer l'inégalité de \textsc{Hölder}.
            \[\left|\langle x, y\rangle\right| \leq \sum |x_ky_k| \leq \|x\|_p\|y\|_q\]
            \boxans{On commence à remarquer que si \(\|x\|_p=\|y\|_q=1\) alors \(\langle x,y\rangle \leq 1\) qui provient directement de la question précédente en appliquant l'inégalité à chaque membre de la somme.
            On obtient alors la partie droite de l'inégalité de \textsc{Hölder} en posant \(x'=\frac{x}{\|x\|_p}\) et \(y'=\frac{y}{\|y\|_q}\). La partie gauche s'obtient de l'inégalité triangulaire sur la valeur absolue.
            On a ainsi une formule générale qui se ramène à \textsc{Cauchy-Schwarz} pour \(p=2\).}
        \end{q}
        \begin{q}{c}
            Utiliser la question précédente pour démontrer la formule variationelle :
            \[\forall x \in \K^n \quad \|x\|_p=\sup\{|\sum x_iz_i|\ z\in\K^n, \|z\|_q\leq 1\}\]
            \boxans{On commence par observer que \(\|\cdot\|\geq \|\cdot\|_p\) pour \(p\geq 1\) ce qui implique \(\|x\|_p\leq\|x\|\)
            On considère ensuite \(z\) sur la droite \(\lambda x\) tel que \(\|z\|=1\), alors \(|\langle x,z\rangle|= \left|\sum x_iz_i\right|=\|x\|\).
            On peut prendre \(z'=\frac{z\|x\|_p}{\|x\|}\) qui vérifie bien \(\|z'\|_q\leq 1\) car \(\|\cdot\|_q\leq\|\cdot\|\).
            \(|\langle x, z'\rangle|=\left|\sum x_i z'_i\right| = \|x\|_p\). L'inégalité de \textsc{Hölder} nous assure que cette valeur est le sup.}
        \end{q}
        \begin{q}{d}
            Démontrer que \(\|\cdot\|_p\) vérifie l'inégalité triangulaire (inégalité de \textsc{Minkowski})
            \boxans{Ce résultat s'obtient à l'aide de l'inégalité triangulaire sur la valeur absolue dans la formule variationelle.}
        \end{q}
        \begin{q}{e}
            Démontrer alors que l'application \(\|\cdot\|_p\) est une norme pour \(p\geq 1\).
            \boxans{La séparation et la positivité sont triviales, l'homogénéité provient de la linéarité de la somme et l'inégalité triangulaire de la question précédente.}
        \end{q}
    \end{q}
    \begin{q}{2}
        Démontrer que pour \(p\in ]0,1[\) l'application \(\|\cdot\|_p\) n'est pas une norme.
        \boxans{L'application \(t\mapsto t^p\) est maintenant concave et plus
        convexe, elle est donc au dessus de ses cordes et ainsi ne vérifie pas
        l'inégalité triangulaire.}
    \end{q}
    \begin{q}{3}
        On rappelle que \(\|\cdot\|_\infty\) est une norme sur \(\K^n\)
        \begin{q}{a}
            Justifier que la formule variationelle est encore vraie si \(p=1\) ou \(p=+\infty\).
            \boxans{On commence par remarquer l'inégalité de \textsc{Young} est vérifiée pour \(p=1, q=+\infty\) ce qui, de par la suite de questions, donne le résultat souhaité.}
        \end{q}
        \begin{q}{b}
            Montrer que, pour tout \(p\in [1,+\infty]\) et \(x\in\K^n\) : \(\|x\|_\infty\|\leq\|x\|_p\leq n^\frac{1}{p}\|x\|_\infty\)
            \boxans{Le plus grand écart entre les boules se situe aux coins
            de la boule de \(\|\cdot\|_\infty\) donc pour \(x=(1,1,\dots)\)
            or pour ce \(x\) on a \(\|x\|_p=n^\frac{1}{p}\) ce qui livre
            l'inégalité de droite, celle de gauche s'obtenant par inclusion
            de boule.}
        \end{q}
        \begin{q}{c}
            Soit \(x\in\K^n\) montrer que l'application \(p\mapsto \|x\|_p\) admet une limite en \(+\infty\) égale à \(\|x\|_\infty\)
            \boxans{Ce résultat s'obtient en appliquant le théorème des gendarmes au résultat précédent.}
        \end{q}
        \begin{q}{d}
            Prouver que les normes \(\left(\|\cdot\|_p\right)_{p\geq 1}\) sont deux à deux équivalentes.
            \boxans{Les normes sont équivalentes car on est en dimension finie.}
        \end{q}
    \end{q}
\end{exo}

\begin{exo}
    Soient \(n\in\N^*\) et \(\mnr\) le \(\R\)-ev des matrices de taille \(n\times n\)
    \begin{q}{1}
        On note \(\left\langle\cdot,\cdot\right\rangle\) l'application de \(\mnr^2\)
        dans \(\R\) telle que \(\langle A,B\rangle=\Tr(^tAB)\). Démontrer que l'application
        ainsi définie est un produit scalaire sur \(\mnr\) de norme associée \(\|\cdot\|\).
        \boxans{L'application est bien multilinéaire et définie positive comme somme de
        carrés de réels, on a donc bien un produit scalaire auquel on peut associer une norme
        à l'aide des formules de polarisation usuelle.}
    \end{q}
    \begin{q}{2}
        Démontrer que l'application \(\|\cdot\|_\infty : A \mapsto \sup |a_{i,j}|\)
        est une norme sur \(\mnr\).
        \boxans{L'application est trivialement homogène et séparée, l'inégalité triangulaire
        provient directement de celle sur la valeur absolue.}
    \end{q}
    \begin{q}{3}
        Montrer que les normes sont équivalentes et trouver les meilleures constantes pour l'encadrement.
        \boxans{Les normes sont équivalentes avec \(\|\cdot\|_\infty\leq\|\cdot\|\leq n^2\|\cdot\|\infty\)
        cer bornes sont atteintes à gauche avec les \(E_{i,j}\) et à droite avec \(J_n\) la matrice
        ne contenant que des \(1\).}
    \end{q}
\end{exo}

\begin{exo}
    Soient \(V\) un \(\K\)-ev et une application \(N\) de \(V\) dans \(\R_+\) telle que :
    \begin{enumerate}
        \itt Pour tout \(\left(t,v\right)\in\K\times V, N(tv)=|t|N(v)\)
        \itt Pour tout \(v\in V, N(v)=0 \Leftrightarrow v = 0_V\)
    \end{enumerate}
    On note \(B_N = \{v\in V\mid N(v)\leq 1\}\). Prouver que \(N\) est une norme si et seulement si \(B_N\) est convexe.
    \boxans{
        \begin{enumerate}
            \itt Supposons \(B_n\) convexe alors soient \(x,y\in B_n\) tels que \(N(x)=N(y)=1\)
            alors \(\frac{x+y}{2}\in B_n\) par convexité donc \(N(x+y)\leq 2 = N(x)+N(y)\).
            L'absolue homogénéité de \(N(\cdot)\) assure alors que l'application
            vérifie l'inégalité triangulaire et est donc une norme
            \itt La réciproque est évidente, si \(N(\cdot)\) est une norme alors elle vérifie
            l'inégalité triangulaire et \(B_N\) est alors convexe de par l'argument ci-dessus.
        \end{enumerate}
    }
\end{exo}

\begin{exo}
    \boxans{Cet exercice étant majoritairement graphique, aucune correction ne sera proposée ici}
\end{exo}

\begin{exo}
    Soit \(E\) un \(\K\)-ev normé et \(u\in\mathcal{L}(E)\).
    \begin{q}{1}
        L'application \(\|\cdot\|\circ u\) est-elle une norme sur \(E\) ?
        \boxans{On remarque que la seule propriété qui pourrait poser problème est la séparation,
        il faut donc que \(u\) soit injective afin d'avoir une norme. La linéarité permet de
        s'assurer que l'inégalité triangulaire ne pose pas problème si \(u\) est in isomorphisme.}
    \end{q}
    \begin{q}{2}
        On choisit par exemple \(E=\left(\R^2,\|\cdot\|_1\right)\) et \(u\) la rotation
        d'angle \(\frac\pi4\). Expliciter la norme sur \(\R^2\) ainsi obtenue.
        \boxans{La norme obtenue est \(\|\cdot\|_\star : \left(x,y\right) \mapsto
        |\arctan\left(\frac{y}{x}+\frac\pi4\right)|\|(x,y)\|_1\)}
    \end{q}
\end{exo}

\begin{exo}
    Soit \(n\in\N^*\) et \(\left(a_i\right)_{k\leq n} \in\R_+^n\) on note \(M_1\) et \(M_\infty\)
    les applications de \(\R^n\) dans \(\R\) telles que:
    \[ M_1\left(x\right)=\sum_k a_k\left|x_k\right|\quad\text{et}\quad M_\infty\left(x\right)
    =\sup a_k\left|x_k\right|\]
    \begin{q}{1}
        Les applications définies sont elles des normes ? Si non, sous quelles conditions le deviennent-elles ?
        \boxans{On commence par remarquer que si l'un des \(a_k\) est nul on peut créer un \(x\) tel que \(M\)
        ne vérifie pas la séparation. Si tous sont non nuls, on a bien la séparétion, l'homogénéité
        provient de la linéarité de la somme et su passage au sup et l'inégalité triangulaire de celle
        sur la valeur absolue.}
    \end{q}
    \begin{q}{2}
        Soit \(C:\R^2\to\R\) telle que \(C(x)=|x_1|\mathbb{I}_{x_1\neq 0}+|x_2|\mathbb{I}_{x_2\neq 0}\)
        \boxans{Si \(x_1\neq 0\) et \(x_2 \neq 0\) alors \(C(x)=\|x\|_1\), les deux applications
        coincident aussi lorsqu'au moins une des deux variable est nulle, on a donc \(C(\cdot)=\|\cdot\|_1\).}
    \end{q}
\end{exo}

\begin{exo}
    Soit \(X\) un ensemble non vide. On note \(\R^X\) le \(R\)-ev des applications de
    \(X\) dans \(\R\) et \(l^\infty(X,\R)\) l'ensemble des applications bornée de \(X\)
    dans \(\R\).
    \begin{q}{1}
        Prouver que \(l^\infty(X,\R)\) est un sev de \(\R^X\)
        \boxans{L'espace étudié contient bien la fonction nulle et est stable par combinaison
        linéaire sans le moindre problème, c'est donc bien un sev.}
    \end{q}
    \begin{q}{2}
        Démontrer que \(\|\cdot\|_\infty\) est une norme sur \(l^\infty(X,\R)\).
        \boxans{L'homogénéité et la positivité étant évidentes on commence la séparation,
        supposons \(\|f(x)\|=0\) alors \(f(x)=0\) pour tout \(x\in X\) ainsi \(f\) est nulle.
        Pour ce qui est de l'inégalité triangulaire elle est une conséquance immédiate
        de celle sur la valeur absolue en chaque \(x\).}
    \end{q}
\end{exo}

\begin{exo}
    Soient \(n\in\N^*\), on pose \(\|\cdot\|_\infty : \mnk \to \K\colon
    A \mapsto \sup_{k\leq n}\sum_{l\leq n} |a_{k,l}|\) la plus grande somme
    sur une colonne.
    \begin{q}{1}
        Prouver que \(\|\cdot\|_\infty\) est une norme d'algèbre sur \(\mnk\).
        \boxans{Si \(\|A\|_\infty=0\) alors chaque colonne est pleine de \(0\) donc la matrice est nulle.
        L'application est trivialement positive et homogène. L'inégalité triangulaire
        provient encore une fois de celle de la valeur absolue. L'application est donc bien une norme.
        \[\|AB\|_\infty=\sup_k\sum_l[AB]_{k,l}=\sup_k\sum_l\sum_i[A]_{k,i}[B]_{i,l}
        =\sup_k\sum_i[A]_{k,i}\sum_l[B]_{i,l}=\|A\|_\infty\sum_l [B]_{i,l}\]
        or on a \(\sum_l [B]_{i,l}\leq\|B\|_\infty\) par définition de l'application.
        Mézalor \(\|\cdot\|_\infty\) est bien une norme d'algèbre.
        \textit{(on a pu intervertir les sommes car elles sont finies)}}
    \end{q}
    \begin{q}{2}
        Soit \(A\in\mnk\), démontrer que pour \(\lambda\) valeur propre de \(A\)
        on a \(|\lambda|\leq\|A\|_\infty\).
        \boxans{Si \(\lambda=0\) alors l'inégalité est évidente, sinon on peut écrire
        \(A = \begin{pmatrix}B&0\\0&0\end{pmatrix}\) avec \(B\) inversible.
        On a donc que l'image et le noyau de \(A\) sont en somme directe, ce qui
        permet de diagonaliser \(A\) de sorte que \(B\) soit une diagonale de valeurs propres.
        ainsi on trouve \(\|A\|_\infty\) est le sup des valeurs propres.
        On a bien l'inégalité souhaitée.}
    \end{q}
\end{exo}

\begin{exo}
    Étudier sur des exemples si toute norme \(N\) sur le \(\R\)-ev \(\R^2\)
    vérifie la propriété suivante : \(\left(|x_1|\leq|x_2|\right)\land
    \left(|y_1|\leq|y_2|\right)\Rightarrow N\left(x_1,y_1\right)\leq N\left(x_2,y_2\right)\)
\end{exo}

\begin{exo}
    Soit \(\left(a,b\right)\in\R^2\) tel que \(a<b\). On note \(J=[a,b]\).
    \begin{q}{1}
        Justifier que \(\CC(J,\K)\) est un sev de \(l^\infty(J,K)\)
        \boxans{On est sur un fermé, ainsi d'après \textsc{Bolzano-Weiestrass} toute
        fonction continue est bornée et atteint ses bornes. On a donc l'inclusion,
        la stabilité par combinaison linéaire est évidente.}
    \end{q}
    \begin{q}{2}
        Prouver que \(\|\cdot\|_{l^1}\) est une norme sur \(\CC(J,\K)\)
        \boxans{La continuité et la positivité de \(|f(t)|\) donnent là séparation,
        l'homogénéité provient de la linéarité de l'intégrale, l'inégalité triangulaire
        s'obtient en poussant la craie avec celle de la valeur absolue.}
    \end{q}
    \begin{q}{3}
        Comparer la norme \(l^1\) et la norme infinie.
        \boxans{\[\|f\|_{l^1}\leq\int_J |f(t)| \leq \int_J \sup_J|f| \leq (b-a)\|f\|_\infty\]
        On ne peut cependant avoir d'inégalité dans l'autre sens, les deux normes ne sont pas équivalentes,
        en effet la suite de fonctions \(g_n: x\mapsto \left(x-a-\frac{1}{n}\right)^{-n}\)
        vérifie \(\lim_{n\to\infty} \frac{\|g_n\|_\infty}{\|g_n\|_{l^1}}=+\infty\)}
    \end{q}
\end{exo}

\begin{exo}
    Dans cet exercice on étudiera le lien entre des normes et leur boule unité.
    \begin{q}{1}
        Soit \(E\) un \(\K\)-ev, on note \(B(N)=\left\{ x\in E\mid N(x)<1 \right\}\)
        la boule unité ouverte associée à \(N\). Soient \(N_1\) et \(N_2\) des normes sur \(E\),
        montrer que \(N_1\leq N_2\) si et seulement si \(B(N_2)\subseteq B(N_1)\).
    \end{q}
    Pour toit \(p\in[1,+\infty)\) on note \(B_n^p=\left\{ v\in\R^n\mid \|v\|_p<1 \right\}\)
    c'est à dire la boule ouverte de centre \(0\) et rayon \(1\) pour chaque norme \(p\).
    \begin{q}{2}
        Dessiner sur le plan euclidien les ensembles \(B_2^p\) pour \(p\in\{1,2,\infty\}\).
        Décrire au mieux l'évolution de \(B_2^p\) lorsque \(p\) croît.
    \end{q}
    \begin{q}{3}
        Soient \(\left(p,p'\right)\in[1,+\infty]^2\) prouver l'équivalence
        \(B_n^p\subseteq B_n^{p'} \Leftrightarrow p\leq p'\)
    \end{q}
    \begin{q}{4}
        Soit \(v\in B_n^\infty\backslash B_n^1\) montrer qu'il existe \(p\in(1,+\infty)\) tel que \(\|v\|_p\)
    \end{q}
\end{exo}

\end{document}