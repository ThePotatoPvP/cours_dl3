\documentclass{report}
\usepackage[T1]{fontenc}
\usepackage[utf8x]{inputenc}
\usepackage{lmodern}
\usepackage{amsmath,amsthm,amsfonts,amssymb}
\usepackage{graphicx}
\usepackage[shortlabels]{enumitem}
\usepackage{xcolor}
\usepackage{accents}
\usepackage{titlesec}
\usepackage{etoolbox}
\usepackage{bookmark}
\usepackage{stmaryrd}
\usepackage{fancyhdr}
\usepackage[margin=25mm]{geometry}
\usepackage[frenchb]{babel}
\usepackage{changepage}
\usepackage{listings}
\usepackage[most,breakable,listings]{tcolorbox}

\date{\today}
\author{Corentin Sallin \thanks{Cours de L3}}
\title{Calcul Différentiel et topologie}


\renewcommand{\theenumi}{\Alph{enumi}}

\newcommand{\N}{\mathbb{N}}
\newcommand{\Z}{\mathbb{Z}}
\newcommand{\R}{\mathbb{R}}
\newcommand{\Q}{\mathbb{Q}}
\newcommand{\C}{\mathbb{C}}
\newcommand{\K}{\mathbb{K}}
\renewcommand{\L}{\mathcal{L}}
\newcommand{\M}{\mathcal{M}}
\newcommand{\T}{\mathcal{T}}
\newcommand{\Diag}{\mathcal{D}}
\newcommand{\curlyv}{\mathcal{V}}

\newcommand{\esp}{\mathbb{E}}
\newcommand{\proba}{\mathbb{P}}


\newcommand{\CC}{\mathcal{C}}
\newcommand{\Int}{\mathrm{Int}}
\newcommand{\id}{\mathrm{id}}
\newcommand{\eps}{\varepsilon}
\newcommand{\mnr}{\mathcal{M}_n(\R)}
\newcommand{\mnk}{\mathcal{M}_n(\K)}
\newcommand{\glnr}{\mathrm{GL}_n(\R)}
\newcommand{\glnk}{\mathrm{GL}_n(\K)}

\newcommand{\D}{\mathop{}\!\mathrm{d}}
\newcommand{\ds}{\displaystyle}
\newcommand*{\ensemble}[3][]{#1\{ #2 \mid #3 #1\}}
\newcommand{\Vvert}{\vert\kern-0.25ex\vert\kern-0.25ex\vert}
\newcommand{\tnorm}[1]{\Vvert #1 \Vvert}

\DeclareMathOperator{\Ker}{Ker}
\DeclareMathOperator{\Isom}{Isom}
\DeclareMathOperator{\Tr}{Tr}

\pagestyle{fancy}
\fancyhf{}
\fancyhead[L]{Corentin Sallin}
\fancyhead[R]{}
\fancyfoot[L]{\leftmark}
\fancyfoot[R]{Page : \thepage}

\renewcommand{\headrulewidth}{2pt}
\renewcommand{\footrulewidth}{1pt}


\theoremstyle{definition}
\newtheorem{exo}{Exercice}
%\newtheorem{q}{\(\quad\)}[exo]

\newtheorem{theorem}{Théorème}[section]
\newtheorem{lemma}[theorem]{Lemme}
\newtheorem{definition}[theorem]{Définition}
\newtheorem{lemme}[theorem]{Lemme}
\newtheorem*{corollary}{Corollaire}

\newtheorem{proposition}{Proposition}[theorem]
\newtheorem{rk}[proposition]{Remarque}
\newtheorem*{example}{Exemple}

\definecolor{main1white} {RGB}{121, 129, 134}
\definecolor{main1white2} {RGB}{204, 204, 204}

\newenvironment{q}[1]{
    \begin{adjustwidth}{1cm}{}
    \textbf{#1} : }{
    \end{adjustwidth}}

\titleformat{\chapter}[display]
    {\normalfont\bfseries}{}{0pt}{\Huge}

\patchcmd{\chapter}{\thispagestyle{plain}}{\thispagestyle{fancy}}{}{}

\newcommand{\boxans}[1]{
    \begin{tcolorbox}[
            breakable,
            enhanced,
            interior style      = {
                left color      = main1white2!65!gray!8,
                middle color    = main1white2!50!gray!7,
                right color     = main1white2!35!gray!6
            },
            %borderline north    = {.3pt}{0pt}{main1white!10},
            %borderline south    = {.3pt}{0pt}{main1white!10},
            frame hidden,
            borderline west     = {2pt}{0pt}{main1white!30},
            sharp corners       = downhill,
            arc                 = 0 cm,
            boxrule             = 0 cm,
            %nobeforeafter,
            %before={},
            %nobeforeSTYLE,
            %noafterSTYLE,
            %after=\par\nointerlineskip
            %source=remy
        ]
        #1
    \end{tcolorbox}
}

\newcommand{\itt}{\item[\(\triangleright\)]}

\begin{document}

\begin{exo}
    Soient \(n \in\N^*\), pour tout \(x=(x_k)_{1\leq i\leq n} \in \R^n\) on définit:
    \[\|x\|_1 = \sum_{i=0}^n |x_i| \quad \|x\|_2=\left(\sum_{i=0}^n |x_i|^2\right)^\frac{1}{2} \quad \|x\|_\infty = \sup_{1\leq i \leq n}\left|x_i\right|\]

    \begin{q}{1}
        Démontrer que \(\|\cdot\|_1, \|\cdot\|_2, \|\cdot\|_\infty\) sont des normes sur \(\R^n\).
        \boxans{
            \begin{enumerate}[leftmargin=1cm]
                \item[S] \begin{enumerate}
                    \itt Supposons \(\|x\|_1=0\) alors pour tout \(i\) on a \(|x_i|=0\Rightarrow x_i=0\)
                    \itt Supposons \(\|x\|_2=0\) alors pour tout \(i\) on a \(|x_i|=0\Rightarrow x_i=0\)
                    \itt Supposons \(\|x\|_\infty=0\) alors pour tout \(i\) on a \(|x_i|=0\Rightarrow x_i=0\)
                \end{enumerate}
                \item[H] \begin{enumerate}
                    \itt \(\|\lambda x\|_1 = \sum \|\lambda x_i\| = |\lambda|\sum\|x_i\| = |\lambda|\|x\|_1\)
                    \itt \(\|\lambda x\|_2 = \sum \left(\|\lambda x_i\|^2\right)^\frac{1}{2} = |\lambda|\left(\sum\|x_i\|^2\right)^\frac{1}{2} = |\lambda|\|x\|_1\)
                    \itt \(\|\lambda x\|_\infty =  \sup_{1\leq i \leq n}\left|\lambda x_i\right| = |\lambda|\|x\|\)
                \end{enumerate}
                \item[IT] Les inégalités triangulaires s'obtiennent trivialement du fait que \(|\cdot|\) est une norme.
            \end{enumerate}
        }
    \end{q}
    \begin{q}{2}
        Décrire géométriquement la boule fermée de centre \(0\) et de rayon \(1\) pour chaque norme.
        \boxans{
            \begin{enumerate}
                \itt Pour \(\|\cdot\|_1\) la boule a une forme d'octaèdre.
                \itt Pour \(\|\cdot\|_2\) la boule a une forme de boule usuelle.
                \itt Pour \(\|\cdot\|_\infty\) la boule a une forme d'hypercube.
            \end{enumerate}
        }
    \end{q}
    \begin{q}{3}
        Soit \(x\in\R^n\), prouver les inégalités :
        \[\|x\|_2\leq\|x\|_1\leq\sqrt{n}\|x\|_2\quad\|x\|_\infty\leq\|x\|_1\leq n\|x\|_\infty\quad\|x\|_\infty\leq \|x\|_2\leq\sqrt{n}\|x\|\]
        \boxans{
            \begin{enumerate}
                \itt En regardant les boules, l'inclusions donne l'inégalité de gauche, la plus grande distance entre les boules se situant au milieu des arêtes on a la deuxième inégalité pour \(x=\left(\frac{1}{n}\right)_k\).
                \itt En regardant encore une fois les boules, l'inégalité de gauche est évidente, celle de droite s'obtient pareillement au milieu des arêtes de la boules incluses pour le vecteur explicité plus haut.
                \itt Le dernier encadrement provient directement des deux premiers.
            \end{enumerate}
        }
    \end{q}
    \begin{q}{4}
        Prouver grâce à des exemples bien choisis que les inégalités ci-dessus sont optimales.
        \boxans{Déjà repondu à la question précédente.}
    \end{q}
\end{exo}

\begin{exo}
    TODO
\end{exo}

\begin{exo}
    On cherche dans cet exercice à démontrer des résultats classiques sur \(\K^n\)
    \begin{q}{1}
        Soit \(\left(p,q\right)\) un couple de réels conjugués positifs.
        \begin{q}{a}
            Prouver l'inégalité de \textsc{Young} : \(ab\leq \frac{1}{p}a^p + \frac{1}{q}b^q\)
            \boxans{\(\ln\left(ab\right)=\frac{1}{p}\ln\left(a^p\right)+\frac{1}{q}\ln\left(b^q\right)\leq\ln\left(\frac{a^p}{p}+\frac{b^q}{q}\right)\)
            où la concavité du logarithme donne l'inégalité et la croissance donne le résultat souhaité.}
        \end{q}
        \begin{q}{b}
            Soit \(\left(x,y\right)\in\K^n\), démontrer l'inégalité de \textsc{Hölder}.
            \[\left|\langle x, y\rangle\right| \leq \sum |x_ky_k| \leq \|x\|_p\|y\|_q\]
            \boxans{On commence à remarquer que si \(\|x\|_p=\|y\|_q=1\) alors \(\langle x,y\rangle \leq 1\) qui provient directement de la question précédente en appliquant l'inégalité à chaque membre de la somme.
            On obtient alors la partie droite de l'inégalité de \textsc{Hölder} en posant \(x'=\frac{x}{\|x\|_p}\) et \(y'=\frac{y}{\|y\|_q}\). La partie gauche s'obtient de l'inégalité triangulaire sur la valeur absolue.
            On a ainsi une formule générale qui se ramène à \textsc{Cauchy-Schwarz} pour \(p=2\).}
        \end{q}
        \begin{q}{c}
            Utiliser la question précédente pour démontrer la formule variationelle :
            \[\forall x \in \K^n \quad \|x\|_p=\sup\{|\sum x_iz_i|\ z\in\K^n, \|z\|_q\leq 1\}\]
            \boxans{On commence par observer que \(\|\cdot\|\geq \|\cdot\|_p\) pour \(p\geq 1\) ce qui implique \(\|x\|_p\leq\|x\|\)
            On considère ensuite \(z\) sur la droite \(\lambda x\) tel que \(\|z\|=1\), alors \(|\langle x,z\rangle|= \left|\sum x_iz_i\right|=\|x\|\).
            On peut prendre \(z'=\frac{z\|x\|_p}{\|x\|}\) qui vérifie bien \(\|z'\|_q\leq 1\) car \(\|\cdot\|_q\leq\|\cdot\|\).
            \(|\langle x, z'\rangle|=\left|\sum x_i z'_i\right| = \|x\|_p\). L'inégalité de \textsc{Hölder} nous assure que cette valeur est le sup.}
        \end{q}
        \begin{q}{d}
            Démontrer que \(\|\cdot\|_p\) vérifie l'inégalité triangulaire (inégalité de \textsc{Minkowski})
            \boxans{Ce résultat s'obtient à l'aide de l'inégalité triangulaire sur la valeur absolue sur chaque terme de la suite.}
        \end{q}
        \begin{q}{e}
            Démontrer alors que l'application \(\|\cdot\|_p\) est une norme pour \(p\geq 1\).
            \boxans{La séparation et la positivité sont triviales, l'homogénéité provient de la linéarité de la somme et l'inégalité triangulaire de la question précédente.}
        \end{q}
    \end{q}
    \begin{q}{2}
        Démontrer que pour \(p\in ]0,1[\) l'application \(\|\cdot\|_p\) n'est pas une norme.
        \boxans{L'application \(t\mapsto t^p\) est maintenant concave et plus
        convexe, elle est donc au dessus de ses cordes et ainsi ne vérifie pas
        l'inégalité triangulaire.}
    \end{q}
    \begin{q}{3}
        On rappelle que \(\|\cdot\|_\infty\) est une norme sur \(\K^n\)
        \begin{q}{a}
            Justifier que la formule variationelle est encore vraie si \(p=1\) ou \(p=+\infty\).
            \boxans{On commence par remarquer l'inégalité de \textsc{Young} est vérifier pour \(p=1, q=+\infty\) ce qui, de par la suite de questions, donne le résultat souhaité.}
        \end{q}
        \begin{q}{b}
            Montrer que, pour tout \(p\in [1,+\infty]\) et \(x\in\K^n\) : \(\|x\|_\infty\|\leq\|x\|_p\leq n^\frac{1}{p}\|x\|_\infty\)
            \boxans{Le plus grand écart entre les boules se situe aux coins
            de la boule de \(\|\cdot\|_\infty\) donc pour \(x=(1,1,\dots)\)
            or pour ce \(x\) on a \(\|x\|_p=n^\frac{1}{p}\) ce qui livre
            l'inégalité de droite, celle de gauche s'obtenant par inclusion
            de boule.}
        \end{q}
        \begin{q}{c}
            Soit \(x\in\K^n\) montrer que l'application \(p\mapsto \|x\|_p\) admet une limite en \(+\infty\) égale à \(\|x\|_\infty\)
            \boxans{Ce résultat s'obtient en appliquant le théorème des gendarmes au résultat précédent.}
        \end{q}
        \begin{q}{d}
            Prouver que les normes \(\left(\|\cdot\|_p\right)_{p\geq 1}\) sont deux à deux équivalentes.
            \boxans{Les normes sont équivalentes car on est en dimension finie.}
        \end{q}
    \end{q}
\end{exo}

\begin{exo}
    Soient \(n\in\N^*\) et \(\mnr\) le \(\R\)-ev des matrices de taille \(n\times n\)
    \begin{q}{1}
        On note \(\left\langle\cdot,\cdot\right\rangle\) l'application de \(\mnr^2\)
        dans \(\R\) telle que \(\langle A,B\rangle=\Tr(^tAB)\). Démontrer que l'application
        ainsi définie est un produit scalaire sur \(\mnr\) de norme associée \(\|\cdot\|\).
    \end{q}
    \begin{q}{2}
        Démontrer que l'application \(\|\cdot\|_\infty : A \mapsto \sup |a_{i,j}|\)
        est une norme sur \(\mnr\).
    \end{q}
    \begin{q}{3}
        Montrer que les normes sont équivalentes et trouver les meilleures constantes pour l'encadrement.
    \end{q}
\end{exo}

\begin{exo}
    Soient \(V\) un \(\K\)-ev et une application \(N\) de \(V\) dans \(\R_+\) telle que :
    \begin{enumerate}
        \itt Pour tout \(\left(t,v\right)\in\K\times V, N(tv)=|t|N(v)\)
        \itt Pour tout \(v\in V, N(v)=0 \Leftrightarrow v = 0_V\)
    \end{enumerate}
    On note \(B_N = \{v\in V\mid N(v)\leq 1\}\). Prouver que \(N\) est une norme si et seulement si \(B_N\) est convexe.
    \boxans{
        \begin{enumerate}
            \itt Supposons \(B_n\) convexe alors soient \(x,y\in B_n\) tels que \(N(x)=N(y)=1\)
            alors \(\frac{x+y}{2}\in B_n\) par convexité donc \(N(x+y)\leq 2 = N(x)+N(y)\).
            L'absolue homogénéité de \(N(\cdot)\) assure alors que l'application
            vérifie l'inégalite triangulaire et est donc une norme
            \itt La réciproque est évidente, si \(N(\cdot)\) est une norme alors elle vérifie
            l'inégalité triangulaire et \(B_N\) est alors convexe de par l'argument ci-dessus.
        \end{enumerate}
    }
\end{exo}

\end{document}