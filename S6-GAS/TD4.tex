\documentclass[french]{article}
\usepackage{../exercices}
\usetikzlibrary{arrows,automata, positioning}
\begin{document}

% jeremy.ledent@irif.fr
\begin{center}
    \huge{\textbf{S6- Langages et expressions rationnelles}}
\end{center}

Dans tous les exercices de cette page (sauf les deux premiers), les minuscules sont les terminaux,
les majuscules sont les non-terminaux, et \(Z\) est l'axiome. Le \(\$\) correspondra à la fin
de fichier. SI votre grammaire n'a pas un axiome avec une unique production du genre \(Z\to S\$\),
il faut l'ajouter.

\begin{exo}
    Montrer que chacun des langages ci-dessous est algébrique.
    \begin{q}{1}
        \(L_1=\{a^nb^m|n=2m\}\)
    \end{q}
    \begin{q}{2}
        \(L_2=\{a^nb^m|n\neq 2m\}\)
    \end{q}
    \begin{q}{3}
        \(L_3=\{a^nb^m|n\leq m+3\}\)
    \end{q}
    \begin{q}{4}
        \(L_4=a^nb^m|n\neq m-1\)
    \end{q}
    \begin{q}{5}
        \(L_5=\{a^nb^m| \frac{n}{2}\leq m\leq \frac{3n}{2}\}\)
    \end{q}
\end{exo}

\begin{exo}
    Pour chaque grammaire suivante trouvez la grammaire réduite correspondante. On rappelle
    que dans une grammaire réduite tous les non-terminaux sont accessibles et productifs.
    \begin{q}{1}
    \end{q}
    \begin{q}{2}
    \end{q}
\end{exo}

\begin{exo}
    Soit la grammaire suivante, définie sur le vocabulaire terminal \(\{b,e,i,;,\$\}\):
    \begin{q}{1}
        \(Z\to S\$\quad S\to bT\quad T\to i;T|e\) cette grammaire est-elle LL(1) ? Quel
        langage génère-t-elle ?
    \end{q}
    \begin{q}{2}
        On ajoute la règle \(T\to ST\). La grammaire obtenue est-elle LL(1) ? Donner une dérivation
        de \("bi;bi;i;ebei;i;e\$"\).
    \end{q}
\end{exo}

\begin{exo}
    Soit la grammaire suivante, définie sur le vocabulaire terminal \(\{:,=,i,e,;\}\)
    \[Z\to S\$\quad S\to V:=e|LS L\to i:\quad V\to i\]
    Cette grammaire est-elle LL(1) ? Pourquoi ? Sinon proposez une grammaire LL(1) qui
    engendre le même langage.
\end{exo}

\begin{exo}
    Soit la grammaire suivante, définie sur le vocabulaire terminal \(\{\$,a,b,c,d\}\) :
    \[Z\to S\$\quad S\to X|Yc\quad X\to Xa|\eps\quad Y\to Yb|d\]
    Décrivez le langage engendré par cette grammaire. Cette est-elle LL(1) ? Sinon proposez une
    grammaire LL(1) qui engendre le même langage. (On raplle qu'avec les définitions actuelles, on
    cherchera d'abord à se débarasser de \(\eps\))
\end{exo}

\begin{exo}
    Donner les ensembles FIRST\(_1\) des membres droits de chacune des règles ci-dessous.
    \[Z\to S\$\]
\end{exo}

\end{document}
