\documentclass[french]{article}
\usepackage{../exercices}
\usetikzlibrary{arrows,automata, positioning}
\begin{document}

% jeremy.ledent@irif.fr
\begin{center}
    \huge{\textbf{S6- Langages et expressions rationnelles}}
\end{center}

\begin{exo}
    Soit G la grammaire donnée par les règles suivantes, où les capitales sont des
    non-terminaux, les autres caractères sont des terminaux, et S est l'axiome de G.
    \[S\to F\quad S\to(F+S)\quad F\to a\]
    \begin{q}{1}
        Donner une dérivation gauche pour les mots \((a+a)\) et \((a+(a+a))\).
        Donner une dérivation droite pour le mot \((a+a)\).
    \end{q}
    \begin{q}{2}
        Les mots \((a+a))\) et \(((a+a)+a)\) sont-ils dans le langage engendré pa
        \(G\) ?
    \end{q}
    \begin{q}{3}
        Donner les arbres de dérivation des mots que la question 1.
    \end{q}
\end{exo}

\begin{exo}
    Décrire les langages engendrés par les grammaires suivantes :
    \begin{q}{1}
        \(S\to\eps|aSa|bSb|a|b\)
        \boxans{Tous les palindromes sur l'alphabet \(\{a,b\}\).}
    \end{q}
    \begin{q}{2}
        \(S\to\eps|[S]S\)
        \boxans{Tous les mots bien parenthésées sur l'alphabet \(\{[,]\}\)}
    \end{q}
    \begin{q}{3}
        \(S\to\eps|(S| (S)S\)
        \boxans{Tous les mots bien prenthésés, avec potentiellement des ouvrantes en trop.}
    \end{q}
    Ces grammaires sont-elles ambigües ?
\end{exo}

\begin{exo}
    Montrer que les langages suivants sont algébrique en donnant une grammaire
    qui les engendre.
    \begin{q}{1}
        \(L_1=\{a^nb^n|n\in\N\}\)
        \boxans{\(S\to \eps|aSb\) génère le langage.}
    \end{q}
    \begin{q}{2}
        \(L_2=\{a^nb^m\mid m\geq n\geq 0\}\)
        \boxans{\(S\to\eps|aS|aSb\) génère le langage}
    \end{q}
    \begin{q}{3}
        \((*)L_3=\{a^nb^*c^n\mid n\in\N\}\)
        \boxans{\(S\to\eps|aSc|R \quad R\to \eps|bR\) génère le langage}
    \end{q}
    \begin{q}{4}
        \(L_4=\{a^nb^mc^k\mid n=m+k\}\)
        \boxans{\(S\to \eps|aSc|R\quad R\to aRb\to\eps\) génère le langage.}
    \end{q}
    \begin{q}{5}
        \(L_5=\{a^nb^mc^md^n\mid n,m\in\N\}\)
        \boxans{\(S\to R|aSd\quad R\to\eps|bcb\) génère le langage.}
    \end{q}
    \begin{q}{6}
        \(L_6=\{\omega\in\{a,b\}^*\mid \textrm{pas un palindrome}\}\)
        \boxans{\(S\to aSa|bSb|aRb|bRa\quad R\to aRa|bRb|aRb|bRa|b|a|\eps\) génère le langage.}
    \end{q}
    L'un d'entre eux est-il rationnel ?
    \boxans{Non, soit par besoin d'un compteur (donc facilement démontrable par le lemme de
    l'étoile) soit comme contre langage d'un classique de la non rationalité.}
\end{exo}

\begin{exo}
    La grammaire suivante engendre le langage \(\{a^nb^nc^m\mid n,m\in\N\}\cup
    \{a^mb^nc^n\mid m,n\in\N\}\).
    \[S\to UV|XY \quad U\to\eps|aUb\quad V\to\eps|cV\quad X\to\eps|aX\quad Y\to\eps|bYc\]
    Montrer que cette grammaire est ambigüe.
    \boxans{La règle S sépare les deux ensembles comme s'ils étaient distincts, ils ne
    le sont pas, en effet si \(m=n\) alors on a clairement deux arbres de dérivation.}
\end{exo}

\begin{exo}
    On considère la grammaire suivante qui engendre les expressions polnaises inverses:
    \[S\to E\quad E\to EE+|EE*|EE-|EE/|v|i\]
    Donner une dérivation gauche, une dérivation droite et un arbre de dérivation
    pour l'expression : \(iv+i*\).
\end{exo}
\end{document}